\section*{Conclusion}

Dans ce chapitre, nous avons présenté des modèles de programmation
linéaire en nombres entiers pour le \CECSP. Pour ce problème, trois
modèles sont présentés,  
un modèle utilisant une discrétisation de l'horizon de temps et deux
modèles basés sur une représentation des événements pertinents du
problème. 

Enfin, des améliorations de ces modèles sont proposées dans la
dernière partie du chapitre. Ces améliorations sont basées sur le
raisonnement énergétique, la mise en place d'inégalités valides et des
études polyédrales. De plus, les avantages et inconvénients de chacun
des modèles sont décrits ce qui permet de justifier l'intérêt de ces
améliorations. 

Des résultats numériques évaluant les performances de ces formulations
ainsi que l'intérêt de chaque amélioration sur diverses instances du
\CECSP~et du \RCPSP~feront l'objet d'un paragraphe dans le chapitre
portant sur les expérimentations (cf. Chapitre~\ref{sec:expe}). 