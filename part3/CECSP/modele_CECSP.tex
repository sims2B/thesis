%!TeX root =../main_file.tex
\section{Modèles de programmation linéaire en nombres entiers pour le
\CECSP}

Le \CECSP ~et le \RCPSP~étant deux problèmes relativement proches -
utilisation d'une ou plusieurs ressources cumulatives, de fenêtres de
temps - la modélisation du \CECSP~à l'aide de la programmation
linéaire en nombres entiers est une approche naturelle pour
sa résolution. 

De plus, le théorème~\ref{theo_LPM_CECSP} nous assurant que toute solution
$S$ du \CECSP~peut être transformée en une solution $S'$ ayant la
propriété que $\forall i \in \A,\ b_i(t)$ soit constante par morceaux,
le problème peut être modélisé à l'aide de la programmation linéaire
mixte. Nous pouvons aussi remarquer que, comme la transformation
proposée par le théorème n'augmente ni la date de début des activités,
ni leur date de fin, ni leur consommation totale de ressource, tout modèle
ayant un objectif impliquant seulement la minimisation de ces trois
quantités sera valide pour le \CECSP.

Dans cette section, nous commençons par décrire un modèle indexé par
le temps, puis, nous présentons deux modèles à événement. Ces trois
modèles sont adaptés des modèles pour le \RCPSP~présentés dans la
section~\ref{sec:PLNE_ordo_res} et sont présentés
dans~\cite{Nattaf_ORSpectrum}.  


\subsection{Modèle indexé par le temps}

La première formulation proposée est une formulation indexée par
le temps. Comme pour le modèle à temps discret du le \RCPSP,
ici, l'horizon de temps est divisé en intervalles de taille unitaire. Le
calcul d'une borne supérieure $T$ sur la durée totale du projet
est ici trivial: il suffit de prendre la plus grande date échue,
i.e. $T=\max_{i \in \A} \LE$. L'horizon de temps, noté
$\H$, peut donc être défini par: $\H=\{0,\dots,T\}$.
Notons que, par translation, nous pouvons toujours supposer que
$\min_{i\in \A} \ES=0$.

Une des principales différences entre le modèle à temps discret du
\RCPSP~et de celui du \CECSP~ repose sur le fait que, dans le
second, la durée des activités n'est pas connue à l'avance et doit
être déterminée par le modèle. De ce fait, nous devons définir,
pour chaque activité $i \in \A$ et pour chaque instant $t
\in \H$, deux variables binaires $x_{it}$ et $y_{it}$ pour
modéliser le début et la fin des activités. La variable $x_{it}$
(resp. $y_{it}$) prendra la valeur $1$ si et seulement si
l'activité $i$ commence (finit) à l'instant $t$.

Une seconde différence entre les deux modèles est le calcul des 
fenêtres de temps est effectué de manière triviale pour le \CECSP. 
En effet, pour une activité $i$,
la fenêtre correspondante à sa date de début est $[\ES,\LS]$, avec
$\LS=\LE-W_i/f_i(\bmax)$, et celle correspondante à sa date de fin
$[\EE,\LE]$, avec $\EE=\ES+W_i/f_i(\bmax)$. Enfin, l'ajout des
activités fictives marquant le début et la fin du projet n'est 
pas nécessaire ici.

\paragraph{Fonction de rendement identité}

Dans le cas où la fonction de rendement de chaque activité est la
fonction identité, i.e. $f_i(b_i(t))=b_i(t),\ \forall i \in \A$, nous
définissons une variable $b_{it}$, pour chaque activité $i \in \A$ et
pour chaque période de temps $t \in \H$, qui représentera la quantité
de ressource consommée par l'activité $i$ dans la période de temps
$t$, i.e. dans l'intervalle $[t,t+1]$.

Ceci conduit à la formulation suivante:
{\small
 \begin{align} &\text{min }
\sum_{i\in \A}\sum_{t \in \H} b_{it}
\label{obj_CECSP_TI}\\ &\sum_{t=\ES}^{\LS} x_{it} = 1 & &\forall i
\in \A \label{start_CECSP_TI}\\ &\sum_{t=\EE}^{\LE} y_{it} =
1 & & \forall i \in \A
\label{end_CECSP_TI}\\&\left(\sum_{\tau=\ES}^{t} x_{i\tau}
-\sum_{\tau=\ES+1}^{t} y_{i\tau}\right)\bmin \le b_{it} & &
\forall t \in \{\ES,\dots,\LE-1\},\ \forall i \in \A
\label{bmin_CECSP_TI}\\ &\left(\sum_{\tau=\ES}^{t} x_{i\tau} -
\sum_{\tau=\ES+1}^{t} y_{i\tau}\right)\bmax\ge b_{it}& & \forall t
\in \H ,\ \forall i \in \A \label{bmax_CECSP_TI}\\
&\sum_{t=\ES}^{\LE} b_{it} \ge W_i & & \forall i \in \A
\label{nrj_CECSP_TI}\\ &\sum_{i \in \A} b_{it} \le R & &
\forall t \in \H \label{res_CECSP_TI}\\ &b_{it} = 0 & &
\forall t \not\in \{\ES,\dots,\LE-1\},\ \forall i \in \A
\label{consNul_CECSP_TI}\\ &x_{it} = 0 & & \forall t \not\in
\{\ES,\dots,\LS\},\ \forall i \in \A \label{twx_CECSP_TI}\\
&y_{it} = 0 & & \forall t \not\in \{\EE,\dots,\LE\},\ \forall i
\in \A \label{twy_CECSP_TI}\\ & b_{it} \ge 0 & & \forall t
\in \H; \forall i \in \A \label{Bb_CECSP_TI}\\
&x_{it}\in \{0,1\},\ y_{it} \in \{0,1\} & & \forall t \in {\cal
H},\ \forall i \in \A \label{Bxy_CECSP_TI} \end{align}
}
Ce modèle s'inspire grandement du modèle
de~\cite{ALR}, la principale différence résidant
dans la considération de fenêtres de temps plus fine pour les
variables, i.e. $[\ES,\LS]$ pour $x_{it}$ au lieu de
$[\ES,\LE-1]$ et $[\EE,\LE]$ pour $y_{it}$ au lieu de
$[\ES-1,\LE]$. Ces fenêtres de temps pouvant être raffinées à
l'aide d'heuristiques, de la programmation par contraintes (cf.
section~\ref{sec:adjustment_tw}) ou par le biais d'autres méthodes,
ces modifications peuvent améliorer grandement les performances du
modèle.

Dans cette formulation, l'objectif est décrit
par~\eqref{obj_CECSP_TI}. Ici, l'objectif est de minimiser la
consommation totale de la ressource durant tout le projet. Cet
objectif a moins d'impact dans le cas où la fonction de rendement
de chaque activité est la fonction identité mais se révèle très
pertinent pour d'autres types de fonctions de rendement.
Cependant, comme la formulation à temps discret ne nous permet pas
de s'assurer que la quantité de ressource consommée par une
activité soit exactement égale à la quantité nécessaire pour
apporter l'énergie requise, i.e. $\sum_{t=\ES}^{\LE} b_{it} \ge
W_i$, cet objectif reste valide dans le cas des fonctions
identités.

Si l'on souhaite modifier l'objectif pour minimiser la date de fin
du projet, il suffit d'introduire une variable $C_{max}$ ainsi que
la contrainte suivante: 
\begin{equation} 
C_{max} \ge \sum_{i \in \A} ty_{it} \quad \forall t \in \H \notag
\end{equation} 
et alors l'objectif s'écrit facilement comme: 
\begin{equation}
\text{min } C_{max} \notag
\end{equation} 
Les contraintes~\eqref{start_CECSP_TI} et~\eqref{end_CECSP_TI}
stipulent que chaque activité est exécutée une et une seule fois
durant la durée du projet. En effet, grâce à ces contraintes, chaque
activité n'a qu'une et une seule date de début et une et une seule
date de fin.

Les contraintes~\eqref{bmin_CECSP_TI} et~\eqref{bmax_CECSP_TI}
permettent de s'assurer que la consommation instantanée de chaque
activité est bien comprise entre $\bmin$ et $\bmax$ durant toute sa
durée d'exécution. Pour s'en assurer, il suffit de remarquer que
$\sum_{\tau=\ES}^{t} x_{i\tau} -\sum_{\tau=\ES+1}^{t}y_{i\tau}=1$ si
et seulement si l'activité $i$ est en cours à l'instant $t$. Les
autres configurations possibles sont $\sum_{\tau=\ES}^{t} x_{i\tau}
-\sum_{\tau=\ES+1}^{t} y_{i\tau}=$0 et $\sum_{\tau=\ES}^{t} x_{i\tau}
-\sum_{\tau=\ES+1}^{t} y_{i\tau}=-1$. Dans le premier cas, ceci
signifie que l'activité n'a pas encore débuté et les inégalités
imposent que $b_{it} \le 0$ et donc $b_{it}=0$. Dans le second cas,
l'équation devient $-\bmax \ge b_{it}$, ce qui est impossible. Notons
que ce dernier cas nous assure ainsi qu'une activité ne peut finir avant
d'avoir commencé. Cependant, afin de renforcer la relaxation linéaire
du modèle, des contraintes spécifiques, empêchant que le début d'une
activité ne se produise avant sa fin, peuvent être ajoutées. Ces
inégalités sont de la forme: 
\begin{equation}
 \sum_{\tau=\ES}^{t} x_{i\tau}
-\sum_{\tau=\ES+1}^{t} y_{i\tau} \le 0 \quad \forall t \in \{\ES,\dots,\LE-1\}
\end{equation}
L'intérêt de l'ajout de ces équations sera discuté dans la
section~\ref{section:expe}. 

La contrainte~\eqref{nrj_CECSP_TI} modélise le fait qu'une activité
doit recevoir au moins la quantité d'énergie requise par la donnée du
problème.

La contrainte~\eqref{res_CECSP_TI} limite la quantité de ressource
utilisée simultanément à la capacité de cette dernière.

Enfin, les contraintes~\eqref{consNul_CECSP_TI},~\eqref{twx_CECSP_TI}
et~\eqref{twy_CECSP_TI} fixent la valeur des variables à zéro en
dehors de leurs fenêtres de temps, i.e. $[\ES,\LE]$ pour $b_{it}$,
$[\ES,\LS]$ pour $x_{it}$ et $[\EE,\LE]$ pour $y_{it}$.

Cette formulation possède $2nT$ variables binaires, $nT$ variables
continues et au plus $3n+T*(5n+1)$ contraintes.

\paragraph{Fonction de rendement affine}

Dans le cas où les fonctions de rendement étaient toutes la fonction
identité, l'énergie apportée à une activité durant une certaine
période était égale à la quantité de ressource consommée par celle-ci
durant la même période de temps. Or, lorsque les fonctions de
rendement deviennent affines, ce n'est plus le cas.  Pour s'assurer
qu'une activité $i$ reçoive bien l'énergie requise, nous devons
déclarer une nouvelle variable qui permettra de mesurer l'énergie
apportée à cette dernière durant la période $t$, $w_{it},\ \forall
(i,t) \in \A \times \H$. Dans le cas
précédent, nous avions $w_{it}=b_{it}$.

Nous devons donc vérifier, pour chacune des contraintes du modèle
précédent impliquant la variable $b_{it}$, si cette dernière
modélisait une quantité de ressource ou d'énergie, i.e. si nous devons
la remplacer par $w_{it}$. De plus, une contrainte liant les variables
$b_{it}$ et $w_{it}$ devra être rajoutée pour assurer la bonne
conversion entre les deux quantités.

La contrainte~\eqref{bmin_CECSP_TI} (resp.~\eqref{bmax_CECSP_TI})
représentant la quantité minimale (resp. maximale) de ressource qu'une
activité doit consommer à chaque instant de son exécution, reste donc
inchangée. De même, la contrainte~\eqref{res_CECSP_TI}, modélisant la
contrainte sur la capacité de la ressource, n'a pas besoin d’être
modifiée.

La contrainte~\eqref{nrj_CECSP_TI}, qui s'assure qu'une activité
reçoive bien la quantité d'énergie requise doit être réécrite en
utilisant la variable $w_{it}$. Ce qui nous donne l'inégalité
suivante:
\begin{equation} 
\sum_{t=\ES}^{\LE} w_{it} \ge W_i\quad \forall i \in \A \tag{\ref{nrj_CECSP_TI}'} \label{nrj2_CECSP_TI}
\end{equation} 
De plus, nous devons ajouter une contrainte permettant de lier la
quantité de ressource consommée par une activité et la quantité
d'énergie apportée à celle-ci. Nous ajoutons donc la contrainte
suivante au modèle:
\begin{equation}
w_{it}=a_ib_{it}+c_i\left(\sum_{\tau=\ES}^t
x_{i\tau}-\sum_{\tau=\ES+1}^t y_{i\tau}\right) \quad \forall t\in
\H,\ \forall i \in \A \label{conv_CECSP_TI}
\end{equation} 

Cette contrainte nous permet de modéliser la fonction de rendement
$f_i,\ \forall i \in \A$. En effet, $\left(\sum_{\tau=\ES}^t
x_{i\tau}-\sum_{\tau=\ES+1}^t y_{i\tau}\right) $ est égale à $1$ si et
seulement si l'activité $i$ est en cours à l'instant $t$.  Dans ce cas
là, la valeur de l'énergie apportée à $i$ est bien
$w_{it}=a_ib_{it}+c_i$. Le second cas se produit quand l'activité $i$
n'est pas en cours à $t$. Dans ce cas, la
contrainte~\eqref{bmax_CECSP_TI} nous dit que $b_{it}=0$ et donc
$w_{it}=0$.

Le modèle possède donc $2nT$ variables binaires, $2nT$ variables
continues et au plus $3n+T*(6n+1)$ contraintes.

\paragraph{Fonction de rendement concaves affines par morceaux}

Lorsque les fonctions de rendements sont des fonctions concaves
affines par morceaux, nous utilisons aussi la variable $w_{it}$ pour
représenter l'énergie reçue par l'activité $i$ dans la période $t$. La
contrainte~\eqref{nrj2_CECSP_TI} est utilisée pour modéliser le fait
que cette activité doive recevoir une quantité d'énergie $W_i$ durant
son exécution. 

Pour s'assurer de la bonne conversion entre la quantité de ressource
consommée par une activité $i$ dans une période $t$ et la quantité
d'énergie reçue par cette dernière dans la même période,
l'inégalité~\eqref{conv_CECSP_TI} est remplacée par l'inégalité
suivante: 
\begin{equation}
w_{it} \le a_{ip}b_{it} + c_{ip}\left(\sum_{\tau=\ES}^t
x_{i\tau}-\sum_{\tau=\ES+1}^t y_{i\tau}\right) \quad  \forall i \in
\A,\ \forall p \in \P_i,\ \forall t \in \H 
\label{conv_CECSP_TI}
\end{equation}

Notons que l'utilisation d'une inégalité peut impliquer la variable
$w_{it}$ ne représente pas exactement la quantité d'énergie apportée à
$i$ dans la période $t$. Cependant, la contrainte nous assure que
$w_{it} \le f_i(b_{it})$. De ce fait, à une solution optimale donnée
par le modèle correspond toujours une solution optimale du \CECSP à
date de début et de fin entières. En effet, suposons que dans une
solution optimale renvoyée par le PLNE, il existe $(i,t)$ tel que
$w_{it} < f_i(b_{it})$, alors, deux cas sont possibles:
\begin{itemize}
\item $\bmin \ge f_i^{-1}(w_{it})$ et dans ce cas là, $b_{it}$ peut
  prendre la valeur $f_i^{-1}(w_{it})$ et cette solution aura un coût,
  i.e. une consommation totale de ressource, plus faible que la
  solution précédente et ceci contredit l'optimalité de la solution.
\item $\bmin < f_i^{-1}(w_{it})$ et dans ce cas là, $b_i(t)$ prend la
  valeur $\bmin$ et $et_i=\min\{t \in\mathbb{N}\ |\ \int_{\tau=st_i}^{t}
  f_i(b_i(t))dt \le W_i\}$. Ici, deux cas sont possibles: le premier
  correspond au cas où $et_i=\{t\ | \ y_{it}=1\}$ et la solution
  renvoyée par le PLNE est optimale; dans le second cas $et_i<\{t\ |\
  y_{it}=1\}$ mais alors, $y_{it-1}$ peut prendre la valeur $1$ et
  $y_{it}$ la valeur $0$ et cette solution est de plus faible coût que
  la solution précédente et ceci contredit l'optimalité de la solution
renvoyée par le PLNE.
\end{itemize}


Le modèle ainsi défini possède alors $2nT$ variables binaires, $2nT$
variables continues et au plus $3n + (5n +nP+1)T$ contraintes, avec
$P=max_{i\in A}|\P_i|$.  

\subsection{Modèles à événement}

Dans cette section, nous proposons deux formulations à événement pour
le \CECSP. Ces deux formulations sont grandement inspirées des
formulations start/end et on/off du \RCPSP, présentées dans la
section~\ref{sec:PLNE_ordo_res} et issues
de~\cite{modele_RCPSP}. Comme dans le cas du \RCPSP, ces modèles se
justifient par leur nombres polynomial de variables et de contraintes
qui peut s'avérer très pertinent dans la cas de grand horizon de
temps. Pour le \CECSP, un argument supplémentaire vient s'ajouter pour
le \CECSP. En effet, il peut arriver qu'une instance de ce problème ne
possède que des solutions à valeur non entières et ce, même si toutes
les données sont entières (voir exemple~\ref{exemple_NE},
page~\pageref{exemple_NE}).


Comme dans le cas de la formulation à temps discret, ici, les dates de
fin des activités ne sont plus totalement définies par leur date de
début. De ce fait, dans les formulations à événements du \CECSP, nous
avons besoin de modéliser deux types d'événements, les débuts et les
fins des activités, soit, au plus, $2n$ événements. Ces événements,
indexés par l'ensemble $\E=\{1,\dots,2n\}$, sont représentés par un
ensemble de variables continues, notées $t_e$.

Nous rappelons aussi les notations suivantes: $T=\max_{i \in \A}
\LE$ est une borne supérieure sur la date de fin du projet,
$[\ES,\LS]$, avec $\LS= \LE-W_i/f_i(\bmax)$ est la fenêtre de temps
dans laquelle l'activité $i$ peut débuter et, de même, $[\EE,\LE]$,
avec $\EE=\ES + W_i/f_i(\bmax)$, la fenêtre de temps durant laquelle
l'activité $i$ peut finir.

Nous allons présenter, en premier lieu, la formulation start/end du
\CECSP~avec fonctions de rendement identité dont nous dériverons 
les cas de fonctions de rendement affines et affines par morceaux.
Ensuite, nous présenterons la formulation on/off du même problème et
de ses dérivés.

\subsubsection{Modèle start/end}

Dans le modèle start/end, deux variables binaires $x_{ie}$ et
$y_{ie}$, $\forall (i,e) \in \A\times \E$, 
servent à affecter les dates des différents événements, modélisés par
les variables $t_e$, aux débuts et fins des activités. En effet, la
variable $x_{ie}$ prendra la valeur $1$ si et seulement si l'événement
$e$ correspond au début de l'activité $i$, i.e. l'activité $i$
commence à la date $t_e$, et sera égale à $0$ sinon. De même, la
variable $y_{ie}$ sera égale à $1$ si et seulement si l'événement $e$
correspond à la fin de l'activité $i$, et vaudra $0$ sinon.

\paragraph{Fonction de rendement identité}

Dans la cas où toutes les fonctions de rendement sont la fonction
identité, un seul ensemble de variables est nécessaire pour modéliser
la consommation de la ressource. En effet, comme la quantité de
ressource consommée par une activité $i$ est égale à la quantité
d'énergie apportée à cette même activité, il n'est pas utile de
définir une variable représentant cette quantité d'énergie.

Un ensemble de variables supplémentaire, $b_{ie}$, est donc
défini. Cet ensemble de variables représente la quantité de ressource
consommée par une activité $i$ entre les dates $t_e$ et
$t_{e+1}$. Notons que nous avons seulement besoin de définir $b_{ie},\
\forall i \in \A$ et $\forall e \in \E \setminus\{2n\}$. Ceci nous permet de
modéliser le problème de la façon suivante:
{\small
\begin{align}
& \text{min } \sum_{i\in A}\ \sum_{e\in \E\setminus\{2n\}} b_{ie} 
\label{obj_CECSP_SE}\\
&t_e \le t_{e+1} & & \forall e \in
\E\setminus\{2n\} \label{ordre_CECSP_SE}\\
 &\sum_{e\in \E} x_{ie} =1 & & \forall i \in
A \label{start_CECSP_SE}\\
 &\sum_{e\in \E} y_{ie} =1 & & \forall i \in A \label{end_CECSP_SE}\\
 &x_{ie}\ES \le t_e & & \forall i \in A,\ \forall e \in
\E \label{twx1_CECSP_SE}\\
 &t_e \le x_{ie}\LS+ \left(1-x_{ie}\right)T & & \forall i \in A,\
\forall e \in \E \label{twx2_CECSP_SE}\\
 &t_e \ge y_{ie}\EE & & \forall i \in A ,\ \forall e \in \E 
 \label{twy1_CECSP_SE}\\
 &\LE y_{ie} + \left(1-y_{ie}\right)T \ge t_e & & \forall i \in A,\ \forall e \in \E 
 \label{twy2_CECSP_SE}\\
 &\sum_{i \in A} b_{ie} \le R \left(t_{e+1}- t_e\right) & & 
 \forall e \in \E\setminus\{2n\} \label{res_CECSP_SE}\\
 &t_f \ge t_e +  \left(x_{ie} + y_{if} -1\right) W_i/f_i(\bmax) & & \forall i \in A,\ 
 \forall e,f \in \E\ ; f\ge e \label{dur_CECSP_SE}\\
 &\sum_{e\in \E\setminus\{2n\}} b_{ie} = W_i  & & \forall i \in A 
 \label{nrj_CECSP_SE}\\
&b_{ie} \ge \bmin \left(t_{e+1}-t_e\right) 
- M \left(1 - \sum_{f=0}^e x_{if} +\sum_{f=0}^e y_{if}\right) 
& &  \forall i \in A,\ \forall e \in \E\setminus\{2n\} \label{bmin_CECSP_SE}\\
&b_{ie} \le \bmax  \left(t_{e+1} - t_e\right) & &
\forall i \in A,\ \forall e \in \E\setminus\{2n\} \label{bmax_CECSP_SE}\\
& \left(\sum_{f=0}^{e} x_{if} - \sum_{f=0}^e y_{if}\right)M\ge b_{ie} & &
 \forall i \in A,\ \forall e \in \E\setminus\{2n\} \label{res0_CECSP_SE}\\
&t_e \ge 0 & & \forall e \in \E \label{eq36}\\
& b_{ie} \ge 0 & & \forall i \in A,\ \forall e \in \E\setminus\{2n\} 
\label{Bb_CECSP_SE}\\
&x_{ie} \in \{0,1\},\ y_{ie} \in \{0,1\} & & 
\forall i \in A,\ \forall e \in \E \label{eq39}
\end{align}
}
L'objectif, décrit par l'équation~\eqref{obj_CECSP_SE}, est de
minimiser la consommation totale de la ressource, i.e. la somme des
consommations de toutes les tâches. On peut facilement modifier cet
objectif afin de minimiser la date de fin du projet en remplaçant
l'objectif par:
\begin{equation}
\text{min } t_{|\E|} \notag
\end{equation}
La contrainte~\eqref{ordre_CECSP_SE} ordonne les événements. La
contrainte~\eqref{start_CECSP_SE} (resp.~\eqref{end_CECSP_SE})
s'assure que chaque activité ne commence (resp. finisse) qu'une et une
seule fois. En effet, chaque début (resp. fin) d'activité ne peut être
associé qu'à un et un seul événement.
 
Les contraintes~\eqref{twx1_CECSP_SE} et~\eqref{twx2_CECSP_SE}
garantissent que la date d'un événement correspondant au début d'une
activité soit bien comprise dans sa fenêtre de temps, i.e. dans
l'intervalle $[\ES,\LS]$. En effet, si l'événement $e$ correspond au
début de l'activité $i$, i.e. $x_{ie}=1$, alors la première inégalité
devient $\ES \le t_e$ et la seconde $t_e \le \LS$. Les autres
configurations donnent: $0 \le t_e \le T$ et ceci est trivialement
vérifié. De même, les contraintes~\eqref{twy1_CECSP_SE}
et~\eqref{twy2_CECSP_SE} s'assurent que, si un événement $e$
correspond à la fin d'une activité $i$, alors $t_e$ est bien compris
entre $\EE$ et $\LE$.

La contrainte~\eqref{res_CECSP_SE} s'assure que, dans la période de
temps comprise entre $t_e$ et $t_{e+1}$, la quantité de ressource
consommée n’excède pas la capacité de cette dernière.

La contrainte~\eqref{dur_CECSP_SE} modélise le fait que si l'activité
$i$ commence à l'événement $e$ et se termine à l'événement $f$, alors
les dates correspondant à ces deux événements sont au moins séparées
par la durée de cette tâche. Or, comme nous ne connaissons pas cette
durée, nous utilisons une borne inférieure: $W_i/f_i(\bmax)$. 

La contrainte~\eqref{nrj_CECSP_SE} modélise le fait qu'une activité
doit recevoir au moins la quantité d'énergie requise par la donnée du
problème.

Les contraintes~\eqref{bmin_CECSP_SE} et~\eqref{bmax_CECSP_SE}
permettent de s'assurer que la consommation instantanée de chaque
activité est bien comprise entre $\bmin$ et $\bmax$ durant toute sa
durée d'exécution. Pour s'en assurer, il suffit de remarquer que, dans
la première inégalité, $\sum_{f=0}^{e} x_{if}-\sum_{f=0}^{e}y_{if}=1$
si et seulement si l'activité $i$ est en cours entre les événements
$e$ et $e+1$.  L'inégalité devient donc $b_{ie} \ge
\bmin(t_{e+1}-t_e)$. Les autres configurations possibles donnent
$b_{ie} \ge \bmin(t_{e+1}-t_e)-M$, avec $M$ une constante
suffisamment grande pour dépasser $\bmin(t_{e+1}-t_e)$.

La contrainte~\eqref{res0_CECSP_SE} garantit que si l'activité $i$
n'est pas en cours entre les événements $e$ et $e+1$, alors
$b_{ie}=0$. En effet, soit $M$ est une constante suffisamment grande
pour servir de borne supérieure à $b_{ie}$, alors la contrainte
s'écrit $M\ge b_{ie}$ quand $\sum_{f=0}^{e} x_{if}
-\sum_{f=0}^{e}y_{if}=1$, i.e. quand l'activité $i$ est en cours, et
s'écrit $0\ge b_{ie}$ quand $\sum_{f=0}^{e}
x_{if}-\sum_{f=0}^{e}y_{if}=0$, i.e. quand l'activité $i$ n'est pas en
cours. Notons aussi que ces contraintes empêchent qu'une activité ne
puisse commencée avant d'avoir finie. De plus, pour renforcer la
formulation, nous pouvons ajouter les contraintes~\eqref{xby_RCPSP_SE}
modélisant explicitement la fait qu'un événement ``début d'activité''
se produit forcément avant l'événement ``fin d'activité'' lui
correspondant. L'intérêt de l'ajout de telles contraintes sera discuté
dans le chapitre~\ref{section:expe}.

Cette formulation possède $4n^2$ variables binaires, $2n^2+n$
variables continues et $2n^3+13n^2+4n-2$ contraintes.

\paragraph{Fonction de rendement affine}

Pour adapter cette formulation au cas des fonctions de rendement
affines, nous avons, dans un premier temps, besoin de définir une
variable $w_{ie},\ \forall i \in \A$ et $\forall e \in \E
\setminus\{2n\}$. Cette variable sert à modéliser la quantité
d'énergie apportée à une activité $i$ durant l'intervalle de temps
$[t_e,t_{e+1}]$. De plus, nous devons identifier les contraintes
impliquant la variable $b_{ie}$ pour lesquelles cette variable doit
être remplacée par la variable $w_{ie}$. En d'autres termes, nous
devons identifier les contraintes traitant de la ressource et les
contraintes liées à l'énergie.

La contrainte~\eqref{res_CECSP_SE}, modélisant la contrainte de
capacité de la ressource, n'est pas modifiée. De même, les
contraintes~\eqref{bmin_CECSP_SE} et~\eqref{bmax_CECSP_SE},
garantissant la contrainte de consommation maximale et minimale de la
ressource, restent inchangées. Enfin, la
contrainte~\eqref{res0_CECSP_SE}, s'assurant de la non-consommation de
ressource des activités en dehors de leur exécution, ne nécessite
aucun changement.

La contrainte~\eqref{nrj_CECSP_SE} cependant, étant en charge du
respect de l'apport requis en énergie de chaque activité, doit être
réécrite en utilisant la variable adéquate, $w_{ie}$. Ceci donne
l'inégalité suivante:
\begin{equation} 
\sum_{e\in \E\setminus\{2n\}} w_{ie} = W_i\quad \forall i \in \A
\tag{\ref{nrj_CECSP_SE}'}
\label{nrj2_CECSP_SE}
\end{equation} 

Il nous reste à écrire les contraintes permettant de lier les deux
variables, i.e.  de s'assurer de la bonne conversion
ressource/énergie. Nous ajoutons donc les contraintes suivantes au
modèle:
\begin{align}
  &w_{ie} \le a_ib_{ie}+c_i(t_{e+1}-t_e) & & \forall i \in A,\ \forall
  e \in \E \setminus\{2n\} \label{conv1_CECSP_SE}\\
     &w_{ie} \le W_i (\sum_{f=0}^{e} x_{if} -\sum_{f=0}^{e} y_{if}) &
    & \forall i \in A,\ \forall e \in {\cal
      E}\setminus\{2n\} \label{conv2_CECSP_SE}\\
& w_{ie} \ge 0 & & \forall i \in A,\ \forall e \in {\cal
      E}\setminus\{2n\}
\label{Bw_CECSP_SE}
\end{align}

La contrainte~\eqref{conv1_CECSP_SE} permet de modéliser l'énergie
apportée à l'activité $i$ en fonction de la quantité de ressource
consommée par cette même activité dans l'intervalle
$[t_e,t_{e+1}]$. Notons d'abord l'utilisation d'une inégalité et non
d'une égalité. Ceci est dû au fait que, lorsque l'activité n'est pas
en cours entre $t_e$ et $t_{e+1}$, i.e. $b_{ie}=0$, la contrainte
devient $w_{ie} \le c_i(t_{e+1}-t_e)$. Dans le cas d'une égalité, ceci
aurait été faux puisqu'on devrait avoir $w_{ie}=0$. Le rôle de la
seconde contrainte est donc de s'assurer que lorsque l'activité n'est
pas en cours, on ait bien $w_{ie}=0$. En effet, lorsque l'activité
n'est pas en cours, la contrainte devient $w_{ie}\le 0$ et dans le cas
contraire, l'inégalité s'écrit $w_{ie} \le W_i$.

La seconde remarque qui peut être faite est que, puisque la
contrainte~\eqref{conv1_CECSP_SE} utilise une inégalité, $w_{ie}$ peut
ne pas être réellement égale à la quantité d'énergie apportée à $i$
entre $t_e$ et $t_{e+1}$. En fait, grâce à
l'objectif~\eqref{obj_CECSP_SE}, ceci ne peut arriver.

\begin{theo}
  \label{th:conv}
  Soit $S$ une solution optimale du modèle décrit
  par~\eqref{obj_CECSP_SE} -- \eqref{dur_CECSP_SE},
  \eqref{nrj2_CECSP_SE} et
  \eqref{bmin_CECSP_SE}--\eqref{Bw_CECSP_SE}. Alors $S$ vérifie la
  condition suivante: $\forall i \in \A,\ \forall e \in \E\setminus\{2n\},\
  w_{ie}=f_i\left(\frac{b_{ie}}{t_{e+1}-t_e}\right)(t_{e+1}-t_e)$.
\end{theo}

\begin{proof}
Pour prouver le théorème, commençons par remarquer que la
contrainte~\eqref{conv1_CECSP_SE} implique
$w_{ie} \le f_i\left(\frac{b_{ie}}{t_{e+1}-t_e}\right)(t_{e+1}-t_e)$. En effet, 
\begin{align*}
  f_i\left(\frac{b_{ie}}{t_{e+1}-t_e}\right)(t_{e+1}-t_e)&
  =\left(a_i\frac{b_{ie}}{t_{e+1}-t_e}+c_i\right)(t_{e+1}-t_e)\\  
   &=a_ib_{ie}+c_i(t_{e+1}-t_e)
\end{align*}
Il nous reste donc à montrer que $w_{ie}\ge
f_i\left(\frac{b_{ie}}{t_{e+1}-t_e}\right)(t_{e+1}-t_e)$. Par l'absurde,
supposons que $\exists i \in \A,\ \exists e \in \E\setminus\{2n\}$
tel que $w_{ie} < 
f_i\left(\frac{b_{ie}}{t_{e+1}-t_e}\right)(t_{e+1}-t_e)$. Soit $\S_C$
l'ensemble des solutions du \CECSP~et soit  $c :\S_C
\rightarrow \mathbb{R}$ la fonction qui associe à une solution $S$ sa
consommation de ressource $c(S)$. Nous allons créer une nouvelle
solution $S'$ qui vérifie que $c(S)>c(S')$, ce qui invalidera
l'optimalité de $S$.

Pour cela, nous considérons la solution du \CECSP~associée à
$S$. Notons $\widehat{S}$ cette solution. $\widehat{S}$ est obtenue à
partir de $S$ de la manière suivante, $\forall (i,e)
\in \A \times \E$: 
\begin{itemize}
\item si $x_{ie}=1$ alors $st_i$ est fixé à $t_e$,
\item si $y_{ie}=1$ alors $et_i$ est fixé à $t_e$,
\item $b_i(t)=\frac{b_{ie}}{t_{e+1}-t_e}, \forall t \in [t_e,t_{e+1}]$. 
\end{itemize}

$S'$ est définie de la façon suivante: 
\begin{itemize}
\item $st_i'=st_i$,
\item $et_i'=\min(t \in \H | \int_{st_i}^t f_i(b_i(t)) = W_i)$,
\item $b_i'(t)=\left\{ \begin{array}{ll}
                         b_i(t)& \text{si } st_i' \le t \le et_i'\\ 
                         0 & \text{sinon}
                       \end{array}
                     \right.$
\end{itemize}
Clairement, $S'$ est une solution pour le \CECSP. Nous montrons
maintenant que $c(S) = c(\widehat{S})> c(S')$. Pour cela, nous allons montrer qu'il
existe une tâche $i$ telle que $et_i > et_i'$ et donc
$\int_{st_i}^{et_i} b_i(t)dt > \int_{st_i'}^{et_i'} b_i(t)dt$. De ce
fait, nous aurons bien que $\sum_{i \in \A} \int_{st_i}^{et_i}
b_i(t)dt > \sum_{i \in \A}\int_{st_i'}^{et_i'} b_i(t)dt$. 

On sait que $\exists (i,e) \in \A \times \E$ tel que:
\begin{align*}
  &w_{ie} < f_i\left(\frac{b_{ie}}{t_{e+1}-t_e}\right)(t_{e+1}-t_e)\\  
  \Rightarrow &\sum_{e \in  \E} w_{ie} < \sum_{e \in \E} f_i\left(
                \frac{b_{ie}}{t_{e+1}-t_e}\right)(t_{e+1}-t_e)\\ 
  \Rightarrow & W_i=\sum_{e \in  \E} w_{ie} < \sum_{e \in \E}
                f_i(b_i(t))(t_{e+1}-t_e)=\int_{st_i}^{et_i}
                f_i(b_i(t))dt \\
  \Rightarrow & \min\left(t \in \H |
                \int_{st_i}^{t}f_i(b_i(t))dt\right) = et_i' < et_i                 
\end{align*}

Donc nous avons bien $et_i > et_i'$ et ceci achève la démonstration.
\end{proof}

Notons que pour d'autres  objectifs ce peut ne pas être valide. Mais, la
solution obtenue par le modèle peut toujours être transformée en une
solution du \CECSP~en temps polynomial. Pour cela, il suffit de suivre
la transformation proposée dans la preuve du théorème~\ref{th:conv}.

Le modèle ainsi défini possède $4n^2$ variables binaires, $4n^2$
variables continues et $2n^3+17n^2+2n-2$ contraintes.

\paragraph{Fonction de rendement affine par morceaux}

Lorsque les fonctions de rendement sont des fonctions concaves
affines par morceaux, nous utilisons aussi la variable $w_{ie}$ pour
représenter l'énergie reçue par l'activité $i$ dans la période
$[t_e,t_{e+1}]$. La contrainte~\eqref{nrj2_CECSP_SE} est utilisée pour
modéliser le fait que cette activité doive recevoir une quantité
d'énergie $W_i$ durant son exécution.

Pour s'assurer de la bonne conversion entre la quantité de ressource
consommée par une activité $i$ dans une période $[t_e,t_{e+1}]$ et la
quantité d'énergie reçue par cette dernière dans la même période,
l'inégalité~\eqref{conv1_CECSP_SE} est remplacée par l'inégalité
suivante:
\begin{equation}
w_{ie} \le a_{ip}b_{ie} + c_{ip}\left( t_{e+1} - t_e\right) \quad  
\forall i \in \A,\ \forall p \in \P_i,\ \forall e \in \E\setminus\{2n\}  
\label{conv2_CECSP_SE}
\end{equation}

Notons que la même argumentation que celle de la preuve du
théorème~\ref{th:conv} permet de nous assurer de la bonne conversion
entre énergie et ressource du modèle. 

Le modèle ainsi défini possède $4n^2$ variables binaires, $4n^2$
variables continues et $2n^3+ (15 + 2P)n^2+\left(1-P\right)n-2$
contraintes, avec $P=max_{i\in A}|\P_i|$.

\subsubsection{Modèle on/off}
\label{sssection:OO_CECSP}
Dans le modèle on/off, une variable binaire $z_{ie}$ se charge
d'assigner à chaque événement les dates de début et de fin des
activités. La variable $z_{ie}$ prendra la valeur $1$ si et seulement
si l'activité $i$ est en cours d'exécution dans la période
$[t_e,t_{e+1}]$ et sera égale à $0$ sinon. Ceci permet de diviser par
deux le nombre de variables binaires utilisées par rapport à la
formulation start/end.

\paragraph{Fonction de rendement identité}

Dans le cas où les fonctions de rendement sont la fonction identité,
nous définissons la variable $b_{ie}$ qui modélise la quantité de
ressource consommée par l'activité $i$ entre les dates $t_e$ et
$t_{e+1}$. Comme pour le modèle start/end, nous avons seulement besion
de définir la variable $b_{ie},\ \forall i \in \A$ et $\forall e \in
\Em$. Ceci conduit à la formulation suivante: 
{\small
\begin{align}
& \text{min } \sum_{i\in A}\ \sum_{e\in \E\setminus
    \{2n\}}b_{ie}
 \label{obj_CECSP_OO}\\ 
&t_e \le t_{e+1} & &\forall e \in \E\setminus
 \{2n\} \label{ordre_CECSP_OO}\\
&\sum_{e \in \E} z_{ie} \ge 1 & & \forall i \in {\cal
   A}\label{start_CECSP_OO}\\
& \ES z_{ie}\le t_e \le \LS(z_{ie}-z_{i,e-1})+(1-(z_{ie}-z_{i,e-1}))T
 & & \forall e \in \E\setminus \{1\},\ \forall i \in {\cal
   A}\label{twx_CECSP_OO}\\
&\EE(z_{i,e-1}-z_{ie})\le t_e & & \forall e \in \E\setminus
 \{1\},\ \forall i \in \A\label{twy1_CECSP_OO}\\
&t_e \le \LE(z_{i,e-1}-z_{ie})+(1-(z_{i,e-1}-z_{ie}))T & & \forall e
 \in \E\setminus \{1\},\ \forall i \in {\cal
   A}\label{twy2_CECSP_OO}\\
&t_f\ge t_e +((z_{ie}-z_{i,e-1})-(z_{if}-z_{i,f-1})-1)W_i/f_i(\bmax) &
 & \forall e,f \in \E,\ f>e\neq 1,\ \forall i \in {\cal
   A} \label{dur_CECSP_OO}\\
&\sum_{e'=1}^{e} z_{ie'} \le e(1-(z_{ie}-z_{i,e-1})) & & \forall e \in
       \E\setminus \{1\},\ \forall i \in \A
\label{preem1_CECSP_OO}\\
&\sum_{e'=e}^{2n} z_{ie'} \le (2n-e)(1+(z_{ie}-z_{i,e-1})) & & \forall
e \in \E\setminus \{1\},\ \forall i \in \A
\label{preem2_CECSP_OO}\\
&\sum_{i\in \A} b_{ie} \le R(t_{e+1}-t_e) & & \forall e \in
      \E\setminus \{2n\}\label{res_CECSP_OO}\\
&\sum_{e \in \E\setminus \{2n\}} b_{ie} = W_i & & \forall i \in
        \A\label{nrj_CECSP_OO}\\
&b_{ie} \ge \bmin(t_{e+1}-t_e) - M(1-z_{ie}) & & \forall e \in {\cal
  E}\setminus \{2n\}, \ \forall i \in \A\label{bmin_CECSP_OO}\\
&b_{ie} \le \bmax(t_{e+1}-t_e) & &\forall e \in \E\setminus
\{2n\},\ \forall i \in \A\label{bmax_CECSP_OO}\\
&z_{ie}M\ge b_{ie} & & \forall e\in \E\setminus \{2n\},
\ \forall i \in \A\label{res0_CECSP_OO}\\
&t_e \ge 0 & & \forall e \in \E \label{eq54}\\
& b_{ie} \ge 0 & & \forall i \in A,\ \forall e \in \E\setminus
\{2n\}
 \label{Bb_CECSP_OO}\\
&z_{ie} \in \{0,1\} & & \forall i \in A,\ \forall e \in {\cal
   E} \label{eq57}
\end{align}
}
Dans cette formulation, l'objectif est décrit
par~\eqref{obj_CECSP_OO}. Ici, l'objectif est de minimiser la
consommation totale de la ressource durant tout le projet. Si 
l'on souhaite minimiser la date de fin du projet, l'objectif s'écrit 
facilement comme: 
\begin{equation}
\text{min } t_{|\E|} \notag
\end{equation} 

La contrainte~\eqref{ordre_CECSP_OO} joue le même rôle que dans la
formulation start/end, à savoir, ordonner les événements.

La contrainte~\eqref{start_CECSP_OO} stipule que chaque activité doit
être ordonnancée une fois dans toute la durée du projet.
 
La contrainte~\eqref{twx_CECSP_OO} s'assure que la date d'un événement
correspondant au début d'une activité soit bien comprise dans sa
fenêtre de temps.  En effet, si l'événement $e$ correspond au début de
l'activité $i$, i.e. $z_{ie}=1$ et $z_{i,e-1}=0$ , l'inégalité devient
alors: $\ES \le t_e \le \LS$. Notons que pour le cas $e=1$,
$z_{i,e-1}-z_{ie}$ est remplacé par $z_{ie}$. De même, les
contraintes~\eqref{twy1_CECSP_OO} et~\eqref{twy2_CECSP_OO}
garantissent qu'un événement correspondant à la fin d'une activité
soit bien compris entre $\EE$ et $\LE$. Ici, il n'est pas
n'écessaire de considérer le cas $e=1$ car cet événement ne peut pas
correspondre à la fin d'une activité. 

La contrainte~\eqref{dur_CECSP_OO} assure une séparation suffisante
entre un événement correspondant au début d'une activité et un
événement correspondant à la fin de cette même activité. Ici, comme
nous ne connaissons pas la durée d'une activité, nous utilisons une
borne inférieure sur cette dernière. Pour le cas où $e=1$, il suffit
de remplacer $z_{i,e-1}-z_{ie}$ par $z_{ie}$.

Les contraintes~\eqref{preem1_CECSP_OO} et~\eqref{preem2_CECSP_OO},
appelées {\it contraintes de contiguïté}, assure la non-préemption des
activités. Une preuve formelle de la validité de ces contraintes peut
être trouvée dans~\cite{modele_RCPSP}. Ici aussi, le cas $e=1$ ,est
traité en remplaçant $z_{i,e-1}-z_{ie}$ par $z_{ie}$.

La contrainte~\eqref{res_CECSP_OO} représente la limite de capacité de
la ressource tandis que la contrainte~\eqref{nrj_CECSP_OO} s'assure
que chaque activité reçoit bien la quantité d'énergie requise par la
donnée du problème.

Les contraintes~\eqref{bmin_CECSP_OO},~\eqref{bmax_CECSP_OO} et
\eqref{res0_CECSP_OO} - permettant respectivement de modéliser les
contraintes de consommation minimale, de consommation maximale et de
consommation nulle en dehors de l'intervalle d'exécution de $i$ - se
déduisent directement des
contraintes~\eqref{bmin_CECSP_SE},~\eqref{bmax_CECSP_SE} et
\eqref{res0_CECSP_SE}. Dans un premier temps, remarquons que nous
avons la relation suivante:
\begin{equation}
\label{SEversOO}
z_{ie} = \sum_{f=1}^e x_{if} -  \sum_{f=1}^e y_{if} \quad \forall i
\in \A,\ \forall e\in \Em
\end{equation}
En effet, $z_{ie}$ vaut $1$ si et seulement si l'activité $i$ est en
cours entre $t_e$ et $t_{e+1}$, c'est à dire si l'activité a commencé à
l'événement $e$ ou avant, i.e. $\sum_{f=1}^ex_{if}=1$, et si elle
ne finit pas à l'événement $e$ ou avant, i.e. $\sum_{f=1}^ey_{if}=
0$. Dans ce cas, nous avons bien $\sum_{f=1}^ex_{if} -
\sum_{f=1}^ey_{if}= 1 = z_{ie}$. Les
contraintes~\eqref{bmin_CECSP_OO},~\eqref{bmax_CECSP_OO} et
\eqref{res0_CECSP_OO} s'obtiennent donc facilement à partir
de~\eqref{bmin_CECSP_SE},~\eqref{bmax_CECSP_SE} et
\eqref{res0_CECSP_SE} en remplaçant $\sum_{f=0}^{e}
x_{if}-\sum_{f=0}^{e}y_{if}$ par $z_{ie}$.


Le modèle ainsi défini possède $2n^2-n$ variables binaires, $2n^2+n$
variables continues et $2n^3+13n^2-n-2$ contraintes.    

\paragraph{Fonction de rendement affine}

Pour modéliser la conversion de la ressource en énergie, nous
définissons une variable $w_{ie}$, $\forall (i,e) \in \A \times \Em$,
représentant la quantité d'énergie 
apportée à l'activité $i$ dans l'intervalle $[t_e,t_{e+1}]$. Puis,
nous identifions les contraintes du précédent modèle qui ont besoin
d'être réécrites à l'aide de cette variable. Ici, seule la
contrainte~\eqref{nrj_CECSP_OO} est concernée. La réécriture donne la
contrainte suivante:
\begin{equation}
\sum_{e\in \E} w_{ie}=W_i \quad \forall i \in \A
\tag{\ref{nrj_CECSP_OO}'}
\end{equation}

Dans un second temps nous écrivons les contraintes liant les variables
$b_{ie}$ et $w_{ie}$:
\begin{align}
  &w_{ie} \le a_ib_{ie}+c_i(t_{e+1}-t_e) & & \forall i \in A,\ \forall
  e \in \E \setminus\{2n\} \label{conv1_CECSP_OO}\\
  &w_{ie} \le W_i z_{ie} & & \forall i \in A,\ \forall e \in \E\setminus\{2n\} \label{conv2_CECSP_OO}\\
  & w_{ie} \ge 0 & & \forall i \in A,\ \forall e \in \E\setminus\{2n\}
  \label{Bw_CECSP_OO}
\end{align}

Notons que la même argumentation que celle de la preuve du
théorème~\ref{th:conv} permet de nous assurer de la bonne conversion
entre énergie et ressource du modèle. 

Le modèle ainsi défini possède $2n^2-n$ variables binaires, $4n^2$
variables continues et $2n^3+17n^2-3n-2$ contraintes.

\paragraph{Fonction de rendement affine par morceaux}

Lorsque les fonctions de rendements sont des fonctions concaves
affines par morceaux, nous utilisons aussi la variable $w_{ie}$ pour
représenter l'énergie reçue par l'activité $i$ dans la période
$[t_e,t_{e+1}]$. La contrainte~\eqref{nrj2_CECSP_SE} est utilisée pour
modéliser le fait que cette activité doive recevoir une quantité
d'énergie $W_i$ durant son exécution.

Pour s'assurer de la bonne conversion entre la quantité de ressource
consommée par une activité $i$ dans une période $[t_e,t_{e+1}]$ et la
quantité d'énergie reçue par cette dernière dans la même période,
l'inégalité~\eqref{conv1_CECSP_SE} est remplacée par l'inégalité
suivante:
\begin{equation}
w_{ie} \le a_{ip}b_{ie} + c_{ip}\left( t_{e+1} - t_e\right) \quad  
\forall i \in \A,\ \forall p \in \P_i,\ \forall e \in \Em
\label{conv2_CECSP_SE}
\end{equation}

Le modèle ainsi défini possède $2n^2-n$ variables binaires, $4n^2$
variables continues et $2n^3+(15 + 2P)n^2+\left(2-P\right)n-2$
contraintes, avec $P=max_{i\in A}|\P_i|$.    
