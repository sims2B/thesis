\section*{Conclusion}

Dans ce chapitre, nous avons rappelé les principales notions de la
programmation linéaire, puis de la programmation linéaire mixte et en
nombres entiers. Nous avons ensuite vu plusieurs méthodes de
modélisation pour un problème d'ordonnancement avec contraintes de
ressource: le \RCPSP. Pour ce problème, nous avons vu deux types de
modèles. Le premier fait partie des modèles indexés par le temps. Ces
derniers possèdent de relativement bonnes relaxation, ce qui en fait
un des types de modèles les plus efficaces pour le \RCPSP~et en
particulier sur les célèbres instances de la PSPLIB~\cite{PSPLIB}. 

Cependant, ces modèles souffrent de limitations dues à leur taille,
dépendante de l'horizon de temps du projet. De ce fait, quand
l'horizon de temps des instances devient grand, la tailles des modèles
associés augmente et leurs performances décroissent. Pour pallier ce
problème, un autre type de modèle est mis en place: les modèles à
événements. L'avantage de ces modèles est leur taille, polynomiale en
fonction de la taille de l'instance et surtout indépendante de
l'horizon de temps. Dans~\cite{modele_RCPSP}, Kone {\it et al.}
comparent les performances de ces modèles avec ceux indexés par le
temps sur des instances ayant de grand horizon de temps. De plus, ces
modèles permettent la modélisation de problèmes continus. De ce fait,
nous allons adapter ces modèles dans le cadre du \CECSP~dans le
chapitre suivant. De plus, un modèle indexés par le temps est aussi
présenté, permettant d'avoir des solutions plus rapidement (mais pas
forcément optimales).