%!TeX root =../main_file.tex 
\section{Application à l'ordonnancement de projet sous contraintes de
  ressource}   
\label{sec:PLNE_ordo_res}

La facilité de modélisation des problèmes d'ordonnancement sous
contraintes de ressource à l'aide de la programmation linéaire en a
fait une des principales méthodes de résolution pour ces
derniers. De ce fait, de nombreuses techniques de modélisation ont été
développées et la littérature sur ce sujet est très vaste. Dans cette
section, nous nous intéressons principalement aux modèles développés
pour le problème d'ordonnancement de projet sous contraintes de
ressource, le \RCPSP. Ces modéles sont regroupées en plusieurs
familles: 
\begin{itemize}
\item les formulations indexées par le temps (ou à temps discret)
sont, en général, des formulations purement discrètes. Ces
formulations ont la particularité d'avoir les meilleures relaxations
linéaires au détriment du nombre de
variables~\cite{CAVT,ex_RCPSP_discret};
\item les formulations avec variables de séquencement sont des
formulations qui ont l'avantage d'être plus compactes que celles
indexées par le temps mais possèdent de moins bonnes relaxations
linéaires que ces dernières~\cite{AVT,AMR};
\item les formulations à événements, plus résentes, possèdent aussi de
faibles relaxations linéaires mais ont un nombre de variables
polynomiales, moins élevé que pour les formulations à temps
discret~\cite{modele_RCPSP}.
\end{itemize}

Dans cette section, nous présentons quelques modèles de
programmation linéaire mixte développés pour les problèmes
d'ordonnancement sous contraintes de ressource. Nous nous intéressons
particulièrement aux modèles décrits dans~\cite{modele_RCPSP} 
pour le \RCPSP. En effet, ce sont ces modèles qui seront adaptés dans
le cas du \CECSP. Nous commençons donc, dans un premier temps, par
présenter les modèles indexés par le temps puis, dans un second temps, nous
présentons les modèles à événement. Les modèles avec variables de
séquencement ne sont pas traités ici. Des exemples de telles
formulations dans le cadre du \RCPSP~sont décrites dans~\cite{ADN}.

\subsection{Modèles indexés par le temps}
\label{sec:time_RCPSP}

Une des méthodes permettant la modélisation des problèmes
d'ordonnacement de projet consiste à discrétiser l'horizon de
temps. Ces modèles de programmation linéaire en nombres entiers
indexés par le temps, aussi appelés modèles à temps discret, ont été
largement étudiés dans le cadre des problèmes d'ordonnancement en
général. Ceci étant principalement dû à leur relativment forte
relaxation linéaire et à la facilité d’extension à de nombreux
objectifs et contraintes.

Dans ces formulations, l'horizon de temps est discrétisé en période de
temps, i.e en intervalle, généralement unitaire, et un
ensemble de variables est défini pour modéliser le satus d'une
activité $i$ dans la période $t$ - en cours, commencé ou terminé. Pour
chaque activité $i$ et pour chaque période $t$ (discret), une variable
$x_{it} \in \{0,1\}$ est donc définie.

Dans la formulation présentée pour le \RCPSP, cette variable prendra
la valeur $1$ si et seulement si l'activité $i$ commence à au début de
la période $t$. Le nombre de ces variables dépend donc de la taille de
l'horizon de temps du problème, i.e. du nombre de périodes, qui peut
être, pour certains problèmes, aussi grand que la somme de toutes les
durées des activités. Le calcul d'une borne supérieure raisonnable
pour la durée du projet $T$ est donc indispensable.  Dans la suite,
nous dénoterons par ${\cal H}=\{0,\dots,T-1\}$ le nombre de période du
problème.

En pratique, pour chaque activité, nous calculons l'ensemble des
périodes de temps pendant lesquels l'activité peut commencer. Pour
cela, nous ajoutons deux activités fictives au problème: $0$ et
$n+1$. Ces activités ont les caractéristiques suivantes: leur
durée est égale à 0, elles ne consomment aucune ressource durant
leur exécution et l'activité $0$ (resp. $n+1$) doit être
ordonnancée avant (resp. après) toutes les autres activités. Ces
deux activités fictives vont nous permettre d'associer à chaque
activité $i$, une date de début au plus tôt $\ES$ et une date de
début au plus tard $\LS$. Donc, l'intervalle $[\ES,\LS]$ est la
fenêtre de temps pendant laquelle l'activité $i$ peut commencer.

Pour calculer ces fenêtres de temps, remarquons que, si chaque
arc $(i,j)$ du graphe des précédences $G$ est pondéré par
$p_i$, la date de début au plus tôt de $i$, $\ES$ peut prendre la
valeur du plus long chemin entre l'activité $0$ et l'activité $i$
et la date de début au plus tard $\LS$, $T$ moins la valeur du
plus long chemin entre $i$ et $n+1$.

Dans la suite, par ${\cal A}$ nous dénotons l'ensemble des
activités réelles du projet et par $V={\cal A} \cup \{0,n+1\}$
l'ensemble des activités réelles et fictives du projet.

Nous pouvons maintenant exhiber la formulation à temps discret
suivante~\cite{ex_RCPSP_discret}:
{\small \begin{align} &\text{min }
\sum_{t \in {\cal H}} tx_{n+1,t} \label{obj_RCPSP_discret}\\
&\sum_{t \in \H} tx_{jt} - \sum_{t \in {\cal H}} tx_{it} \ge p_i &
&\forall (i,j) \in E \label{prec_RCPSP_discret}\\ &\sum_{i \in
V}\,\sum_{\tau=\max(0,t-p_i+1)}^t r_{ik}x_{i\tau} \le R_k& &\forall t\in
{\cal H},\ \forall k \in {\cal R} \label{res_RCPSP_discret}\\
&\sum_{t\in {\cal H}} x_{it}=1& & \forall i \in V
\label{preem_RCPSP_discret}\\ & x_{it} \in \{0,1\} & &\forall i
\in V,\ \forall t \in {\cal H} \end{align}
 } 
Dans cette formulation l'objectif est donné par
l'expression~\eqref{obj_RCPSP_discret}. Cette expression signifie que
l'on cherche à minimiser la date de début de l'activité $n+1$. Or
comme cette activité est forcément placée à la fin de
l'ordonnancement du fait des relations de précédences, ceci
revient bien à minimiser la durée totale du projet.

La contrainte~\eqref{prec_RCPSP_discret} modélise les relations de
précédences entre les activités. En effet, toute paire d'activités
$i$ et $j$ ayant la propriété que $i$ doit précéder $j$ vérifie la
relation suivante: la date de début de $j$ est supérieure ou égale
à la date de début de $i$ plus sa durée, i.e. la date de fin de
$i$.

La contrainte~\eqref{res_RCPSP_discret} formalise les contraintes de
ressource. En effet, la contrainte s'assure que, pour chaque ressource
$k \in {\cal R}$, la somme des consommations instantanées sur $k$ des
activités en cours durant la période $t$ est bien inférieure ou égale à la
capacité de cette ressource.

La contrainte~\eqref{preem_RCPSP_discret} impose que chaque activité
n'ait qu'une et une seule date de début. Ceci peut aussi être vu comme
une contrainte de non-préemption pusique ceci revient à empêcher que
l'activité soit interrompue et redémarrée plus tard dans
l'ordonnancement final.

Le modèle possède donc $(n+2)(T+1)$ variables binaires et
$|E|+(T+1)*m+n$ contraintes.

Il existe d'autres formulations à temps discret comme la formulation
désagrégée de Christofides et al.~\cite{CAVT} ou la formulation basée
sur les ensembles réalisables de Mingozzi et al.~\cite{MMRB}.

\subsection{Modèles à événements}
\label{sec:event_RCPSP}

Nous allons présenter deux formulations à événements: une formulation
dite start/end et une formulation dite on/off.  Cette section montre,
dans un premier temps, l'intérêt de l'utilisation des modèles à
événement à la place des modèles à temps discret, en comparant
notamment le nombre de variables et de contraintes de ces derniers.
Dans un second temps, les modèles seront détaillés.

\subsubsection{Motivations des modèles à événement}

Parmi les formulations existantes pour le \RCPSP, celles basées sur des
modèles indexés par le temps sont les plus utilisées dû au fait de la
qualité (relative) de leur relaxation.  Cependant, comme la taille de
ces modèles, et donc leur complexité, dépend grandement de la taille
de l'horizon de temps du problème, ces modèles peuvent s’avérer moins
efficaces sur certains types d'instance~\cite{modele_RCPSP}. Pour
pallier ce problème, des formulations basées sur des {\it événements}
ont été mises en place.

La notion d'événement permet de caractériser seulement les dates
"importantes" de l'ordonnancement, i.e. les dates de début et de
fin de chaque activité. Ainsi, la considération de chaque date de
l'horizon de temps n'est plus nécessaire. Cela permet de réduire
le nombre de variables qui ne dépend alors que du nombre
d'activités à ordonnancer. De plus, pour le \RCPSP, seuls les
événements correspondant au début d'une activité ont besoin d'être
considérés. En effet, si on considère que les activités sont
ordonnancées au plus tôt, i.e. dès que les ressources requises sont
disponibles, alors la date de début de chaque activité correspond soit
à l'instant $0$, soit à la date de fin d'une tâche.

Les formulations à événements possèdent aussi la caractéristique
suivante: elles permettent de résoudre des instances pouvant
contenir des valeurs (de paramètre et/ou de solution) non
entières.

Dans~\cite{modele_RCPSP}, les auteurs montrent les limitations des
modèles indexés par le temps pour le \RCPSP. Les instances considérées
sont des instances pour lesquelles l'horizon de temps, i.e. $T$, est
très grand. Dans ce cas là, le nombre de variables et de contraintes
des modèles à temps discret est nettement supérieur à ceux des modèles
à événements. 

Par exemple, la famille d'instances KSD15\_d~\cite{theseOumar} possède
480 instances de $15$ activités et avec $4$ ressources dont l'horizon
de temps peut varier entre $187$ et $999$ et possédant entre $77$ et
$144$ relations de précédence. Pour le \RCPSP, le modèle à temps
discret peut alors posséder jusqu'à $17000$ variables binaires et
entre $4200$ contraintes. Les modèles à événements possèdent quant à
eux jusqu'à $500$ variables binaires, $16$ variables continues et
$4200$ contraintes pour le modèle start/end et jusqu'à $250$ variables
binaires, $16$ variables continues et $4000$ contraintes.

Considérant des instances de taille similaire pour le \CECSP~ avec
fonction de rendement affine, le modèle à temps discret peut alors
posséder jusqu'à $30000$ variables binaires, le même nombre de
variables continues et $90000$ contraintes.  Les modèles à événements
possèdent quant à eux jusqu'à $900$ variables binaires, le même nombre
de variables continues et $7200$ contraintes pour le modèle start/end
et jusqu'à $450$ variables binaires, $900$ variables continues et
$7100$ contraintes.

Dans ce cas là, la faiblesse de la relaxation linéaires des modèles à
événements est largement compensée par la grande différence de taille
avec les modèles indexés par le temps. 

Dans ce contexte, l'amélioration des modèles à événement s'avère être
une direction de recherche nécessaire dans l'élaboration de méthode de
résolution efficaces pour le \RCPSP. 



\subsubsection{Modèle start/end}

Dans cette formulation, la date de chaque événement est représentée
par une variable continue $t_e$, pour tout $e \in {\cal E}$, avec
${\cal E}$ l'ensemble des indices des événements. Afin d'associer
ces dates aux débuts et fins des activités, nous définissons, pour
toute activité $i \in \A$ et pour tout événement $e \in \E$, deux
variables binaires, $x_{ie}$ et $y_{ie}$, ayant les propriétés
suivantes: 
\begin{itemize}
 \item $x_{ie}=1$ si et seulement si
l'activité $i$ commence à l'événement $e$, c'est à dire commence à
la date $t_e$. 
\item $y_{ie}=1$ si et seulement si l'activité $i$
finit à l'événement $e$, c'est à dire finit à la date $t_e$.
\end{itemize} 
Notons que, les événements correspondant à la fin
d'une activité correspondent aussi au début d'une autre activité
(à l'exception de l'événement correspondant à la date de fin de la
dernière activité), le nombre d'événements à considérer est $n+1$. De
ce fait, nous avons comme ensemble d'événements ${\cal
  E}=\{1,\dots,n+1\}$. 

Nous définissons aussi une variable continue $b_{ek}$, pour chaque
événement $e \in \E$ et pour chaque ressource $k \in \R$, modélisant
la consommation totale de la ressource $k$ à l'événement $e$ (et
donc durant tout l'intervalle compris entre $t_e$ et $t_{e+1}$).
Ceci nous permet de présenter la formulation
suivante issue de~\cite{modele_RCPSP} et corrigées dans~\cite{ABKKLM}: 
{\small
\begin{align} 
& \text{min } t_{n+1}
\label{obj_RCPSP_SE}\\ 
& t_1 =0 & & \label{t0_RCPSP_SE}\\ 
&t_e \le
t_{e+1} & & \forall e \in {\cal E} \label{ordre_RCPSP_SE}\\
&\sum_{e\in {\cal E}} x_{ie} =1 & & \forall i \in \A
\label{start_RCPSP_SE}\\ 
&\sum_{e\in {\cal E}} y_{ie} =1 & &
\forall i \in \A\label{end_RCPSP_SE}\\ 
&x_{ie}\ES \le t_e \le \LS
x_{ie}+\LS[n+1]\left(1-x_{ie}\right) & & \forall i \in \A,\ \forall
e \in {\cal E} \label{twx_RCPSP_SE}\\ 
&\left(\LS+p_i\right)y_{ie}
+\left(1-y_{ie}\right)\LS[n+1] \ge t_e & & \forall i \in \A ,\
\forall e \in {\cal E} \label{twy1_RCPSP_SE}\\ 
& t_e \ge
y_{ie}\left(\ES+p_i\right) & & \forall i \in \A ,\ \forall e \in
{\cal E} \label{twy2_RCPSP_SE}\\ 
&b_{1k} - \sum_{i \in \A}
r_{ik}x_{i1}=0 & & \forall k \in {\cal R} \label{res0_RCPSP_SE}\\
& b_{ek} - b_{e-1,k} + \sum_{i\in \A}r_{ik}
\left(y_{ie}-x_{ie}\right)=0 & & \forall e \in {\cal E}
\setminus{1},\ \forall k \in {\cal R} \label{resCons_RCPSP_SE}\\ 
&
b_{ek} \le R_k & & \forall e \in {\cal E},\ \forall k\in {\cal R}
\label{res_RCPSP_SE}\\
 &t_f \ge t_e + (x_{ie} + y_{if} -1) p_i & &
\forall i \in \A,\ \forall (e,f) \in {\cal E}^2,\ f>e
\label{dur_RCPSP_SE}\\ 
&\sum_{f=1}^e y_{if} +\sum_{f=e}^n x_{if}
\le 1 & & \forall i \in \A,\ \forall e \in {\cal E}
\label{xby_RCPSP_SE}\\
 &\sum_{f=e}^n y_{if} - \sum_{f=1}^{e-1}
x_{jf} \le 1 & & \forall (i,j) \in E,\ \forall e \in {\cal E}
\label{prec_RCPSP_SE}\\ 
& \ES[n+1] \le t_n \le \LS[n+1]&
&\label{Btn_RCPSP_SE}\\ 
&t_e \ge 0 & & \forall e \in {\cal E}
\label{Bte_RCPSP_SE}\\ 
& b_{ek} \ge 0 & & \forall k \in {\cal R},\
\forall e \in {\cal E} \label{Bbek_RCPSP_SE}\\ 
&x_{ie} \in
\{0,1\},\ y_{ie} \in \{0,1\} & & \forall i \in \A,\ \forall e \in
{\cal E} \label{Bxy_RCPSP_SE} \end{align}
 }
Dans cette formulation, l'objectif est donné par
l'équation~\eqref{obj_RCPSP_SE}. En effet, comme pour la
formulation à temps discret, on cherche à minimiser la date de fin
du projet. Or, l'événement $n+1$ représente, par définition, le
dernier événement, c'est à dire le début de l'activité fictive $n+1$
et donc la fin de l'ordonnancment.

La contrainte~\eqref{t0_RCPSP_SE} fixe le premier événement à la date
0, tandis que la contrainte~\eqref{ordre_RCPSP_SE} ordonne les dates
des événements par ordre croissant.

La contrainte~\eqref{start_RCPSP_SE} (resp.~\eqref{end_RCPSP_SE})
stipule que chaque activité ne peut commencer (resp. finir) qu'une
et une seule fois. En effet, chaque début (resp. fin) d'activité
ne peut être associé qu'à un et un seul événement.
 
La contrainte~\eqref{twx_RCPSP_SE} garantit que la date d'un
événement correspondant au début d'une activité soit bien comprise
dans sa fenêtre de temps, i.e. entre sa date de début au plus tôt
$\ES$ et sa date de début au plus tard $\LS$. En effet, si
l'événement $e$ correspond au début de l'activité $i$, i.e.
$x_{ie}=1$, alors l'inégalité devient: $\ES \le t_e \le \LS$. Au
contraire, si $x_{ie}=0$, i.e. $e$ ne correspond pas au début de
$i$, alors l'inégalité devient: $0 \le t_e \le \LS[n+1]$ et, grâce
aux contraintes~\eqref{ordre_RCPSP_SE} et~\eqref{Btn_RCPSP_SE},
ceci est vrai pour tout $e \in {\cal E}$. De même, les
contraintes~\eqref{twy1_RCPSP_SE} et~\eqref{twy2_RCPSP_SE}
s'assurent que, si un événement $f$ correspond à la fin d'une
activité $i$, alors $t_f$ est bien compris entre $\ES+p_i$ et
$\LS+p_i$. 

Les contraintes~\eqref{res0_RCPSP_SE}
et~\eqref{resCons_RCPSP_SE} représentent les contraintes de
conservation des ressources. La contrainte~\eqref{res0_RCPSP_SE}
modélise la consommation totale de chaque ressource $k$ à
l'événement $1$. Pour une ressource $k$, cette quantité est égale
à la somme des consommations sur $k$ de chaque activité commençant
à l'événement $1$. La contrainte \eqref{resCons_RCPSP_SE} donne
la consommation totale de chaque ressource $k$ à l'événement $e$,
$b_{ek}$. En effet, pour chaque événement $e$, cette quantité est
égale à la consommation totale à l'événement $e-1$, i.e.
$b_{e-1,k}$, à laquelle on retranche les consommations des
activités finissant à l'événement $e$ et on ajoute les
consommations de chaque activité commençant à l'événement $e$.
Enfin, la contrainte~\eqref{res_RCPSP_SE} limite la demande totale
sur chaque ressource à sa capacité.

La contrainte~\eqref{dur_RCPSP_SE} assure que si l'activité $i$
commence à l'événement $e$ et se termine à l'événement $f$, alors
les dates correspondant à ces deux événements sont séparées par
au moins la durée de cette tâche, i.e. $p_i$. Dans ce cas,
l'inégalité s'écrit $t_f \ge t_e+p_i$, et pour toute autre
combinaison de $x_{ie}$ et $y_{if}$, on obtient soit $t_f \ge
t_e$ ou $t_f\ge t_e- p_i$ ce qui est vérifié par la
contrainte~\eqref{ordre_RCPSP_SE}.

La contrainte~\eqref{xby_RCPSP_SE} garantit qu'une activité ne peut se
terminer avant d'avoir commencé. Si l'activité $i$ finit entre
l'événement $1$ et $e$, i.e. $sum_{f=1}^{e} y_{if}=1$, alors elle ne
peut pas commencer entre l'événement $e$ et $n$, i.e. $sum_{f=e}^{n}
x_{if}=0$, et inversement.

Enfin, la contrainte~\eqref{prec_RCPSP_SE} modélise les relations
de précédences entre les activités: si une activité $i$, devant
précéder une activité $j$, finit à l'événement $e$ ou après, alors
l'activité $j$ ne peut commencer avant l'événement $e$, i.e.
$\sum_{f=e}^n x_{if}=1 \Rightarrow \sum_{f=1}^{e-1} y_{jf}=0$.

Le modèle possède $2n^2+n$ variables binaires, $n+1$ variables
continues et $(n+1)(m+n^2/2+|E|)+3n$ contraintes. Le nombre de
variables et de contraintes du modèle est donc bien polynomial en
fonction du nombre d'activités et de ressources.
 
\subsubsection{Modèle on/off} 

Dans le modèle on/off, comme pour la
formulation start/end, une variable $t_e$ qui représente la date
de chaque événement, pour tout $e \in {\cal E}$. Pour associer ces
dates aux débuts et fins de chaque activité, nous définissons, pour
chaque activité $i \in \A$ et pour chaque événement $e \in \E$, la
variable $z_{ie} \in \{0,1\}$. Cette variable prendra la valeur
$1$ si et seulement si l'activité $i$ est en cours durant
l'intervalle $[t_e,t_{e+1}]$, et 0 sinon. Ainsi, une activité
commencera (resp. finira) à l'événement $e$ si
$z_{ie}-z_{i,e-1}=1$ (resp.$z_{i,e-1}-z_{ie}=1$).

Notons que, comme dans le cas du modèle start/end, nous pouvons
limiter le nombre d'événements au début de chaque activité. Ainsi,
${\cal E}=\{1,\dots,n\}$; l'utilisation d'activités fictives
n'est, ici, plus nécessaire pour modéliser l'événement de fin de
la dernière activité. De plus, la variable $z_{ie}$ modélisant le fait
qu'un variable soit en cours entre $t_e$ et $t_{e+1}$, nous pouvons
limiter le nombre de ces variables en ne les définissant que pour les
événements $e \in \Em[n]$. Ceci conduit à la formulation
suivante~\cite{modele_RCPSP}: {\small \begin{align} & \text{min } C_{max}
\label{obj_RCPSP_OO}\\ &C_{max} \ge t_e+ (z_{ie}-z_{i,e-1})p_i & &
\forall e \in {\cal E}\setminus\{1\},\ \forall i \in {\cal A}
\label{Cmax_RCPSP_OO}\\ &t_1=0 & & \label{t0_RCPSP_OO}\\ &t_e \le
t_{e+1} & &\forall e \in {\cal E}\setminus\{n\}
\label{ordre_RCPSP_OO}\\ &\sum_{e \in \Em[n]} z_{ie} \ge 1 & &
\forall i \in {\cal A} \label{start_RCPSP_OO}\\ & \ES z_{ie}\le
t_e & & \forall e \in \Em[n],\ \forall i \in {\cal A}
\label{ES_RCPSP_OO} \\ & t_e \le
\LS(z_{ie}-z_{i,e-1})+(1-(z_{ie}-z_{i,e-1}))\LS[n+1] & & \forall e
\in \Em[n]\setminus{1},\ \forall i \in {\cal A}
\label{LS_RCPSP_OO}\\ &t_f\ge t_e
+\left((z_{ie}-z_{i,e-1})-(z_{if}-z_{i,f-1})-1\right)p_i & & \forall
e>f \in (\Em[1,n])^2,\ \forall i \in {\cal
A}\label{dur_RCPSP_OO}\\ &\sum_{f=1}^{e} z_{if} \le
e(1-(z_{ie}-z_{i,e-1})) & & \forall e \in \Em[1,n],\
\forall i \in {\cal A}\label{preem1_RCPSP_OO}\\ &\sum_{f=e}^{n-1}
z_{if} \le (n-e)(1+(z_{ie}-z_{i,e-1})) & & \forall e \in \Em[1,n],\ 
\forall i \in {\cal A}\label{preem2_RCPSP_OO}\\
&\sum_{i \in {\cal A}} r_{ik}z_{ie} \le R_k & &\forall e \in \Em[n]
,\ \forall k \in {\cal R} \label{res_RCPSP_OO}\\
&z_{ie}+\sum_{f=1}^e z_{jf} \le 1 + (1-z_{ie})e & & \forall e \in
\Em[1,n],\ \forall (i,j) \in E \label{prec_RCPSP_OO}
\end{align}
}
L'objectif, donné par l'équation~\eqref{obj_RCPSP_OO}, 
consiste à minimiser la date de fin du projet, ici
représentée par la variable $C_{max}$. La
contrainte~\eqref{Cmax_RCPSP_OO} s'assure que $C_{max}$ prend bien
la valeur de la date de fin du projet.

Les contraintes~\eqref{t0_RCPSP_OO} et~\eqref{ordre_RCPSP_OO} jouent
le même rôle que dans la formulation start/end, à savoir, ordonner
les événements.

La contrainte~\eqref{start_RCPSP_OO} stipule que chaque activité
doit être ordonnancée sur au moins un événement dans toute la durée du projet.
 
Les contraintes~\eqref{ES_RCPSP_OO} et~\eqref{LS_RCPSP_OO}
garantissent que la date d'un événement correspondant au début
d'une activité soit bien comprise dans sa fenêtre de temps, i.e.
entre sa date de début au plus tôt $\ES$ et sa date de début au
plus tard $\LS$. En effet, si l'événement $e$ correspond au début
de l'activité$i$, i.e. $z_{ie}=1$ et $z_{i,e-1}=0$ , l'inégalité
devient alors: $\ES \le t_e \le \LS$. Nous distinguons 3 autres
cas: \begin{itemize} \item {\it $z_{ie}=z_{i,e-1}=1$ :} l'activité
est en cours entre les événements $t_{e-1}$ ($z_{i,e-1}=1$) et
$t_{e+1}$ ($z_{ie}=1$). L'inégalité devient: $\ES\le t_e \le
\LS[n+1]$. Comme l'activité $i$ a déjà commencé à un événement $f
<e$, on a: $\ES \le t_f \le t_e$. L'autre inégalité est triviale.
\item {\it$z_{ie}=0$ et $z_{i,e-1}=1$:} l'activité se termine à
l'événement $e$. L'inégalité donne alors $0 \le t_e \le
2\LS[n+1]-\LS$. Or, $2\LS[n+1] -\LS \ge \LS[n+1]$. Donc
l'inégalité est vérifiée. \item {\it $z_{ie}=z_{i,e-1}=0$:}
l'activité n'est pas en cours entre les événements $t_{e-1}$ et
$t_{e+1}$. L'inégalité devient $0 \le t_e \le \LS[n+1]$ et est
trivialement vérifiée. \end{itemize}

La contrainte~\eqref{dur_RCPSP_OO} assure une séparation
suffisante, i.e. la durée de l'activité, entre un événement
correspondant au début d'une activité et un événement
correspondant à la fin de cette même activité. La validité de
cette contrainte suit la même logique que pour la
contrainte~\eqref{dur_RCPSP_SE} du modèle précédent.

Les contraintes~\eqref{preem1_RCPSP_OO}
et~\eqref{preem2_RCPSP_OO}, appelées {\it contraintes de
contiguïté}, assure la non-préemption des activités. La validité
de cette contrainte n'est pas détaillée ici mais une preuve
formelle peut être trouvée dans~\cite{modele_RCPSP}.

La contrainte~\eqref{res_RCPSP_OO} représente les limites de
capacité de chaque ressource. En effet, une activité $i$ consomme
une quantité $r_{ik}$ de la ressource $k$ entre $t_e$ et $t_{e+1}$
si et seulement si elle est en cours entre ces deux dates, i.e.
$z_{ie}=1$. On pourra remarquer que dans ce modèle le nombre de
contraintes nécessaires pour modéliser les contraintes de capacité
des ressources est nettement inférieur à celui du modèle précédent.

Enfin, la contrainte~\eqref{prec_RCPSP_OO} modélise les relations
de précédences entre les activités: si une activité $i$, devant
précéder une activité $j$, est en cours entre les événements $e$
et $e+1$, i.e. $z_{ie}=1$, alors l'activité $j$ ne peut être en
cours avant l'événement $e$, i.e. $z_{ie}=1 \Rightarrow
\sum_{f=0}^e z_{jf}=0 $.

Le modèle possède $n^2$ variables binaires, deux fois moins que
pour le modèle start/end, $n+1$ variables continues et
$(n-1)(3+|E|+m+n^2/2)+n^2+n$ contraintes. Le nombre de variables
et de contraintes du modèle est donc bien polynomial en fonction
du nombre d'activités et de ressources.



\subsection{Approches polyèdrales et inégalités valides}
\label{sec:ineg_RCPSP}