%!TeX root =../main_file.tex

\section{La programmation linéaire en nombres entiers}

Dans ce paragraphe, nous présentons les concepts de base de la
programmation linéaire ainsi que les notions et outils qui seront
utilisés dans la suite de cette partie. Cette présentation n'étant que
partielle, nous renvoyons le lecteur à~\cite{LP} pour une description
plus détaillée des différentes techniques existantes.

\subsection{La programmation linéaire}

La programmation linéaire vise à résoudre des problèmes
d'optimisation ayant la particularité de pouvoir s'exprimer à
l'aide de contraintes, se présentant sous la forme d'inégalités
et/ou d'égalités, linéaires en fonction des variables du problème.
De plus, la fonction objectif du problème, i.e. ce que l'on
cherche à maximiser/minimiser, s'écrit aussi sous la forme d'une
fonction linéaire.

Sous forme canonique, un programme linéaire (PL) s'écrit de la manière
suivante: 
 \[ \begin{array}{lcl}
\text{maximiser } & & \displaystyle cx\\ 
\text{tel que }& & \displaystyle Ax \le b\\
 & & \displaystyle x \in \mathbb{R}^n_+
 \end{array}
\]
avec $c \in \mathbb{R}^n$, $b \in \mathbb{R}^m$ et $A \in
\mathbb{R}^{m,n}$. 

Nous pouvons supposer, sans perte de généralité, que les variables du
problème sont positives et que la fonction objectif $x\rightarrow cx$
doit être maximisée. Les vecteurs $x$ de $\mathbb{R}^n$ qui vérifient
$Ax \le b$ sont appelés les solutions réalisables du problème et les
vecteurs $x^*$ qui, en plus, maximisent le critère $cx^*$ sont appelés
solutions optimales.    

La force de ces formulations repose sur le fait que l'ensemble des
contraintes du problème définit alors un polyèdre convexe nommé
polyèdre des solutions réalisables. De plus, si ce polyèdre est non
vide, alors toute solution optimale se trouve forcément sur un sommet,
appelé point extrême, de celui-ci. Les solutions optimales peuvent aussi
se trouver sur une face du polyèdre, i.e. l'intersection de celui-ci
avec un plan défini par une des contraintes du programme linéaire $\{x
\in \mathbb{R}^n\ | \ A_jx=b_j\}$. 

Ceci a permis la mise en place de plusieurs algorithmes pour résoudre
ces programmes linéaires. Un des plus efficaces en pratique et donc
des plus utilisés est l'algorithme du Simplexe. Le principe de cet
algorithme est le parcours "intelligent" des points extrêmes du
polyèdre des solutions admissibles: à chaque itération, l'algorithme
se "déplace" vers un sommet adjacent de meilleur (ou même)
coût. L'algorithme se termine donc lorsque tous les sommets adjacents
possèdent un coût moins bon que le sommet courant.

Un des inconvénients de cet algorithme est qu'il a, dans le pire des
cas, une complexité exponentielle. D'autres algorithmes permettent de
résoudre un programme linéaire en temps polynomial. Cependant, en
pratique, l'algorithme du simplexe reste plus efficace et c'est
souvent ce dernier qui est implémenté dans les solveurs d'optimisation
linéaire. 

\subsection{Contrainte d'intégrité}

De nombreux problèmes d'optimisation combinatoire ne peuvent s'écrire
sous la forme d'un programme linéaire. En effet, pour des problèmes
tels que les problèmes d'ordonnancement, tout ou partie des
variables doivent être restreintes à prendre des valeurs entières.  

Quand toutes les variables du problème sont contraintes à prendre des
valeurs entières, on parle de programme linéaire en nombres
entiers (PLNE). Lorsque seulement une partie des variables sont
autorisées à prendre des valeurs réelles, on parle de programme
linéaire mixte (PLM). 

De manière générale, un programme linéaire mixte 
peut s'écrire de la façon suivante: 

\[ \begin{array}{lcl}
\text{maximiser } & & \displaystyle \sum_{i=1}^p c_ix_i +
\sum_{j=1}^{n-p} f_jy_j\\ \text{tel que }& & \displaystyle
\sum_{i=1}^p a_{ki}x_i + \sum_{j=1}^{n-p} d_{kj}y_j \le b_k, \quad
k=1,\dots,m\\ & & \displaystyle x_i \ge0,\quad i=1,\dots,p\\ & &
\displaystyle y_j \in \mathbb{N},\quad j=1,\dots,n-p 
\end{array}
\] $y_j,\ j=1,\dots,n-p$ sont les variables de décisions ne
pouvant prendre que des valeurs entières tandis que les variables
$x_i,\ i=1,\dots,p$ peuvent prendre n'importe quelles valeurs
réelles positives.

La difficulté de la résolution de ces programmes repose sur le fait
que l'ensemble des solutions réalisables du problème ne forme plus un
polyèdre convexe et les algorithmes traditionnels ne peuvent être
appliqués directement dans ce cas. Le problème de décision associé à
un PLNE, ou à un PLM, est NP-complet~\cite{NP_bible}. De ce fait,
plusieurs techniques permettant de trouver la solution optimale en
temps raisonnable ont été développées.

\subsection{Techniques de résolution}

Parmi les techniques les plus utilisées, on retrouve les techniques
utilisant des algorithmes de recherche arborescente par séparation et
évaluation ainsi que des méthodes de coupes.

Le principe de la {\it recherche arborescente par séparation et
évaluation} repose sur deux idées principales: 
\begin{itemize}
\item {\bf la séparation} qui consiste à décomposer selon une
partition l'ensemble des solutions en plusieurs sous-ensembles;
\item {\bf l'évaluation} qui consiste à examiner la qualité des
solutions d'un sous-ensemble de façon optimiste, i.e. trouver une
borne supérieure (resp. inférieure) de la meilleure solution de ce
sous-ensemble lorsqu'il s'agit d'un problème de maximisation
(resp. minimisation).
\end{itemize} 
L’algorithme propose de parcourir l’arborescence des solutions
possibles en évaluant chaque sous-ensemble de solutions de façon
optimiste. Lors de ce parcours, il maintient la valeur de la meilleure
solution trouvée jusqu’à présent. Quand l’évaluation d’un
sous-ensemble donne une valeur plus faible (plus forte pour un
problème de minimisation) ou égale, il est inutile d’explorer plus loin ce
sous-ensemble. L’ensemble des solutions admissibles est ainsi
représenté par une arborescence dans laquelle un grand nombre de nœuds
peuvent être éliminés. L’évaluation des sous-problèmes se fait
typiquement par {\it relaxation linéaire} (on ignore la contrainte
d’intégrité) en utilisant le Simplexe.

Les {\it méthodes de coupes} cherchent à caractériser le plus petit
polyèdre contenant l'ensemble de toutes les solutions
admissibles. Pour cela, un ensemble de coupes, i.e. des inégalités
permettant de supprimer une partie du polyèdre ne contenant que des
solutions non entières, est généré. Si ces coupes sont assez fortes,
le polyèdre des solutions admissibles du modèle devient alors
exactement l'enveloppe convexe des solutions admissibles
entières. Dans ce cas, la résolution de la relaxation du programme
linéaire  réduit à cet ensemble donne une solution optimale
entière.

