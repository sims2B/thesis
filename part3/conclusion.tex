\chapter*{Conclusion}

Cette partie consacrée à la programmation linéaire mixte et en nombres
entiers a commencé par décrire les concepts fondamentaux de cette
théorie. Puis, dans un second temps, nous avons présenté plusieurs
modèles pour un problème d'ordonnancement à contraintes de ressource:
le \RCPSP. Pour ce problème, plusieurs types de modèles ont été
présentés et, pour chacun d'eux, nous avons discuté leurs avantages et
inconvénients. Les modèles indexés par le temps sont plus performants
sur des instances ayant de petits horizon de temps tandis que les
modèles à événements s'avèrent plus efficaces sur des instances
disposant de plus grands horizon de temps. 

Ces deux types de modèles ont ensuite été adaptés pour être appliquer
au \CECSP. Pour ce problème, en plus des avantages décrits ci-dessus,
les modèles à événements disposent d'un avantage supplémentaire. En
effet, pour le \CECSP, il est possible qu'une instance ne possède que
des solutions rationnelles et ce même si cette dernière est seulement
pourvu de données entières. Les modèles indexés par le temps ne
permettent d'obtenir que des solutions à date de début et fin
d'activités entières. A l'inverse, les modèles à événements permettant
d'obtenir des solutions non entières, sont donc plus adaptés au cas du
\CECSP. De ce fait, une grande partie de nos travaux a porté sur le
renforcement de ces modèles par le biais de l'ajout de coupes et
d'inégalités valides. De plus, nous avons montré qu'un de ces
ensembles d'inégalités permettait de décrire exactement l'enveloppe
convexes des vecteurs binaires solutions du modèle. 

Des inégalités valides pour le modèle indexé par le temps ont aussi
été décrites. Ces inégalités sont directement déduites d'un algorithme
présenté dans la partie dédiée à la programmation par contrainte de
ce manuscrit: le raisonnement énergétique. Dans la continuité de ces
travaux, d'autres raisonnements issues de la programmation par
contraintes pourraient être envisagés afin de déduire des ensembles
d'inégalités pouvant être ajoutés au modèles (indexé par le temps ou à
événements). De plus, les modèles présentées souffrent d'un grand
nombre de symétrie. Ces dernières pourraient être éviter par l'ajout
de nouvelles contraintes. La définition de telle contraintes ainsi
qu'une étude du polyèdre associé fait aussi partie des poursuites de
recherche sur ce sujet. La dernière perspective dans ce domaine
seraient l'établissement de nouveaux modèles indexés par le temps, il
en existe beaucoup d'autres pour le \RCPSP, ou le mise en place de
nouvelles formulations étendues.
