\documentclass{article}

\begin{document}

The expression of the mandatory consumption is :
\[ \underline{b}(i,t_1,t_2)= max( 
b_i^{min} * \underline{w}(i,t_1,t_2)/f_i(b_i^{min}),
max_p(1/a_{ip} (\underline{w}(i,t_1,t_2) -c_i{ip}|J|)))
\]

Hence, we know that whenever $\underline{b}(i,t_1,t_2)=
b_i^{min} * \underline{w}(i,t_1,t_2)/f_i(b_i^{min})$ for the linear case, 
$\underline{b}(i,t_1,t_2)$ has the same expression for the piecewise linear case. 

Whenever $\underline{b}(i,t_1,t_2)=
1/a_{i} (\underline{w}(i,t_1,t_2) -c_i{i}|J|$ for the linear case, 
$\underline{b}(i,t_1,t_2)=
max_p(1/a_{ip} (\underline{w}(i,t_1,t_2) -c_i{ip}|J|))$ for the piecewise 
linear case.

Thus, to prove that the relevant intervals for the linear case are the same 
that for the piecewise concave linear case, we have to prove that function: 
$max_p(1/a_{ip} (\underline{w}(i,t_1,t_2) -c_i{ip}|J|))$ is piecewise linear 
and that the serie of slope of this function is increasing. Indeed, if this 
serie is increasing, then the left derivative of the function cannot be greater 
than its right.

We first prove that the serie is of the form: $S_{ip}= \frac{f_i(b_i^{max})- c_{ip}}
{a_{ip}}$ and then, we prove that the serie is increasing.

{\color{red} \Huge TODO: proof (use efficiency ratio and concavity)}


Now, we prove that the serie is increasing. To do so, we prove that 
$\frac{f_i(b_i^{max}) - c_{ip}}{a_{ip}} < 
\frac{f_i(b_i^{max}) - c_{ip+1}}{a_{ip+1}}$.

We study the sign of $\frac{f_i(b_i^{max}) - c_{ip}}{a_{ip}} - 
\frac{f_i(b_i^{max}) - c_{ip+1}}{a_{ip+1}}$.

\begin{align*}
 & & sign(\frac{f_i(b_i^{max}) - c_{ip}}{a_{ip}} - 
\frac{f_i(b_i^{max}) - c_{ip+1}}{a_{ip+1}})\\
 &=& sign(\frac{a_{ip+1}f_i(b_i^{max}) - a_{ip+1}c_{ip}-a_{ip}f_i(b_i^{max}) + a_{ip}c_{ip+1}}{a_{ip+1}a_{ip}})\\ 
 &=& sign(a_{ip+1}f_i(b_i^{max}) - a_{ip+1}c_{ip}-a_{ip}f_i(b_i^{max}) + a_{ip}c_{ip+1})
 \end{align*}

Let ${[}\gamma_p,\gamma_{p+1}{]}$ and ${[}\gamma_{p+1},\gamma_{p+2}{]}$, with 
$\gamma_{p+1}\neq b_i^{max}$, be 
the interval over which the function $f_i$ is defined by $a_{ip}*b+c_{ip}$ and 
$a_{ip+1}*b+c_{ip+1}$ respectively.

Since, $f_i(b)$ is continuous, we have $f_i(\gamma_{p})=a_{ip}*\gamma_{p+1}+c_{ip}=
a_{ip+1}*\gamma_{p+1}+c_{ip+1}$ and then $c_{ip+1}=
(a_{ip}-a_{ip+1})*\gamma_{p+1}+c_{ip}$. 

Therefore,
\begin{align*}
 & & sign(a_{ip+1}f_i(b_i^{max}) - a_{ip+1}c_{ip}-a_{ip}f_i(b_i^{max}) + a_{ip}c_{ip+1})\\
 &=& sign(a_{ip+1}f_i(b_i^{max}) - a_{ip+1}c_{ip}-a_{ip}f_i(b_i^{max}) + a_{ip}((a_{ip}-a_{ip+1})*\gamma_{p+1}+c_{ip}))\\
 &=& sign(a_{ip+1}f_i(b_i^{max}) - a_{ip+1}c_{ip}-a_{ip}f_i(b_i^{max}) + a_{ip}^2\gamma_{p+1}-a_{ip}a_{ip+1}\gamma_{p+1}+a_{ip}c_{ip}))\\
 &=& sign(a_{ip+1}(f_i(b_i^{max}) -c_{ip}-a_{ip}\gamma_{p+1})-a_{ip}(f_i(b_i^{max}) - a_{ip}\gamma_{p+1}-a_{ip}c_{ip}))\\
 &=& sign((a_{ip+1}-a_{ip})(f_i(b_i^{max})-f_i(\gamma_{p+1})))
 \end{align*}
 
And, since $f_i$ is concave, we have $a_{ip+1}<a_{ip}$ and, since $f_i$ is 
non-decreasing and $\gamma_{p+1}< b_i^{max}$, we have $f_i(b_i^{max})>
f_i(\gamma_{p+1})$. 

Hence, $\frac{f_i(b_i^{max}) - c_{ip}}{a_{ip}} < 
\frac{f_i(b_i^{max}) - c_{ip+1}}{a_{ip+1}}$ and this prove the property.

\end{document}
