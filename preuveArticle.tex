\documentclass{report}

\usepackage[utf8]{inputenc}
\usepackage[T1]{fontenc}
\usepackage[francais]{babel}
\usepackage{xargs}

\usepackage{amsmath}
\usepackage{amssymb}
\usepackage{amsthm}

\usepackage[svgnames]{xcolor}

\usepackage{array}

\usepackage{tikz}
\usetikzlibrary{patterns}
\usepackage{caption}
\usepackage{subcaption}

\newcommandx{\bb}[2][1=t_1,2=t_2]{\underline{b}(i,#1,#2)}
\newcommandx{\wb}[2][1=t_1,2=t_2]{\underline{w}(i,#1,#2)}
\renewcommandx{\wp}[2][1=t_1,2=t_2]{\underline{w}^{+}_i(#1,#2)}
\newcommandx{\wm}[2][1=t_1,2=t_2]{\underline{w}^{-}_i(#1,#2)}
\newcommandx{\w}[2][1=t_1,2=t_2]{\underline{w}_i(#1,#2)}
\newcommandx{\bp}[2][1=t_1,2=t_2]{\underline{b}^{+}_i(#1,#2)}
\newcommandx{\bm}[2][1=t_1,2=t_2]{\underline{b}^{-}_i(#1,#2)}
\renewcommandx{\b}[2][1=t_1,2=t_2]{\underline{b}_i(#1,#2)}
\newcommand{\bmin}{b_i^{min}}
\newcommand{\bmax}{b_i^{max}}
\newcommandx{\inter}[2][1=t_1,2=t_2]{{[}#1,#2{]}}
\newcommandx{\ga}{\gamma}

\newtheorem{Th}{Théorème}
\newtheorem{Lem}{Lemme}



\begin{document}

Nous allons prouver que la fonction $\wb$ est indépendante de $t_1,\ 
\forall t_1 \le r_i$ et indépendante de $t_2,\ \forall t_2 \ge d_i$.\\

Pour cela nous commençons par introduire les notations suivantes:
\begin{itemize}
\item $\wp=\max(0,W_i-f_i(\bmax)(\max(0,t_1-r_i))$ est l'énergie apportée à $i$ 
  après $t_1$ si la tâche est calée à gauche.
\item $\wm =\max(0,W_i-f_i(\bmax)(\max(0,d_i-t_2))$ est l'énergie apportée à $i$ 
  avant $t_2$ si la tâche est calée à droite.	
\item $\w =\max(f_i(\bmin)(\min(d_i,t_2)-\max(r_i,t_1)),
  W_i-f_i(\bmax)(\max(0,t_1-r_i)+\max(0,d_i-t_2)))$ est l'énergie apportée à $i$ 
  entre $t_1$ et $t_2$ si la tâche est centrée(calée à gauche et à droite ou 
  exécutée à $\bmin$).\\
\end{itemize}


Avec ces notations, l'expression de l'énergie minimale dans l'intervalle 
$\inter$ devient: $\wb=\min(\wp,\wm,\w)$.\\

Nous allons maintenant montrer le lemme suivant:
\begin{Lem}
  \begin{align}
    \wb=\wb[r_i][t_2],\ \forall t_1\le r_i \label{indept1}\\
    \wb=\wb[t_1][d_i],\ \forall t_2\ge d_i \label{indept2}
  \end{align}
\end{Lem}

\begin{proof}
  Nous allons prouver seulement (\ref{indept1}), la preuve pour (\ref{indept2}) 
  étant similaire.\\
  Supposons donc $t_1\le r_i$, nous avons alors :
  $\wp=W_i=\wp[r_i][t_2])$. De plus, $\wm$ étant indépendant de $t_1$ et $r_i$, 
  $\wm=\wm[r_i][t_2]$. Il reste donc à prouver que $\w=\w[r_i][t_2]$. Comme 
  $t_1\le r_i,\ \w=\max(f_i(\bmin)(\min(d_i,t_2)- r_i) , W_i-f_i(\bmax)(\max(0,d_i-t_2)))=
  \w[r_i][t_2]$. Donc $\wb=\min(\wp,\wm,\w)=\min(\wp[r_i][t_2],\wm[r_i][t_2],
  \w[r_i][t_2])=\wb[r_i][t_2]$. Donc, nous avons bien $\wb=\wb[r_i][t_2],\ 
  \forall t_1\le r_i$.
  
\end{proof}

Nous pouvons aussi prouver que la fonction $\bb$ possède la même propriété. Pour cela, nous définissons les 3 notations suivantes (similaires à celles définies pour $\wb$):

\begin{itemize}
\item $\bp=\max(\bmin \frac{\wp}{f_i(\bmin)},\frac{1}{a_i}(\wp - c_i(\max(0,\min(0,t_1-r_i)+d_i-t_1))))$ est la quantité de ressource que doit 
consommer $i$ après $t_1$ si la tâche est calée à gauche.
\item $\bm=\max(\bmin \frac{\wm}{f_i(\bmin)},\frac{1}{a_i}(\wm - c_i(\max(0,\min(0,d_i-t_2)+t_2-r_i))))$ est la quantité de ressource que doit 
consommer $i$ avant $t_2$ si la tâche est calée à droite.	
\item $\b =\max(\bmin \frac{\w}{f_i(\bmin)},
  \frac{1}{a_i}(\w - c_i(\min(d_i,t_2)-\max(r_i,t_1))))$ est la quantité 
  de ressource que doit consommer $i$ entre $t_1$ et $t_2$ si la tâche 
  est centrée.\\
\end{itemize}


Avec ces notations, l'expression de la consommation minimale de la ressource dans l'intervalle 
$\inter$ devient: $\bb=\min(\bp,\bm,\b)$.\\

\begin{Lem}
	\begin{align}
	\bb=\bb[r_i][t_2],\ \forall t_1\le r_i \label{indepbt1}\\
	\bb=\bb[t_1][d_i],\ \forall t_2\ge d_i \label{indepbt2}
	\end{align}
\end{Lem}

\begin{proof}
Supposons $t_1\le r_i$, nous avons donc:\\
 $
 \begin{array}{rl}
 \bp & =\max(\bmin\frac{\wp}{f_i(\bmin)}, \frac{1}{a_i}(\wp - c_i(d_i-r_i)))\\
  & =\max(\bmin\frac{\wp[r_i][t_2]}{f_i(\bmin)}, \frac{1}{a_i}(\wp[r_i][t_2] - c_i(d_i-r_i)))\\
   & =\wp[r_i][t_2]
   \end{array}
   $\\
   De plus, $\bm$ étant indépendant de $t_1$ et $\wm=\wm[r_i][t_2]$, $\bm=\bm[r_i][t_2]$. Il nous reste donc à prouver que $\b=\b[r_i][t_2]$.\\
   
  \noindent
Comme $t_1\le r_i$, \\
	$
	\begin{array}{rl}
	\b & =\max(\bmin \frac{\w}{f_i(\bmin)},
  	\frac{1}{a_i}(\w - c_i(\min(d_i,t_2)-r_i)))\\
  	 & =\max(\bmin \frac{\w[r_i][t_2]}{f_i(\bmin)},
  	\frac{1}{a_i}(\w[r_i][t_2] - c_i(\min(d_i,t_2)-r_i)))\\
  	 & =\b[r_i][t_2]
  	\end{array}
  	$\\
  Par conséquent,\\
  	$
	\begin{array}{rl}
  \bb= & \min(\bp,\bm,\b)\\
   = & \min(\bp[r_i][t_2],\bm[r_i][t_2],
  \b[r_i][t_2])\\
  =&\bb[r_i][t_2]
  \end{array}
  $\\
  Donc, nous avons bien $\bb=\bb[r_i][t_2],\ 
  \forall t_1\le r_i$.
\end{proof}

Nous allons maintenant montrer que la fonction $\wb$ est symétrique par 
rapport à l'axe $t_1+t_2=r_i+d_i$. C'est l'énoncé du lemme suivant:

\begin{Lem}
  $\wb=\wb[t_1'][t_2']$ avec $t_1'=d_i+r_i-t_2$ et $t_2'=d_i+r_i-t_1,\ \forall t_1 \le 0$ et $\forall t_2 \le 0$.
\end{Lem} 

\begin{proof}~\\
  \begin{itemize}
  \item $\wp[t_1'][t_2']=\max(0,W_i-f_i(\bmax)(\max(0,d_i+r_i-t_2-r_i)))=\wm$
    \vspace{0.2cm}
  \item $\wm[t_1'][t_2']=\max(0,W_i-f_i(f_i(\bmax))(\max(0,d_i-d_i-r_i+t_1)))=\wp$
    \newpage
  \item ~\\
    $
    %\setlength{\extrarowheight}{0.5cm}
    \begin{array}{rl}
      \w[t_1'][t_2']=&\max\left(
      \begin{array}{l}
	f_i(\bmin) (d_i+\min(0,r_i-t_1) - r_i- \max(d_i-t_2,0)),\\
	W_i - f_i(\bmax)(\max(0,d_i-t_2) + \max (0,t_1-r_i))
      \end{array}
      \right)\\
      & \\
      = &\max\left(
      \begin{array}{l}
	f_i(\bmin) (\min(-r_i,r_i-r_i-t_1) - \max(d_i-d_i-t_2,-d_i)),\\
	W_i - f_i(\bmax)(\max(0,d_i-t_2) + \max (0,t_1-r_i))
      \end{array}
	\right)\\
	& \\
	= &\max\left(
	\begin{array}{l}
	  f_i(\bmin) (\min(d_i,t_2) - \max(r_i,t_1)),\\
	  W_i - f_i(\bmax)(\max(0,d_i-t_2) + \max (0,t_1-r_i))
	\end{array}
	\right)\\
	& \\
	= & \w
	
    \end{array}$
  \end{itemize}
  Et donc $\wb=\wb[t_1'][t_2']$ avec $t_1'=d_i+r_i-t_2$ et $t_2'=d_i+r_i-t_1,\ \forall t_1 \le 0$ et $\forall t_2 \le 0$
\end{proof}

Comme précédemment, nous pouvons prouver cette propriété pour le fonction $\bb$.

\begin{Lem}
  $\bb=\bb[t_1'][t_2']$ avec $t_1'=d_i+r_i-t_2$ et $t_2'=d_i+r_i-t_1,\ \forall t_1 \le 0$ et $\forall t_2 \le 0$.
\end{Lem}

\begin{proof}~\\
\begin{itemize}
\item $
	\bp[t_1'][t_2']= \max
	\left(
	\begin{array}{l}
	\bmin\frac{\wp[t_1'][t_2']}{f_i(b_i^{min})},\\
	\frac{1}{a_i}(\wp[t_1'][t_2'] - c_i (\max(0,\min(0,d_i-t_2)+t_2-r_i))
	\end{array}
	\right)$
	 et comme $\wp[t_1'][t_2']=\wm$, $\bp[t_1'][t_2']=\bm$
	 \vspace{0.2cm}
	\item $
	\bm[t_1'][t_2']= \max
	\left(
	\begin{array}{l}
	\bmin\frac{\wm[t_1'][t_2']}{f_i(b_i^{min})},\\
	\frac{1}{a_i}(\wm[t_1'][t_2'] - c_i (\max(0,\min(0,t_1-r_i)+d_i-t_1))
	\end{array}
	\right)$
	 et comme $\wm[t_1'][t_2']=\wp$, $\bm[t_1'][t_2']=\bp$
	 \vspace{0.2cm}
	\item $
	\b[t_1'][t_2']= \max
	\left(
	\begin{array}{l}
	\bmin\frac{\w[t_1'][t_2']}{f_i(b_i^{min})},\\
	\frac{1}{a_i}(\w[t_1'][t_2'] - c_i (\min(d_i,d_i+r_i-t_1)-\max(r_i,r_i+d_i-t_2)))
	\end{array}
	\right)$
	et comme $\min(d_i,d_i+r_i-t_1)-\max(r_i,r_i+d_i-t_2)=\min(d_i,t_2)-\max(r_i,t_1)$, $\b[t_1'][t_2']=\b$

\end{itemize}
\end{proof}
\end{document}
