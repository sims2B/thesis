\chapter*{Introduction\markboth{INTRODUCTION}{}}

De nos jours, de nombreuses tâches, auparavant fastidieuses, ont vu
leur complexité largement diminuée grâce à la mise en place d'outils
informatiques permettant leur traitement. Ces outils sont constamment
améliorés et l'augmentation des performances des ordinateurs les
rendent de plus en plus efficaces. Cependant, nombre de ces outils
reposent sur des problèmes complexes dont la résolution demande un
grand nombre d'opérations. Le plus souvent, ce nombre d'opérations est
exponentiel en fonction de la taille du problème. Dans ce cas, la
résolution de tels problèmes peut prendre plusieurs centaines
d'années. 

La mise en place de techniques dédiées permettant de prendre en
considération les propriétés intrinsèques du problème est donc un
axe de recherche majeur dans le domaine de l'informatique. Parmi les
classes de problèmes les plus étudiés, on retrouve les problèmes
d'ordonnancement, les problèmes de tournée de véhicules ou les
problèmes d'affectation. 

Dans cette thèse, nous nous sommes intéressé aux problèmes
d'ordonnancement et, plus particulièrement, aux problèmes
d'ordonnancement avec contraintes de ressource. Parmi les problèmes
les plus étudiés dans la littérature, nous retrouvons le problème
d'ordonnancement de projet avec contraintes de ressource et le problème
cumulatif. 

Dans le problème d'ordonnancement de projet avec contraintes de
ressource, nous devons ordonnancer un ensemble d'activités, chacune
d'entre elles consommant une partie d'une ou plusieurs ressources (de
capacité limitée) et étant liées par des relations de précédence. Le
plus souvent, ces activités doivent être ordonnancées de manière à
minimiser la date de fin du projet mais de nombreux autres objectifs,
tels que la minimisation du coût ou des retards peuvent être trouvés 
dans la littérature.

Dans le problème cumulatif, les activités consomment une quantité
d'une et une seule ressource. Il s'agit de la même ressource pour
toutes les activités. Dans ce problème, il n'y a pas de relation de
précédence entre les activités mais chaque activité dispose d'une
fenêtre de temps dans laquelle elle doit s'exécuter. 

La plus grande limites de ces problème est que les activités sont
supposées de durée fixes et consommant une quantité de ressource
constante au cours du temps. Cependant, il existe de nombreuses
applications pratiques dans lesquelles ces contraintes ne sont pas
respectées~\cite{HaitArtiguesLopez,Blaz,W80}. En effet, la
possibilité, pour une activité, de consommer plus de ressource
(respectivement moins) afin de finir plus rapidement (resp. lentement)
n'est pas modélisée dans ces problèmes. Les activités satisfaisant
cette propriété sont dites {\it malléables} dans le sens où leur
forme, définie par leur durée et consommation de ressource pendant
leur exécution, doit être décidée pendant le processus de résolution. 

Cette thèse étudie une nouvelle modélisation des activités malléables
représentée par un problème appelé le problème d'ordonnancement
continu avec contraintes énergétiques. Dans ce problème, un ensemble
d'activités,  utilisant une ressource continue et cumulative de
capacité limitée, doit être ordonnancé. La quantité de ressource
nécessaire à l'exécution d'une tâche n'est pas fixée mais doit
être déterminé. Une fois la tâche commencée et jusqu'à sa date de fin,
la quantité de ressource consommée par l'activité doit être comprise
entre une valeur maximale et une valeur minimale. De plus, la
consommation, à un instant donnée,  d'une partie de la ressource
permet à l'activité de recevoir une certaine quantité d'énergie,
calculée par le biais d'une fonction de rendement. Cette énergie nous
permet de savoir quand l'activité est terminée, i.e. quand elle a
reçue une quantité suffisante d'énergie. 

Pour le problème cumulatif et le problème d'ordonnancement de projet
avec contraintes de ressource, différentes techniques permettant de
trouver des solutions ont été mises en place. Ces techniques utilisent
des concepts et théorie pouvant être très variés. Cependant, deux de
ceux figurant parmi les plus utilisées demeurent les techniques issues
de la programmation par contraintes et de la programmation linéaire
mixte (ou en nombres entiers). En effet, ces techniques se sont
révélées très efficace dans le processus de résolution de ces deux
problèmes. Ce sont quelques unes de ces méthodes que nous nous
proposons d'appliquer dans cette thèse.

Le plan de la thèse est le suivant: 
\begin{itemize}
\item le chapitre~\ref{sec:chapter1} commence par détailler les
principales caractéristiques des problèmes d'ordonnancement
(paragraphe~\ref{sec:ordo}). Ensuite, une définition formelle des
problèmes d'ordonnancement cumulatifs et d'ordonnancement de projet
avec contrainte de ressource ainsi qu'une description des principales
limitations de ces problèmes sont données
(paragraphe~\ref{sec:ordo_res}).  Enfin, le
paragraphe~\ref{sec:ordo_nrj} décrit le problème étudié dans ce
manuscrit: le problème d'ordonnancement continu avec contraintes
énergétiques. De plus, ce paragraphe d'écrit les modélisations
pré-existantes des activités malléables tout en expliquant en quoi ces
modélisations ne suffisaient pas à la modélisation de tous les problèmes
réels. La dernière chose abordée dans ce chapitre est la présentation
d'un certains nombres de propriétés satisfaites par ce problème.
\item les chapitres~\ref{sec:PPC_CUSP} et~\ref{sec:PPC_CECSP} sont
  dédiés aux méthodes issues de la programmation par
  contraintes. Après une brève introduction à la programmation par
  contraintes (paragraphe~\ref{sec:PPC}) et à l'ordonnancement en
  programmation par contraintes (paragraphe~\ref{sec:cumu_ordo}), nous
  présentons les principaux algorithmes de filtrages mis en place pour
  le problème cumulatif (paragraphe~\ref{sec:cumu_propag}). Une partie
  de ces algorithmes est ensuite adapté au cas du problème
  d'ordonnancement continu avec contraintes énergétiques dans le
  chapitre~\ref{sec:PPC_CECSP}. 
\item les chapitres~\ref{sec:PLNE_RCPSP} et~\ref{sec:PLNE_CECSP}
  présentent les techniques de résolution issues de la programmation
  linéaire mixte. Dans un premier temps, nous décrivons les concepts généraux
  de la programmation linéaire mixte (paragraphe~\ref{sec:PLNE}). Nous
  présentons ensuite un certain nombre de modèles mis en place pour
  résoudre le problème d'ordonnancement de projet avec contraintes de
  ressource (paragraphe~\ref{sec:PLNE_ordo_res}). Ces modèles sont
  ensuite adaptés dans le cadre du problème d'ordonnancement continu
avec contraintes énergétiques (paragraphe~\ref{sec:modele_CECSP}). De
plus, des résultats venant renforcer ces modèles sont présentés dans
le paragraphe~\ref{sec:amelioration_modele}.
\item le chapitre~\ref{sec:expe} présente les résultats expérimentaux
  conduits pour valider les notions théoriques décrites dans les
  chapitres précédents. 
\item les annexes~\ref{ann:JFPC} et~\ref{ann:IESM} présentent des
  travaux réalisés pendant la thèse mais pas assez aboutis ou trop
  éloignés du sujet de ce manuscrit pour y figurer à part entière. 
\end{itemize}










