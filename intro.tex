\chapter*{Introduction\markboth{INTRODUCTION}{}}

De nos jours, de nombreuses tâches, auparavant fastidieuses, ont vu
leur complexité largement diminuée grâce à la mise en place d'outils
informatiques permettant leur traitement. Ces outils sont constamment
améliorés et l'augmentation des performances des ordinateurs les
rendent de plus en plus efficaces. Cependant, nombre de ces outils
reposent sur la résolution de problèmes complexes demandant un
grand nombre d'opérations. Dans le cas où la variante décisionnelle du
problème est NP-complète, ce nombre d'opérations est
exponentiel en fonction de la taille du problème. Dans ce cas, la
résolution de tels problèmes peut prendre plusieurs centaines
d'années. 

La mise en place de techniques dédiées permettant de prendre en
considération les propriétés intrinsèques du problème est donc un
axe de recherche majeur dans le domaine de l'informatique. Parmi les
classes de problèmes les plus étudiés, on retrouve les problèmes
d'ordonnancement, les problèmes de tournées de véhicules ou les
problèmes d'affectation. 

Dans cette thèse, nous nous sommes intéressé aux problèmes
d'ordonnancement et, plus particulièrement, aux problèmes
d'ordonnancement avec contraintes de ressource. Parmi les problèmes
les plus étudiés dans la littérature, nous retrouvons le problème
d'ordonnancement de projet avec contraintes de ressource et le problème
cumulatif.

Dans le problème d'ordonnancement de projet avec contraintes de
ressource, nous devons ordonnancer un ensemble d'activités, chacune
d'entre elles consommant une partie d'une ou plusieurs ressources (de
capacité limitée) et étant liées par des relations de précédence. Le
plus souvent, ces activités doivent être ordonnancées de manière à
minimiser la date de fin du projet mais de nombreux autres objectifs,
tels que la minimisation du coût ou des retards peuvent être
rencontrés dans la littérature.

Dans le problème cumulatif, les activités consomment une quantité
d'une et une seule ressource. Il s'agit de la même ressource pour
toutes les activités. Dans ce problème, il n'y a pas de relation de
précédence entre les activités mais chaque activité dispose d'une
fenêtre de temps dans laquelle elle doit s'exécuter. Ce problème
correspond à une relaxation de la variante décisionnelle du problème
d'ordonnancement de projet avec contraintes de ressource.

La plus grande limite des problèmes ci-dessus est que les activités sont
supposées de durée fixe et consommant une quantité de ressource
constante au cours du temps. Cependant, il existe de nombreuses
applications pratiques dans lesquelles ces hypothèses ne sont pas
respectées~\cite{HaitArtiguesLopez,Blaz,W80}. En effet, la
possibilité, pour une activité, de consommer plus de ressource
(respectivement moins) afin de finir plus rapidement (resp. lentement)
n'est pas modélisée dans ces problèmes. Les activités satisfaisant
cette propriété sont dites {\it malléables} dans le sens où leur
forme, définie par leur durée et consommation de ressource pendant
leur exécution, doit être décidée pendant le processus de
résolution. Des exemples de telles activités sont variés lorsque les
activités doivent consommer, par exemple, une ressource de nature 
énergétique comme 
l'électricité.  La quantité de ressource 
allouée à une activité peut alors être modulée à tout instant pour
accélérer l'activité 
ou au contraire la ralentir et diminuer sa consommation (souvent
pour réduire les coûts 
énergétiques). Un autre exemple intervient dans l'ordonnancement de
projet où la durée d'une activité dépend de la quantité de ressource
qui lui est attribuée (p.e. personnel). 

Cette thèse étudie une nouvelle modélisation des activités malléables
représentée par un problème appelé le problème d'ordonnancement
continu avec contraintes énergétiques. Dans ce problème, un ensemble
d'activités,  utilisant une ressource continue et cumulative de
capacité limitée, doit être ordonnancé. La quantité de ressource
nécessaire à l'exécution d'une activité n'est pas fixée mais doit
être déterminée à chaque instant. Une fois la tâche commencée et
jusqu'à sa date de fin, la quantité de ressource consommée par
l'activité doit être comprise entre une valeur maximale et une valeur
minimale. De telles bornes dans des problèmes pratiques
peuvent représenter, par exemple, la quantité d'énergie
minimale requise pour qu'une réaction chimique ou thermodynamique
puisse se faire dans les conditions prescrites ou encore le nombre
minimum et 
maximum d'employés pouvant être affectés à une activité.
De plus, la consommation, à un instant donné, d'une partie
de la ressource permet à l'activité de recevoir une certaine quantité
d'énergie, calculée par le biais d'une fonction de rendement. Ces
fonctions de rendement peuvent, par exemple, représenter les pertes
dues à la conversion de la ressource en énergie (p.e. AC/DC) ou les
coûts de communication dans des architectures multiprocesseurs. 
La connaissance de l'énergie reçue par une activité nous permet de
savoir quand 
l'activité est terminée, i.e. quand elle a reçu une quantité
suffisante d'énergie. 

Pour le problème cumulatif et le problème d'ordonnancement de projet
avec contraintes de ressource, différentes techniques permettant de
trouver des solutions ont été mises en place dans la littérature. Ces
techniques utilisent des concepts et théories pouvant être très
variés. Cependant, deux de celles figurant parmi les plus utilisées
demeurent les techniques issues de la programmation par contraintes et
de la programmation linéaire mixte (ou en nombres entiers). En effet,
ces techniques se sont révélées très efficaces dans le processus de
résolution de ces deux problèmes. Ce sont quelques-unes de ces
méthodes que nous nous proposons d'étendre au problème considéré dans
cette thèse.

Le plan de la thèse est le suivant: 
\begin{itemize}
\item le chapitre~\ref{sec:chapter1} commence par détailler les
principales caractéristiques des problèmes d'ordonnancement
(sous-section~\ref{sec:ordo_def}). Ensuite, une définition formelle des
problèmes d'ordonnancement cumulatif et de projet
avec contraintes de ressource ainsi qu'une description des principales
limitations de ces problèmes en termes de modélisation de certaines
ressources sont données
(sous-section~\ref{sec:ordo_res}).  Enfin, la
section~\ref{sec:ordo_nrj} décrit le problème étudié dans ce 
manuscrit: le problème d'ordonnancement continu avec contraintes
énergétiques. De plus, cette section décrit les modélisations
préexistantes des activités malléables tout en expliquant en quoi ces
modélisations ne suffisaient pas à la modélisation de certains problèmes
réels. La dernière partie du chapitre est consacrée à la présentation
d'un certain nombre de propriétés remarquables satisfaites par ce problème.
\item les chapitres~\ref{sec:PPC_CUSP} et~\ref{sec:PPC_CECSP} sont
  dédiés aux méthodes de programmation par
  contraintes. Après une brève introduction à la programmation par
  contraintes (section~\ref{sec:PPC}) et à l'ordonnancement en
  programmation par contraintes, nous
  présentons les principaux algorithmes de filtrage mis en place pour
  le problème cumulatif (section~\ref{sec:cumu}). Dans le
  chapitre~\ref{sec:PPC_CECSP}, nous adaptons une partie de ces
  algorithmes au problème d'ordonnancement continu avec contraintes
  énergétiques. Dans un premier temps, nous montrons qu'une partie de
  ces algorithmes peut facilement être adaptée en considérant
  l'ordonnancement des activités dans le pire des cas en termes de
  durée ou de consommation de ressource (voir 
  section~\ref{sec:time_CECSP}). La section~\ref{sec:ER_CECSP} est
  consacrée à l'adaptation du raisonnement énergétique. Pour ce
  raisonnement, nous présentons plusieurs méthodes permettant de
  caractériser les intervalles sur lesquels appliquer ce
  raisonnement. Nous attirons ici l'attention du lecteur sur
  l'utilisation du terme {\it énergie}. En effet, dans ce manuscrit
  nous utiliserons à la fois ce terme pour le raisonnement
  énergétique (algorithme de filtrage pour la contrainte cumulative)
  mais aussi dans un problème qui modélise des ressources énergétiques
  telles que l'électricité.
\item les chapitres~\ref{sec:PLNE_RCPSP} et~\ref{sec:PLNE_CECSP}
  présentent les techniques de résolution issues de la programmation
  linéaire mixte. Dans un premier temps, nous décrivons les concepts généraux
  de la programmation linéaire mixte (section~\ref{sec:PLNE}). Nous
  présentons ensuite trois modèles mis en place pour
  résoudre le problème d'ordonnancement de projet avec contraintes de
  ressource (section~\ref{sec:PLNE_ordo_res}). Le premier est un
  modèle indexé par le temps et les deux autres utilisent des
  formulations basées sur les événements. Ces modèles sont
  ensuite adaptés au problème d'ordonnancement continu
avec contraintes énergétiques (section~\ref{sec:modele_CECSP}). La 
section~\ref{sec:amelioration_modele} présente plusieurs
ensembles d'inégalités permettant de renforcer les modèles
présentés. Pour le modèle indexé par le temps, des inégalités
directement déduites du raisonnement énergétique sont exhibées. Pour
les modèles à événements, cinq jeux d'inégalités sont présentés et
l'un d'entre eux est utilisé pour donner une description minimale de
l'enveloppe convexe du polyèdre formé par toutes les affectations
possibles des variables binaires correspondant à une seule activité.
\item le chapitre~\ref{sec:expe} présente les résultats expérimentaux
  conduits pour valider les notions théoriques décrites dans les
  chapitres précédents. Dans un premier temps, nous présentons les
instances sur lesquelles les algorithmes ont été appliqués (voir
section~\ref{sec:instance}). La section~\ref{sec:expe_PLNE} présente
les performances des différents modèles de programmation linéaire
mixte définis pour le problème d'ordonnancement continu à contrainte
énergétique. De plus, l'influence des inégalités définies pour
renforcer les modèles est évaluée. La section suivante
(\ref{sec:expe_PPC}) présente les résultats obtenus lors des
expérimentations portant sur la programmation par contraintes. Les
raisonnements présentés dans le manuscrit sont intégrés dans une
méthode de branchement hybride utilisant un modèle de programmation
linéaire.
\item les annexes~\ref{ann:JFPC} et~\ref{ann:IESM} présentent des
  travaux réalisés pendant la thèse mais pas assez aboutis ou trop
  éloignés du sujet de ce manuscrit pour y figurer à part entière.
  L'annexe~\ref{ann:JFPC} présente l'étude du cas discret du problème
  (article publié aux Journées Francophones de Programmation par
  Contraintes~\cite{Nattaf_JFPC}). L'annexe~\ref{ann:IESM} porte sur
  la mise en place d'une matheuristique pour un problème industriel
  d'ordonnancement avec contraintes et objectifs énergétiques. Ce
  travail s'inscrit dans le cadre d'une collaboration pour un projet
  franco-chilien et a été publié dans la conférence IESM (International
  Conference on Industrial Engineering and Systems
  Management~\cite{Nattaf_IESM}). 
\end{itemize}










