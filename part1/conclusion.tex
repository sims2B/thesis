
\chapter*{Conclusion}

Dans cette partie, nous avons introduit les principales
caractéristiques des problèmes d'ordonnancement avant de nous
intéresser en particulier aux problèmes cumulatifs. Nous avons ensuite
présenté deux des principaux problèmes étudiés en ordonnancement sous
contraintes de ressource: le \RCPSP~et le \CUSP. Les limitations de
ces problèmes en termes de modélisation d'activités à profils
variables ont ensuite été démontrées et une nouvelle modélisation de
la consommation de ressource nous a permis de définir un nouveau
problème: le \CECSP. 

Dans un premier temps, nous avons comparé ce problème avec les
problèmes existant dans la littérature. Cette comparaison nous a
permis de montrer que les techniques de résolution existantes ne
pouvaient pas s'appliquer directement dans le cadre du \CECSP. De
plus, le profil variable de consommation de la ressource rendaient la
résolution de ce problème délicates. En effet, dans le cas général, la
quantité de ressource allouée à une activité peut varier à tout
moment ce qui accroît la difficulté de résolution. 

Nous avons donc défini un ensemble de propriétés permettant de
faciliter le résolution du \CECSP. Dans le cas général, ce problème
est NP-complet mais il existe des cas particuliers pour lequel ce
problème est polynomial. C'est le cas, par exemple, de la version
préemptive du \CECSP. De plus, nous avons montré que, dans le cas où
l'on considère des fonctions de rendement concaves, la fonction
d'allocation de ressource est une fonction constante par morceaux et
que ces points de rupture correspondent aux dates de début et de fin
des activités. Cette propriété nous permet de définir des méthodes de
résolution pour le \CECSP et aussi d'adapter plusieurs méthodes
existantes définies pour d'autres problèmes. Ces méthodes peuvent être
classées en deux catégories: 
\begin{itemize}
\item les techniques adaptées du \CUSP~et issues de la programmation
  par contraintes. Ces techniques seront détaillées dans la
  partie~\ref{part:PPC}.
\item les techniques adaptées du \RCPSP~et issues de la programmation
  linéaire. Ces techniques seront détaillées dans la
  partie~\ref{part:PLNE}.
\end{itemize}
