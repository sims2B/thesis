
\section{Conclusion}

Dans ce chapitre, nous avons d'abord introduit les principales
caractéristiques des problèmes d'ordonnancement avant de nous intéresser
en particulier aux problèmes cumulatifs. Nous avons ensuite présenté
deux des principaux problèmes étudiés en ordonnancement sous
contrainte de ressource. Les limitations de ces problèmes en termes de
modélisation d'activités à profils variables ont ensuite
été démontrées et une nouvelle modélisation de la consommation de
ressource nous a permis de définir un nouveau problème: le
\CECSP. 

Dans un premier temps, nous avons comparé ce problème avec les
problème existant dans la littérature, puis nous avons présenté un
ensemble de propriétés qui va nous permettre, dans la suite de ce
manuscrit, de décrire des techniques pour sa résolution. La plupart de
ces techniques sont adaptées de techniques existantes et peuvent être
classées en deux catégories: 
\begin{itemize}
\item les techniques adaptées du \CUSP~et issues de la programmation
par contraintes. Ces techniques seront détaillées dans la
partie~\ref{part:PPC}.
\item les techniques adaptées du \RCPSP~et issues de la programmation
linéaire. Ces techniques seront détaillées dans la
partie~\ref{part:PLNE}.
\end{itemize}