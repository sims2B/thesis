\section{Ordonnancement et contraintes de ressources}
\label{sec:ordo}
\subsection{L'ordonnancement}
\label{sec:ordo_def}
La théorie de l'ordonnancement s'intéresse au calcul de dates
d'exécution d'un ensemble d'activités. Dans cette optique,
l'utilisation d'une ou plusieurs ressources peut être nécessaire et
l'exécution d'une activité implique souvent une telle consommation. Un
problème d'ordonnancement peut alors être vu comme l'organisation dans
le temps de la réalisation d'activités soumises à des contraintes de
temps et de ressource. Dans la plupart des cas, un ou plusieurs
objectifs sont définis et une solution au problème d'ordonnancement 
vise à optimiser ces objectifs.
 
\subsubsection{Les activités}

Une activité peut être définie par une date de début $st_i$ et une date
de fin $et_i$ et une durée $p_i$ vérifiant $et_i=st_i+p_i$. Si
l'activité utilise une ou plusieurs ressources durant son exécution, il
est nécessaire d'ajouter à cette définition une fonction permettant de
modéliser l'allocation de chaque ressource $k$ à chaque activité
$i$. Si cette fonction est donnée dans les paramètres du 
problème, on la note $r_{ik}(t)$. Si, au contraire, elle fait partie
des variables de décision du problème, on la note $b_{ik}(t)$. Enfin,
si cette fonction ne dépend pas du temps, i.e. est constante, le
paramètre $t$ pourra être omis. 

Selon les problèmes, une activité peut être contrainte à s'exécuter en
un seul morceau. On parle alors d'activité non préemptive et dans le
cas contraire, i.e. les activités peuvent être exécutées en plusieurs
morceaux, on parle d'activité préemptive.

Une activité peut être utilisée pour représenter, par exemple, une
opération dans un processus de production, le décollage/atterrissage
d'un avion ou encore une étape d'un projet de construction.

\subsubsection{Les ressources}

Une ressource $k$ est un moyen technique ou humain requis pour la
réalisation d'une activité et est disponible en quantité
limitée. Cette quantité, appelée {\it disponibilité de la ressource ou
  capacité}, peut être soit constante ou varier au cours du
temps. Dans ce manuscrit, nous considérons des ressources à capacité
constante et cette capacité est notée $R_k$. Les ressources utilisées
par les activités peuvent être de nature diverse. Parmi elles, on peut
distinguer:
\begin{itemize}
\item les ressources renouvelables: ces ressources peuvent être
réutilisées dès lors qu'elles sont libérées. Il s'agit, en fait, de
ressources qui, après avoir été utilisées par une ou plusieurs
activités, sont de nouveau disponible en même quantité. Ces ressources
peuvent, par exemple, représenter la main d'\oe uvre d'une entreprise,
des machines, de l'électricité ou des équipements.
\item à l'inverse, les ressources consommables sont des ressources
dont la consommation globale est limitée au cours du temps. Il peut
s'agir, par exemple, de matières premières ou d'un budget.
\end{itemize}

Parmi les ressources renouvelables, on distingue par ailleurs, les
ressources disjonctives qui ne peuvent exécuter qu'une activité à la
fois -- e.g. pistes de décollage, salles -- et les ressources cumulatives
qui peuvent, elles,  être utilisées par plusieurs activités en
parallèle mais sont disponibles en quantité limitée -- e.g. main d'\oe
uvre,
processeurs.


Du point de vue de leur divisibilité, les ressources peuvent aussi
être divisées selon deux catégories: 
\begin{itemize}
\item les ressources continues, i.e. divisibles en temps ou en
  quantité continu: dans le premier cas, il s'agit de ressources
  pouvant être ré-allouées à tout instant $t \in [0,T]$, où $T$ est une
  borne supérieure sur la date de fin de l'ordonnancement; dans le
  second cas, il s'agit de ressources pouvant être allouées en quantité
  continue, i.e. non discrète. Ce type de ressource permet, par
  exemple, de modéliser l'électricité, l'essence, l'énergie hydraulique...
\item les ressources discrètes, i.e. divisibles en temps ou en quantité
  discret: à l'inverse, le premier cas décrit une ressource où la
  ré-allocation de cette dernière ne peut être 
  exécutée qu'à des temps discrets $t \in \{0,\dots,T\}$. Ce cas
  correspond généralement au cas où l'horizon de temps a été
  discrétisé, soit pour faciliter le problème, soit sans perte de
génralité. Le second cas correspond aux ressources ne pouvant être
attribuées aux activités qu'en quantité discrète, e.g. employés,
machines...
\end{itemize}

\subsubsection{Les contraintes}

Une contrainte permet d'exprimer des restrictions sur les valeurs que
peuvent prendre une ou plusieurs variables du problème. Parmi les
principales, on distingue:
\begin{itemize}
\item les contraintes de temps: elles intègrent les contraintes de
  temps alloué, issues généralement d'impératifs de gestion et
  relatives aux dates limites des activités (e.g. dates
    de livraison) ou à la durée totale d'un projet mais aussi les
    contraintes d'enchaînement qui décrivent des
    positionnements relatifs devant être respectés entre les
    activités. Ces contraintes peuvent, par exemple, modéliser des
    contraintes de précédence entre les activités, i.e. une activité
    ne peut commencer avant qu'une autre n'ait été achevée, ou des temps
    de transition à respecter entre les activités.

    {\'E}tant donné une activité $i$, la date à partir de laquelle
    l'activité $i$ peut être exécutée est appelée {\it date de début au
      plus tôt} et est notée $\ES$ (earliest start time en anglais). De
    même, la date avant laquelle l'activité $i$ doit avoir été
    complètement exécutée sera appelée {\it date de fin au plus tard} et
    notée $\LE$ (latest end time en anglais).

\item les contraintes de ressources:  ce sont des contraintes d'utilisation
  des ressources qui expriment la nature et la quantité de moyens
  utilisés par les activités, ainsi que les caractéristiques
  d'utilisation de ces moyens. Ces contraintes peuvent aussi représenter
  des contraintes de disponibilité des ressources qui précisent la
  nature et la quantité de moyens disponibles au cours du temps.
\end{itemize}

\subsubsection{Les objectifs}

Lors de la résolution d'un problème d'ordonnancement, deux buts
différents peuvent être poursuivis. Le premier vise à trouver une solution
réalisable pour le problème tandis que le second cherche à trouver une
solution optimisant un ou plusieurs critères ou objectifs.

Ces objectifs peuvent être liés à différents aspect de la solution. On
distingue par exemple:
\begin{itemize}
\item les objectifs liés au temps: le temps total d'exécution ou le temps moyen
  d'achèvement d'un ensemble d'activités peuvent être minimisés, mais
  aussi  les retards (maximum, moyen, somme...) par rapport
  aux dates de fin au plus tard fixées par le problème.
\item les objectifs liés aux ressources: la quantité (maximale,
  moyenne, pondérée...) de ressources nécessaires pour réaliser un
  ensemble d'activités peut, par exemple, être minimisée.
\item les objectifs liés aux coûts de lancement, de production, de
  transport, de stockage ou liés aux revenus, aux retours
  d'investissements... 
\item les objectifs liés à une énergie, un débit...
\end{itemize}

Deux exemples de problèmes d'ordonnancement sont présentés dans la
sous-section suivante: le \RCPSP~et le \CUSP. 