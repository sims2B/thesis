
\section{L'ordonnancement sous contraintes énergétiques}
 
La modélisation présentée dans cette section repose sur le problème
d'ordonnancement continu à contraintes
énergétiques~\cite{ArtiguesLopez}, le \CECSP.

\subsubsection{Définition du problème}

Dans ce problème, un ensemble d'activités non-préemptives
$\A=\{1,\dots,n\}$ utilisant une ressource continue, cumulative et
renouvelable, de capacité $R$ doit être ordonnancé. Durant son
exécution, une activité consomme une quantité variable $b_i(t)$ de la
ressource qui doit être comprise entre une valeur minimale, $\bmin \in
[0,R]$, et une valeur maximale, $\bmax \in [\bmin,R]$. De plus, la fin
d'une activité correspond au moment où cette dernière a reçu une
certaine quantité d'énergie $W_i$. Cette énergie est reçue via la
ressource et calculée à l'aide d'une fonction $f_i: \{0\} \cup
[\bmin,\bmax]\longrightarrow\{0\} \cup [f(\bmin),f(\bmax)]$, supposée
continue et croissante et appelée fonction de rendement. La quantité
d'énergie reçue par $i$ à l'instant $t$ est donc $\int_{0}^t
f_i(b_i(s))ds$. La dernière contrainte du problème précise que chaque activité doit
être exécutée dans sa fenêtre de temps $[\ES,\LE]$.

L'objectif du \CECSP~est donc de déterminer la date de début $st_i$ et
de fin $et_i$ de chaque activité $i \in \A$, ainsi que la fonction
d'allocation de ressource, $b_i(t)$, associée à
cette activité telle que: 
\begin{itemize}
\item la fenêtre de temps de chaque activité est respectée, i.e. 
  \begin{equation} 
    \forall i \in \A,\ \ES \le st_i < et_i \le \LE \label{tw_CECSP}
  \end{equation}
  Les activités de durée nulle ne sont pas considérées. 
\item la capacité de la ressource n'est excédée à aucun moment du
  projet, i.e.
  \begin{equation} 
    \forall t \in \H,\sum_{\substack{i\in \A\\ t \in
        [st_i,et_i]}} b_i(t) \le  R \label{res_CECSP}
  \end{equation}
  où $\H=\{0,\dots,T\}$ est l'horizon de temps du projet et $T=\max_{i
    \in \A} \LE$.
\item si une activité est en cours à l'instant $t$ alors les
  contraintes de consommation minimale et maximale doivent être
  respectées, i.e.  
  \begin{equation}
    \forall i \in \A,\ \forall t \in [st_i,et_i],\ \bmin \le b_i(t) \le
    \bmax \label{req_CECSP}
  \end{equation}
\item si l'activité n'est pas en cours, alors elle ne consomme pas de
  ressource, i.e.
  \begin{equation}
    \label{nulleConso_CECSP}
    \forall i \in \A,\ \forall t \not\in [st_i,et_i],\  b_i(t)=0 
  \end{equation}
\item l'énergie requise doit être apportée à chaque activité, i.e. 
  \begin{equation}
    \forall i \in \A,\ \int_{st_i}^{et_i}f_i(b_i(t))dt=W_i \label{nrj_CECSP}
  \end{equation}
  Dans certains cas, nous pourrons remplacer cette contrainte par la
  contrainte suivante:
  \begin{equation}
    \forall i \in \A,\ \int_{st_i}^{et_i}f_i(b_i(t))dt \ge W_i \tag{\ref{nrj_CECSP}a}
  \end{equation}
\end{itemize}

Dans ce manuscrit, nous considérons les trois cas
suivants:
\begin{itemize}
\item $f_i$ est la fonction identité, $\forall i \in \A$,
\item $f_i$ est une fonction affine, $\forall i \in \A$,
\item $f_i$ est une fonction concave et affine par morceaux, $\forall
  i \in \A$.
\end{itemize}

L'intérêt de considérer de telles fonctions de rendement est qu'elles
nous permettent d'approcher un grand nombre de fonctions de
rendements réelles non linéaires. Un exemple présentant de telles
approximations sera présenté (cf. exemple\ref{ex_approx_CECSP}). 


Soit $P_i$ le nombre d'intervalles de définition de la fonction
$f_i$, i.e. le nombre d'intervalles où la fonction $f_i$ a une
expression différente. La fonction $f_i$ peut alors s'écrire de la
manière suivante:  
\[f_i(b)=\left\{
    \begin{array}{ll}
      0 & \quad \text{if }b=0\\
      a_{ip_1}*b+c_{ip_1} &\quad \text{si }\bmin=0\text{ et }b \in ]\bmin,x_{p_2}] \\
      a_{ip_1}*b+c_{ip_1} &\quad \text{si }\bmin\neq 0 \text{ et }b \in
                            [\bmin,x_{p_2}] \\
      a_{ip_\ell}*b+c_{ip_\ell} &\quad \text{si } b \in
                                  ]x_{p_\ell},x_{p_{\ell+1}}], \ell \in \P_i=\{2,\dots,P_i-1\} 
    \end{array}
  \right.\]
De plus, nous considérons que la fonction $f_i$ satisfait les
propriétés suivantes: 
\begin{itemize}
\item $a_{ip_1} >a_{ip_2} > \dots > a_{ip_{P_i}}>0$ et $c_{ip_1}
  <c_{ip_2} < \dots < c_{ip_{P_i}}$ pour assurer la croissance et la
  concavité de la fonction; 
\item $-a_{ip_1}*\bmin \ge c_{ip_1}$  afin de s'assurer que $f_i(b) \ge
  0,\ \forall b \in [\bmin,\bmax]$;
\item $a_{ip_\ell}*b+c_{ip_\ell}=a_{ip_{\ell+1}}*b+c_{ip_{\ell+1}}$
  pour assurer la continuité de la fonction.
\end{itemize}

\begin{ex}
  \label{ex_approx_CECSP}
  Considérons l'instance à  quatre activités suivante:
  \begin{itemize}
  \item $B=2$
  \item $\ES={[}0,2,0,5{]}$
  \item $\LE={[}6,10,9,13{]}$
  \item $\bmin={[}0,0.25,2,1{]}$
  \item $\bmax={[}1,1,2,1.5{]}$
  \item $W={[}1,5,7,8{]}$
  \item $f(b)={[}b,\sqrt{b},b,\sqrt{b}{]}$
  \end{itemize}

  Nous devons approcher les fonctions $f_2(b)$ et $f_4(b)$. Commençons
  par le cas où la fonction est approchée par une fonction affine. Pour
  cela, nous calculons le coefficient directeur de la tangente en
  $\bmin + (\bmax-\bmin)/ 2 =0.625$. Ce coefficient directeur est égal à
  $\frac{1}{2\sqrt{0.625}}$, et donc,
  $f'_2(b)=\frac{1}{2\sqrt{0.625}}*b+ \frac{\sqrt{0.625}}{2}$ (cf. figure~\ref{approx_aff}). 

  \begin{figure}[!htb]
    \centering
    \subcaptionbox{Approximation par une fonction affine
      \label{approx_aff}}[0.45\linewidth]{
      \begin{tikzpicture}
        [xscale=4.5,yscale=2.5]
        \node (O) at (0,0) {};

        \draw[->] (0,0) -- (1.28,0) node[below] {$b$};
        \draw[->] (0,0) -- (0,1.3) node[left] {$f_i(b)$};

        \draw (0,1) -- (-0.02,1) node[left] {$1$};
        \draw (0,0.5) -- (-0.02,0.5) node[left] {$\sqrt{0.25}$};

        \draw (1,0) -- (1,-0.02) node[below] {$1$};
        \draw (0.25,0) -- (0.25,-0.02) node[below] {$0.25$};

        \draw[color=gray,domain=0:1.28,samples=50] plot ({\x},{sqrt(\x)});
        \draw[dashed,thick,domain=0.25:1,samples=50] plot
        ({\x},{\x/(2*sqrt(0.625))+sqrt(0.625)/2});


        \draw[dotted] (0.25,0)-- (0.25,1.3);
        \draw[dotted] (1,0) -- (1,1.3);
      \end{tikzpicture}}
    \hfill
    \subcaptionbox{Approximation par une fonction concave, affine par
      morceaux
      \label{approx_affparmorceau}}[0.45\linewidth]{
      \begin{tikzpicture}
        [xscale=4.5,yscale=2.5]
        \node (O) at (0,0) {};

        \draw[->] (0,0) -- (1.28,0) node[below] {$b$};
        \draw[->] (0,0) -- (0,1.3) node[left] {$f_i(b)$};

        \draw (0,1) -- (-0.02,1) node[left] {$1$};
        \draw (0,0.5) -- (-0.02,0.5) node[left] {$\sqrt{0.25}$};

        \draw (1,0) -- (1,-0.02) node[below] {$1$};
        \draw (0.25,0) -- (0.25,-0.02) node[below] {$0.25$};

        \draw[color=gray,domain=0:1.28,samples=50] plot ({\x},{sqrt(\x)});
        
        \draw[ dashed,thick,domain=0.25:0.5,samples=50] plot
        ({\x},{\x/(2*sqrt(0.375))+sqrt(0.375)/2});
        \draw[dashed, thick,domain=0.5:0.75,samples=50] plot
        ({\x},{\x/(2*sqrt(0.625))+sqrt(0.625)/2});
        \draw[dashed, thick,domain=0.75:1,samples=50] plot
        ({\x},{\x/(2*sqrt(0.875))+sqrt(0.875)/2});

        \draw[dotted] (0.5,0)-- (0.5,1.3);
        \draw[dotted] (0.25,0) -- (0.25,1.3);
        \draw[dotted] (0.75,0) -- (0.75,1.3);
        \draw[dotted] (1,0) -- (1,1.3);
      \end{tikzpicture}}
    \caption{Approximation d'une fonction de rendement non linéaire.}
    \label{approx}
  \end{figure}

  De même, pour l'activité $4$, nous avons
  $f'_4(b)=\frac{1}{2\sqrt{1.25}}*b+ \frac{\sqrt{1.25}}{2}$.

  Approchons maintenant la fonction $f_2(b)$  par une
  fonction concave et affine par morceaux. Dans un premier temps,
  nous devons choisir le pas d'approximation $\epsilon$, i.e. la taille
  des intervalles pour lesquels la fonction $f_i$ a une
  expression différente. Dans cet, exemple, nous choisissons,
  $\epsilon=1/4$. Le nombre d'intervalles de définition de la fonction
  $f_2$ est alors $(\bmax-\bmin)/\epsilon=3$. Pour chacun de ces
  intervalles, nous appliquons la procédure utilisée pour
  l'approximation de $f_2$ par une fonction affine. Nous obtenons donc
  l'approximation suivante (cf. figure~\ref{approx_affparmorceau}):
  \[f_2=\left\{ 
      \begin{array}{lll}
        \frac{1}{2\sqrt{3/8}}*b + \frac{\sqrt{3/8}}{2}& & \text{si } b \in
                                                          [0.25,0.5]\\
        \frac{1}{2\sqrt{5/8}}*b + \frac{\sqrt{5/8}}{2}& & \text{si } b \in [0.5,0.75]\\
        \frac{1}{2\sqrt{7/8}}*b + \frac{\sqrt{7/8}}{2}& & \text{si } b \in [0.75,1]
      \end{array}
    \right.\]
  
\end{ex}

Avec une telle approche, il peut arriver que la fonction de rendement
$f_i$ ne vérifie pas $\bmin=0 \Rightarrow f_i(\bmin)=0$. Dans ce cas,
la valeur de $f_i(0)$ est mise à $0$. La fonction n'est donc plus
continue sur tout son intervalle de définition. La
contrainte~\eqref{nrj_CECSP} est donc remplacée par:
\begin{equation}
  \int_{st_i}^{et_i}{\bf 1}_{NZ}(t)f_i(b_i(t))dt = W_i \tag{\ref{nrj_CECSP}b}
\end{equation}
\noindent 
où ${\bf 1}_{NZ}(t):=\left\{
  \begin{array}{ll}
    1 & \text{si }t \in NZ:=\{t|b_i(t)\neq0\}\\
    0 & \text{sinon}
  \end{array}
\right.$ est la fonction caractéristique de l'ensemble $\mathbb{R}^+$.

La difficulté du \CECSP~repose, entre autre, sur le fait que la
fonction d'allocation de ressource peut n'être ni constante, ni
constante par morceau. De ce fait, la représentation temps/ressource
d'une activité peut prendre n'importe quelle forme
(cf. figure~\ref{figure_forme_conso}). 

\begin{figure}[!htb]
  \centering
  \subcaptionbox{$b_i(t)$ constante}[0.45\linewidth]{
    \begin{tikzpicture}
      [xscale=0.75,yscale=0.5]
      \node[] (O) at (0,0) {};
      
      
      \draw (0.5,0) node[below] {$\ES$};
      \draw (6,0) node[below] {$\LE$};
      \draw  (0,1) node[left] {$\bmin$};
      \draw (0,4) node[left] {$\bmax$};

      \draw[dotted] (0,1) -- (6.5,1);
      \draw[dotted] (0,4) -- (6.5,4);
      \draw[dotted] (0.5,0) -- (0.5,4);
      \draw[dotted] (6,0) -- (6,4);

      \draw[->] (O.center) -- (0,4.5) node[above] {$b_i(t)$};
      \draw[->] (O.center) -- (6.5,0) node[right] {$t$};
      
      \draw (5.7,0) -- (5.7,2) -- (1,2) -- (1,0);
    \end{tikzpicture}
  }
  \hfill
  \subcaptionbox{$b_i(t)$ constante par morceaux}[0.45\linewidth]{
    \begin{tikzpicture}
      [xscale=0.75,yscale=0.5]
      \node[] (O) at (0,0) {};
      
      
      \draw (0.5,0) node[below] {$\ES$};
      \draw (6,0) node[below] {$\LE$};
      \draw  (0,1) node[left] {$\bmin$};
      \draw (0,4) node[left] {$\bmax$};

      \draw[dotted] (0,1) -- (6.5,1);
      \draw[dotted] (0,4) -- (6.5,4);
      \draw[dotted] (0.5,0) -- (0.5,4);
      \draw[dotted] (6,0) -- (6,4);

      \draw[->] (O.center) -- (0,4.5) node[above] {$b_i(t)$};
      \draw[->] (O.center) -- (6.5,0) node[right] {$t$};
      
      \draw (5.7,0) -- (5.7,2) -- (4,2) -- (4,4) -- (2,4) -- (2,1) -- (1,1) --  (1,0);
    \end{tikzpicture}
  }\\
  \vspace{0.2cm}
  \subcaptionbox{$b_i(t)$ quelconque}[0.9\linewidth]{
    \begin{tikzpicture}
      [xscale=0.75,yscale=0.5]
      \node (O) at (0,0) {};

      \path[draw] (1,0) -- (1,4) parabola [bend at end] (5.7,1) -- (5.7,0); 

      \draw (0.5,0) node[below] {$\ES$};
      \draw (6,0) node[below] {$\LE$};
      \draw  (0,1) node[left] {$\bmin$};
      \draw (0,4) node[left] {$\bmax$};

      \draw[dotted] (0,1) -- (6.5,1);
      \draw[dotted] (0,4) -- (6.5,4);
      \draw[dotted] (0.5,0) -- (0.5,4);
      \draw[dotted] (6,0) -- (6,4);
      
      
      \draw[->] (O.center) -- (0,4.5) node[above] {$b_i(t)$};
      \draw[->] (O.center) -- (6.5,0) node[right] {$t$};
      
      \draw (1,4) -- (1,0);
      
      
      \path[draw] (1,4) parabola [bend at end] (5.7,1); 
      
    \end{tikzpicture}
    \hfill
    \begin{tikzpicture}
      [xscale=0.75,yscale=0.5]
      \node (O) at (0,0) {};
      
      \draw (0.5,0) node[below] {$\ES$};
      \draw (6,0) node[below] {$\LE$};
      \draw  (0,1) node[left] {$\bmin$};
      \draw (0,4) node[left] {$\bmax$};

      \draw[dotted] (0,1) -- (6.5,1);
      \draw[dotted] (0,4) -- (6.5,4);
      \draw[dotted] (0.5,0) -- (0.5,4);
      \draw[dotted] (6,0) -- (6,4);
      
      
      \draw[->] (O.center) -- (0,4.5) node[above] {$b_i(t)$};
      \draw[->] (O.center) -- (6.5,0) node[right] {$t$};

      \draw (1,1) parabola bend (1.8,3)(2.5,2); 
      \draw (2.5,2) parabola bend (3.2,1) (4.5,3);
      \draw (4.5,3) parabola bend (5.2,4) (5.7,2);
      \draw (1,0) -- (1,1);
      \draw (5.7,0) -- (5.7,2);
    \end{tikzpicture}
  }
\caption{Différentes formes de fonction d'allocation de ressource pour
le \CECSP.}
\label{figure_forme_conso}
\end{figure} 

Nous allons maintenant décrire un exemple d'instance et de solution
pour le \CECSP. Cependant, par souci de clarté, nous présentons un
exemple où il existe une solution dans laquelle toutes les fonctions
$b_i(t)$ sont constantes par morceaux.

\begin{ex}
Considérons l'instance à $3$ activités du \CECSP~suivante:
\begin{itemize}
\item $R=5$
\item cf. figure~\ref{ex_CECSP}
\item la fonction $f_2(b)$ est définie par l'expression suivante: 
\[f_2(b)=\left\{
\begin{array}{lll}
2b & & b \in [3,4]\\
b+4 & & b \in [4,5]
\end{array}
\right.\]
\end{itemize}
\begin{figure}[!htb]
\centering
\subcaptionbox{Fonction $f_2(b)$ \label{fonction_ex_CECSP}}[0.4\linewidth]{
\begin{tikzpicture}
[xscale=1.65,yscale=0.56]
\node (O) at (2,5) {};
\draw[->] (2,5) -- (5.5,5);
\draw[->] (2,5) -- (2,10);

\path[draw] (3,6) -- (4,8) -- (5,9) ;

\draw[dotted] (3,5) node[below] {\footnotesize $3$} -- (3,10);
\draw[dotted,color=gray!70] (4,5) node[below,color=black] {\footnotesize $4$}
-- (4,10);
\draw[dotted] (5,5) node[below] {\footnotesize $5$} -- (5,10);

\draw (2,6) node[left] {\footnotesize $6$};
\draw (2,8) node[left] {\footnotesize $8$};
\draw (2,9) node[left] {\footnotesize $9$};
\end{tikzpicture}}
\hfill
\subcaptionbox{Données de l'instance}[0.55\linewidth]{
  \begin{tabular}{|M{0.6cm}|M{0.6cm}M{0.6cm}M{0.6cm}M{0.6cm}M{0.6cm}M{1.2cm}|}
    \hline
    $i$ & $r_i$ & $d_i$ & $W_i$ & $\bmin$ & $\bmax$ & $f_i(b)$\\[2mm]
\hline
    1 & 0 & 2 & 6 & 3 & 3 & $b$\\[2mm]
    2 & 1 & 5 & 22 & 3 & 4 & fig.~\ref{fonction_ex_CECSP}\\[2mm]
    3 & 0 & 6 & 39 & 1 & 5 & $3b$\\[2mm]
    \hline
    \multicolumn{7}{c}{}
  \end{tabular}} 
\caption{Données de l'instance de l'exemple du \CECSP.}
\label{ex_CECSP}
\end{figure}
La figure~\ref{sol_ex_CECSP} présente une solution réalisable pour le
\CECSP. Dans cette figure, nous pouvons voir que l'énergie reçue par une
activité n'est, à priori, pas égale à la quantité de ressource
consommée par cette dernière. En effet, regardons l'activité $2$. Sa
consommation de ressource est égale à $3 + 4 * 2 = 11$ tandis que
l'énergie qu'elle reçoit est $f_2(3)+ f_2(4) * 2 = 6 + 8 * 2 =22$.

\begin{figure}[!htb]
\centering
\begin{tikzpicture}
[xscale=0.75,yscale=0.56]
\node (O) at (0,0) {};
\draw[->] (0,0) -- (6.5,0);
\draw[->] (0,0) -- (0,5.5);

\draw (0,2) rectangle (2,5) node[midway] {$1$};
\path[draw] (0,2) -- (3,2) -- (3,1) node[left=0.4cm] {$3$} -- (5,1) -- (5,5) -- (6,5)  -- (6,0);
\draw (2,5) -- (5,5) node[midway,below=0.5cm] {$2$};

\draw (0,1) node[left] {\footnotesize $1$};
\draw (0,2) node[left] {\footnotesize $2$};
\draw (0,5) node[left] {\footnotesize $5$};


\draw (2,0) node[below] {\footnotesize $2$};
\draw (3,0) node[below] {\footnotesize $3$};
\draw (5,0) node[below] {\footnotesize $5$};

\foreach \i in {1,...,5}{
\draw (\i,-0.1) -- (\i,0);
\draw (-0.1,\i) -- (0,\i);
}
\end{tikzpicture}
\caption{Solution de l'instance du \CECSP.}
\label{sol_ex_CECSP}
\end{figure}
\end{ex}

Le paragraphe suivant présente les différentes modélisations des
activités non-rectangulaires présentes dans la littérature. 

\subsubsection{Contexte}


Dans un premier temps, nous nous intéressons aux extensions du
\RCPSP. Une des extensions les plus célèbres est le problème
d'ordonnancement de projet multimode (MRCPSP). Dans ce problème, un
choix de différents modes est disponible pour chaque activité et une
activité doit être exécutée selon un de ces modes. Un mode correspond
à une combinaison formée d'un temps d'exécution constant et d'une
consommation de ressource qui permet d'apporter à l'activité au moins
la quantité d'énergie requise. Même si de nombreux problèmes basés sur
ce concept de mode existent~\cite{DDH,RK,RDK,DD} et que des méthodes
de résolution efficaces ont été mises en place pour résoudre le
MRCPSP~\cite{PV}, cette modélisation peut amener à une mauvaise
allocation de la ressource.

Si nous reprenons l'exemple de la peinture d'un bateau,
décrit au paragraphe~\ref{sec:limit_CUSP}, l'activité avait besoin de
$3$ unités d'énergie pour s'exécuter. Dans le contexte du MRCPSP,
seulement $3$ modes seraient décrits: $(3,1),\ (2,2)$ et $(1,3)$. Or,
dans le second cas, on donne une unité de trop à l'activité et la
possibilité d'allouer $2$ unités de ressource pendant une période de
temps et $1$ unité pendant la seconde n'est pas représentée ici. 

La principale limitation du MRCPSP est donc que les activités sont
contraintes à être rectangulaire, i.e. consommation de ressource
constante. 

D'autres extensions du \RCPSP~existent. C'est le cas par exemple des
problèmes d'ordonnancement de projet avec une ressource de type {\it
  work-content}~\cite{FT} ou du problème d'ordonnancement de projet
avec des profils de ressource flexibles (FRCPSP)~\cite{NK}. Dans ces
problèmes, plusieurs types de ressources sont considérées: 
\begin{itemize}
\item principale (ou work-content dans~\cite{FT}): il s'agit de la
  ressource via laquelle la quantité d'énergie requise est donnée à
  l'activité. C'est elle qui sert à déterminer la durée de l'activité.
\item les ressources dépendantes: l'utilisation de ces ressources
  dépendent de l'utilisation de la ressource principale.
\item les ressources indépendantes: la consommation de ces ressources
  est indépendante des consommations des autres ressources mais ces
  utilisations doivent être synchrone. 
\end{itemize}
Bien que plusieurs différences existent entre ces problèmes et le
\CECSP - l'utilisation de plusieurs ressources, le temps discret
pour~\cite{FT}... - les principales sont les suivantes: la longueur
minimale des blocs et les fonctions de rendements. La première
correspond au temps minimal qu'il faut attendre entre deux
ré-allocations de la ressource, que les auteurs appellent longueur
minimale de bloc, tandis que la seconde fait référence à la
non-présence de fonction de rendement, similaire à celles présentes
dans le cas du \CECSP. 

Enfin, la dernière extension du \RCPSP~présentée est celle où les
activités ont une intensité variable~\cite{Kis}. Ici, chaque activité
requiert une certaine quantité d'énergie durant son exécution et, dans
chaque période de temps, il est possible de donner une certaine partie
de cette quantité à l'activité, proportionnelle à l'{\it intensité} à
laquelle est exécutée l'activité. Dans ce cas, on peut introduire des
fonctions de rendement mais ces fonctions seraient alors contraintes à
être linéaire, i.e. $b \rightarrow a*b$. De plus, aucune borne
inférieure sur le rendement d'une activité n'est considérée. 

Dans le cadre du \CUSP, d'autre variantes ainsi que des algorithmes de
filtrages dédiés ont été proposés. Parmi ceux-ci, on retrouve le cas
des activités complètement/partiellement élastiques de Baptiste et
al.~\cite{BLN}. Dans le premier cas, les activités ont une demande en
énergie constante mais la quantité de ressource consommée par
une activité à chaque instant (discret) peut varier entre $0$ et la
capacité de la ressource. Dans le second cas, les mêmes conditions
sont présentes mais les auteurs définissent des contraintes permettant
de limiter les variations dans l'utilisation de la ressource. Aucun de
ces deux problèmes ne considère de fonctions de rendement. 

Dans~\cite{BP}, les auteurs définissent une activité comme une
séquence de sous-activités trapézoïdales ayant des durées et hauteurs
(consommations) variables. Enfin, Vil{\`i}m~\cite{V09} considère des
activités pour lesquelles la durée et la hauteur sont définies par des
intervalles. Pour ces deux problèmes, aucune demande en énergie n'est
définie pour les activités. 

Enfin, le \CECSP~ est aussi lié à d'autres problèmes à contraintes
d'énergie avec ressources continues~\cite{Blaz,Wali}.
Dans~\cite{Blaz}, plusieurs modèles représentant le temps d'exécution
d'une activité en fonction de la ressource qui lui est allouée sont
présentés. En particulier, les auteurs considèrent un problème où 
un ensemble de processeurs identiques et parallèles jouent le rôle de
la ressource. De plus, des fonctions représentant le temps d'exécution
d'une activité en fonction du nombre de processeurs qui lui est
allouée sont définies. Ce nombre de processeurs peut varier
continuellement au cours du temps et donc ces fonctions sont
équivalentes au fonctions de rendement définies dans le cadre du
\CECSP. De plus, dans~\cite{Blaz,Wali}, l'énergie est calculée en
intégrant une fonction de rendement sur tout l'horizon de temps.
Cependant, aucune contrainte de rendement maximum et minimum n'est
considérée dans ces problèmes.

Le tableau~\ref{tab:dif_CECSP} récapitule les principales différences
entre tous ces problèmes et le \CECSP. 


\begin{table}[!htb]
  \centering
  \begin{tabular}{|c|cc>{\centering\arraybackslash} p{1.9cm}>{\centering\arraybackslash} p{1.5cm}>{\centering\arraybackslash} p{1.3cm}>{\centering\arraybackslash} p{1.2cm}>{\centering\arraybackslash} p{2.4cm}|}
    \hline    
    Problème & $\bmin$ & $\bmax$ & fonction de rend. ($f_i$) &
                                                                   activités
                                                                   non-rect.
    & 
                                                                 énergie
                                                                   ($W_i$)
    & res. cont. & autre différence \\
\hline
MRCPSP & X & X & X &    & X &  & \\
\hline
 FRCPSP & X & X &    & X & X & X & long. de bloc\\  
\hline
work-content & X & X & & X & X &  & long. de bloc\\  
\hline
intensité variable&  & X&  & X & X & & \\   
\hline
part. élastique & & & & X & X & &\\
\hline
compl. élastique & & & X & X & & &\\
\hline
Trapézoïdales act. & & & & X & & &\#trapèzes fixe\\
\hline
Intervalles repr.& X & X & & & & & achat d'énergie\\ 
\hline
Modèle proces. & &  & X& X& X& X& \\ 
\hline
Intervalles repr.&  & &X &X & X& X& res. discrète et cont.\\ 
\hline
  \end{tabular}
  \caption{Principales différences entre les extensions des problèmes
    cumulatifs et le \CECSP.}
  \label{tab:dif_CECSP}
\end{table}

Le \CECSP~est donc un nouveau problème et ces différences avec les
problèmes existants ne nous permettent pas d'appliquer directement des
techniques déjà définies pour d'autres problèmes. Cependant, certaines
techniques existantes peuvent être adaptées dans le cadre du
\CECSP. Ces techniques seront présentées plus tard dans le manuscrit.

Le paragraphe suivant présente des propriétés du \CECSP~qui seront 
utilisées dans la suite de ce manuscrit. 

\subsubsection{Propriétés du \CECSP}

Dans ce paragraphe, nous allons commencer par présenter la preuve de
NP-complétude du \CECSP. Ce problème pouvant être vu comme une
généralisation du \CUSP, nous utilisons ce problème pour montrer la
difficulté du \CECSP. 

\begin{theo}[\cite{Nattaf_Constraints}]
Le \CECSP~est NP-complet.
\end{theo}

\begin{proof}
Nous réduisons donc le \CUSP~vers le \CECSP. Soit $\Pi$ une instance
du \CUSP. Nous réduisons $\Pi$ en une instance du \CECSP, $\Pi'$, de
la manière suivante, $\forall i \in \A$:
\begin{itemize}
\item $ \bmin=\bmax=r_i$
\item $f_i(b)=b$
\item $W_i=p_ir_i$
\item $R,\ \ES$ and $\LE$ restent inchangés. 
\end{itemize}

Il reste à montrer que $\Pi$ est une instance positive du \CUSP~si
et seulement si $\Pi'$ est une instance positive du \CECSP. Et ceci
est trivialement vrai. 
\end{proof}

Le problème de décision associé au \CECSP~est donc NP-complet. Dans ce
manuscrit, nous avons donc considérer ce problème sans fonction
objectif mais aussi avec la fonction objectif suivante: 
\[\text{minimiser } \sum_{i \in \A} \int_{st_i}^{et_i} b_i(t)dt\]
Cette fonction consiste en la minimisation de la consommation totale
de ressource. Dans la suite, si rien n'est précisé, cela veut dire que
nous considérons le \CECSP~ sans fonction objectif.



 Pour le \CECSP, nous présentons un exemple d'une instance ne
comprenant que des données entières et ne possédant que des solutions
non-entières. Ceci permet, en plus des arguments présentés pour le
\RCPSP, de démontrer l'importance de l'amélioration des modèles à
événement.

\begin{ex}
  \label{exemple_NE}
  Dans cet exemple, nous considérons une instance à deux activités et
  une ressource de capacité $2$. Le tableau~\ref{instance_exemple_NE}
  décrit les données de l'instance. 
  \begin{table}[!htb]
    \centering
    \begin{tabularx}{12cm}{|>{\centering\arraybackslash}p{0.6cm}|
        *5{>{\centering\arraybackslash}X}>{\centering\arraybackslash}p{2cm}|}
      \hline
      $i$ & $\ES$ & $\LE$ & $W_i$ & $\bmin$ & $\bmax$ & $f_i(b_i(t))$ \\
      \hline
      $1$ & $0$ & $2$ & $18$ & $2$ & $2$ & $3b_i(t)+6$\\
      $2$ & $1$ & $3$ & $3$ & $1$ & $2$ & $b_i(t)$\\
      \hline
    \end{tabularx}
    \caption{Données de l'instance de l'exemple~\ref{exemple_NE}}
    \label{instance_exemple_NE}
  \end{table}

  La seule solution est décrite par la figure~\ref{figure_exemple_NE}.
  \begin{figure}[!htb]
    \centering
    \begin{tikzpicture}
      [xscale=2]
      \node (O) at (0,0) {} node[below=0.1cm] {$0$};
      \draw (1.5,0)  node[below=0.1cm] {$1.5$};
      \draw (3,0) node[below=0.1cm] {$3$};
      \node (T) at (3.5,0) {};
      \node at (0.75,1) {0};
      \node at (2.25,1) {1};
      \draw[dashed] (-0.5,2) -- (3.5,2) node[right,node
      distance=1.5pt] {$R=2$}; 
      \draw (O) rectangle (1.5,2);
      \draw (1.5,2) rectangle (3,0);
      \draw[->] (-0.5,0) -- (T);
    \end{tikzpicture}
    \caption{Solution de l'instance de l'exemple~\ref{exemple_NE}}
    \label{figure_exemple_NE}
  \end{figure}

Dans cette solution, la première activité doit finir au temps $t=1.5$
pour que la seconde activité puisse finir avant sa date échue
$\LE[2]=3$. En effet, l'activité $2$ doit commencer avant sa date de
début au plus tard, ici $\LS[2]= \LE[2] - W_i/f_i(\bmax) = 3 - 3/2=
1.5$, et l'activité $1$ ne peut finir avant sa date de fin au plus
tôt, $\EE[1]= \ES[1] + W_i / f_i(\bmax) = 0 + 18/12 = 1.5$.
\end{ex}

De ce fait, l'espace des solutions peut être réduit par l'utilisation
du modèle à temps discret et ceci peut conduire à des infaisabilité ou
à des résultats sous-optimaux.

Une solution pour palier à ce problème est de mettre à l'échelle les
instances, i.e. multiplier les données par un certain coefficient $\alpha$
afin de s'assurer de l'existence d'une solution optimale entière,
avant de les résoudre. Cependant, l'utilisation d'un coefficient trop
grand peut conduire à une augmentation de la taille des modèles trop
importante pour permettre leur résolution. 


Le théorème suivant présente une des propriétés majeures du \CECSP. En
effet, il stipule que quelque soit l'instance considérée, il existe
toujours une solution de cette instance où les fonctions $b_i(t)$ sont
constantes par morceaux. 

\begin{theo}[\cite{Nattaf_CPDP}]
\label{theo_LPM_CECSP}
Soit $\Pi$ une instance réalisable du \CECSP~telle que:
\begin{itemize}
\item $\forall i \in \A,\ f_i$ est croissante, continue, concave et
  affine par morceaux. 
\item $\forall i \in \A,\ f_i(0)=0.$
\end{itemize}
Une solution ayant la propriété que, $\forall i \in \A,\ b_i(t)$ soit
constante par morceaux existe.
\end{theo}

Afin de prouver le théorème~\ref{theo_LPM_CECSP}, nous commençons par
prouver que, pour chaque intervalle $[t_1,t_2]$ tel que $b_i(t), t \in
[t_1,t_2]$, n'est pas constante, il existe une constante $b_{iq}$ pour
laquelle exécuter $i$ à $b_{iq}$ dans $[t_1,t_2]$ apporte au moins
autant d'énergie tout en consommant la même quantité de
ressource. C'est ce qu'affirme le lemme suivant:

\begin{lemma}
\label{lemmaEn}
Soit $b_{iq}= \frac{\int_{t_1}^{t_2}b_i(t)dt}{t_2-t_1}$. Alors, nous
avons:
\begin{align}
  &\int_{t_1}^{t_2}b_{iq}dt = \int_{t_1}^{t_2} b_i(t) dt \label{eq_LPM_res} \\
  & \int_{t_1}^{t_2}f_i(b_{iq})dt \ge \int_{t_1}^{t_2} f_i(b_i(t)) dt 
    \label{eq_LPM_nrj}
\end{align}
\end{lemma}

\begin{proof}
  L'équation~\eqref{eq_LPM_res} est trivialement vérifiée en remplaçant
  $b_{iq}$ par sa valeur. En effet, nous avons:
  \begin{align*}
    \int_{t_1}^{t_2}b_{iq}dt =&
    \int_{t_1}^{t_2}\left(\frac{\int_{t_1}^{t_2}b_i(t)dt}{t_2-t_1}\right)dt\\
    =& (t_2-t_1)\left(\frac{\int_{t_1}^{t_2}b_i(t)dt}{t_2-t_1}\right)\\
    =&\int_{t_1}^{t_2}b_i(t)dt
  \end{align*}

  Pour prouver que l'équation~\eqref{eq_LPM_nrj} est satisfaite,
  nous utilisons le théorème suivant:  
  \begin{theo}[\cite{Jensen}]
    Soit $\alpha(t)$ et $g(t)$ deux fonctions intégrables sur
    $[t_1,t_2] \subseteq \mathbb{R}$ telles que $\alpha(t) \ge 0,\
    \forall t \in [t_1,t_2]$. Alors, nous avons la propriété suivante: 
    \begin{equation}
      \phi\left( \frac{\int_{t_1}^{t_2} \alpha(t)g(t)dt }
        {\int_{t_1}^{t_2} \alpha(t)dt} \right) \ge
      \frac{\int_{t_1}^{t_2} \alpha(t)\phi(g(t))dt }
      {\int_{t_1}^{t_2} \alpha(t)dt}
    \end{equation}
    où $\phi$ est une fonction continue, concave sur $[\min_{t \in
      [t_1,t_2]} g(t),\max_{t \in [t_1,t_2]} g(t)]$. 
  \end{theo}
  Si nous remplaçons $\phi(t)$ par $f_i(t),\ g(t)$ par $b_i(t)$ et
  $\alpha(t)$ par la fonction constante égale à $1$, nous obtenons:
  \begin{align*}
    & f_i\left( \frac{\int_{t_1}^{t_2}b_i(t)dt }
      {t_2-t_1} \right) \ge
      \frac{\int_{t_1}^{t_2}f_i(b_i(t))dt }
      {t_2-t_1} \\
    & \Leftrightarrow (t_2-t_1)f_i\left( 
      b_{iq} \right) \ge
      \int_{t_1}^{t_2}f_i(b_i(t))dt\\
    & \Leftrightarrow \int_{t_1}^{t_2}f_i\left( 
      b_{iq} \right)dt \ge
      \int_{t_1}^{t_2}f_i(b_i(t))dt
  \end{align*}
Et donc, l'équation~\eqref{eq_LPM_nrj} est satisfaite.  
\end{proof}  

Nous pouvons maintenant prouver le théorème~\ref{theo_LPM_CECSP}. Pour
cela, nous allons montrer que, soit $S$ une solution d'une instance
$\Pi$, alors nous pouvons transformer $S$ en une solution $S'$ ayant
la propriété que chaque fonction $b'_i(t)$ est constante par morceaux.

\begin{proof}[Preuve du théorème~\ref{theo_LPM_CECSP}]  
  Soit $S$ une solution réalisable de $\Pi$ et soit
  $(t_q)_{q=1..Q}$ la suite des des différentes dates de
  début et de fin d'activité triées par ordre croissant. Clairement,
  nous avons $Q\le 2n$. 

  Par souci de clarté, nous définissons la fonction intermédiaire
  $\tilde{b}_i(t),\ \forall i \in \A$, de la façon suivante:  

    \[\tilde{b}_i(t) =\left\{
        \begin{array}{lll}
          b_{i0} & & \text{si $t \in [t_0,t_1]$}\\
          \multicolumn{2}{c}{\vdots} &   \\
          b_{i(Q-1)} & & \text{si $t \in [t_{Q-1},t_Q]$}
        \end{array}
      \right.\]
    avec $b_{iq}=\frac{\int_{t_q}^{t_{q+1}} b_i(t) dt}{t_{q+1}-t_q}$.

    La solution $S'$ est alors construite de la manière suivante: 
    \begin{itemize}
    \item $st'_i=st_i$ 
    \item $b'_i(t)= \left\{ 
        \begin{array}{lll}
          \tilde{b}_i(t) &\quad& \text{si $t \in [st_i,et'_i]$}\\
          0 &\quad& \text{sinon}\\
        \end{array}
      \right.$
  \item $et'_i=\min(\tau | \int_{st_i}^{\tau} f_i(\tilde{b}_i(t))dt=W_i)$
  \end{itemize}

  Il est facile de voir que $S'$ satisfait les contrainte de fenêtre de
  temps~\eqref{tw_CECSP}, puisque, par le Lemme~\ref{lemmaEn},
  $et'_i\le et_i$. De plus, $S'$ vérifie la contrainte 
  d'énergie~\eqref{nrj_CECSP} puisqu'elle est définie de cette
  façon. Enfin, $S'$ vérifie aussi la contrainte de capacité de la
  ressource~\eqref{res_CECSP}. En effet, comme $S$ est une solution
  réalisable, nous avons $\forall q \in \{1,\dots,Q\}$ et $\forall t
  \in [t_q,t_{q+1}]$:  
  $\sum_{i\in \A}b_i(t) \le R \Rightarrow  
  \sum_{i\in \A} \int_{t_q}^{t_{q+1}} b_i(t)dt \le B(t_{q+1}-t_q)$.
 
  Donc, 
  \begin{align*}
    \sum_{i\in \A}b'_i(t) &\le 
                            \sum_{i\in \A} \tilde{b}_i(t)\\
                          &= 
                            \sum_{i\in \A} b_{iq}\\
                          &=
                            \sum_{i\in \A} \frac{\int_{t_q}^{t_{q+1}} b_i(t)dt}{t_{q+1}-t_q} \\
                          &\le B
  \end{align*}
  Nous pouvons montrer que $S'$ vérifie les contraintes de
  consommation minimale et maximale de la ressource d'une façon
  similaire. 
\end{proof}

Une remarque intéressante peut être faite à partir de la preuve du
théorème précédent. En effet, la nouvelle solution $S'$ possède la
propriété suivante: l'ensemble des points $t \in \H$ coïncidant avec
une variation de la consommation de ressource d'une activité $i$,
i.e. $\{t \in \H \ |\ \forall \epsilon>0,\ b_i(t) \neq b_i(t+\epsilon)\}$,
est contenue dans l'ensemble formé de toutes les dates de début et de
fin des activités. C'est ce qu'affirme le corollaire suivant: 

\begin{coro}
$\{t \in \H \ |\ \forall \epsilon>0,\ b_i(t) \neq b_i(t+\epsilon)\}
\subseteq \{st_i,et_i\ |\ i \in \A \}$.
\end{coro}

De plus, nous pouvons en déduire que le \CECSP~à date de début et de
fin fixées peut être résolu en temps polynomial. 

\begin{prop}
Soit $\Pi$ une instance du \CECSP~avec des dates de début, $st_i$, et
des dates de fin, $et_i$, fixées. On peut vérifier que $\Pi$ est réalisable en temps 
polynomial en la taille de l'instance. 
\end{prop}

En effet, dans ce cas là, il suffit de décider pour chaque intervalle
composé de deux dates de début/fin consécutifs, i.e. de la forme
$[st_i,st_j],\ [st_i,et_j],\ [et_i,et_j]$ ou $[et_i,st_j]$, la
quantité de ressource consommée par chaque activité à l'intérieur de
cet intervalle. Ce problème peut facilement être modélisé par un
programme linéaire.

Soit $(t_q)_{q=1..Q}$ la suite définie dans la preuve du
théorème~\ref{theo_LPM_CECSP} et $b_{iq}$ (respectivement $w_{iq}$),
$\forall (i,q) \in \A\times\{1,\dots,Q-1\}$, la
quantité de ressource consommée par (resp. la quantité d'énergie
apportée à) l'activité $i$ dans l'intervalle
$[t_q,t_{q+1}]$. Rappelons que $Q \le 2n$. Le programme linéaire
s'écrit alors de la manière suivante:
{\small
\begin{align}
&\sum_{\i \in A} b_{iq} \le R & \forall q\in
\{1..Q-1\} \label{poly2}\\ & b_{iq} \le \bmax & \forall i \in \A,\
\forall q \in \{1..Q-1\} |\ t_q \in [st_i,et_i[\label{poly3}\\ &
b_{iq} \ge \bmin& \forall i \in \A,\ \forall q \in \{1..Q-1\} |\ t_q
\in [st_i,et_i[\label{poly4}\\ & b_{iq}=0 & \forall i \in \A,\ \forall
q \in \{1..Q-1\} |\ t_q \not\in [st_i,et_i[\label{poly5}\\ &
\sum_{q=1}^{Q-1} w_{iq}(t_{q+1}-t_q) = W_i & \forall i \in
\A\label{poly6}\\ & w_{iq} \le a_{ip}b_{iq} + c_{ip} & \forall i \in
\A,\ \forall p \in \P_i,\ \forall q \in \{1..Q-1\}\label{poly7}\\ &
w_{iq} \le Mb_{iq} & \forall i \in \A,\ \forall q \in
\{1..Q-1\}\label{poly8}
\end{align} }
pour $M$ une constante suffisamment grande et $\P_i=\{1,\dots,P_i\}$
le nombre d'intervalles de définition de la fonction $f_i$. La
contrainte~\eqref{poly2} modélise la contrainte de capacité de la
ressource. Les contraintes~\eqref{poly3} et~\eqref{poly4} assurent les
contraintes de consommation minimale et maximale de la ressource
tandis que la contrainte~\eqref{poly5} fixe la consommation de la
ressource à $0$ si l'activité n'est pas en cours. La
contrainte~\eqref{poly6} stipule que chaque activité doit recevoir la
quantité d'énergie requise. Enfin, les contraintes~\eqref{poly7} et
\eqref{poly8} assure la conversion ressource/énergie. De plus la
contrainte~\eqref{poly8} fixe $w_{iq}$ à $0$ si $b_{iq}=0$,
i.e. modélise $f_i(0)=0$.

On peut remarquer que si $\forall i \in \A,\ \bmin=0$, alors le
problème devient polynomial. En effet, il suffit de prendre 
$(t_q)_{q=1..Q}$ la suite des différentes dates de début (resp. fin)
au plus tôt (resp. tard). Alors, le programme linéaire précédent nous
donne une solution réalisable. 

\begin{theo}
Le \CECSP~préemptif ($\forall i \in \A,\ \bmin=0$) peut être résolu en
temps polynomial.
\end{theo}

De ce fait, dans la suite, nous considérerons que $\exists i \in \A$
tel que $\bmin\neq 0$.
