\message{ !name(ordo_nrj.tex)}
\message{ !name(ordo_nrj.tex) !offset(431) }
\subsubsection{Contexte}


Dans un premier temps, nous nous intéressons aux extensions du
\RCPSP. Une des extensions les plus célèbres est le \RCPSP~multimode 
(MRCPSP). Dans ce problème, un choix de différents modes est
disponible pour chaque activité et une activité doit être exécutée
selon un de ces modes. Un mode correspond à une combinaison formée
d'un temps d'exécution constant et d'une consommation de ressource qui
permet d'apporter à l'activité au moins la quantité d'énergie
requise. Même si de nombreux problèmes basés sur ce concept de mode
existent~\cite{DDH,RK,RDK,DD} et que des méthodes de résolution
efficaces ont été mises en place pour résoudre le MRCPSP~\cite{PV},
cette modélisation peut amener à une mauvaise allocation de la
ressource. 

Si nous reprenons l'exemple de la peinture d'un bateau,
décrit au paragraphe~\ref{sec:limit_CUSP}, l'activité avait besoin de
$3$ unités d'énergie pour s'exécuter. Dans le contexte du MRCPSP,
seulement $3$ modes seraient décrits: $(3,1),\ (2,2)$ et $(1,3)$. Or,
dans le second cas, on donne une unité de trop à l'activité et la
possibilité d'allouer $2$ unités de ressource pendant une période de
temps et $1$ unité pendant la seconde n'est pas représentée ici. 

La principale limitation du MRCPSP est donc que les activités sont
contraintes à être rectangulaire, i.e. consommation de ressource
constante. 

D'autres extensions du \RCPSP~existent. C'est le cas par exemple des
problèmes d'ordonnancement de projet avec une ressource de type {\it
  work-content}~\cite{FT} ou du problème d'ordonnancement de projet
avec des profils de ressource flexibles (FRCPSP)~\cite{NK}. 


\begin{table}[!htb]
  \centering
  \begin{tabular}{|c|cc>{\centering\arraybackslash} p{2cm}>{\centering\arraybackslash} p{2cm}>{\centering\arraybackslash} p{2cm}>{\centering\arraybackslash} p{2cm}>{\centering\arraybackslash} p{2cm}|}
    \hline    
    Problème & $\bmin$ & $\bmax$ & fonction de rendement ($f_i$) &
                                                                 énergie
                                                                   ($W_i$)
    & ressource continue & autre différence \\
\hline
    

  \end{tabular}
  \caption{Principales différences entre les extensions des problèmes
    cumulatifs et le \CECSP.}
  \label{tab:dif_CECSP}
\end{table}
Le paragraphe suivant présente des propriétés du \CECSP~qui seront 
utilisées dans la suite de ce manuscrit. 

\message{ !name(ordo_nrj.tex) !offset(-56) }
