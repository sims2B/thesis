\chapter*{Conclusion\markboth{CONCLUSION}{}}
Dans cette partie présentant des méthodes issues de la programmation
par contraintes, nous avons commencé par brièvement rappeler les
principales notions sur lesquelles reposent ce paradigme. De plus, une
description des principales techniques de modélisation des problèmes
d'ordonnancement à été donnée. 

Ces deux paragraphes introductifs nous ont ensuite permis de décrire
un problème d'ordonnancement particulier, le problème cumulatif. Pour
ce problème, un grand nombre d'algorithmes de vérification et de
filtrage ont été présentés. Ces algorithmes se classent en trois
catégories: les algorithmes impliquant des raisonnements simples, ceux
utilisant le concept d'énergie et enfin ceux combinant plusieurs
raisonnements, souvent simples, pour améliorer leurs
performances. Pour chacune de ces catégories, au moins un exemple d'un
tel algorithme est détaillé: le Time-Table et le raisonnement
disjonctif dans le cas des algorithmes simples, le raisonnement
énergétique pour les algorithmes utilisant le concept d'énergie et
enfin le Time-Table disjonctif dans le dernier cas. Pour le
raisonnement énergétique, une étude de Derrien {\it et al.}~\cite{DP}
permettant de caractériser les intervalles d'intérêt du raisonnement
énergétique pour la contrainte cumulative est présentée.

Les raisonnements présentés pour la contrainte cumulative sont ensuite
adapté dans le cas du \CECSP. Le premier raisonnement que nous avons
adapté au \CECSP~est le raisonnement énergétique. En effet,
l'adaptation d'un raisonnement pour la contrainte cumulative dans le
cas du \CECSP~est généralement beaucoup moins fort. De ce fait, nous
avons choisi de commencer par adapter un des raisonnements les plus
forts au \CECSP. Pour ce raisonnement, nous avons aussi appliqué au
cas du \CECSP~les travaux de Derrien {\it et al.}~\cite{DP} pour la
contrainte cumulative, ce qui a permis de prouver la polynomialité de
ce raisonnement dans le cadre du \CECSP. De plus, deux autres
méthodes de calcul de ces intervalles pour l'algorithme de
vérification sont présentées. Ces deux méthodes reposent sur une étude
des segments de définition des fonctions de consommation individuelle
des activités.

D'autres adaptations des raisonnements cumulatifs sont aussi
détaillés. Ces raisonnements s'adapte aisément au cas du \CECSP. Ils
restent cependant beaucoup moins puissants en terme de déduction que
leur homologues cumulatifs. De ce fait, un nouvel algorithme de
vérification couplant un programme linéaire adapté d'un problème de
flot et le raisonnement Time-Table est présenté. Ce raisonnement
n'étant pas dominé par le raisonnement énergétique peut être couplé
avec ce dernier afin de déduire plus d'incohérence. Il peut aussi être
couplé avec des algorithmes de filtrage tels que celui du raisonnement
disjonctif ou du Time-Table. 

Pour la poursuite de ces travaux, il est raisonnable de chercher à
déterminer des règles ou propriétés selon lesquelles un raisonnement a
plus de chance d'appliquer des ajustements qu'un autre. De plus,
l'adaptation d'autres raisonnements définis pour la contrainte
cumulative peut être envisagée. Cependant, la mise en place de
raisonnements dédiés semble être une perspective plus prometteuse. 
