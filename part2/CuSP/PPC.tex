\section{La programmation par contraintes}
\label{sec:PPC}
Cette section s'intéresse à la présentation des concepts de base de la
programmation par contraintes (PPC). La programmation par contraintes
vise à résoudre des problèmes de satisfaction de contraintes (CSP)
mais aussi des problèmes d'optimisation (ce dernier cas ne sera pas
traité dans ce manuscrit). Pour cela, un problème est
modélisé à l'aide d'un réseau de contraintes et la recherche d'une
solution tend à trouver une affectation des variables satisfaisant
toutes les contraintes de ce réseau. Une présentation formelle des
problèmes de satisfaction de contraintes ainsi qu'un aperçu de
quelques méthodes permettant leur résolution sont présentés dans les
sous-sections suivantes. 

\subsection{Problème de satisfaction de contraintes}

Une instance d'un {\it problème de satisfaction de contraintes}, ou
CSP\footnote{en anglais, Constraint Satisfaction Problem.} est
la donnée d'un triplet $(\X,\D,\C)$ où: 
\begin{itemize}
\item $\X=\{x_1,x_2,\dots,x_n\}$ est l'ensemble des variables du problème;
\item $\D=\{D_1,D_2,\dots,D_n\}$ est l'ensemble des domaines de ces
  variables, i.e. $x_i \in D_i,\ i=1,\dots,n$;
\item $\C=\{c_1,c_2,\dots,c_m\}$ est l'ensemble des contraintes du
  problème où chaque $c_j$ définit un sous-ensemble du produit
  cartésien des domaines des variables sur lesquelles elle porte:

  $c_j(x_{j1},x_{j2},\dots,x_{jk}) \subseteq D_{j1} \times D_{j2}
  \times \dots \times D_{jk}$
\end{itemize}

La notion de domaine désigne l'ensemble des valeurs que peut prendre
une variable. La nature de ces domaines peut potentiellement être très
différente. Par exemple:
\begin{itemize}
\item un ou plusieurs intervalle d'entiers;
\item un ou plusieurs intervalle de réels;
\item un ensemble d'entiers non contigus: il est possible d'utiliser
  un ensemble d'entiers quelconque, e.g. $D=\{4,9,26\}$;
\item un ensemble de valeurs symboliques: on peut vouloir représenter
  des couleurs, ou encore des jours de la semaine... 
\end{itemize}
~

Une fois les variables définies, nous pouvons ajouter des contraintes
les liant entre elles. Formellement, une contrainte peut être définie
comme une relation portant sur un ensemble de variables.  Ici aussi,
plusieurs types de contraintes existent. Nous détaillons trois d'entre
elles.

Les premières contraintes présentées sont les contraintes en {\it
  extension}. Pour ces contraintes, étant donné un sous-ensemble de
variables, on définit explicitement la liste des tuples
autorisés. Suivant les cas, ces contraintes peuvent aussi être définies
comme une liste de tuples interdits. Les deuxièmes contraintes
détaillées sont les contraintes en {\it intention}. Dans ce cas,
chaque contrainte est décrite sous la forme d'une expression arithmétique
définissant  une relation entre les variables. Enfin, les dernières
contraintes décrites sont les contraintes globales. Ces contraintes
sont des relations prédéfinies, ayant une signification précise. Des
exemples de telles contraintes sont décrits dans
l'exemple~\ref{ex:contrainte}. Notons que si les variables sont à
valeurs dans $\mathbb{R}$ alors une définition en extension de la
contraintes peut comprendre un nombre infini de tuples.

\begin{ex}
\label{ex:contrainte}
  Soient trois variables $x_0,\ x_1$ et $x_2$ de domaines respectifs
  $D_0=[0,2],\ 
  D_1=[0,2]$ et $D_2=[1,2]$. 
  Nous considérons les contraintes $c_0,\ c_1$ et $c_2$ suivantes:
  \begin{itemize}
  \item $c_0$ porte sur les variables $x_0$ et $x_1$ et est décrite en
    intention: $x_0 \le x_1$; 
  \item $c_1$ porte sur $x_1$ et $x_2$ et est décrite en extension:
    $\{(0,1) , (0,2), (1,2), (2,2)\}$; 
  \item $c_2$ est une contrainte  globale portant sur les variable
    $x_0,x_1$ et $x_2$: allDifferent$(x_0,x_1,x_2)$. Cette contrainte
    stipule que les valeurs affectées à chaque variable sont
    différentes les unes des autres. 
  \end{itemize}
  Les domaines des variables étant continus, un nombre infini de
tuples aurait été nécessaire pour écrire $c_0$ en extension.
\end{ex}

Pour trouver une solution à un CSP, il faut instancier toutes les
variables du problème de telle sorte que toutes les contraintes
soient satisfaites. Une variable est dite {\it instanciée} quand on
lui assigne une valeur de son domaine. Pour trouver une telle
instanciation, un certain nombre de techniques ont été mises en
place. Ces techniques reposent principalement sur deux éléments
centraux: le filtrage des domaines et l'exploration de l'espace de
recherche. Ces deux concepts sont donc décrits dans les sous-sections 
suivantes. 



\subsection{Exploration de l'espace de recherche}
\label{sec:PPC_rech}

{\'E}tant donné un CSP, différentes techniques d'exploration de
l'espace de recherche peuvent être employées pour trouver des
solutions. Cette exploration se fait en général par {\it séparation et
  évaluation}, i.e. on sépare le problème difficile à résoudre en deux
sous-problèmes, plus petits, jusqu'à obtenir un problème que l'on sera
capable de résoudre en temps raisonnable.  

Une représentation usuelle du processus de recherche d'une solution
est l'{\it arbre de recherche}. La racine de cet arbre représente le
problème que l'on cherche à résoudre et les sommets sont des problèmes
réduits, obtenus en décomposant le domaine d'une des variables du
problème père. Les feuilles de cet arbre correspondent donc à des
instanciations de toutes les variables du problème.

Une première approche consiste alors à générer tous les
tuples de valeurs possibles et de tester s'ils sont solution du
problème, i.e. si cette instanciation satisfait bien toutes les
contraintes du problème. Cela revient à considérer toutes les
affectations possibles des variables et ce nombre, qui peut ne pas
être fini dans le cas des variables continues, est égal au produit
cartésien des cardinaux des domaines des variables impliquées dans le
CSP pour des variables dans $\mathbb{Z}$.

Classiquement, une méthode de séparation, ou de branchement, consiste
à fixer une variable à une valeur pour le premier sous-problème et de
retirer cette valeur du domaine de la variable dans le second
sous-problème. On va alors, à l'aide d'un parcours en profondeur, créer
un premier tuple, i.e. une solution candidate, qui pourra ainsi être
évalué. Si le tuple satisfait toutes les contraintes du problème,
alors c'est une solution et l'algorithme peut s'arrêter. Dans le cas 
contraire, on évaluera le second sous-problème créé lors de la
dernière séparation. Un tel parcours est décrit dans
l'exemple~\ref{ex:branchement}.

Dans le cas de CSP comprenant des variables continues, l'énumération
de tous les domaines des variables n'est pas possible. Une première
approche serait de considérer qu'une variable est instanciée quand son
domaine est réduit à un intervalle de taille suffisamment petite. Mais
même avec cette restriction supplémentaire, il est très coûteux
d'énumérer chacun de ces intervalles. L'idée est alors la suivante: au
lieu, à chaque étape de l'algorithme de séparation, d'instancier une
variable à une valeur de son domaine, nous séparons le domaine de
cette variable en deux (ou plusieurs) sous-domaines.

\begin{ex}
  \label{ex:branchement}
  Soient $3$ variables, $x_0,\ x_1$ et $x_2$, de domaines
  respectifs: $\{1,2\},\ \{1,2\},\ \{1,2,3\}$. Nous considérons les
  contraintes suivantes: $c_0: x_0<x_1$ et $c_1 : x_1 <x_2$.   
  \begin{figure}[!htb]
    \centering
    \begin{tikzpicture}
      [level 1/.style={sibling distance=7cm,level distance=1.8cm},
      level 2/.style={sibling distance=3.5cm,level distance=1.8cm},
      level 3/.style={sibling distance=2.5cm,level distance=1.8cm},
      level 4/.style={sibling distance=1.5cm,level distance=1.8cm},
      every node/.style={color=black},%
      dot/.style={circle,fill=black,minimum size=4pt,inner sep=0pt,%
        outer sep=-1pt},
      cross/.style={path picture={ 
          \draw
          (path picture bounding box.south east) -- (path picture bounding box.north west) (path picture bounding box.south west) -- (path picture bounding box.north east);
        }}]

      \node[tree] {} %racine
      child{ node [tree]{}  %l1
        child{ node[tree] {}%l2
          child{ node[leaf,cross] {} %l3
         edge from parent node[sloped,above]{$x_2=1$}}
          child{ node[tree] {} %l3
            child{ node[leaf,cross] {}
              edge from parent node[sloped,above]{$x_2=2$}}
            child{ node[leaf,cross] {}
              edge from parent node[sloped,above]{$x_2=3$}}
            edge from parent node[sloped,above]{$x_2\neq 1$}
          }
          edge from parent node[sloped,above]{$x_1=1$} }
        child{ node[tree] {} %l2
          child{ node[leaf,cross] {}%l3
            edge from parent node[sloped,above]{$x_2=1$}
          }
          child{ node[tree] {}%l3
            child{ node[leaf,cross] {}
              edge from parent node[sloped,above]{$x_2=2$}
            }
            child{ node[sol] {$\checkmark$}
              edge from parent node[sloped,above]{$x_2=3$}
            }          
            edge from parent node[sloped,above]{$x_2 \neq 1$}}
          edge from parent node[sloped,above]{$x_1=2$} 
        }
        % etc.
        edge from parent node[sloped,above] {$x_0=1$}
      }
      child{ node [tree]{}  %l1
        child{ node[tree] {}%l2
          child{ node[leaf,cross] {} %l3
         edge from parent node[sloped,above]{$x_2=1$}}
          child{ node[tree] {} %l3
            child{ node[leaf,cross] {}
              edge from parent node[sloped,above]{$x_2=2$}}
            child{ node[leaf,cross] {}
              edge from parent node[sloped,above]{$x_2=3$}}
            edge from parent node[sloped,above]{$x_2\neq 1$}
          }
          edge from parent node[sloped,above]{$x_1=1$} }
        child{ node[tree] {} %l2
          child{ node[leaf,cross] {}%l3
            edge from parent node[sloped,above]{$x_2=1$}
          }
          child{ node[tree] {}%l3
            child{ node[leaf,cross] {}
              edge from parent node[sloped,above]{$x_2=2$}
            }
            child{ node[leaf,cross] {}
              edge from parent node[sloped,above]{$x_2=3$}
            }          
            edge from parent node[sloped,above]{$x_2 \neq 1$}}
          edge from parent node[sloped,above]{$x_1=2$} 
        }
        edge from parent node[sloped,above]{$x_0=2$}
        % etc.
      };
    \end{tikzpicture}
    \caption{Exemple de parcours d'un arbre de recherche.}
    \label{fig:ex_tree}
  \end{figure}

La figure~\ref{fig:ex_tree} montre un parcours complet de l'espace de
recherche. Les noeuds internes de l'arbre (en gris) sont les noeuds
pour lesquels toutes les variables ne sont pas instanciées. Les
feuilles de l'arbre correspondent bien à des instanciations complètes
des variables. Parmi celles-ci, celles marquées d'une croix sont celles
qui ne satisfont pas la contrainte, tandis que celle représentée par
un noeud carré correspond à une solution réalisable du problème.
\end{ex}

Ces combinaisons peuvent être générées de différentes manières et donc
testées dans des ordres différents. Cet ordre peut avoir une influence
déterminante dans le processus de résolution d'un problème. Pour
définir cet ordre, on utilise des {\it heuristiques} de choix de
variable et de valeur. Une {\it heuristique de choix de
  variable} détermine la variable que l'on va instancier
prioritairement. Une {\it heuristique de choix de
  valeur} détermine la ou les valeurs à affecter en priorité à
cette variable. 

Nous donnons ci-après des exemples de telles heuristiques. 
\begin{description}
\item[Heuristiques de choix de variable]~

  \begin{itemize}
  \item instanciation des variables dans l'ordre lexicographique,
    c'est l'heuristique utilisée dans la figure~\ref{fig:ex_tree};
  \item instanciation des variables suivant un ordre aléatoire;
  \item instanciation de la variable de plus petit domaine~\cite{min_dom}...
  \end{itemize}
\item[Heuristiques de choix de valeur] ~

  \begin{itemize}
  \item sélectionner la valeur minimale (resp. maximale) du domaine de
    la variable courante, c'est l'heuristique utilisée dans la
    figure~\ref{fig:ex_tree}; 
  \item sélectionner une valeur aléatoire dans le domaine de la
    variable courante;
  \item dans le cas de variables continues, on peut choisir de séparer
    l'intervalle en deux ou plus de morceaux et le séparer au milieu,
    au tiers...
  \end{itemize}
\end{description}
Dans la suite du manuscrit, nous présenterons d'autres heuristiques
que nous appliquerons dans notre processus de recherche. 

Même si plusieurs heuristiques efficaces existent, on ne peut être sûr
de ne pas devoir explorer tout l'espace de recherche pour trouver une
solution. Comme il n'est pas concevable, en pratique, de devoir
explorer un tel espace, des techniques permettant de réduire l'arbre
de recherche ont donc été mises en place. Les techniques les plus
célèbres permettant une telle réduction sont les algorithmes de
filtrage. Ces algorithmes sont décrits dans la sous-section suivante.

\subsection{Détection d'incohérence et Filtrage}
\label{sec:PPC_propag}

Afin de réduire l'espace de recherche en programmation par contraintes,
il est indispensable de mettre en place des techniques permettant de
vérifier la validité d'une solution afin de ne pas considérer un tuple
comme étant cohérent s'il ne l'est pas. Ces {\it algorithmes de
détection d'incohérences}, aussi appelés checkers, permettent de
refuser une instanciation des variables si cette dernière n'est pas
cohérente, i.e. si elle viole un des contraintes. Un algorithme de
vérification est donc associé à chaque contrainte.

De plus, ces algorithmes peuvent aussi être étendus de manière à
détecter une incohérence en cours de recherche si aucune solution
n'est contenue dans le sous-arbre courant.
\begin{ex}
\label{ex:ex_checker}  Soit l'instance définie dans
l'exemple~\ref{ex:branchement}.    
  \begin{figure}[!htb]

    \centering
    \begin{tikzpicture}
      [level 1/.style={sibling distance=7cm,level distance=1.8cm},
      level 2/.style={sibling distance=3.5cm,level distance=1.8cm},
      level 3/.style={sibling distance=2.5cm,level distance=1.8cm},
      level 4/.style={sibling distance=1.5cm,level distance=1.8cm},
      every node/.style={color=black},%
      dot/.style={circle,fill=black,minimum size=4pt,inner sep=0pt,%
        outer sep=-1pt},
      cross/.style={path picture={ 
          \draw
          (path picture bounding box.south east) -- (path picture bounding box.north west) (path picture bounding box.south west) -- (path picture bounding box.north east);
        }}]

      \node[tree] {} %racine
      child{ node [tree]{}  %l1
        child{ node[tree,fill=black] {}%l2
          edge from parent node[sloped,above]{$x_1=1$}}
        child{ node[tree] {} %l2
          child{ node[leaf,cross] {}%l3
            edge from parent node[sloped,above]{$x_2=1$}
          }
          child{ node[tree] {}%l3
            child{ node[leaf,cross] {}
              edge from parent node[sloped,above]{$x_2=2$}
            }
            child{ node[sol] {$\checkmark$}
              edge from parent node[sloped,above]{$x_2=3$}
            }          
            edge from parent node[sloped,above]{$x_2 \neq 1$}}
          edge from parent node[sloped,above]{$x_1=2$} 
        }
        % etc.
        edge from parent node[sloped,above] {$x_0=1$}
      }
      child{ node [tree]{}  %l1
        child{ node[tree,fill=black] {}%l2
          edge from parent node[sloped,above]{$x_1=1$} }
        child{ node[tree,fill=black] {} %l2
          edge from parent node[sloped,above]{$x_1=2$} 
        }
        edge from parent node[sloped,above]{$x_0=2$}
        % etc.
      };
    \end{tikzpicture}
    \caption{Exemple de parcours d'un arbre de recherche avec
      détection d'incohérences.}
    \label{fig:ex_tree_check}
  \end{figure}

La figure~\ref{fig:ex_tree_check} montre un parcours de l'espace de
recherche si l'algorithme utilisé est muni d'un algorithme de
détection d'incohérence. Les noeuds noirs correspondent à des noeuds
pour lesquels aucune solution ne peut exister dans les sous-arbres
ayant un de ces noeuds pour racine. En effet, l'instanciation
partielle des variables faites à ce niveau est déjà incohérente avec
les contraintes $c_0$ ($x_0 < x_1$) et $c_1$ ($x_1 < x_2$).
\end{ex}


Une deuxième technique permettant la réduction de l'espace de
recherche est l'utilisation d'algortihme de {\it filtrage}. Un
algorithme de filtrage est aussi défini pour chaque contrainte et cet
algorithme consiste à détecter et à retirer des domaines des
variables, des valeurs qui ne sont pas cohérentes avec la contrainte.

 La mise en place de techniques de filtrage performantes,
i.e. qui retirent un maximum de valeurs du domaine des variables,
jouent un rôle important dans la résolution de problème en
programmation par contraintes. Un exemple simple de filtrage est
présenté dans l'exemple~\ref{ex:ex_filtrage}.

\begin{ex}
\label{ex:ex_filtrage}  Soit l'instance définie dans
l'exemple~\ref{ex:branchement}.       
  \begin{figure}[!htb]

    \centering
    \begin{tikzpicture}
      [level 1/.style={sibling distance=7cm,level distance=1.8cm},
      level 2/.style={sibling distance=3.5cm,level distance=1.8cm},
      level 3/.style={sibling distance=2.5cm,level distance=1.8cm},
      level 4/.style={sibling distance=1.5cm,level distance=1.8cm},
      every node/.style={color=black},%
      dot/.style={circle,fill=black,minimum size=4pt,inner sep=0pt,%
        outer sep=-1pt},
      cross/.style={path picture={ 
          \draw
          (path picture bounding box.south east) -- (path picture bounding box.north west) (path picture bounding box.south west) -- (path picture bounding box.north east);
        }}]

      \node[tree] {} %racine
      child{ node [tree]{}  %l1
        child{ node[] {}%l2
          edge from parent[white]
          node[rectangle, draw,above left=1cm,text width=2.6cm,align=center] (exp1) {la valeur $1$ est supprimée des
            domaines de $x_1$ et $x_2$} {};
          \draw[->] (exp1.east) -- (\tikzparentnode.west);}
        child{ node[tree] {} %l2
          child{ node[] {}%l3
            edge from parent[white] {}
          }
          child{ node[tree] {}%l3
            child{ node[] {}
              edge from parent[white] {} 
            }
            child{ node[sol] {$\checkmark$}
              edge from parent node[sloped,above]{$x_2=3$}
            }          
            edge from parent node[sloped,above]{$x_2 \neq 1$}}
          edge from parent node[sloped,above]{$x_1=2$} 
        }
        % etc.
        edge from parent node[sloped,above] {$x_0=1$} 
      }
      child{ node [tree]{}  %l1
        child{ node[] {}
          edge from parent[white]
          node[rectangle, draw,above right=0.75cm and 2.55cm,text
          width=2.6cm,align=center] (exp2) {les valeurs $1$ et $2$ sont supprimées du
            domaine de  $x_1$} {};
          \draw[->] (exp2.west) -- (\tikzparentnode.east);}
        child{ node[] {}
          edge from parent[white]}
        edge from parent node[sloped,above]{$x_0=2$}
        % etc.
      };
    \end{tikzpicture}
    \caption{Exemple de parcours d'un arbre de recherche avec fitrage
      des domaines.}
    \label{fig:ex_tree_filtrage}
  \end{figure}

La figure~\ref{fig:ex_tree_filtrage} montre un parcours de l'espace de
recherche si l'algorithme utilisé est muni d'un algorithme de filtrage
des domaines. Les contraintes sont considérées une par une.  A
la racine, on ne peut rien déduire, mais une fois que $x_0$ est
instanciée à $1$, alors nous pouvons retirer cette valeur du domaine
des autres variables car, pour satisfaire les contraintes $c_0$ et
$c_1$, les variables $x_1$ et $x_2$ doivent être strictement
supérieures à $1$. Puis, on instancie $x_1$ à $2$ et on supprime cette
valeur du domaine de $x_2$. La seule valeur possible pour $x_2$ est
alors $3$ et on obtient une solution.

L'autre cas correspond à l'instanciation de $x_0$ à $2$. Dans ce
cas-là, les valeurs $1$ et $2$ sont rétirées du domaines de $x_1$ qui
devient vide. Donc il n'existe pas de solution avec $x_0=2$.
\end{ex}

Dans l'exemple~\ref{ex:ex_filtrage}, nous pouvons remarquer que si on
avait considéré simultanément les $2$ contraintes, on
aurait pu déduire la solution au noeud racine. 

Quand les contraintes sont binaires, i.e. elles ne concernent que deux
variables à la fois, on peut représenter schématiquement le CSP par un
graphe. Dans ce graphe, chaque noeud correspond à une variable et
chaque arc à une contrainte. Si on vérifie la cohérence des
contraintes une par une, on parle alors de cohérence d'arcs. Cette
notion peut être généralisée à la considération simultanée de
plusieurs contraintes mais aussi à des contraintes contenant plus de
deux variables.

Enfin, pour certains problèmes, il n'est pas possible de retirer
toutes les valeurs incohérentes du domaines des variables en temps
raisonnables. En effet, il peut arriver que les problèmes sous-jacents
des contraintes que l'on est amené à résoudre soient NP-complets. Dans
ce cas-là, assurer de retirer les valeurs incohérentes signifie que
les valeurs restantes sont cohérentes, donc qu'il existe une
solution, ce qui ne peut être fait en temps raisonnable puisque le
problème est NP-complet.  Dans ce cas, il devient crucial de mettre
en place des algorithmes qui s'exécutent en temps raisonnable mais qui
n'assure pas un filtrage complet du domaine des variables. De tels
algorithmes seront présentés dans la sous-section~\ref{sec:cumu_propag}
dans le but de résoudre le \CUSP. La section suivante est quant à
elle dédiée à la modélisation des problèmes d'ordonnancement en PPC.