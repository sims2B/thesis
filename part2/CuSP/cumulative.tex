\section{L'ordonnancement cumulatif}
\label{sec:cumu}

\subsection{L'ordonnancement en programmation par contrainte}
\label{sec:cumu_ordo}

Les problèmes d'ordonnancement étudiés en programmation par
contraintes sont majoritairement regroupés en trois catégories: les
problème préemptifs, les problèmes non préemptifs et les problèmes
élastiques. Les problèmes préemptifs, respectivement non préemptifs,
sont des problèmes pour lesquels la préemption des activités est
autorisée, respectivement non autorisée. Une définition d'une activité
préemptive peut être trouvée dans le paragraphe~\ref{sec:ordo}. Les
{\it problèmes d'ordonnancement élastiques} correspondent aux
problèmes où la quantité de ressource attribuée à une activité peut
varier, à tout instant $t$ de l'horizon de temps, avec la contrainte
que la quantité totale de ressource consommée par l'activité durant
son exécution soit égale à une certaine valeur appelée {\it
  énergie}. Cette dernière notion peut aisément être étendue dans le
cas où l'énergie reçue par une activité n'est pas égale à la quantité
de ressource consommée mais où des fonctions de rendement modélisent 
cette conversion. Clairement, le \CECSP est un exemple typique d'un
tel problème. 

La majorité des problèmes d'ordonnancement non-préemptifs classiques
peuvent être modélisés à l'aide d'un problème de satisfaction de
contraintes. En général, trois variables représentant respectivement
la date de début d'une activité, notée $st_i$, sa date de fin, notée
$et_i$ et sa durée, notée $p_i$, sont définies. Les domaines de
chacune de ces variables sont définis par les données du problème. En
effet, pour chaque activité, nous pouvons calculer une date de début
au plus tôt, $\ES$, et au plus tard, $\LS$; ainsi, le domaine de la
variable $st_i$ est $[\ES,\LS]$. De même, le domaine de la variable
$et_i$ est $[\EE,\LE]$, avec $\EE$ la date de fin au plus tôt de $i$
et $\LE$ sa date de fin au plus tard. La durée d'une activité est
quant à elle définie comme la différence entre sa date de fin et sa
date de début, i.e. $p_i=et_i-st_i$.

Les problèmes préemptifs sont plus difficiles à modéliser. En effet,
dans ce cas-là, un ordonnancement valide ne peut seulement être
représenté par une date de début, de fin et une durée pour chaque
activité. Pour ces problèmes, au moins deux modélisations différentes
existent. La première consiste à associer à chaque activité une
variable d'ensemble, i.e. une variable dont la valeur sera un
ensemble. Cet variable représente l'ensemble des instants $t$ où
l'activité est en cours, définie comme un ensemble d'intervalles ou de
temps $t$ discret. Une second possibilité est de définir une variable
binaire, pour chaque activité et chaque instant $t$, qui prendra la
valeur $1$ si l'activité est en cours à l'instant $t$. Notons que,
dans le cas d'un problème d'ordonnancement continu, une telle solution
ne peut être envisageable car cela conduirait à un nombre infini de
telles variables.

Les problèmes élastiques sont, quant à eux modéliser, à travers les
contraintes de ressources. Il existe deux principaux types de
contraintes de ressource en PPC. Le premier permet de modéliser les
ressources disjonctives, i.e. qui ne peuvent exécuter qu'une activité
en parallèle, et le second sert à la modélisation des ressources
cumulatives. Ici, nous nous intéressons seulement au second cas.

{\'E}tant donné une activité et une ressource de capacité $R$, une
variable $r_i$ sert généralement à modéliser la consommation de
l'activité sur cette ressource. Dans le cas des tâches élastiques,
cette variable est remplacée par une variable $W_i$ représentant
l'énergie requise par l'activité. Pour représenter un ordonnancement,
un ensemble de variables représentant la quantité de ressource
utilisée par une activité à l'instant $t$ est introduit. Ces variables
sont ensuite liés entre elles par un ensemble de contraintes
modélisant le fait que chaque activité doit recevoir une énergie
$W_i$. 

Ces différents concepts peuvent être étendus pour modéliser d'autre
types de contraintes telles que des contraintes de ressources
alternatives, des temps de préparation entre les activités, des
activités optionnelles ou des contraintes de réservoirs.

Dans la suite de ce manuscrit, nous nous intéressons aux problèmes
d'ordonnancement non-préemptifs avec ressource cumulative et aux
problèmes d'ordonnancement élastiques. Le premier cas correspond au
\CUSP, décrit dans le paragraphe~\ref{sec:ordo_res}. En PPC, le
problème est modélisé à l'aide d'une contrainte globale appelées
contrainte cumulative. Cette contrainte ainsi que différents
algorithmes de filtrage mis en place pour cette dernière sont
présentés dans les deux paragraphes suivant. 

Les problèmes d'ordonnancement élastiques seront représentés par le
\CECSP, pour lequel une partie des algorithmes de filtrage pour la
contrainte cumulative sont adaptés et un modèle de PPC dans le cas de
la restriction discrète du \CECSP~seront présentés dans le
chapitre~\ref{sec:PPC_CECSP}.   

\subsection{La contrainte cumulative}
\label{sec:cumu_cume}

En PPC, le \CUSP~est modélisé par la contrainte cumulative. 

\subsection{Les filtrages de la contrainte cumulative}
\label{sec:cumu_propag}


