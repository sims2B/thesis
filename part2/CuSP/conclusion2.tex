\section*{Conclusion}

Dans ce chapitre, nous avons d'abord présenté les bases de la
programmation par contraintes. En particulier, nous avons vu que les
problèmes d'ordonnancement se traduisent naturellement en problème de
satisfaction de contraintes. 

Ce paradigme est ensuite appliqué au \CUSP, un des plus célèbres
problèmes d'ordonnancement traité en PPC. De ce fait, nous avons pu
présenter de nombreux algorithmes de filtrage, i.e. des algorithmes
filtrant les valeurs incohérentes des domaines des variables, pour ce
problème. Ces algorithmes, issus de la littérature, sont classés en
trois classes: les algorithmes de filtrage simples, ceux basés sur le
concept d'énergie et les algorithmes de filtrage étendus. Pour chacune
de ces trois classes, au moins un algorithme de propagation est
présenté en détail et de nombreux autres exemples sont cités.

Dans le paragraphe concernant les règles de filtrages simples, nous
avons présenté le raisonnement Time-Table et le raisonnement
disjonctif. Ces raisonnements sont parmi les moins forts pour le
\CUSP~mais leur facilité d'implémentation et leur rapidité en font des
algorithmes très utilisés en pratique. 

Le paragraphe suivant a présenté les règles de filtrage 
énergétiques. Parmi elles, sont mentionnées le raisonnement
Edge-Finding, les activités élastiques et le raisonnement énergétique
qui est présenté en détail. Ce dernier est, à ce jour, un des
raisonnements les plus forts pour le \CUSP. Cependant, la complexité
élevée de ce dernier fait de lui un des algorithmes les moins utilisés 
en pratique.

Enfin, le dernier paragraphe explique comment certains auteurs ont
étendu ou couplé des règles de filtrage existantes pour définir de
nouvelles règles de filtrage plus fortes que prises
individuellement. Parmi celles citées, on trouve le raisonnement
Edge-Fiding étendu, le Time-Table-Edge-Finding et le Time-Table
disjonctif. A l'inverse du raisonnement énergétique, ces règles
sont très utilisées en pratique car leur complexité est relativement
faible. 

Dans le prochain chapitre, nous présenterons l'adaptation de plusieurs
de ces raisonnements dans le cadre du \CECSP. Un nouveau raisonnement,
basé sur le Time-Table, sera aussi présenté.