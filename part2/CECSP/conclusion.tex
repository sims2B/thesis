\section*{Conclusion}

Dans ce chapitre, nous avons présenté plusieurs adaptations de
raisonnements existants ainsi que de nouveaux raisonnements dans le
cadre du \CECSP. Dans un premier temps, nous avons montré que certains
des raisonnements pour la contrainte cumulative pouvaient être
directement adaptés au \CECSP. En effet, dans le \CECSP~une activité
peut être vue comme deux sous-activités, une rectangulaire et une
représentant la partie malléable de l'activité. Considérer seulement
la partie rectangulaire de l'activité dans le pire des cas, en temps
ou en consommation de ressource, permet souvent l'adaptation directe
de plusieurs raisonnements. Pour illustrer ce propos, nous avons
présenté l'adaptation des raisonnements Time-Table, disjonctif et
Time-Table disjonctif. Cette technique nous a aussi permis de
développer un nouvel algorithme de vérification basée sur les flots et
utilisant la programmation linéaire. Cet algorithme peut être couplé
avec d'autres algorithmes, d'ajustement ou de vérification, comme le
Time-Table, disjonctif ou non, ou le raisonnement énergétique.

Cependant, d'autres raisonnements ne peuvent être adaptés
directement. C'est le cas du raisonnement énergétique pour lequel nous
avons présenté l'adaptation au \CECSP~de l'algorithme de vérification
ainsi que des ajustements des fenêtres de temps. De plus, une
caractérisation totale des intervalles à considérer dans le cas de
l'algorithme de vérification et des ajustements a été détaillée. Dans
le cadre de l'algorithme de vérification, plusieurs méthodes
permettant de calculer ces intervalles ont été présentées. Les
comparaisons de ces différentes méthodes ainsi que les performances
des algorithmes de filtrage seront présentées dans le
chapitre~\ref{sec:expe}. 