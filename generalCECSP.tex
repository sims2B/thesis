\documentclass{report}

\usepackage[utf8]{inputenc}
\usepackage[T1]{fontenc}
\usepackage[francais]{babel}
\usepackage{amsmath}
\usepackage{amssymb}
\usepackage{amsthm}
\usepackage{tikz}
\usetikzlibrary{patterns}
\usepackage{array}
\usepackage{caption}
\usepackage{subcaption}

\newcommand{\bb}{\underline{b}(i,t_1,t_2)}
\newcommand{\wb}{\underline{w}(i,t_1,t_2)}

\newtheorem{Th}{Théorème}
\newtheorem{Lem}{Lemme}

\begin{document}

\section{Condition nécessaire d'existence}

%TODO intro + peut etre definir le raisonnement energetique en general (ou sur CUSP) et ensuite l'appliquer???

%TODO preuve pour les fonction linéaire par morceaux à améliorer (on peut faire seulement avec un seul point de cassure)!!!!!! MAYBEEEEEEEEEEEEEEEEEEEEEEEEEe


\subsection{Cumulative Scheduling Problem (CuSP)}
	
Le problème CuSP se définit comme suit: nous avons en entrée un ensemble de $n$ tâches, chacune d'entre elles ayant un intervalle de temps ${[}r_i,d_i{]}$ durant lequel elle peut être exécutée et une durée $p_i$. De plus, chaque tâche consomme, durant son exécution, une quantité $b_i$ d'une ressource cumulative et renouvelable disponible en quantité $B$. Le but du problème est d'arriver à ordonnancer toutes les tâches dans leur fenêtre de temps et sans jamais dépasser la capacité de la ressource.\\
	
Ce problème, connu NP-complet, a été l'objet de nombreuses recherches. Parmi celle-ci, on retrouve une condition nécessaire d'existence, calculable en temps polynomial %TODO ref
Cette condition se base sur un raisonnement appelé raisonnement énergétique %TODO ref

	
\subsection{Définition du problème}
	
Dans le problème CECSP, nous voulons ordonnancer un ensemble $A=\{1, \cdots , n\}$ de $n$ tâches sur une machine. Chaque tâche s'exécute en utilisant une ressource cumulative et renouvelable, disponible en quantité $B$. De plus, une tâche $i$ doit être exécutée dans une fenêtre de temps ${[}r_i,d_i{]}$.\\
La particularité de ce problème est que, au lieu d'affecter à chaque tâche une durée, on lui affecte une quantité d'énergie $W_i$ qui devra lui être fournie durant son exécution. Pour cela, on donne à chaque tâche, à tout temps $t$, une quantité $b_i(t)$ de la ressource comprise entre une valeur minimum $b_i^{min}$ et une valeur maximum $b_i^{max}$. L'énergie fournie à la tâche $i$ est alors une fonction de $b_i(t)$.\\
Résoudre CECSP revient donc à trouver, pour chaque tâche $i$, sa date de début $st_i$, sa date de fin $ft_i$ et la fonction $b_i(t)$ de telle sorte que les conditions suivantes soient vérifiées:

\begin{align}
  \label{eq1}
  r_i \le st_i \le ft_i \le d_i & & (i \in A)\\
  \label{eq2}
  b_i^{min} \le b_i(t) \le b_i^{max} & & (i \in A; t\in {[}st_i,ft_i{]})\\
  \label{eq3}
  b_i(t)=0 & & (i \in A; i \in {\cal T} \setminus {[}st_i,ft_i{]}\\
  \label{eq4}
  \int_{st_i}^{ft_i} f_i(b_i(t))dt = W_i & & (i \in A)\\
  \label{eq5}
  \sum \limits_{i\in A} b_i(t) \le B & & (t \in {\cal T})
\end{align}

où ${\cal T}={[}\min_i r_i,\max_i d_i{]}$ et $f_i(b)$ est une fonction croissante et continue.\\

Ce problème est NP-complet car c'est une généralisation du problème CuSP, connu NP-complet. En effet, une instance de CuSP s'obtient en posant $b_i^{min}=b_i^{max}=b_i$, $f_i(b)=b$ et la durée d'une tâche $i$ est alors $p_i=W_i/f_i(b_i)$. De plus, on peut vérifier que, si toutes les entrées du problème sont entières, alors la solution est aussi entière. Donc CECSP est bien une généralisation de CuSP.\\

\subsection{Raisonnement énergétique}


Avant de donner la condition d'existence basée sur le raisonnement énergétique, nous allons définir une condition permettant de vérifier que la donnée du problème est consistante. Pour cela, remarquons d'abord que, comme $f_i$ est croissante, ordonnancer la tâche avec une puissance $b_i^{max}$ donne toujours le maximum d'énergie. La condition se présente alors comme suit:

\begin{Th}
  S'il existe $i \in A$ tel que $W_i>(d_i-r_i)f_i(b_i^{max})$ alors le problème CECSP ne peut pas avoir de solution.
\end{Th}

En effet, si ordonnancer la tâche à $b_i^{max}$ durant tout l'intervalle ${[}r_i,d_i{]}$ ne permet pas de satisfaire la contrainte d'énergie alors aucun ordonnancement ne le peut.\\

Nous allons maintenant présenter le raisonnement énergétique sur le problème CECSP. Pour cela, nous introduisons deux nouvelles fonctions $\wb$ et $\bb$, correspondant respectivement à la quantité minimale d'énergie et à la consommation minimale de la ressource durant l'intervalle ${[}t_1,t_2{]}$ et définies comme suit:

\begin{align*}
  \wb= \min \int_{t_1}^{t_2} f_i(b_i(t))dt & & \text{subject to (\ref{eq1})-(\ref{eq4})}\\
  \bb= \min \int_{t_1}^{t_2} b_i(t)dt & & \text{subject to (\ref{eq1})-(\ref{eq4})}
\end{align*}
	
Nous allons utiliser ces deux fonctions pour calculer une fonction mesurant l'écart entre la quantité de ressource disponible et les consommations minimales de toutes les tâches dans l'intervalle ${[}t_1,t_2{]}$. Cette fonction, notée $SL(t_1,t_2)$, est définie de la manière suivante:
\[ SL(t_1,t_2)=B(t_2-t_1)-\sum\limits_{i \in A} \bb \]

Nous pouvons maintenant énoncer la condition nécessaire d'existence d'une solution:

\begin{Th}
  S'il existe $t_1,\ t_2 \in {\cal T}$ tel que $SL(t_1,t_2) <0$ alors CECSP ne peut pas avoir de solution.
\end{Th}
	
En d'autre termes, s'il existe un intervalle pour lequel les consommations minimales de la ressource dépassent la capacité totale de la ressource sur l'intervalle, alors il n'y a pas de solution.\\
	
Nous avons donc établi une condition nécessaire d'existence de solution pour CECSP. Il nous reste à montrer que ce test peut se faire en temps polynomial. Pour ce faire, nous allons montrer qu'il suffit de faire ce test sur un nombre polynomial d'intervalle ${[}t_1,t_2{]}$. Nous allons d'abord analyser les différentes configurations de la consommation minimale d'une tâche.\\
Remarquons que les seules configurations possibles sont celles où la tâche est soit collée à gauche (s'exécute à son rendement maximal $b_i^{max}$ entre $r_i$ et $t_1$), soit collée à droite (s'exécute à son rendement maximal $b_i^{max}$ entre $t_2$ et $d_i$), soit les deux. Nous présentons ces différents cas dans la figure \ref{configuration}.

\begin{figure}[!h]
  \[
\begin{array}{ccc}
  \begin{tikzpicture}
  [scale=0.4]
    \node[] (O) at (0,0) {};
    \node[label={[shift={(-0.4,-0.4)}]$b_i^{min}$}] (bmin) at (0,1) {};
    \node[label={[shift={(-0.4,-0.4)}]$b_i^{max}$}] (bmax) at (0,4) {};
    \node[label={[shift={(0,-0.8)}]$t_1$}] (t1) at (0.5,0) {};
    \node[label={[shift={(0,-0.8)}]$t_2$}] (t2) at (1.5,0) {};
    \node[label={[shift={(0,-0.8)}]$d_i$}] (di) at (7,0) {};
    
    \draw[->] (O.center) -- (8,0);
    \draw (O.south) -- (bmax.north);
    \draw (bmin.center) -- (8,1);
    \draw (bmax.center) -- (8,4);
    \draw[fill=white] (3,0) rectangle (7,4);
    \draw[pattern=north west lines] (3,0) rectangle (7,4);
    \draw(di.south) -- (di.center);
    \draw[thick] (t1.south) -- (0.5,4.1);
    \draw[thick] (t2.south) -- (1.5,4.1);
  \end{tikzpicture}


&
  \begin{tikzpicture}
  [scale=0.4]
   \node[] (O) at (0,0) {};
    \node[label={[shift={(-0.4,-0.4)}]$b_i^{min}$}] (bmin) at (0,1) {};
    \node[label={[shift={(-0.4,-0.4)}]$b_i^{max}$}] (bmax) at (0,4) {};
    \node[label={[shift={(0,-0.8)}]$t_1$}] (t1) at (6,0) {}; 
    \node[label={[shift={(0,-0.8)}]$r_i$}] (ri) at (1,0) {};
    \node[label={[shift={(0,-0.8)}]$t_2$}] (t2) at (7,0) {};

    \draw[->] (O.center) -- (8,0);
    \draw (O.south) -- (bmax.north);
    \draw (bmin.center) -- (8,1);
    \draw (bmax.center) -- (8,4);
    \draw(ri.south) -- (ri.center);
    \draw[fill=white] (ri.center) rectangle (5,4);
    \draw[pattern=north west lines] (ri.center) rectangle (5,4);
    \draw[thick] (t1.south) -- (6,4.1);
    \draw[thick] (t2.south) -- (7,4.1);
  \end{tikzpicture}


&
\begin{tikzpicture}
 [scale=0.4]
 \node[] (O) at (0,0) {};
 \node[label={[shift={(-0.4,-0.4)}]$b_i^{min}$}] (bmin) at (0,1) {};
 \node[label={[shift={(-0.4,-0.4)}]$b_i^{max}$}] (bmax) at (0,4) {};
 \node[label={[shift={(0,-0.8)}]$d_i$}] (di) at (7,0) {};
 \node[label={[shift={(0,-0.8)}]$t_1$}] (t1) at (1,0) {}; 
 \node[label={[shift={(0,-0.8)}]$r_i$}] (ri) at (2.5,0) {};
 \node[label={[shift={(0,-0.8)}]$t_2$}] (t2) at (4,0) {};
 
 \draw[->] (O.center) -- (8,0);
 \draw (O.south) -- (bmax.north);
 \draw (bmin.center) -- (8,1);
 \draw (bmax.center) -- (8,4);
 \draw(di.south) -- (di.center);
 \draw(ri.south) -- (ri.center);
 \draw[fill=white] (t2.center) rectangle (7,4);
 \draw[pattern=north west lines] (t2.center) rectangle (7,4);
 \draw[fill=white] (t2.center) rectangle (3.2,2);
 \draw[pattern=north west lines] (t2.center) rectangle (3.2,2);
 \draw[thick] (t1.south) -- (1,4.1);
 \draw[thick] (t2.south) -- (4,4.1);

 \end{tikzpicture}

\\
\text{(1)} &\text{(2)} &\text{(3)}
\\

  \begin{tikzpicture}
  [scale=0.4]
    \node[] (O) at (0,0) {};
    \node[label={[shift={(-0.4,-0.4)}]$b_i^{min}$}] (bmin) at (0,1) {};
    \node[label={[shift={(-0.4,-0.4)}]$b_i^{max}$}] (bmax) at (0,4) {};
    \node[label={[shift={(0,-0.8)}]$t_1$}] (t1) at (4,0) {};
    \node[label={[shift={(0,-0.8)}]$t_2$}] (t2) at (7,0) {};
    \node[label={[shift={(0,-0.8)}]$d_i$}] (di) at (6,0) {};
    \node[label={[shift={(0,-0.8)}]$r_i$}] (ri) at (1,0) {};
    
    \draw[->] (O.center) -- (8,0);
    \draw (O.south) -- (bmax.north);
    \draw (bmin.center) -- (8,1);
    \draw (bmax.center) -- (8,4);
    \draw[fill=white] (ri.center) rectangle (4,4);
    \draw[pattern=north west lines] (ri.center) rectangle (4,4);
    \draw[fill=white] (t1.center) rectangle (5,1.5);
    \draw[pattern=north west lines] (t1.center) rectangle (5,1.5);
    \draw(di.south) -- (di.center);
    \draw(ri.south) -- (ri.center);
    \draw[thick] (t1.south) -- (4,4.1);
    \draw[thick] (t2.south) -- (7,4.1);
  \end{tikzpicture}

&
  \begin{tikzpicture}
  [scale=0.4]
   \node[] (O) at (0,0) {};
    \node[label={[shift={(-0.4,-0.4)}]$b_i^{min}$}] (bmin) at (0,1) {};
    \node[label={[shift={(-0.4,-0.4)}]$b_i^{max}$}] (bmax) at (0,4) {};
    \node[label={[shift={(0,-0.8)}]$t_1$}] (t1) at (1,0) {}; 
    \node[label={[shift={(0,-0.8)}]$r_i$}] (ri) at (2,0) {};
    \node[label={[shift={(0,-0.8)}]$t_2$}] (t2) at (7,0) {};
    \node[label={[shift={(0,-0.8)}]$d_i$}] (di) at (6,0) {};

    \draw[->] (O.center) -- (8,0);
    \draw (O.south) -- (bmax.north);
    \draw (bmin.center) -- (8,1);
    \draw (bmax.center) -- (8,4);
    \draw(ri.south) -- (ri.center);
    \draw(di.south) -- (di.center);
    \draw[fill=white] (2.5,0) rectangle (5.5,4);
    \draw[pattern=north west lines] (2.5,0) rectangle (5.5,4);
    \draw[thick] (t1.south) -- (1,4.1);
    \draw[thick] (t2.south) -- (7,4.1);
  \end{tikzpicture}


&
\begin{tikzpicture}
 [scale=0.4]
 \node[] (O) at (0,0) {};
 \node[label={[shift={(-0.4,-0.4)}]$b_i^{min}$}] (bmin) at (0,1) {};
 \node[label={[shift={(-0.4,-0.4)}]$b_i^{max}$}] (bmax) at (0,4) {};
 \node[label={[shift={(0,-0.8)}]$d_i$}] (di) at (7,0) {};
 \node[label={[shift={(0,-0.8)}]$t_1$}] (t1) at (2.5,0) {}; 
 \node[label={[shift={(0,-0.8)}]$r_i$}] (ri) at (1,0) {};
 \node[label={[shift={(0,-0.8)}]$t_2$}] (t2) at (4,0) {};
 
 \draw[->] (O.center) -- (8,0);
 \draw (O.south) -- (bmax.north);
 \draw (bmin.center) -- (8,1);
 \draw (bmax.center) -- (8,4);
 \draw(di.south) -- (di.center);
 \draw(ri.south) -- (ri.center);
 \draw[fill=white] (t2.center) rectangle (7,4);
 \draw[pattern=north west lines] (t2.center) rectangle (7,4);
 \draw[fill=white] (t2.center) rectangle (3.2,1);
 \draw[pattern=north west lines] (t2.center) rectangle (3.2,1);
 \draw[thick] (t1.south) -- (2.5,4.1);
 \draw[thick] (t2.south) -- (4,4.1);
 \end{tikzpicture}


\\
\text{(4)} &\text{(5)} &\text{(6-a)}
\\

  \begin{tikzpicture}
  [scale=0.4]
    \node[] (O) at (0,0) {};
    \node[label={[shift={(-0.4,-0.4)}]$b_i^{min}$}] (bmin) at (0,1) {};
    \node[label={[shift={(-0.4,-0.4)}]$b_i^{max}$}] (bmax) at (0,4) {};
    \node[label={[shift={(0,-0.8)}]$t_1$}] (t1) at (3,0) {};
    \node[label={[shift={(0,-0.8)}]$t_2$}] (t2) at (5,0) {};
    \node[label={[shift={(0,-0.8)}]$d_i$}] (di) at (7,0) {};
    \node[label={[shift={(0,-0.8)}]$r_i$}] (ri) at (1,0) {};
    
    \draw[->] (O.center) -- (8,0);
    \draw (O.south) -- (bmax.north);
    \draw (bmin.center) -- (8,1);
    \draw (bmax.center) -- (8,4);
    \draw[fill=white] (t1.center) rectangle (1,4);
    \draw[pattern=north west lines] (t1.center) rectangle (1,4);
    \draw[fill=white] (t1.center) rectangle (3.7,1);
    \draw[pattern=north west lines] (t1.center) rectangle (3.7,1);
    \draw(di.south) -- (di.center);
    \draw(ri.south) -- (ri.center);
    \draw[thick] (t1.south) -- (3,4.1);
    \draw[thick] (t2.south) -- (5,4.1);
  \end{tikzpicture}

&
  \begin{tikzpicture}
  [scale=0.4]
   \node[] (O) at (0,0) {};
    \node[label={[shift={(-0.4,-0.4)}]$b_i^{min}$}] (bmin) at (0,1) {};
    \node[label={[shift={(-0.4,-0.4)}]$b_i^{max}$}] (bmax) at (0,4) {};
    \node[label={[shift={(0,-0.8)}]$t_1$}] (t1) at (2.5,0) {}; 
    \node[label={[shift={(0,-0.8)}]$r_i$}] (ri) at (1.5,0) {};
    \node[label={[shift={(0,-0.8)}]$t_2$}] (t2) at (6,0) {};
    \node[label={[shift={(0,-0.8)}]$d_i$}] (di) at (7,0) {};

    \draw[->] (O.center) -- (8,0);
    \draw (O.south) -- (bmax.north);
    \draw (bmin.center) -- (8,1);
    \draw (bmax.center) -- (8,4);
    \draw(ri.south) -- (ri.center);
    \draw(di.south) -- (di.center);
    \draw[fill=white] (2.5,0) rectangle (2,2.7);
    \draw[pattern=north west lines] (2.5,0) rectangle (2,2.7);
    \draw[fill=white] (2.5,0) rectangle (6,1);
    \draw[pattern=north west lines] (2.5,0) rectangle (6,1);
    \draw[fill=white] (6,0) rectangle (7,4);
    \draw[pattern=north west lines] (6,0) rectangle (7,4);
    \draw[thick] (t1.south) -- (2.5,4.1);
    \draw[thick] (t2.south) -- (6,4.1);
  \end{tikzpicture}


&
\begin{tikzpicture}
 [scale=0.4]
 \node[] (O) at (0,0) {};
 \node[label={[shift={(-0.4,-0.4)}]$b_i^{min}$}] (bmin) at (0,1) {};
 \node[label={[shift={(-0.4,-0.4)}]$b_i^{max}$}] (bmax) at (0,4) {};
 \node[label={[shift={(0,-0.8)}]$t_1$}] (t1) at (2.5,0) {}; 
 \node[label={[shift={(0,-0.8)}]$r_i$}] (ri) at (1.5,0) {};
 \node[label={[shift={(0,-0.8)}]$t_2$}] (t2) at (6,0) {};
 \node[label={[shift={(0,-0.8)}]$d_i$}] (di) at (7,0) {};
 
  \draw[->] (O.center) -- (8,0);
  \draw (O.south) -- (bmax.north);
  \draw (bmin.center) -- (8,1);
  \draw (bmax.center) -- (8,4);
  \draw(ri.south) -- (ri.center);
  \draw(di.south) -- (di.center);
  \draw[fill=white] (2.5,0) rectangle (1.5,4);
  \draw[pattern=north west lines] (2.5,0) rectangle (1.5,4);
  \draw[fill=white] (2.5,0) rectangle (6,2);
  \draw[pattern=north west lines] (2.5,0) rectangle (6,2);
  \draw[fill=white] (6,0) rectangle (7,4);
  \draw[pattern=north west lines] (6,0) rectangle (7,4);
  \draw[thick] (t1.south) -- (2.5,4.1);
  \draw[thick] (t2.south) -- (6,4.1);
 \end{tikzpicture}
\\
\text{(6-a)} &\text{(6-a)} &\text{(6-b)}

\end{array}
\]

  \caption{les différentes configurations}
  \label{configuration}
\end{figure}
	

Nous allons maintenant détailler les différentes expressions de $\wb$ et $\bb$ en fonction des positions relatives de $t_1,\ t_2,\ r_i$ et $d_i$ et en fonction de l'expression de $f_i(b_i(t))$.

\subsection{Consommation minimale}

En premier lieu, nous allons commencer par donner l'expression de $\wb$ pour chacun des cas de la figure \ref{configuration}, puis nous détaillerons l'expression de $\bb$ pour deux cas : celui où la fonction $f_i(b_i(t))$ est linéaire et celui où la fonction est linéaire par morceau. 

\subsubsection{{\'E}nergie minimale dans un intervalle}
	
Il est facile de calculer l'expression de la consommation minimale d'énergie dans un intervalle pour une fonction $f_i$ croissante. En effet, les différentes configurations possibles étant toujours celle où la tâche est exécutée à son rendement maximum en dehors de l'intervalle ${[}t_1,t_2{]}$, il suffit de retrancher à $W_i$ l'énergie produite par l'exécution de la tâche en dehors de ${[}t_1,t_2{]}$. Il existe une exception à cette règle, produite par la contrainte du rendement minimal, mais ce cas est facilement traiter puisque l'énergie minimale correspond alors à la configuration où la tâche est exécutée à son rendement minimal durant ${[}t_1,t_2{]}$. Les différents cas sont détaillés ci-dessous:
	
\begin{enumerate}
\item Si $t_1 \le t_2 \le d-i - \frac{W_i}{f_i(b_i^{max})}$ alors $\wb=0$
\item Si $r_i+\frac{W_i}{f_i(b_i^{max})} \le t_1 \le t_2$ alors $\wb=0$
\item Si $t_1 \le r_i \le t_2 \le d_i$ et $t_2 \ge d_i - \frac{W_i}{f_i(b_i^{max})}$ alors $\wb=W_i-(d_i-t_2)f_i(b_i^{max})$
\item Si $r_i \le t_1 \le d_i \le t_2$ et $t_1 \le r_i+ \frac{W_i}{f_i(b_i^{max})}$ alors $\wb=W_i- (t_1-r_i)f_i(b_i^{max})$
\item Si $t_1 \le r_i \le d_i \le t_2$ alors $\wb=W_i$ 
\item Si $r_i \le t_1 \le r_i + \frac{W_i}{f_i(b_i^{max})}$ et $d_i-\frac{W_i}{b-i^{max}} \le t_2 \le d_i$ alors les configurations possibles sont les suivantes:
	
  \begin{description}
  \item[(a)] Si la tâche peut être exécutée à son rendement minimal dans l'intervalle ${[}t_1,t_2{]}$, i.e. $W_i -f_i(b_i^{max})(t_1-r_i+d_i-t_2) \le f_i(b_i^{min})(t_2-t_1)$ alors \\
		$\wb=\min {(f_i(b_i^{min})(t_2-t_1),W_i - f_i(b_i^{max})(d_i-t_2),W_i - f_i(b_i^{max})(t_1-r_i))}$ et $\bb$
		
  \item[(b)] Si $W_i -f_i(b_i^{max})(t_1-r_i+d_i-t_2) \ge f_i(b_i^{min})(t_2-t_1)$ alors $\wb=W_i- (t_1-r_i+d_i-t_2)f_i(b_i^{max})$ 
  \end{description}
\end{enumerate}
	
Nous allons maintenant utiliser l'expression de $\wb$ et la fonction $f_i$ pour calculer $\bb$.

\subsubsection{Consommation minimale de la ressource}
	
Pour calculer l'expression de $\bb$, nous allons utiliser les propriétés de la fonction $f_i$. En effet, une fois que nous avons calculer $\wb$, nous voulons savoir quelle est la puissance minimale que nous devons fournir à la tâche pour obtenir cette quantité d'énergie dans l'intervalle ${[}t_1,t_2{]}$.\\
Nous allons commencer par détailler le cas où cette fonction est linéaire, i.e. $f_i$ est de la forme $a_ib+c_i$ avec $a_i >0$ et $c_i \ge 0$, puis nous traiterons le cas où $f_i$ est linéaire par morceaux.
	
\paragraph{Fonction linéaire}
	
%TODO expliquer que bimin c'est bien
Pour les fonctions linéaires, commençons par remarquer que, pour une énergie donnée, exécuter la tâche à son rendement minimal $b_i{min}$ produit cette énergie en consommant une quantité minimale de la ressource ($\frac{f_i(b_i^{min})}{b_i^{min}}$ maximum). Nous avons alors deux cas à considérer:
	
\begin{itemize}
\item Soit l'intervalle $I={[}t_1,t_2{]}\cap {[}r_i,d_i{]}$ est suffisamment grand pour exécuter la tâche $i$ à son rendement minimal, i.e. $|I| \ge \frac{\wb}{f_i(b_i^{min})}$ et donc $\bb=b_i^{min}\frac{\wb}{f_i(b_i^{min})}$
	
\item Soit $|I| < \frac{\wb}{f_i(b_i^{min})}$ et nous devons résoudre le programme linéaire suivant:
  \begin{alignat*}{2}
    \text{minimiser}   & \int_{I} b_i(t)dt  \\
    \text{sous} & \int_{I} a_ib_i(t)+c_idt \ge \wb
  \end{alignat*}
  Alors la contrainte s'écrit $\int_{I} b_i(t)dt \ge \frac{1}{a_i}(\wb-|I|c_i)$ et donc, nous avons $\bb= \frac{1}{a_i}(\wb-|I|c_i)$.\\
\end{itemize}

Les différents cas de figures sont présentés ci-dessous:
\begin{enumerate}
\item Si $t_1 \le t_2 \le d_i - \frac{W_i}{f_i(b_i^{max})}$ alors $\bb=0$
\item Si $r_i+\frac{W_i}{f_i(b_i^{max})} \le t_1 \le t_2$ alors $\bb=0$
\item Si $t_1 \le r_i \le t_2 \le d_i$ et $t_2 \ge d_i - \frac{W_i}{f_i(b_i^{max})}$ alors $\bb=\max (\frac{1}{a_i}(\wb-c_i(t_2-r_i)),b_i^{min}\frac{\wb}{f_i(b_i^{min})})$
\item Si $r_i \le t_1 \le d_i \le t_2$ et $t_1 \le r_i+ \frac{W_i}{f_i(b_i^{max})}$ alors $\bb= \max (\frac{1}{a_i} (\wb-c_i(d_i-t_1)),b_i^{min}\frac{W_i}{f_i(b_i^{min})})$
\item Si $t_1 \le r_i \le d_i \le t_2$ alors $\bb= \max (\frac{1}{a_i} (\wb-c_i(d_i-r_i)),b_i^{min}\frac{W_i}{f_i(b_i^{min})})$
\item Si $r_i \le t_1 \le r_i + \frac{W_i}{f_i(b_i^{max})}$ et $d_i-\frac{W_i}{b-i^{max}} \le t_2 \le d_i$ alors les configurations possibles sont les suivantes:
	
  \begin{description}
  \item[(a)] Si l'activité peut être exécutée à son rendement minimal dans l'intervalle ${[}t_1,t_2{]}$, i.e. $W_i -f_i(b_i^{max})(t_1-r_i+d_i-t_2) \le f_i(b_i^{min})(t_2-t_1)$ alors $\bb$ est défini par:
    
    \begin{itemize}
    \item si $\wb=f_i(b_i^{min})(t_2-t_1)$, $\bb=b_i^{min}(t_2-t_1)$
    \item si $\wb=W_i - f_i(b_i^{max})(d_i-t_2)$, $\bb=\max (\frac{1}{a_i} (\wb-c_i(t_2-r_i)),b_i^{min}\frac{\wb}{f_i(b_i^{min})})$
    \item si $\wb=W_i - f_i(b_i^{max})(t_1-r_i)$, $\bb=\max (\frac{1}{a_i} (\wb-c_i(d_i-t_1)),b_i^{min}\frac{\wb}{f_i(b_i^{min})})$
    \end{itemize}
		
  \item[(b)] Si $W_i -f_i(b_i^{max})(t_1-r_i+d_i-t_2) \ge f_i(b_i^{min})(t_2-t_1)$ alors $\bb= \max (\frac{1}{a_i} (\wb-c_i(t_2-t_1)),b_i^{min}\frac{\wb}{f_i(b_i^{min})})$
  \end{description}
\end{enumerate}
	
	
\paragraph{Fonction linéaire par morceaux}
%TODO expliquer le point de rendement max + il suffit de regarder les points de cassure + programme lineaire à résoudre	
	
Pour les fonctions linéaires par morceaux, ce n'est plus l'exécution de la tâche à $b_i^{min}$ qui a le meilleur rendement. Cependant, nous savons que ce point de meilleur rendement se trouve au niveau d'un point de "cassure" de la fonction, i.e. un point où la définition de la fonction change. Pour le trouver, nous devons comparer tous ces points de cassure. Soit $\gamma_{min}$ ce point. Comme pour les fonctions linéaires, nous avons deux cas à considérer. Le premier est celui où l'intervalle $I={[}t_1,t_2{]}\cap {[}r_i,d_i{]}$ est suffisamment grand pour exécuter la tâche $i$ à cette puissance $\gamma_{min}$. Dans ce cas, la consommation minimale de la ressource est $\bb= \gamma_{min} \frac{\wb}{f_i(\gamma_{min})}$. Dans le cas où la taille de l'intervalle $I$ ne permet pas l'exécution de la tâche à ce rendement, nous devons résoudre le problème suivant: 
\begin{alignat*}{2}
  \text{minimiser}   & \int_{I} b_i(t)dt  \\
  \text{sous} & \int_{I} f_i(b_i(t))+dt \ge \wb
\end{alignat*}
		
Pour résoudre ce problème, nous allons d'abord montrer qu'il est suffisant de regarder les points de cassure de la fonction, que l'on va noter $\{\gamma_i | 0 \le i\le q\}$. En d'autres termes, si dans une solution optimale la tâche est exécutée à une puissance $\beta$, comprise entre $\gamma_j$ et $\gamma_k$, pendant un intervalle ${[}a,b{]}$ alors, on peut trouver une partition de l'intervalle ${[}a,b{]}$ en deux sous-intervalles de telle sorte qu'exécuter la tâche $i$ à $\gamma_j$ pendant la première partie de l'intervalle et à $\gamma_k$ pendant la seconde partie soit équivalent à son exécution à $\beta$ sur ${[}a,b{]}$, i.e. produise la même énergie en consommant la même quantité de la ressource.\\
Nous allons donc prouver le lemme suivant:

\begin{Lem}
  Soit une instance de CECSP, et soient une tâche $i$ appartenant à cette instance telle que $f_i$ est une fonction linéaire par morceaux. Alors résoudre
  \begin{alignat*}{2}
    \text{minimiser}   & \int_{I} b_i(t)dt  \\
    \text{sous} & \int_{I} f_i(b_i(t))+dt \ge \wb
  \end{alignat*}
  est équivalent à résoudre:
  \begin{alignat*}{2}
    \text{minimiser}   & \sum\limits_{l=0}^q x_l\gamma_l \\
    \text{sous} & \sum\limits_{l=0}^q x_lf_i(\gamma_l) \ge \wb \\
    & \sum\limits_{l=0}^q x_l \le |I|
  \end{alignat*}
\end{Lem}

\begin{proof}
  Pour montrer ce lemme, nous allons montrer que si une solution optimale utilise un point $\beta$ tel que $ \gamma_j < \beta < \gamma_k$ (et $j$ et $k$ sont respectivement le plus petit et le plus grand indice vérifiant cette inégalité), alors il existe une solution optimale utilisant seulement $\gamma_j$ et $\gamma_k$ et pas $\beta$.\\
Supposons donc que dans une solution optimale, la tâche $i$ soit exécutée avec une puissance $\beta$ comprise entre $\gamma_j$ et $\gamma_k$ pendant un intervalle ${[}a,b{]}$. L'énergie produite par cette affectation est donc $(b-a)f_i(\beta)$ et la quantité de ressource utilisée est $(b-a)\beta$.\\
Nous allons exprimer le point $(\beta,f_i(\beta))$ comme une combinaison convexe des points $(\gamma_j,f_i(\gamma_j))$ et de $(\gamma_k,f_i(\gamma_k))$:
\[ (\beta,f_i(\beta))= \lambda_j (\gamma_j,f_i(\gamma_j)) + \lambda_k (\gamma_k,f_i(\gamma_k))
\]
avec $\lambda_j + \lambda_k=1$.\\
Donc l'énergie produite est $\lambda_j(b-a)f_i(\gamma_j)+\lambda_k(b-a)f_i(\gamma_k)=(b-a)f_i(\beta)$ et la quantité de ressource utilisée est $\lambda_j(b-a)\gamma_j+\lambda_k(b-a)\gamma_k=(b-a)\beta$

\end{proof}
	
\end{document}
