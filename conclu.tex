\chapter*{Conclusions et Perspectives\markboth{CONCLUSIONS ET PERSPECTIVES}{}}

Beaucoup de problèmes d'ordonnancement cumulatifs définis dans la
littérature souffrent d'une limitation imposant que chaque activité
aille une durée et une consommation de ressource fixe durant son 
exécution. Il arrive cependant, dans de nombreux cas pratique, que
cette limitation empêche la modélisation correcte du problème. De ce
fait, de nouveaux problèmes permettant de modéliser la malléabilité
des activités, i.e. ayant une durée et une consommation de ressource
variable, ont été introduits~\cite{DDH,NK,FT,Kis,BLN}. Cette thèse
introduit un nouveau problème appartenant à cette classe, le \CECSP. 
Une comparaison des différents problèmes existant permettant de
modéliser des activités malléables a été réalisée et a permis de
montrer que le \CECSP~est très différents de ces derniers. 

La principale difficulté du \CECSP~repose sur la combinaison entre
ressource continue et malléabilité des activités. En effet, ces deux
caractéristiques impliquent que les activités peuvent prendre des
formes quasi quelconques. De ce fait, un des premiers travail effectué
dans cette thèse a été une étude détaillée du problème. Cette étude a
permis de décrire des cas particuliers du \CECSP~pouvant être résolus
en temps polynomial mais aussi a permis, dans certains cas, la mise en
place d'une propriété permettant de décomplexifier le problème en
caractérisant les différentes formes que peut prendre une activité. 
Cette simplification du problème a ensuite été utilisée pour mettre en
place des méthodes de résolution pour le \CECSP, souvent adaptées de
méthodes existantes dans le cadre des problèmes d'ordonnancement
cumulatifs. 

Les méthodes de résolution dédiées au \CECSP~et décrites dans ce
manuscrit sont regroupées en deux catégories: les méthodes issues de
la programmation par contraintes et adaptées des méthodes définies
pour la contrainte cumulative et les méthodes issues de la
programmation linéaire mixte et en nombres entiers adaptées des
méthodes définies pour le \RCPSP. 

Plusieurs des techniques de programmation par contraintes utilisées dans le
cadre de la résolution de la contrainte cumulative ont décrites et, en
particulier, les principaux algorithmes de filtrage qui lui sont
dédiés. Une partie de ces algorithmes a été adaptés au cas du
\CECSP. C'est le cas, par exemple, du raisonnement énergétique qui
prend une part importante de ce manuscrit. Ce raisonnement comptant
parmi les plus forts dans le cas de la contrainte cumulative a été le
premier à avoir été adapté. De plus, les récents travaux de Derrien
{\it et al.} permettant l'accélération de ce raisonnement ont aussi pu
être adaptés. D'autres raisonnements comme le Time-Table, le
raisonnement disjonctif et le raisonnement Time-Table disjonctif ont
aussi pu être transformés afin de s'appliquer dans le cas du
\CECSP. Enfin, un nouvel algorithme de détection d'incohérence
utilisant un programme linéaire basé sur un problème de flot couplé
avec le raisonnement Time-Table a été présenté. Les expérimentations
conduites sur ces raisonnements -- inclus à l'intérieur d'une méthode
de branchement hybride -- ont permis de montrer que ce nouvel
algorithme de vérification permet la détection d'un plus grand nombre
d'incohérences que le raisonnement énergétique à lui seul.

Différents modèles de programmation linéaire en nombres entiers pour le
\RCPSP~ont été décrits dans le chapitre~\ref{sec:PLNE_RCPSP}. Parmi
ces modèles, le premier repose sur une formulation indexée par le temps
et les deux autres sur des formulations basées sur les événements. Ces
dernières ont montré leur efficacité dans le cas où l'horizon de temps
des modèles devient très grand. De plus, contrairement aux modèles
indexés par le temps, ils permettent de modéliser des dates de début
et de fin d'activités continue. De ce fait, nous avons adapté ces
modèles au cas du \CECSP. Un modèle indexé par le temps est aussi
détaillé. En effet, ces modèles ont prouvé leur efficacité dans le
cadre du \RCPSP.

Pour chacun des modèles présentés, des inégalités valides et/ou des
techniques de coupes ont été présenté afin de renforcer ces
derniers. Des inégalités directement déduites du raisonnement
énergétique sont introduites pour les modèles indexés par le temps du
\CECSP~et du \RCPSP. Pour les modèles à événements, une comparaison
des relaxations linéaires des deux modèles présentés est
effectuée. Des inégalités permettant de renforcer le modèle ayant les
moins bonnes relaxations, le modèle On/Off, sont présentées. Ces
inégalités, appelées inégalités de non-préemption, ont été montré
comme appartenant à l'enveloppe convexe du polyèdre formé par
l’ensemble de toutes les affectations possibles pour les variables
binaires correspondant à une seule activité. Enfin, plusieurs autres
ensembles d'inégalités pour les modèles à événements ont été présentés
et les performances relatives à l'ajout de différentes combinaisons de
ces inégalités au modèle ont été détaillées. 

Malgré tout, le \CECSP~reste un problème difficile, en particulier
dans sa forme générale. En effet, es résultats présentés dans cette
thèse nous ont permis de résoudre des instances allant jusqu'à $60$
activités pour la version décisionnelle de ce problème et seulement
jusqu'à $30$ pour la version ayant pour objectif la minimisation de la
consommation de la ressource. 

Le \CECSP~est un nouveau problème pour lequel de nombreux travaux
restent à faire. Parmi les perspectives directes des résultats
présentés dans ce manuscrit, on trouve:
\paragraph{La considération de fonction de rendement plus générale.} Même si les fonctions concaves permettent une plus grande
  liberté d'expression du problème, elles restent insuffisante pour
  modéliser certains problèmes réels. Dans un premier temps, la
  considération de fonctions de rendement convexes est une
  continuation naturelle de ce travail. Cependant, ce n'est toujours
  pas suffisant dans certains cas. En effet, beaucoup de fonctions 
  de rendement ne sont ni totalement concave, ni totalement convexe
  mais sont convexes sur certains intervalles et concaves sur les
  autres intervalles. 

\paragraph{La mise en place d'algorithme de filtrage plus
  performants.} Qu'il s'agisse d'algorithme plus performant en termes
de temps de calcul ou plus performant en termes de filtrage, cette
direction de recherche est une perspective importante. L'adaptation
d'autres algorithmes de filtrage mis en place pour la contrainte
cumulative semble une piste intéressante mais l'accélération du temps
de calcul de ces derniers est fait au prix d'un filtrage de moins
bonne qualité. Ceci peut s'avérer critique dans le cas du \CECSP~où
les algorithmes de filtrage pour la contrainte cumulative sont
beaucoup plus faibles. A l'inverse, la mise en place d'algorithmes
dédiés au problème ayant un pouvoir de filtrage plus grand sera
probablement effectuée au prix d'une augmentation importante du temps 
de calcul. 

\paragraph{Amélioration des modèles de programmation linéaire mixte.}
Dans un premier temps, les modèles décrits dans ce manuscrit
pourraient encore être renforcés. En effet, les inégalités
énergétiques décrites pour le modèle indexé par le temps sont un moyen
intéressant de coupler des techniques issues de la programmation par
contraintes à la programmation linéaire mixte. D'autres inégalités de
ce type pourraient être définies ou de meilleures techniques
d'intégration de ces inégalités dans le processus de résolution
pourraient être mises en place. 

L'amélioration des modèles à événement est aussi une piste de
recherche importante puisque, contrairement aux modèles indexés par le
temps, ces derniers permettent d'obtenir des solutions exactes pour le
\CECSP~mais au prix d'un temps de calcul beaucoup plus important. Pour
renforcer ces modèles, d'autres jeux d'inégalités, plus fortes,
peuvent être mises en place. Par exemple, des inégalités décrivant des
facettes du polyèdre formé par l'ensemble des solutions de chaque
modèle peuvent être exhibées.

\paragraph{L'utilisation d'autres méthodes de résolution.}
Les techniques décrites dans ce manuscrit utilisent principalement des
techniques de programmation linéaire mixte et de programmation par
contraintes. Cependant, d'autres paradigmes existent et des techniques
issues de ces derniers pourraient être considérées. De même, des
techniques appliquées sur des problèmes connexes pourraient aussi être
adaptées, quitte à discrétiser le problème. Parmi ces techniques, on
pourra trouver les techniques de génération de colonnes, de
branch-and-cut, la programmation dynamique ou des techniques utilisant
des règles de priorités, etc. 








