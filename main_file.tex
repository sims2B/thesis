\documentclass[11pt,a4paper,svgnames]{book}

\usepackage[hmargin=2.5cm,vmargin=3.5cm]{geometry}
\usepackage[ED=EDSYS - Info, Ets=UT3]{tlsflyleaf}
\usepackage{algpseudocode}
\usepackage[french,ruled,vlined]{algorithm2e}
\usepackage[utf8]{inputenc}
\usepackage[T1]{fontenc}
\usepackage[french]{babel}
\usepackage[utf8]{inputenc}
\usepackage{xcolor}
\usepackage{amsmath}
\usepackage{amssymb}
\usepackage{amsthm}
\usepackage{tikz}
\usetikzlibrary{patterns,shapes,positioning,shapes.misc,decorations.pathreplacing}
\usepackage{array}
\usepackage{xargs}
\usepackage{multirow}
\usepackage{setspace}
\usepackage{tabularx}
\usepackage{caption}
\usepackage{float}
\usepackage[small,compact]{titlesec} 
\usepackage{makeidx}
\usepackage{minitoc} 
\setcounter{minitocdepth}{1}
\usepackage{subcaption} 
\captionsetup[subfigure]{font+=footnotesize}
\allowdisplaybreaks[4]
 \renewcommand{\baselinestretch}{1.2} 
\usepackage{fancyhdr}
\fancyhead{}
\fancyhead[LO]{\rightmark}
\fancyhead[RE]{\leftmark}
\pagestyle{fancy}
\fancypagestyle{plain}{
\fancyhead{}
\fancyhead[LO]{\rightmark}
\fancyhead[RE]{\leftmark}
}

\mtcsettitle{parttoc}{}
\newcommand{\clearemptydoublepage}{%
   \newpage{\pagestyle{empty}\cleardoublepage}}
\newcommand\RCPSP{RCPSP}
\newcommand\CECSP{CECSP}
\newcommand\CUSP{CuSP}
\newcommand\CuSP{probl{\`e}me cumulatif}
\renewcommand\H{{\cal H}}
\newcommand\A{{\cal A}}
\renewcommand\P{{\cal P}}
\renewcommand\S{{\cal S}}
\newcommand\R{{\cal R}}
\newcommand\I{{\cal I}}
\renewcommand\O{{\cal O}}
\newcommand\Q{{\cal Q}}
\newcommand\X{{\cal X}}
\newcommand\C{{\cal C}}
\newcommand\D{{\cal D}}
\newcommandx{\bmin}[1][1=i]{r_{#1}^{min}}
\newcommandx{\bmax}[1][1=i]{r_{#1}^{max}}
\newcommandx{\emin}[1][1=i]{e_{#1}^{min}}
\newcommandx{\smax}[1][1=i]{s_{#1}^{max}}

\newcommandx{\ES}[1][1=i]{est_{#1}}
\newcommandx{\LS}[1][1=i]{lst_{#1}}
\newcommandx{\EE}[1][1=i]{eet_{#1}}
\newcommandx{\LE}[1][1=i]{let_{#1}}

\newcommandx{\inter}[2][1=t_1,2=t_2]{{[}#1,#2{[}}
\newcommandx{\htd}{G_{t_2}} 
\newcommandx{\bdp}[1]{\Delta'(#1)}
\newcommandx{\bup}[1]{\Gamma'(#1)} 
\newcommandx{\bd}[1]{\Delta(#1)}
\newcommandx{\bu}[1]{\Gamma(#1)} 
\newcommandx{\htu}{G_{t_1}}
\newcommandx{\itd}{D_{t_2}} 
\newcommandx{\itu}{D'_{t_1}} 

\newcommandx{\bb}[3][1=i,2=t_1,3=t_2]{\underline{b}(#1,#2,#3)}
\newcommandx{\wb}[3][1=i,2=t_1,3=t_2]{\underline{w}(#1,#2,#3)}
\newcommandx{\conv}[1][1=W_i]{CR\left(#1,t_1,t_2\right)}
\newcommandx{\wbLS}[3][1=i,2=t_1,3=t_2]{\underline{w}_{LS}(#1,#2,#3)}
\newcommandx{\wbRS}[3][1=i,2=t_1,3=t_2]{\underline{w}_{RS}(#1,#2,#3)}
\newcommandx{\wbCS}[3][1=i,2=t_1,3=t_2]{\underline{w}_{CS}(#1,#2,#3)}
\newcommandx{\bbLS}[3][1=i,2=t_1,3=t_2]{\underline{b}_{LS}(#1,#2,#3)}
\newcommandx{\bbRS}[3][1=i,2=t_1,3=t_2]{\underline{b}_{RS}(#1,#2,#3)}
\newcommandx{\bbCS}[3][1=i,2=t_1,3=t_2]{\underline{b}_{CS}(#1,#2,#3)}

\newcommandx{\Em}[1][1=2n]{\E\setminus\{#1\}}
\newcommand\E{{\cal E}}
%%%%%%%%%% INDEX %%%%%%%%%%%%%%%

\newcommand\CuSPidx{CuSP}
\newcommand\RCPSPidx{RCPSP}
%%%%%%%%%%%%% THEOREME %%%%%%%%%%%%%%%%

\newtheorem{theo}{Théorème}[chapter]
\newtheorem{defi}{Définition}[chapter]
\newtheorem{coro}{Corollaire}[theo]
\newtheorem{ex}{Exemple}[section]
\newtheorem{lemma}{Lemme}[chapter]
\newtheorem{prop}{Proposition}[chapter]
\newtheorem{reg}{Règle}[chapter]
%%%%%%%%%% TIKZ %%%%%%%%%%%%%

\tikzset{every node/.style={circle,minimum size=0pt,node
distance=3cm,inner sep=2pt}}

%%%%%%%%%%%%%%%% ARRAY %%%%%%%%%%%

\newcolumntype{M}[1]{>{\centering\arraybackslash}m{#1}}
\newcolumntype{P}[1]{>{$}M{#1}<{$}}
\newcolumntype{Y}{>{\centering\arraybackslash}X}

%%%%%%%%%%%%%%%%%TIKZ STYLE%%%%%%%%%%%
    \tikzstyle{tree}=[circle,draw=gray!70!,minimum width=0.7cm]
    \tikzstyle{leaf}=[circle,draw,minimum width=0.7cm]
    \tikzstyle{sol}=[rectangle,draw,very thick,minimum width=0.6cm,minimum height=0.6cm]
 \parskip=7pt
 \setcounter{topnumber}{1}
 \setcounter{bottomnumber}{1}

\hyphenation{ordon-nan-ce-ment}
\hyphenation{con-train-tes}
\title{\textbf{\large Ordonnancement sous contraintes d'énergie}}
\author{Margaux NATTAF}
\defencedate{18/10/2016}
\lab{Laboratoire d'Analyse et d'Architecture des Systèmes (LAAS)}


\nboss{2}
\makesomeone{boss}{2}{Christian ARTIGUES}{}{}  % Sera affiche en second
\makesomeone{boss}{1}{Pierre LOPEZ}{}{} % Sera afiche en premier

%% Referee
\nreferee{2}
\makesomeone{referee}{1}{Philippe BAPTISTE}{}{}
\makesomeone{referee}{2}{Claude-Guy QUIMPER}{}{}

%% Judges
\njudge{8}
\makesomeone{judge}{1}{Christian ARTIGUES}{Directeur de recherche au
  CNRS, LAAS-CNRS\\
\vspace{0.3cm}}{Directeur de thèse}
\makesomeone{judge}{2}{Philippe BAPTISTE}{Directeur de recherche au CNRS\\

\vspace{0.3cm}}{Rapporteur}
\makesomeone{judge}{3}{Cyril BRIAND}{Professeur,
  Université Toulouse 3-Paul Sabatier\\
\vspace{0.3cm}}{Examinateur}
\makesomeone{judge}{4}{Tam{\'a}s KIS}{Research fellow, MTA SZTAKI, Budapest, Hongrie\\
\vspace{0.3cm}}{Examinateur}
\makesomeone{judge}{5}{Philippe LABORIE}{Principal Scientist, IBM, Paris\\
\vspace{0.3cm}}{Examinateur}
\makesomeone{judge}{6}{Pierre LOPEZ}{Directeur de recherche au CNRS, LAAS-CNRS\\
\vspace{0.3cm}}{Directeur
  de thèse} 
\makesomeone{judge}{7}{Alain QUILLIOT}{Professeur, Université Blaise Pascal, Clermont-Ferrand\\
\vspace{0.3cm}}{Examinateur}
\makesomeone{judge}{8}{Claude-Guy QUIMPER}{Professeur agrégé,
  Université Laval, Québec, Canada}{Rapporteur}

% ============================================================
% DOCUMENT
\begin{document}
\makeflyleaf
\doparttoc
\chapter*{Remerciements}

Je tiens d'abord à remercier mes deux directeurs de thèse, Pierre
Lopez et Christian Artigues, pour m'avoir donné la chance de faire
cette thèse avec eux. Son bon déroulement est grandement dû à leurs
qualités scientifiques et humaines. Merci de m'avoir si bien encadrée
et supportée pendant ces trois dernières années.

Je voudrais aussi remercier Alain Quillot pour l'honneur qu'il m'a
fait en présidant le jury de ma soutenance. 

Je remercie aussi Claude Guy Quimper et Philippe Baptiste d'avoir
accepté d'être rapporteur de cette thèse et pour leur conseil
d'experts sur mes travaux. 

Je tiens aussi à remercier Philippe Laborie, Tam{\'a}s Kis et Cyril
Briand pour avoir accepter de participer à mon jury en temps
qu'examinateur mais aussi pour l'intérêt qu'ils ont montré dans ce
travail.

J'adresse également mes sincères remerciements à tous les membres de
l'équipe ROC et à son personnel administratif grâce à qui ces trois
années de thèse ont pu se dérouler dans de si bonnes conditions.

Merci à tous les gens qui m'ont permis de décompresser dans les
moments difficiles (et les moments faciles aussi). Grâce à eux, ces
trois ans de vie toulousaine ont été un plaisir (tellement que je
reste encore un peu). 

Merci aussi aux doctorants du Laboratoire l'Informatique et de
Robotique et de Micro-électronique de Montpellier qui m'ont aidé sur
les problèmes que j'ai rencontré lors de cette thèse et qui m'ont
toujours gardé une petite place pour mes visite. 

Enfin, merci à tous ceux qui, de près ou de loin, ont contribué à
l'aboutissement de cette thèse. {\'A} tous ceux qui m'ont supporté et
à tous ceux qui n'ont jamais douté que j'y arriverai. %
\clearemptydoublepage%

\tableofcontents
%
\clearemptydoublepage%

\thispagestyle{empty}

\listoffigures
%
\clearemptydoublepage%
\listoftables
%
\clearemptydoublepage%
%\listofalgorithms
\chapter*{Introduction\markboth{INTRODUCTION}{}}

De nos jours, de nombreuses tâches, auparavant fastidieuses, ont vu
leur complexité largement diminuée grâce à la mise en place d'outils
informatiques permettant leur traitement. Ces outils sont constamment
améliorés et l'augmentation des performances des ordinateurs les
rendent de plus en plus efficaces. Cependant, nombre de ces outils
reposent sur des problèmes complexes dont la résolution demande un
grand nombre d'opérations. Le plus souvent, ce nombre d'opérations est
exponentiel en fonction de la taille du problème. Dans ce cas, la
résolution de tels problèmes peut prendre plusieurs centaines
d'années. 

La mise en place de techniques dédiées permettant de prendre en
considération les propriétés intrinsèques du problème est donc un
axe de recherche majeur dans le domaine de l'informatique. Parmi les
classes de problèmes les plus étudiés, on retrouve les problèmes
d'ordonnancement, les problèmes de tournée de véhicules ou les
problèmes d'affectation. 

Dans cette thèse, nous nous sommes intéressé aux problèmes
d'ordonnancement et, plus particulièrement, aux problèmes
d'ordonnancement avec contraintes de ressource. Parmi les problèmes
les plus étudiés dans la littérature, nous retrouvons le problème
d'ordonnancement de projet avec contraintes de ressource et le problème
cumulatif. 

Dans le problème d'ordonnancement de projet avec contraintes de
ressource, nous devons ordonnancer un ensemble d'activités, chacune
d'entre elles consommant une partie d'une ou plusieurs ressources (de
capacité limitée) et étant liées par des relations de précédence. Le
plus souvent, ces activités doivent être ordonnancées de manière à
minimiser la date de fin du projet mais de nombreux autres objectifs,
tels que la minimisation du coût ou des retards peuvent être trouvés 
dans la littérature.

Dans le problème cumulatif, les activités consomment une quantité
d'une et une seule ressource. Il s'agit de la même ressource pour
toutes les activités. Dans ce problème, il n'y a pas de relation de
précédence entre les activités mais chaque activité dispose d'une
fenêtre de temps dans laquelle elle doit s'exécuter. 

La plus grande limites de ces problème est que les activités sont
supposées de durée fixes et consommant une quantité de ressource
constante au cours du temps. Cependant, il existe de nombreuses
applications pratiques dans lesquelles ces contraintes ne sont pas
respectées~\cite{HaitArtiguesLopez,Blaz,W80}. En effet, la
possibilité, pour une activité, de consommer plus de ressource
(respectivement moins) afin de finir plus rapidement (resp. lentement)
n'est pas modélisée dans ces problèmes. Les activités satisfaisant
cette propriété sont dites {\it malléables} dans le sens où leur
forme, définie par leur durée et consommation de ressource pendant
leur exécution, doit être décidée pendant le processus de résolution. 

Cette thèse étudie une nouvelle modélisation des activités malléables
représentée par un problème appelé le problème d'ordonnancement
continu avec contraintes énergétiques. Dans ce problème, un ensemble
d'activités,  utilisant une ressource continue et cumulative de
capacité limitée, doit être ordonnancé. La quantité de ressource
nécessaire à l'exécution d'une tâche n'est pas fixée mais doit
être déterminé. Une fois la tâche commencée et jusqu'à sa date de fin,
la quantité de ressource consommée par l'activité doit être comprise
entre une valeur maximale et une valeur minimale. De plus, la
consommation, à un instant donnée,  d'une partie de la ressource
permet à l'activité de recevoir une certaine quantité d'énergie,
calculée par le biais d'une fonction de rendement. Cette énergie nous
permet de savoir quand l'activité est terminée, i.e. quand elle a
reçue une quantité suffisante d'énergie. 

Pour le problème cumulatif et le problème d'ordonnancement de projet
avec contraintes de ressource, différentes techniques permettant de
trouver des solutions ont été mises en place. Ces techniques utilisent
des concepts et théorie pouvant être très variés. Cependant, deux de
ceux figurant parmi les plus utilisées demeurent les techniques issues
de la programmation par contraintes et de la programmation linéaire
mixte (ou en nombres entiers). En effet, ces techniques se sont
révélées très efficace dans le processus de résolution de ces deux
problèmes. Ce sont quelques unes de ces méthodes que nous nous
proposons d'appliquer dans cette thèse.

Le plan de la thèse est le suivant: 
\begin{itemize}
\item le chapitre~\ref{sec:chapter1} commence par détailler les
principales caractéristiques des problèmes d'ordonnancement
(paragraphe~\ref{sec:ordo}). Ensuite, une définition formelle des
problèmes d'ordonnancement cumulatifs et d'ordonnancement de projet
avec contrainte de ressource ainsi qu'une description des principales
limitations de ces problèmes sont données
(paragraphe~\ref{sec:ordo_res}).  Enfin, le
paragraphe~\ref{sec:ordo_nrj} décrit le problème étudié dans ce
manuscrit: le problème d'ordonnancement continu avec contraintes
énergétiques. De plus, ce paragraphe d'écrit les modélisations
pré-existantes des activités malléables tout en expliquant en quoi ces
modélisations ne suffisaient pas à la modélisation de tous les problèmes
réels. La dernière chose abordée dans ce chapitre est la présentation
d'un certains nombres de propriétés satisfaites par ce problème.
\item les chapitres~\ref{sec:PPC_CUSP} et~\ref{sec:PPC_CECSP} sont
  dédiés aux méthodes issues de la programmation par
  contraintes. Après une brève introduction à la programmation par
  contraintes (paragraphe~\ref{sec:PPC}) et à l'ordonnancement en
  programmation par contraintes (paragraphe~\ref{sec:cumu_ordo}), nous
  présentons les principaux algorithmes de filtrages mis en place pour
  le problème cumulatif (paragraphe~\ref{sec:cumu_propag}). Une partie
  de ces algorithmes est ensuite adapté au cas du problème
  d'ordonnancement continu avec contraintes énergétiques dans le
  chapitre~\ref{sec:PPC_CECSP}. 
\item les chapitres~\ref{sec:PLNE_RCPSP} et~\ref{sec:PLNE_CECSP}
  présentent les techniques de résolution issues de la programmation
  linéaire mixte. Dans un premier temps, nous décrivons les concepts généraux
  de la programmation linéaire mixte (paragraphe~\ref{sec:PLNE}). Nous
  présentons ensuite un certain nombre de modèles mis en place pour
  résoudre le problème d'ordonnancement de projet avec contraintes de
  ressource (paragraphe~\ref{sec:PLNE_ordo_res}). Ces modèles sont
  ensuite adaptés dans le cadre du problème d'ordonnancement continu
avec contraintes énergétiques (paragraphe~\ref{sec:modele_CECSP}). De
plus, des résultats venant renforcer ces modèles sont présentés dans
le paragraphe~\ref{sec:amelioration_modele}.
\item le chapitre~\ref{sec:expe} présente les résultats expérimentaux
  conduits pour valider les notions théoriques décrites dans les
  chapitres précédents. 
\item les annexes~\ref{ann:JFPC} et~\ref{ann:IESM} présentent des
  travaux réalisés pendant la thèse mais pas assez aboutis ou trop
  éloignés du sujet de ce manuscrit pour y figurer à part entière. 
\end{itemize}











%
\clearemptydoublepage%
\cleardoublepage
\begin{minipage}{0.95\linewidth}
  \part{Présentation du problème}
  \vspace{15mm} % l'espacement souhaité
  \parttoc 

\end{minipage}

\newpage
\thispagestyle{empty}
\vspace*{\stretch{1}}
\begin{center}
  \begin{minipage}{\textwidth}
    \hrule
    \vspace{0.5cm}
    {\it  La première partie de cette thèse est consacrée à la présentation de
      la problématique et de son contexte général. Nous commençons donc,
    dans un premier temps, par définir les principales caractéristiques
    d'un problème d'ordonnancement et plus précisément des problèmes
    d'ordonnancement à contraintes de ressource. Pour ces derniers,
    deux exemples de tels problèmes sont présentés: le problème
    d'ordonnancement de projet à contraintes de ressource et une de ses relaxations, le
    problème cumulatif. En effet, plusieurs techniques de résolution
    adaptées de celles définies pour ces problèmes seront présentées
    dans cette thèse. 

    La suite du manuscrit discute des limitations de ces problèmes
    et présente une nouvelle modélisation permettant de pallier ces
    dernières: le problème d'ordonnancement continu à contraintes
    énergétiques~\cite{ArtiguesLopez,Nattaf_ORSpectrum,Nattaf_Constraints}. Ce
    problème est ensuite comparé aux modélisations préexistantes afin de
    montrer la pertinence de celui-ci et de souligner que l'application
    directe des techniques de résolution de ces modélisations à ce
    problème n'est pas possible.

    Enfin, la fin de cette partie est dédiée à la présentation des
    principales propriétés du problème d'ordonnancement continu à
    contraintes énergétiques. Après avoir prouvé la NP-complétude de ce
    problème dans le cas général, nous montrons que, même dans le cas où
    tous les paramètres du problème sont entiers, l'ensemble des solutions
    peut ne comporter que des valeurs fractionnaires. Nous prouvons
    ensuite que, dans certains cas particuliers, l'ensemble des instants
    pour lesquels la ressource doit être réallouée peut être
    restreint aux dates de début et de fin des activités. Ceci nous
    permettra, dans un premier temps, de décrire des cas particuliers
    de ce problème solvable en temps polynomial. Cette propriété sera
    ensuite utilisée pour définir des méthodes de résolution dans les
    parties suivantes de cette thèse. 

    Les propriétés définies dans le paragraphe~\ref{sec:ppte_CECSP}
    ont été publiées
    dans~\cite{Nattaf_ORSpectrum,Nattaf_Constraints,Nattaf_CPDP}.} 
    \vspace{0.5cm}
    \hrule
  \end{minipage}
\end{center}
\vspace*{\stretch{1}}


\chapter{Ordonnancement, ressources cumulatives et énergie}
\label{sec:chapter1}
Dans ce chapitre, nous commençons par définir ce qu'est un problème
d'ordonnancement et quelles sont les principales caractéristiques d'un
tel problème. Ensuite, nous présenterons deux exemples de problèmes
d'ordonnancement cumulatif que nous utiliserons tout au long de ce
manuscrit, le problème d'ordonnancement de projet à contraintes de
ressources (RCPSP) et le problème cumulatif (CuSP). Nous montrerons
ensuite les limites de ces problèmes en termes de modélisation des
modalités d'utilisation de certaines ressources et nous introduirons une nouvelle 
extension du CuSP, le problème d'ordonnancement continu à contraintes
énergétiques (CECSP), pour pallier ces limites. Cette nouvelle
extension sera ensuite comparée aux extensions existantes du \RCPSP~et
du \CUSP.  Enfin, plusieurs propriétés du \CECSP, qui seront utilisées
dans la suite de ce manuscrit, seront présentées.


\section{Ordonnancement et contraintes de ressources}
\label{sec:ordo}
\subsection{L'ordonnancement}

La théorie de l'ordonnancement s'intéresse au calcul de dates
d'exécution d'un ensemble d'activités. Dans cette optique,
l'utilisation d'une ou plusieurs ressources peut être nécessaire et
l'exécution d'une activité implique souvent une telle consommation. Un
problème d'ordonnancement peut alors être vu comme l'organisation dans
le temps de la réalisation d'activités soumises à des contraintes de
temps et de ressource. Dans la plupart des cas, un ou plusieurs
objectifs sont définis et une solution au problème d'ordonnancement 
vise à optimiser ces objectifs.
 
\subsubsection{Les activités}

Une activité peut être définie par une date de début $st_i$ et une date
de fin $et_i$ et une durée $p_i$ vérifiant $et_i=st_i+p_i$. Si
l'activité utilise une ou plusieurs ressources durant son exécution, il
est nécessaire d'ajouter à cette définition une fonction d'allocation
de ressource propre à chaque activité $i$ et à chaque
ressource $k$. Si cette fonction est donnée dans les paramètres du
problème, on la note $r_{ik}(t)$. Si, au contraire, elle fait partie
des variables de décision du problème, on la note $b_{ik}(t)$. Enfin,
si cette fonction ne dépend pas du temps, i.e. est constante, le
paramètre $t$ pourra être omis. 

Selon les problèmes, une activité peut être contrainte à s'exécuter en
un seul morceau. On parle alors d'activité non-préemptive et dans le
cas contraire, i.e. les activités peuvent être exécutées en plusieurs
morceaux, on parle d'activité préemptive.

Une activité peut être utilisée pour représenter, par exemple, une
opération dans un processus de production, le décollage/atterrissage
d'un avion ou encore une étape d'un projet de construction.

\subsubsection{Les ressources}

Une ressource $k$ est un moyen technique ou humain requis pour la
réalisation d'une activité et est disponible en quantité limitée $R_k$. 
Les ressources utilisées par les activités peuvent être de nature
diverse. Parmi elles, on peut distinguer: 
\begin{itemize}
\item les ressources renouvelables: ces ressources peuvent être
réutilisées dès lors qu'elles sont libérées. Il s'agit, en fait, de
ressources qui, après avoir été utilisées par une ou plusieurs
activités, sont de nouveau disponible en même quantité. Ces ressources
peuvent, par exemple, représenter la main d'oeuvre d'une entreprise,
des machines, de l'électricité ou des équipements.
\item à l'inverse, les ressources consommables sont des ressources
dont la consommation globale est limitée au cours du temps. Il peut
s'agir, par exemple, de matières premières ou d'un budget.
\end{itemize}

Parmi les ressources renouvelables, on distingue par ailleurs, les
ressources disjonctives qui ne peuvent exécuter qu'une activité à la
fois - e.g. pistes de décollage, salles - et les ressources cumulatives
qui peuvent, elles,  être utilisées par plusieurs activités en
parallèle mais sont disponibles en quantité limitée - e.g. main d'oeuvre,
processeurs.


Du point de vue de leur divisibilité, les ressources peuvent aussi
être divisées selon deux catégories: 
\begin{itemize}
\item les ressources continues, i.e. divisible en temps ou en
  quantité continu: dans le premier cas, il s'agit de ressource
  pouvant être ré-allouées à tout instant $t \in [0,T]$, où $T$ est une
  borne supérieure sur la date de fin de l'ordonnancement; dans le
  second cas, il s'agit de ressources pouvant être allouée en quantité
  continue, i.e. non discrète. Ce type de ressource permet, par
  exemple, de modéliser l'électricité, l'essence, énergie hydraulique...
\item les ressources discrètes, i.e. divisible en temps ou en quantité
  discret: à l'inverse, le premier cas décrit une ressource où la
  ré-allocation de cette dernière ne peut être 
  exécutée qu'à des temps discrets $t \in \{0,\dots,T\}$; le second
  cas correspond aux ressources ne pouvant être attribuées aux
  activités qu'en quantité discrète,  e.g. employés, machines...
\end{itemize}

\subsubsection{Les contraintes}

Une contrainte permet d'exprimer des restrictions sur les valeurs que
peuvent prendre une ou plusieurs variables du problème. Parmi les
principales, on distingue:
\begin{itemize}
\item les contraintes de temps: elles intègrent les contraintes de
  temps alloué, issues généralement d'impératifs de gestion et
  relatives aux dates limites des activités (e.g. dates
    de livraison) ou à la durée totale d'un projet mais aussi les
    contraintes d'enchaînement qui décrivent des
    positionnements relatifs devant être respectés entre les
    activités. Ces contraintes peuvent, par exemple, modéliser des
    contraintes de précédence entre les activités, i.e. une activité
    ne peut commencer avant qu'une autre n'ait été achevée, ou des temps
    de transition à respecter entre les activités.

    {\'E}tant donné une activité $i$, la date à partir
    de laquelle l'activité $i$ peut être exécutée est appelée {\it
      date de début au plus tôt} et est notée $\ES$ (earliest
    start time en anglais) . De même, la date avant
laquelle l'activité $i$ doit avoir été complètement exécutée sera
appelée {\it date de fin au plus tard} et 
dénotée par $\LE$ (latest end time en anglais).

\item les contraintes de ressources:  ce sont des contraintes d'utilisation
  des ressources qui expriment la nature et la quantité de moyens
  utilisés par les activités, ainsi que les caractéristiques
  d'utilisation de ces moyens. Ces contraintes peuvent aussi représenter
  des contraintes de disponibilité des ressources qui précisent la
  nature et la quantité de moyens disponibles au cours du temps.
\end{itemize}

\subsubsection{Les objectifs}

Lors de la résolution d'un problème d'ordonnancement, deux buts
différents peuvent exister. Le premier vise à trouver une solution
réalisable pour le problème tandis que le second cherche à trouver une
solution optimisant un ou plusieurs critères ou objectifs.

Ces objectifs peuvent être liés à différents aspect de la solution. On
distingue par exemple:
\begin{itemize}
\item les objectifs liés au temps: le temps total d'exécution ou le temps moyen
  d'achèvement d'un ensemble d'activités peuvent être minimisés, mais
  aussi  les retards (maximum, moyen, somme...) par rapport
  aux dates de fin au plus tard fixées par le problème.
\item les objectifs liés aux ressources: la quantité (maximale,
  moyenne, pondérée...) de ressources nécessaires pour réaliser un
  ensemble d'activités peut, par exemple, être minimisée.
\item les objectifs liés aux coûts de lancement, de production, de
  transport, de stockage ou liés aux revenus, aux retours
  d'investissements... 
\item les objectifs liés à une énergie, un débit...
\end{itemize}

Deux exemples de problèmes d'ordonnancement sont présentés dans la
section suivante: le \RCPSP~et le \CUSP.

\subsection{Contraintes de ressources}
\label{sec:ordo_res}
Dans ce manuscrit, nous nous intéressons  principalement  aux problèmes
d'ordonnancement cumulatif. Parmi ces derniers, deux des plus
étudiés sont le problème d'ordonnancement de projet à contraintes de
ressource (RCPSP) et le problème cumulatif (CuSP). Nous allons donc
commencer par présenter ces deux problèmes. En effet, beaucoup des
travaux décrits dans ce manuscrit s'appuient sur des résultats mis en
place pour l'un d'entre eux. 

Dans un second temps, nous montrerons les limites de ces modélisations
pour exprimer certains problèmes cumulatifs réels et
les modèles alternatifs mis en place dans la littérature pour pallier ces
limitations.

\subsubsection{Le problème d'ordonnancement de projet à contraintes de
ressources}

Le problème d'ordonnancement de projet à contraintes de ressources
(\RCPSP) est un problème d'ordonnancement très général, utilisé pour
modéliser certains problèmes pratiques. Le but est
d'ordonnancer un ensemble d'activités de telle sorte que les capacités
des ressources ne soient pas excédées et qu'un certain critière, ou
{\it  fonction objectif}, soit minimisé. Parmi les ressources
modélisées, on trouve des ressources telles que des machines, des
personnes, des salles, de l'argent ou encore de l'énergie. Pour les
fonctions objectif, des quantités telles que la durée totale du
projet, le retard ou les coûts peuvent être minimisées.

Formellement, le \RCPSP~est défini de la manière suivante: nous
considérons un ensemble d'activités non préemptives $\A=\{1,\dots,n\}$
à ordonnancer et un ensemble $\R=\{1,\dots,m\}$ de ressources
discrètes, cumulatives et renouvelables. Chacune de ces ressources $k
\in \R$ est disponible tout au long du projet en quantité $R_k$ et,
durant son exécution, une activité consomme une quantité constante
$r_{ik}$ (pouvant être nulle) de cette ressource. Dans ce problème,
une activité $i \in \A$ a une durée fixe $p_i$ et des relations de
précédence lient les activités entre elles. Ces relations sont
modélisées à l'aide d'un graphe $G=(V,E)$, appelé graphe de
précédence, dans lequel l'ensemble des arcs $(i,j) \in E$ représente
les relations de précédence, i.e. $(i,j) \in E \Leftrightarrow i $
doit être ordonnancée avant $j$ dans toute solution. Dans ce graphe,
l'ensemble des sommets, noté $V=\{0,\dots,n+1\}$, correspond aux $n$
activités auxquelles on ajoute deux activités fictives $0$ et $n+1$
qui représentent respectivement le début et la fin du projet. Ces
activités fictives ne consomment pas de ressource et ont une durée
d'exécution nulle. De plus, $E$ contient les arcs suivants:
\begin{itemize}
\item $(0,i),\ \forall i \in \A$,
\item $(i,n+1),\ \forall i \in \A$.
\end{itemize}

Pour ce problème, la fonction objectif la plus rencontrée dans la
littérature étant la minimisation de la date de fin du projet,
i.e. $C_{max}$, nous considérons principalement cet objectif dans la
suite de ce manuscrit. Si un objectif différent est considéré, nous le
précisons.

L'objectif du problème est donc de déterminer la date de début $st_i$
de chaque activité $i\in \A$ de telle sorte que:
\begin{itemize}
\item à chaque instant $t$, la somme, pour chaque activité, des
  consommations d'une même ressource $k \in \R$ ne doit pas dépasser la
  capacité $R_{k}$ de cette dernière, i.e.
  \begin{equation}\forall t \in \H, \forall k \in \R,\sum_{\substack{i\in \A\\ t \in
        [st_i,st_i+p_i[}} r_{ik} \le R_k\end{equation} 
  avec $\H=\{0,\dots,T\}$ définissant l'horizon de temps du projet. $T$
  est donc une borne supérieure sur la date de fin du projet.
\item les contraintes de précédence sont satisfaites, i.e. 
  \begin{equation} \forall (i,j) \in E,\ st_i+p_i \le p_j \end{equation}
\item la date de fin du projet $C_{max}= \max_{i \in \A} st_i+p_i$
  soit minimale. 
\end{itemize}

\begin{ex}
  \label{ex_RCPSP}
  Considérons l'instance à quatre activités et deux ressources suivante:
  \begin{itemize}
  \item $R_1=5$ et $R_2=7$
  \item cf. figure~\ref{instance_ex_RCPSP}
  \end{itemize}
  \begin{figure}[!htb]
    \centering
    \subcaptionbox{Durée et consommation de chaque activité.}{
      \begin{tabular}{|P{1cm}|P{1cm}P{1cm}P{1cm}|}
        \hline
        i & p_i & r_{i1} & r_{i2}\\
        \hline
        1 & 4 & 2 & 3 \\
        2 & 3 & 1 & 5 \\
        3 & 5 & 2 & 2 \\
        4 & 8 & 2 & 4 \\
        \hline
      \end{tabular}}
    \hfill
    \subcaptionbox{Graphe de précédence des activités.}{
      \begin{tikzpicture}
        [xscale=1.3]
        \node[draw,circle] (O) at (0,0) {\small $0$};
        \node[right of=O,draw,node distance=1.5cm] (D) {\small $2$}; 
        \node[above right of=O,draw] (U) {\small $1$}; 
        \node[below right of=D,draw,node distance =1.5cm] (T) {\small $3$}; 
        \node[below right of=O,draw] (Q) {\small $4$};
        \node[right of=D,draw,node distance=4.5cm] (C) {\small $5$};  

        \draw[->,thick] (O) -- (U.south west);
        \draw[->,thick] (O) -- (D.west);
        \draw[->,thick] (O) -- (T.south west);
        \draw[->,thick] (O) -- (Q.west) ;
        \draw[->,thick] (U) -- (C.north west) ;
        \draw[->,thick] (D) -- (C.west) ;
        \draw[->,thick] (D) -- (T.west) ;
        \draw[->,thick] (T) -- (C.south west) ;
        \draw[->,thick] (Q) -- (C.south) ;
      \end{tikzpicture}}
    \caption{Exemple d'instance pour le \RCPSP.} 
    \label{instance_ex_RCPSP}
  \end{figure}

  La figure~\ref{solution_ex_RCPSP_feas} présente un ordonnancement
  réalisable avec $C_{max}=15$. Cet ordonnancement n'est pas optimal
  puisque si l'activité $1$ est décalée à droite de manière à commencer
  au temps $t=8$, on peut décaler les trois autres activités vers la
  gauche et on obtient un ordonnancement ayant une date de fin
  inférieure à celle de l'ordonnancement précédent, i.e. $C_{max}=12$
  (cf. figure~\ref{solution_ex_RCPSP_opt}) dont on peut prouver
  l'optimalité.

  \begin{figure}[!htb]
    \centering
    \subcaptionbox{Une solution réalisable\label{solution_ex_RCPSP_feas}}[0.4\linewidth]{
      \begin{tikzpicture}
        [xscale=0.4,yscale=0.3]
        \node (O) at (0,0) {};
        \node (bmax) at (0,5) {};

        \draw[->] (O.center) -- (16,0) node[below] {$t$};
        \draw (O.south) -- (bmax.north);

        \draw[dashed] (bmax.center) node[left=0.5pt] {\small $R_1=5$} -- (16,5);
        \draw[fill=white] (0,0) rectangle (4,2) node[midway] {\small $1$};
        \draw[fill=white] (4,0) rectangle (7,1) node[midway] {\small $2$};
        \draw[fill=white] (7,0) rectangle (12,2) node[midway] {\small $3$};
        \draw[fill=white] (7,2) rectangle (15,4) node[midway] {\small $4$};

        \draw (0,0) -- (0,-0.1) node[below=0.5pt] {\small $0$};
        \draw (4,0) -- (4,-0.1) node[below=0.5pt] {\small $4$};
        \draw (7,0) -- (7,-0.1) node[below=0.5pt] {\small $7$};
        \draw (12,0) -- (12,-0.1) node[below=0.5pt] {\small $12$};
        \draw (15,0) -- (15,-0.1) node[below=0.5pt] {\small $15$};

        \foreach \i in {0,...,5}
        {\draw (0,\i) -- (-0.1,\i);}
      \end{tikzpicture}

      \begin{tikzpicture}
        [xscale=0.4,yscale=0.3]
        \node (O) at (0,0) {};
        \node (bmax) at (0,7) {};

        \draw[->] (O.center) -- (16,0) node[below] {$t$};
        \draw (O.south) -- (bmax.north);

        \draw[dashed] (bmax.center) node[left=0.5pt] {\small $R_2=7$} -- (16,7);
        \draw[fill=white] (0,0) rectangle (4,3) node[midway] {\small $1$};
        \draw[fill=white] (4,0) rectangle (7,5) node[midway] {\small $2$};
        \draw[fill=white] (7,0) rectangle (12,2) node[midway] {\small $3$};
        \draw[fill=white] (7,2) rectangle (15,6) node[midway] {\small $4$};

        \draw (0,0) -- (0,-0.1) node[below=0.5pt] {\small $0$};
        \draw (4,0) -- (4,-0.1) node[below=0.5pt] {\small $4$};
        \draw (7,0) -- (7,-0.1) node[below=0.5pt] {\small $7$};
        \draw (12,0) -- (12,-0.1) node[below=0.5pt] {\small $12$};
        \draw (15,0) -- (15,-0.1) node[below=0.5pt] {\small $15$};

        \foreach \i in {0,...,7}
        {\draw (0,\i) -- (-0.1,\i);}
      \end{tikzpicture}}
    \hfill
    \subcaptionbox{La solution optimale\label{solution_ex_RCPSP_opt}}[0.4\linewidth]{
      \begin{tikzpicture}
        [xscale=0.4,yscale=0.3]
        \node (O) at (0,0) {};
        \node (bmax) at (0,5) {};
        \node at (16,0) {};
        \draw[->] (O.center) -- (13,0) node[below] {$t$};
        \draw (O.south) -- (bmax.north);

        \draw[dashed] (bmax.center) node[left=0.5pt] {\small $R_1=5$} -- (13,5);
        \draw[fill=white] (0,0) rectangle (3,1) node[midway] {\small $2$};
        \draw[fill=white] (3,0) rectangle (8,2) node[midway] {\small $3$};
        \draw[fill=white] (3,2) rectangle (11,4) node[midway] {\small $4$};
        \draw[fill=white] (8,0) rectangle (12,2) node[midway] {\small $1$};

        \draw (0,0) -- (0,-0.1) node[below=0.5pt] {\small $0$};
        \draw (3,0) -- (3,-0.1) node[below=0.5pt] {\small $3$};
        \draw (8,0) -- (8,-0.1) node[below=0.5pt] {\small $8$};
        \draw (12,0) -- (12,-0.1) node[below =0.5pt] {\small $12$};

        \foreach \i in {0,...,5}
        {\draw (0,\i) -- (-0.1,\i);}
      \end{tikzpicture}

      \begin{tikzpicture}
        [xscale=0.4,yscale=0.3]
        \node (O) at (0,0) {};
        \node (bmax) at (0,7) {};
        \node at (16,0) {};

        \draw[->] (O.center) -- (13,0) node[below] {$t$};
        \draw (O.south) -- (bmax.north);

        \draw[dashed] (bmax.center) node[left=0.5pt] {\small $R_2=7$} -- (13,7);
        \draw[fill=white] (0,0) rectangle (3,5) node[midway] {\small $2$};
        \draw[fill=white] (3,3) rectangle (11,7) node[midway] {\small $4$};
        \draw[fill=white] (3,0) rectangle (8,2) node[midway] {\small $3$};
        \draw[fill=white] (8,0) rectangle (12,3) node[midway] {\small $1$};

        \draw (0,0) -- (0,-0.1) node[below=0.5pt] {\small $0$};
        \draw (3,0) -- (3,-0.1) node[below=0.5pt] {\small $3$};
        \draw (8,0) -- (8,-0.1) node[below=0.5pt] {\small $8$};    
        \draw (12,0) -- (12,-0.1) node[below=0.5pt] {\small $12$};

        \foreach \i in {0,...,7}
        {\draw (0,\i) -- (-0.1,\i);}
      \end{tikzpicture}}
    \caption{Exemple de solutions réalisables pour le \RCPSP.} 
    \label{solution_ex_RCPSP}
  \end{figure}
\end{ex}

Le \RCPSP~est un problème qui a été prouvé NP-complet au sens
fort~\cite{NP_RCPSP}. Ce problème a donc été très étudié dans la
littérature, notamment pour trouver des méthodes efficaces pour sa
résolution. Dans la section~\ref{sec:PLNE_ordo_res}, nous présentons des
modèles de programmation linéaire permettant de trouver la solution
optimale à ce problème. Ces modèles seront alors adaptés dans le cadre
d'un autre problème d'ordonnancement décrit dans le
paragraphe~\ref{sec:ordo_nrj}. 

\subsubsection{Le problème cumulatif}

Le problème d'ordonnancement cumulatif (CuSP) permet de caractériser
le fait que le projet implique une ressource (ou un sous-ensemble de
ressources) de nature cumulative. Le \CUSP~peut être vu comme un cas
particulier de la variante décisionnelle du RCPSP où l'on ne considère
qu'une ressource et où l'on remplace les contraintes de précédence par
les fenêtres de temps qu'elles induisent.

Formellement, le \CUSP~prend en entrée un ensemble $ \A=\{1,\dots,n\}$
d'activités non préemptives à ordonnancer. Pour s'exécuter, une
activité doit consommer une partie de la ressource $r_i$ et ce jusqu'à
l'arrêt de l'activité, i.e. après un temps $p_i$ correspondant à la
durée de l'activité $i$. Cette ressource est de type cumulatif,
discrète et renouvelable, disponible en quantité $R$.

De plus, chaque activité dispose d'une fenêtre de temps $[\ES,\LE]$
dans laquelle l'activité doit obligatoirement s'exécuter. Nous
rappelons que $\ES$ correspond à la date de début au plus tôt de $i$
et $\LE$ à sa date de fin au plus tard.

L'objectif du \CUSP~est donc de déterminer la date de début $st_i$ de
chaque activité $i \in \A$ telle que:
\begin{itemize}
\item la capacité de la ressource n'est excédée à aucun moment du
  projet, i.e.
  \begin{equation} \forall t \in \H,\sum_{\substack{i\in \A\\ t \in
        [st_i,st_i+p_i[}} r_{i} \le  R
\label{eq:CUSP_res}
\end{equation}
\item la fenêtre de temps de chaque activité est respectée, i.e. 
  \begin{equation} \forall i \in \A,\ \ES \le st_i < st_i+p_i \le \LE \end{equation}
\end{itemize}

Trouver une solution réalisable pour ce problème -- étant
une extension de la variante de décision du problème à une machine
($R=1$ et $r_i=1$) et du problème à $m$ machines ($R=m$ et $r_i=1$) -- 
est NP-complet au sens fort~\cite{NP_bible}. De ce fait, dans la
littérature, ce problème est souvent étudié sans fonction
objectif. Sauf précision du contraire, nous ferons de même dans la
suite du manuscrit. 

\begin{ex}
  \label{CUSP_ex}
  Considérons l'instance à quatre activités suivante:
  \begin{itemize}
  \item $R=4$
  \item cf. table~\ref{instance_CUSP_ex}\begin{table}[!htb]
      \centering
      \begin{tabular}{|P{1cm}|P{1cm}P{1cm}P{1cm}P{1cm}|}
        \hline
        i & p_i & \ES & \LE & r_i\\
        \hline
        1 & 2 & 1 & 5 & 2 \\
        2 & 1 & 3 & 5 & 2\\
        3 & 1 & 3 & 5 & 3\\
        4 & 4 & 1 & 10 & 1 \\
        \hline
      \end{tabular}
      \caption{Exemple d'une instance du \CUSP.}
      \label{instance_CUSP_ex}
    \end{table}
  \end{itemize}
  La figure~\ref{solution_CUSP_ex} présente plusieurs solutions
  réalisables pour cette instance. 
  \begin{figure}
    \begin{minipage}{0.45\linewidth}
      \begin{tikzpicture}
        [xscale=0.6,yscale=0.7]
        \node (O) at (0,0) {};
        \node (bmax) at (0,3) {};
        \node at (9,0) {};

        \draw[->] (O.center) -- (9,0) node[below] {$t$};
        \draw (O.south) -- (bmax.north);

        \draw[dashed] (bmax.center) node[left=0.5pt] {\small $R=3$} -- (9,3);
        \draw[fill=white] (1,0) rectangle (3,2) node[midway] {\small $1$};
        \draw[fill=white] (3,0) rectangle (4,3) node[midway] {\small $3$};
        \draw[fill=white] (4,2) rectangle (8,3) node[midway] {\small $4$};
        \draw[fill=white] (4,0) rectangle (5,2) node[midway] {\small $2$};

        \draw (1,0) -- (1,-0.1) node[below=0.5pt] {\small $1$};
        \draw (3,0) -- (3,-0.1) node[below=0.5pt] {\small $3$};
        \draw (5,0) -- (5,-0.1) node[below=0.5pt] {\small $5$};    
        \draw (4,0) -- (4,-0.1) node[below=0.5pt] {\small $4$}; 
        \draw (8,0) -- (8,-0.1) node[below=0.5pt] {\small $8$};

        \node at (10,0) {};
        \foreach \i in {0,...,3}
        {\draw (0,\i) -- (-0.1,\i);}
      \end{tikzpicture}
    \end{minipage}
    \hfill
    \begin{minipage}{0.45\linewidth}
      \begin{tikzpicture}
        [xscale=0.6,yscale=0.7]
        \node (O) at (0,0) {};
        \node (bmax) at (0,3) {};
        \node at (10,0) {};

        \draw[->] (O.center) -- (10,0) node[below] {$t$};
        \draw (O.south) -- (bmax.north);

        \draw[dashed] (bmax.center) node[left=0.5pt] {\small $R=3$} -- (10,3);
        \draw[fill=white] (1,0) rectangle (3,2) node[midway] {\small $1$};
        \draw[fill=white] (4,0) rectangle (5,3) node[midway] {\small $3$};
        \draw[fill=white] (5,2) rectangle (9,3) node[midway] {\small $4$};
        \draw[fill=white] (3,0) rectangle (4,2) node[midway] {\small $2$};

        \draw (1,0) -- (1,-0.1) node[below=0.5pt] {\small $1$};
        \draw (3,0) -- (3,-0.1) node[below=0.5pt] {\small $3$};
        \draw (5,0) -- (5,-0.1) node[below=0.5pt] {\small $5$};    
        \draw (4,0) -- (4,-0.1) node[below=0.5pt] {\small $4$}; 
        \draw (8,0) -- (8,-0.1) node[below=0.5pt] {\small $8$};

        \foreach \i in {0,...,3}
        {\draw (0,\i) -- (-0.1,\i);}
      \end{tikzpicture}
    \end{minipage}
    \caption{Exemple de solutions réalisables pour le \CUSP.}
    \label{solution_CUSP_ex}
  \end{figure}
\end{ex}

Dans la section~\ref{sec:cumu}, nous présenterons des
méthodes de résolution pour ce problème utilisant la programmation par
contraintes. 

\subsubsection{Limites des problèmes cumulatifs en termes de modélisation}
\label{sec:limit_CUSP}
Une des principales limitations du \RCPSP~et donc du \CUSP~(qui en est
un cas particulier) en termes de modélisation est que, dans ces
problèmes, chaque activité consomme une quantité de ressource fixe,
connue à l'avance, durant toute sa durée d'exécution. De plus, cette
durée est aussi supposée fixe. Cependant, pour de nombreux problèmes
pratiques, ces suppositions reviennent à sur-contraindre le
problème. En effet, considérons l'exemple suivant, décrit
dans~\cite{FT}.  Dans cet exemple, une activité représentant la
peinture d'un bateau est considérée. Pour cette activité, la durée est
remplacée par une quantité de travail, ou énergie, représentant le
travail de $3$ personnes sur une journée. Cette activité peut, au
choix, être exécutée en $3$ jours par $1$ personne, ou en $2$ jours
par $2$ personnes le premier jour et $1$ personne le second, ou encore
par $3$ personnes en seulement un jour. Si l'on avait supposé une
consommation et une durée fixe, l'espace des solutions aurait été
réduit.

L'exemple décrit ci-dessus peut facilement être généralisé à de
nombreux cas pratiques. Un autre exemple venant d'un problème
industriel est présenté dans~\cite{HaitArtiguesLopez}. Dans cet
article, une application dans une fonderie où du métal est fondu dans
des fournaises à induction est détaillé. La puissance de chaque
fournaise utilisée pour faire fondre le métal peut être réglée, à tout
moment, afin d'éviter un dépassement d'un certain niveau d'énergie
(généralement dû à une limite suggérée par le fournisseur
d'énergie). De ce fait, si chaque opération de fonte est vue comme une
activité, nous avons besoin de pouvoir moduler la quantité d'énergie
donnée à cette activité à chaque instant et la durée de l'activité
dépend de cette quantité d'énergie.

Pour représenter ces variations dans le profil de consommation des
activités, nous avons choisi de modéliser la ressource comme une
ressource continue. En effet, de nombreux exemples de ressource
continue existent. C'est le cas de l'électricité, des sources
d'énergie hydrauliques, du carburant ou encore de la mémoire d'un
ordinateur. D'autres ressources telles que des employés ou des
machines, qui sont normalement modélisées à l'aide de ressources
discrètes, peuvent être modélisées par des ressources continues si
l'on suppose qu'un employé ou une machine peut exécuter plusieurs
activités en parallèle~\cite{W80,NK}.

De plus, les problèmes à ressources continues peuvent aussi servir de
relaxation pour les problèmes avec des ressources discrètes. En effet,
le caractère discret du problème peut parfois amener à considérer de
nombreuses possibilités d'affectations de la ressource. Ce grand nombre
d'affectation peut grandement complexifier le problème. Donc,
considérer des ressources continues peut permettre l'agrégation des
raisonnements dédiés à ces problèmes. 

La modélisation complète du problème considéré dans cette thèse est
décrit dans la section suivante. Cette modélisation sera ensuite
comparée aux extensions du \CUSP~et du \RCPSP, déjà existantes dans la
littérature. 



\section{L'ordonnancement sous contraintes énergétiques}
Dans le \RCPSP~et dans le \CUSP, il est supposé qu'une activité
consomme une quantité fixe de ressource durant toute son exécution et
que cette activité a une durée fixe. Cependant, il arrive qu'en
pratique, ce ne soit pas la cas. Un exemple venant d'un problème
industriel est présenté dans~\cite{HaitArtiguesLopez}. Dans cet
article, une application dans une fonderie où du métal est fondu dans
des fournaises à induction est détaillé. La puissance de chaque
fournaise utilisée pour faire fondre le métal peut être réglée, à tout
moment, afin d'éviter un dépassement d'un certain niveau d'énergie
(généralement dû à une limite suggérée par le fournisseur
d'énergie). De ce fait, si chaque opération de fonte est vue comme une
activité, nous avons besoin de pouvoir moduler la quantité d'énergie
donnée à cette activité à chaque instant et la durée de l'activité dépend de cette 
quantité d'énergie. La modélisation de ce problème repose sur le problème 
d'ordonnancement continu à contraintes énergétiques~\cite{ArtiguesLopez}, le
\CECSP. 

\subsubsection{Définition du problème}

Dans ce problème, un ensemble d'activités non-préemptives
$\A=\{1,\dots,n\}$ utilisant une ressource continue, cumulative et
renouvelable, de capacité $R$ doit être ordonnancé. Durant son
exécution, une activité consomme une quantité variable $b_i(t)$ de la
ressource qui doit être comprise entre une valeur minimale, $\bmin \in
[0,R]$, et une valeur maximale, $\bmax \in [\bmin,R]$. De plus, la fin
d'une activité correspond au moment où cette dernière a reçu une
certaine quantité d'énergie $W_i$. Cette énergie est reçue via la
ressource et calculée à l'aide d'une fonction $f_i: \{0\} \cup
[\bmin,\bmax]\longrightarrow\{0\} \cup [f(\bmin),f(\bmax)]$, supposée
continue et croissante et appelée fonction de rendement. La quantité
d'énergie reçue par $i$ à l'instant $t$ est donc $\int_{0}^t
f_i(b_i(s))ds$. La dernière contrainte du problème précise que chaque activité doit
être exécutée dans sa fenêtre de temps $[\ES,\LE]$.

L'objectif du \CECSP~est donc de déterminer la date de début $st_i$ et
de fin $et_i$ de chaque activité $i \in \A$, ainsi que la fonction
d'allocation de ressource, $b_i(t)$, associée à
cette activité telle que: 
\begin{itemize}
\item la fenêtre de temps de chaque activité est respectée, i.e. 
  \begin{equation} 
    \forall i \in \A,\ \ES \le st_i < et_i \le \LE \label{tw_CECSP}
  \end{equation}
  Les activités de durée nulle ne sont pas considérées. 
\item la capacité de la ressource n'est excédée à aucun moment du
  projet, i.e.
  \begin{equation} 
    \forall t \in \H,\sum_{\substack{i\in \A\\ t \in
        [st_i,et_i]}} b_i(t) \le  R \label{res_CECSP}
  \end{equation}
  où $\H=\{0,\dots,T\}$ est l'horizon de temps du projet et $T=\max_{i
    \in \A} \LE$.
\item si une activité est en cours à l'instant $t$ alors les
  contraintes de consommation minimale et maximale doivent être
  respectées, i.e.  
  \begin{equation}
    \forall i \in \A,\ \forall t \in [st_i,et_i],\ \bmin \le b_i(t) \le
    \bmax \label{req_CECSP}
  \end{equation}
\item si l'activité n'est pas en cours, alors elle ne consomme pas de
  ressource, i.e.
  \begin{equation}
    \label{nulleConso_CECSP}
    \forall i \in \A,\ \forall t \not\in [st_i,et_i],\  b_i(t)=0 
  \end{equation}
\item l'énergie requise doit être apportée à chaque activité,, i.e. 
  \begin{equation}
    \forall i \in \A,\ \int_{st_i}^{et_i}f_i(b_i(t))dt=W_i \label{nrj_CECSP}
  \end{equation}
  Dans certains cas, nous pourrons remplacer cette contrainte par la
  contrainte suivante:
  \begin{equation}
    \forall i \in \A,\ \int_{st_i}^{et_i}f_i(b_i(t))dt \ge W_i \tag{\ref{nrj_CECSP}a}
  \end{equation}
\end{itemize}

Dans ce manuscrit, nous considérons les trois cas
suivants:
\begin{itemize}
\item $f_i$ est la fonction identité, $\forall i \in \A$,
\item $f_i$ est une fonction affine, $\forall i \in \A$,
\item $f_i$ est une fonction concave et affine par morceaux, $\forall
  i \in \A$.
\end{itemize}

L'intérêt de considérer de telles fonctions de rendement est qu'elles
nous permettent d'approcher un grand nombre de fonctions de
rendements réelles non linéaires. Un exemple présentant de telles
approximations sera présenté (cf. exemple\ref{ex_approx_CECSP}). 


Soit $P_i$ le nombre d'intervalles de définition de la fonction
$f_i$, i.e. le nombre d'intervalles où la fonction $f_i$ a une
expression différente. La fonction $f_i$ peut alors s'écrire de la
manière suivante:  
\[f_i(b)=\left\{
    \begin{array}{ll}
      0 & \quad \text{if }b=0\\
      a_{ip_1}*b+c_{ip_1} &\quad \text{si }\bmin=0\text{ et }b \in ]\bmin,x_{p_2}] \\
      a_{ip_1}*b+c_{ip_1} &\quad \text{si }\bmin\neq 0 \text{ et }b \in
                            [\bmin,x_{p_2}] \\
      a_{ip_\ell}*b+c_{ip_\ell} &\quad \text{si } b \in
                                  ]x_{p_\ell},x_{p_{\ell+1}}], \ell \in \P_i=\{2,\dots,P_i-1\} 
    \end{array}
  \right.\]
De plus, nous considérons que la fonction $f_i$ satisfait les
propriétés suivantes: 
\begin{itemize}
\item $a_{ip_1} >a_{ip_2} > \dots > a_{ip_{P_i}}>0$ et $c_{ip_1}
  <c_{ip_2} < \dots < c_{ip_{P_i}}$ pour assurer la croissance et la
  concavité de la fonction; 
\item $-a_{ip_1}*\bmin \ge c_{ip_1}$  afin de s'assurer que $f_i(b) \ge
  0,\ \forall b \in [\bmin,\bmax]$;
\item $a_{ip_\ell}*b+c_{ip_\ell}=a_{ip_{\ell+1}}*b+c_{ip_{\ell+1}}$
  pour assurer la continuité de la fonction.
\end{itemize}

\begin{ex}
  \label{ex_approx_CECSP}
  Considérons l'instance à  quatre activités suivante:
  \begin{itemize}
  \item $B=2$
  \item $\ES={[}0,2,0,5{]}$
  \item $\LE={[}6,10,9,13{]}$
  \item $\bmin={[}0,0.25,2,1{]}$
  \item $\bmax={[}1,1,2,1.5{]}$
  \item $W={[}1,5,7,8{]}$
  \item $f(b)={[}b,\sqrt{b},b,\sqrt{b}{]}$
  \end{itemize}

  Nous devons approcher les fonctions $f_2(b)$ et $f_4(b)$. Commençons
  par le cas où la fonction est approchée par une fonction affine. Pour
  cela, nous calculons le coefficient directeur de la tangente en
  $\bmin + (\bmax-\bmin)/ 2 =0.625$. Ce coefficient directeur est égal à
  $\frac{1}{2\sqrt{0.625}}$, et donc,
  $f'_2(b)=\frac{1}{2\sqrt{0.625}}*b+ \frac{\sqrt{0.625}}{2}$ (cf. figure~\ref{approx_aff}). 

  \begin{figure}[!htb]
    \centering
    \subcaptionbox{Approximation par une fonction affine
      \label{approx_aff}}[0.45\linewidth]{
      \begin{tikzpicture}
        [xscale=4.5,yscale=2.5]
        \node (O) at (0,0) {};

        \draw[->] (0,0) -- (1.28,0) node[below] {$b$};
        \draw[->] (0,0) -- (0,1.3) node[left] {$f_i(b)$};

        \draw (0,1) -- (-0.02,1) node[left] {$1$};
        \draw (0,0.5) -- (-0.02,0.5) node[left] {$\sqrt{0.25}$};

        \draw (1,0) -- (1,-0.02) node[below] {$1$};
        \draw (0.25,0) -- (0.25,-0.02) node[below] {$0.25$};

        \draw[color=gray,domain=0:1.28,samples=50] plot ({\x},{sqrt(\x)});
        \draw[dashed,thick,domain=0.25:1,samples=50] plot
        ({\x},{\x/(2*sqrt(0.625))+sqrt(0.625)/2});


        \draw[dotted] (0.25,0)-- (0.25,1.3);
        \draw[dotted] (1,0) -- (1,1.3);
      \end{tikzpicture}}
    \hfill
    \subcaptionbox{Approximation par une fonction concave, affine par
      morceaux
      \label{approx_affparmorceau}}[0.45\linewidth]{
      \begin{tikzpicture}
        [xscale=4.5,yscale=2.5]
        \node (O) at (0,0) {};

        \draw[->] (0,0) -- (1.28,0) node[below] {$b$};
        \draw[->] (0,0) -- (0,1.3) node[left] {$f_i(b)$};

        \draw (0,1) -- (-0.02,1) node[left] {$1$};
        \draw (0,0.5) -- (-0.02,0.5) node[left] {$\sqrt{0.25}$};

        \draw (1,0) -- (1,-0.02) node[below] {$1$};
        \draw (0.25,0) -- (0.25,-0.02) node[below] {$0.25$};

        \draw[color=gray,domain=0:1.28,samples=50] plot ({\x},{sqrt(\x)});
        
        \draw[ dashed,thick,domain=0.25:0.5,samples=50] plot
        ({\x},{\x/(2*sqrt(0.375))+sqrt(0.375)/2});
        \draw[dashed, thick,domain=0.5:0.75,samples=50] plot
        ({\x},{\x/(2*sqrt(0.625))+sqrt(0.625)/2});
        \draw[dashed, thick,domain=0.75:1,samples=50] plot
        ({\x},{\x/(2*sqrt(0.875))+sqrt(0.875)/2});

        \draw[dotted] (0.5,0)-- (0.5,1.3);
        \draw[dotted] (0.25,0) -- (0.25,1.3);
        \draw[dotted] (0.75,0) -- (0.75,1.3);
        \draw[dotted] (1,0) -- (1,1.3);
      \end{tikzpicture}}
    \caption{Approximation d'une fonction de rendement non linéaire.}
    \label{approx}
  \end{figure}

  De même, pour l'activité $4$, nous avons
  $f'_4(b)=\frac{1}{2\sqrt{1.25}}*b+ \frac{\sqrt{1.25}}{2}$.

  Approchons maintenant la fonction $f_2(b)$  par une
  fonction concave et affine par morceaux. Dans un premier temps,
  nous devons choisir le pas d'approximation $\epsilon$, i.e. la taille
  des intervalles pour lesquels la fonction $f_i$ a une
  expression différente. Dans cet, exemple, nous choisissons,
  $\epsilon=1/4$. Le nombre d'intervalles de définition de la fonction
  $f_2$ est alors $(\bmax-\bmin)/\epsilon=3$. Pour chacun de ces
  intervalles, nous appliquons la procédure utilisée pour
  l'approximation de $f_2$ par une fonction affine. Nous obtenons donc
  l'approximation suivante (cf. figure~\ref{approx_affparmorceau}):
  \[f_2=\left\{ 
      \begin{array}{lll}
        \frac{1}{2\sqrt{3/8}}*b + \frac{\sqrt{3/8}}{2}& & \text{si } b \in
                                                          [0.25,0.5]\\
        \frac{1}{2\sqrt{5/8}}*b + \frac{\sqrt{5/8}}{2}& & \text{si } b \in [0.5,0.75]\\
        \frac{1}{2\sqrt{7/8}}*b + \frac{\sqrt{7/8}}{2}& & \text{si } b \in [0.75,1]
      \end{array}
    \right.\]
  
\end{ex}

Avec une telle approche, il peut arriver que la fonction de rendement
$f_i$ ne vérifie pas $\bmin=0 \Rightarrow f_i(\bmin)=0$. Dans ce cas,
la valeur de $f_i(0)$ est mise à $0$. La fonction n'est donc plus
continue sur tout son intervalle de définition. La
contrainte~\eqref{nrj_CECSP} est donc remplacée par:
\begin{equation}
  \int_{st_i}^{et_i}{\bf 1}_{NZ}(t)f_i(b_i(t))dt = W_i \tag{\ref{nrj_CECSP}b}
\end{equation}
\noindent 
où ${\bf 1}_{NZ}(t):=\left\{
  \begin{array}{ll}
    1 & \text{si }t \in NZ:=\{t|b_i(t)\neq0\}\\
    0 & \text{sinon}
  \end{array}
\right.$ est la fonction caractéristique de l'ensemble $\mathbb{R}^+$.

La difficulté du \CECSP~repose, entre autre, sur le fait que la
fonction d'allocation de ressource peut n'être ni constante, ni
constante par morceau. De ce fait, la représentation temps/ressource
d'une activité peut prendre n'importe quelle forme
(cf. figure~\ref{figure_forme_conso}). 

\begin{figure}[!htb]
  \centering
  \subcaptionbox{$b_i(t)$ constante}[0.45\linewidth]{
    \begin{tikzpicture}
      [xscale=0.75,yscale=0.5]
      \node[] (O) at (0,0) {};
      
      
      \draw (0.5,0) node[below] {$\ES$};
      \draw (6,0) node[below] {$\LE$};
      \draw  (0,1) node[left] {$\bmin$};
      \draw (0,4) node[left] {$\bmax$};

      \draw[dotted] (0,1) -- (6.5,1);
      \draw[dotted] (0,4) -- (6.5,4);
      \draw[dotted] (0.5,0) -- (0.5,4);
      \draw[dotted] (6,0) -- (6,4);

      \draw[->] (O.center) -- (0,4.5) node[above] {$b_i(t)$};
      \draw[->] (O.center) -- (6.5,0) node[right] {$t$};
      
      \draw (5.7,0) -- (5.7,2) -- (1,2) -- (1,0);
    \end{tikzpicture}
  }
  \hfill
  \subcaptionbox{$b_i(t)$ constante par morceaux}[0.45\linewidth]{
    \begin{tikzpicture}
      [xscale=0.75,yscale=0.5]
      \node[] (O) at (0,0) {};
      
      
      \draw (0.5,0) node[below] {$\ES$};
      \draw (6,0) node[below] {$\LE$};
      \draw  (0,1) node[left] {$\bmin$};
      \draw (0,4) node[left] {$\bmax$};

      \draw[dotted] (0,1) -- (6.5,1);
      \draw[dotted] (0,4) -- (6.5,4);
      \draw[dotted] (0.5,0) -- (0.5,4);
      \draw[dotted] (6,0) -- (6,4);

      \draw[->] (O.center) -- (0,4.5) node[above] {$b_i(t)$};
      \draw[->] (O.center) -- (6.5,0) node[right] {$t$};
      
      \draw (5.7,0) -- (5.7,2) -- (4,2) -- (4,4) -- (2,4) -- (2,1) -- (1,1) --  (1,0);
    \end{tikzpicture}
  }\\
  \vspace{0.2cm}
  \subcaptionbox{$b_i(t)$ quelconque}[0.9\linewidth]{
    \begin{tikzpicture}
      [xscale=0.75,yscale=0.5]
      \node (O) at (0,0) {};

      \path[draw] (1,0) -- (1,4) parabola [bend at end] (5.7,1) -- (5.7,0); 

      \draw (0.5,0) node[below] {$\ES$};
      \draw (6,0) node[below] {$\LE$};
      \draw  (0,1) node[left] {$\bmin$};
      \draw (0,4) node[left] {$\bmax$};

      \draw[dotted] (0,1) -- (6.5,1);
      \draw[dotted] (0,4) -- (6.5,4);
      \draw[dotted] (0.5,0) -- (0.5,4);
      \draw[dotted] (6,0) -- (6,4);
      
      
      \draw[->] (O.center) -- (0,4.5) node[above] {$b_i(t)$};
      \draw[->] (O.center) -- (6.5,0) node[right] {$t$};
      
      \draw (1,4) -- (1,0);
      
      
      \path[draw] (1,4) parabola [bend at end] (5.7,1); 
      
    \end{tikzpicture}
    \hfill
    \begin{tikzpicture}
      [xscale=0.75,yscale=0.5]
      \node (O) at (0,0) {};
      
      \draw (0.5,0) node[below] {$\ES$};
      \draw (6,0) node[below] {$\LE$};
      \draw  (0,1) node[left] {$\bmin$};
      \draw (0,4) node[left] {$\bmax$};

      \draw[dotted] (0,1) -- (6.5,1);
      \draw[dotted] (0,4) -- (6.5,4);
      \draw[dotted] (0.5,0) -- (0.5,4);
      \draw[dotted] (6,0) -- (6,4);
      
      
      \draw[->] (O.center) -- (0,4.5) node[above] {$b_i(t)$};
      \draw[->] (O.center) -- (6.5,0) node[right] {$t$};

      \draw (1,1) parabola bend (1.8,3)(2.5,2); 
      \draw (2.5,2) parabola bend (3.2,1) (4.5,3);
      \draw (4.5,3) parabola bend (5.2,4) (5.7,2);
      \draw (1,0) -- (1,1);
      \draw (5.7,0) -- (5.7,2);
    \end{tikzpicture}
  }
\caption{Différentes formes de fonction d'allocation de ressource pour
le \CECSP.}
\label{figure_forme_conso}
\end{figure} 

Nous allons maintenant décrire un exemple d'instance et de solution
pour le \CECSP. Cependant, par souci de clarté, nous présentons un
exemple où il existe une solution dans laquelle toutes les fonctions
$b_i(t)$ sont constantes par morceaux.

\begin{ex}
Considérons l'instance à $3$ activités du \CECSP~suivante:
\begin{itemize}
\item $R=5$
\item cf. figure~\ref{ex_CECSP}
\item la fonction $f_2(b)$ est définie par l'expression suivante: 
\[f_2(b)=\left\{
\begin{array}{lll}
2b+1 & & b \in [3,7/2]\\
b+9/2 & & b \in [7/2,4]
\end{array}
\right.\]
\end{itemize}
\begin{figure}[!htb]
\centering
\subcaptionbox{Fonction $f_2(b)$ \label{fonction_ex_CECSP}}[0.4\linewidth]{
\begin{tikzpicture}
[xscale=1.7,yscale=0.56]
\node (O) at (2,5) {};
\draw[->] (2,5) -- (5,5);
\draw[->] (2,5) -- (2,10);

\path[draw] (3,7) -- (3.5,8) -- (4,9.5) ;

\draw[dotted] (3,5) node[below] {\footnotesize $3$} -- (3,10);
\draw[dotted,color=gray!70] (3.5,5.2) node[below=-0.1pt,color=black] {\footnotesize $3.5$}
-- (3.5,10);
\draw[dotted] (4,5) node[below] {\footnotesize $4$} -- (4,10);

\draw (2,7) node[left] {\footnotesize $7$};
\draw (2,8) node[left] {\footnotesize $8$};
\draw (2,9.5) node[left] {\footnotesize $9.5$};
\end{tikzpicture}}
\hfill
\subcaptionbox{}[0.55\linewidth]{
  \begin{tabular}{|M{0.6cm}|M{0.6cm}M{0.6cm}M{0.6cm}M{0.6cm}M{0.6cm}M{1.2cm}|}
    \hline
    $i$ & $r_i$ & $d_i$ & $W_i$ & $\bmin$ & $\bmax$ & $f_i(b)$\\[2mm]
\hline
    1 & 0 & 2 & 6 & 3 & 3 & $b$\\[2mm]
    2 & 1 & 5 & 26 & 3 & 4 & fig.~\ref{fonction_ex_CECSP}\\[2mm]
    3 & 0 & 6 & 39 & 1 & 5 & $3b$\\[2mm]
    \hline
    \multicolumn{7}{c}{}
  \end{tabular}} 
\caption{Données de l'instance de l'exemple du \CECSP.}
\label{ex_CECSP}
\end{figure}
La figure~\ref{sol_ex_CECSP} présente une solution réalisable pour le
\CECSP. Dans cette figure, nous pouvons voir que l'énergie reçue par une
activité n'est, à priori, pas égale à la quantité de ressource
consommée par cette dernière. En effet, regardons l'activité $2$. Sa
consommation de ressource est égale à $3 + 4 * 2 = 11$ tandis que
l'énergie qu'elle reçoit est $f_2(3)+ f_2(4) * 2 = 7 + 19/2 * 2 =26$.

\begin{figure}[!htb]
\centering
\begin{tikzpicture}
[xscale=0.75,yscale=0.56]
\node (O) at (0,0) {};
\draw[->] (0,0) -- (6.5,0);
\draw[->] (0,0) -- (0,5.5);

\draw (0,2) rectangle (2,5) node[midway] {$1$};
\path[draw] (0,2) -- (3,2) -- (3,1) node[left=0.4cm] {$3$} -- (5,1) -- (5,5) -- (6,5)  -- (6,0);
\draw (2,5) -- (5,5) node[midway,below=0.5cm] {$2$};

\draw (0,1) node[left] {\footnotesize $1$};
\draw (0,2) node[left] {\footnotesize $2$};
\draw (0,5) node[left] {\footnotesize $5$};


\draw (2,0) node[below] {\footnotesize $2$};
\draw (3,0) node[below] {\footnotesize $3$};
\draw (5,0) node[below] {\footnotesize $5$};

\foreach \i in {1,...,5}{
\draw (\i,-0.1) -- (\i,0);
\draw (-0.1,\i) -- (0,\i);
}
\end{tikzpicture}
\caption{Solution de l'instance du \CECSP.}
\label{sol_ex_CECSP}
\end{figure}
\end{ex}

Le paragraphe suivant présente des propriétés du \CECSP~qui seront 
utilisées dans la suite de ce manuscrit. 

\subsubsection{Propriétés du \CECSP}

Dans ce paragraphe, nous allons commencer par présenter la preuve de
NP-complétude du \CECSP. Ce problème pouvant être vu comme une
généralisation du \CUSP, nous utilisons ce problème pour montrer la
difficulté du \CECSP. 

\begin{theo}[\cite{Nattaf_Constraints}]
Le \CECSP~est NP-complet.
\end{theo}

\begin{proof}
Nous réduisons donc le \CUSP~vers le \CECSP. Soit $\Pi$ une instance
du \CUSP. Nous réduisons $\Pi$ en une instance du \CECSP, $\Pi'$, de
la manière suivante, $\forall i \in \A$:
\begin{itemize}
\item $ \bmin=\bmax=r_i$
\item $f_i(b)=b$
\item $W_i=p_ir_i$
\item $R,\ \ES$ and $\LE$ restent inchangés. 
\end{itemize}

Il reste à montrer que $\Pi$ est une instance positive du \CUSP~si
et seulement si $\Pi'$ est une instance positive du \CECSP. Et ceci
est trivialement vrai. 
\end{proof}

Le problème de décision associé au \CECSP~est donc NP-complet. Dans ce
manuscrit, nous avons donc considérer ce problème sans fonction
objectif mais aussi avec la fonction objectif suivante: 
\[\text{minimiser } \sum_{i \in \A} \int_{st_i}^{et_i} b_i(t)dt\]
Cette fonction consiste en la minimisation de la consommation totale
de ressource. Dans la suite, si rien n'est précisé, cela veut dire que
nous considérons le \CECSP~ sans fonction objectif.



 Pour le \CECSP, nous présentons un exemple d' une instance ne
comprenant que des données entières et ne possédant que des solutions
non-entières. Ceci permet, en plus des arguments présentés pour le
\RCPSP, de démontrer l'importance de l'amélioration des modèles à
événement.

\begin{ex}
  \label{exemple_NE}
  Dans cet exemple, nous considérons une instance à deux activités et
  une ressource de capacité $2$. Le tableau~\ref{instance_exemple_NE}
  décrit les données de l'instance. 
  \begin{table}[!htb]
    \centering
    \begin{tabularx}{12cm}{|>{\centering\arraybackslash}p{0.6cm}|
        *5{>{\centering\arraybackslash}X}>{\centering\arraybackslash}p{2cm}|}
      \hline
      $i$ & $\ES$ & $\LE$ & $W_i$ & $\bmin$ & $\bmax$ & $f_i(b_i(t))$ \\
      \hline
      $1$ & $0$ & $2$ & $18$ & $2$ & $2$ & $3b_i(t)+6$\\
      $2$ & $1$ & $3$ & $3$ & $1$ & $2$ & $b_i(t)$\\
      \hline
    \end{tabularx}
    \caption{Données de l'instance de l'exemple~\ref{exemple_NE}}
    \label{instance_exemple_NE}
  \end{table}

  La seule solution est décrite par la figure~\ref{figure_exemple_NE}.
  \begin{figure}[!htb]
    \centering
    \begin{tikzpicture}
      [xscale=2]
      \node (O) at (0,0) {} node[below=0.1cm] {$0$};
      \draw (1.5,0)  node[below=0.1cm] {$1.5$};
      \draw (3,0) node[below=0.1cm] {$3$};
      \node (T) at (3.5,0) {};
      \node at (0.75,1) {0};
      \node at (2.25,1) {1};
      \draw[dashed] (-0.5,2) -- (3.5,2) node[right,node
      distance=1.5pt] {$R=2$}; 
      \draw (O) rectangle (1.5,2);
      \draw (1.5,2) rectangle (3,0);
      \draw[->] (-0.5,0) -- (T);
    \end{tikzpicture}
    \caption{Solution de l'instance de l'exemple~\ref{exemple_NE}}
    \label{figure_exemple_NE}
  \end{figure}

Dans cette solution, la première activité doit finir au temps $t=1.5$
pour que la seconde activité puisse finir avant sa date échue
$\LE[2]=3$. En effet, l'activité $2$ doit commencer avant sa date de
début au plus tard, ici $\LS[2]= \LE[2] - W_i/f_i(\bmax) = 3 - 3/2=
1.5$, et l'activité $1$ ne peut finir avant sa date de fin au plus
tôt, $\EE[1]= \ES[1] + W_i / f_i(\bmax) = 0 + 18/12 = 1.5$.
\end{ex}

De ce fait, l'espace des solutions peut être réduit par l'utilisation
du modèle à temps discret et ceci peut conduire à des infaisabilité ou
à des résultats sous-optimaux.

Une solution pour palier à ce problème est de mettre à l'échelle les
instances, i.e. multiplier les données par un certain coefficient $\alpha$
afin de s'assurer de l'existence d'une solution optimale entière,
avant de les résoudre. Cependant, l'utilisation d'un coefficient trop
grand peut conduire à une augmentation de la taille des modèles trop
importante pour permettre leur résolution. 


Le théorème suivant présente une des propriétés majeures du \CECSP. En
effet, il stipule que quelque soit l'instance considérée, il existe
toujours une solution de cette instance où les fonctions $b_i(t)$ sont
constantes par morceaux. 

\begin{theo}[\cite{Nattaf_CPDP}]
\label{theo_LPM_CECSP}
Soit $\Pi$ une instance réalisable du \CECSP~telle que:
\begin{itemize}
\item $\forall i \in \A,\ f_i$ est croissante, continue, concave et
  affine par morceaux. 
\item $\forall i \in \A,\ f_i(0)=0.$
\end{itemize}
Une solution ayant la propriété que, $\forall i \in \A,\ b_i(t)$ soit
constante par morceaux existe.
\end{theo}

Afin de prouver le théorème~\ref{theo_LPM_CECSP}, nous commençons par
prouver que, pour chaque intervalle $[t_1,t_2]$ tel que $b_i(t), t \in
[t_1,t_2]$, n'est pas constante, il existe une constante $b_{iq}$ pour
laquelle exécuter $i$ à $b_{iq}$ dans $[t_1,t_2]$ apporte au moins
autant d'énergie tout en consommant la même quantité de
ressource. C'est ce qu'affirme le lemme suivant:

\begin{lemma}
\label{lemmaEn}
Soit $b_{iq}= \frac{\int_{t_1}^{t_2}b_i(t)dt}{t_2-t_1}$. Alors, nous
avons:
\begin{align}
  &\int_{t_1}^{t_2}b_{iq}dt = \int_{t_1}^{t_2} b_i(t) dt \label{eq_LPM_res} \\
  & \int_{t_1}^{t_2}f_i(b_{iq})dt \ge \int_{t_1}^{t_2} f_i(b_i(t)) dt 
    \label{eq_LPM_nrj}
\end{align}
\end{lemma}

\begin{proof}
  L'équation~\eqref{eq_LPM_res} est trivialement vérifiée en remplaçant
  $b_{iq}$ par sa valeur. En effet, nous avons:
  \begin{align*}
    \int_{t_1}^{t_2}b_{iq}dt =&
    \int_{t_1}^{t_2}\left(\frac{\int_{t_1}^{t_2}b_i(t)dt}{t_2-t_1}\right)dt\\
    =& (t_2-t_1)\left(\frac{\int_{t_1}^{t_2}b_i(t)dt}{t_2-t_1}\right)\\
    =&\int_{t_1}^{t_2}b_i(t)dt
  \end{align*}

  Pour prouver que l'équation~\eqref{eq_LPM_nrj} est satisfaite,
  nous utilisons le théorème suivant:  
  \begin{theo}[\cite{Jensen}]
    Soit $\alpha(t)$ et $g(t)$ deux fonctions intégrables sur
    $[t_1,t_2] \subseteq \mathbb{R}$ telles que $\alpha(t) \ge 0,\
    \forall t \in [t_1,t_2]$. Alors, nous avons la propriété suivante: 
    \begin{equation}
      \phi\left( \frac{\int_{t_1}^{t_2} \alpha(t)g(t)dt }
        {\int_{t_1}^{t_2} \alpha(t)dt} \right) \ge
      \frac{\int_{t_1}^{t_2} \alpha(t)\phi(g(t))dt }
      {\int_{t_1}^{t_2} \alpha(t)dt}
    \end{equation}
    où $\phi$ est une fonction continue, concave sur $[\min_{t \in
      [t_1,t_2]} g(t),\max_{t \in [t_1,t_2]} g(t)]$. 
  \end{theo}
  Si nous remplaçons $\phi(t)$ par $f_i(t),\ g(t)$ par $b_i(t)$ et
  $\alpha(t)$ par la fonction constante égale à $1$, nous obtenons:
  \begin{align*}
    & f_i\left( \frac{\int_{t_1}^{t_2}b_i(t)dt }
      {t_2-t_1} \right) \ge
      \frac{\int_{t_1}^{t_2}f_i(b_i(t))dt }
      {t_2-t_1} \\
    & \Leftrightarrow (t_2-t_1)f_i\left( 
      b_{iq} \right) \ge
      \int_{t_1}^{t_2}f_i(b_i(t))dt\\
    & \Leftrightarrow \int_{t_1}^{t_2}f_i\left( 
      b_{iq} \right)dt \ge
      \int_{t_1}^{t_2}f_i(b_i(t))dt
  \end{align*}
Et donc, l'équation~\eqref{eq_LPM_nrj} est satisfaite.  
\end{proof}  

Nous pouvons maintenant prouver le théorème~\ref{theo_LPM_CECSP}. Pour
cela, nous allons montrer que, soit $S$ une solution d'une instance
$\Pi$, alors nous pouvons transformer $S$ en une solution $S'$ ayant
la propriété que chaque fonction $b'_i(t)$ est constante par morceaux.

\begin{proof}[Preuve du théorème~\ref{theo_LPM_CECSP}]  
  Soit $S$ une solution réalisable de $\Pi$ et soit
  $(t_q)_{q=1..Q}$ la suite des des différentes dates de
  début et de fin d'activité triées par ordre croissant. Clairement,
  nous avons $Q\le 2n$. 

  Par souci de clarté, nous définissons la fonction intermédiaire
  $\tilde{b}_i(t),\ \forall i \in \A$, de la façon suivante:  

    \[\tilde{b}_i(t) =\left\{
        \begin{array}{lll}
          b_{i0} & & \text{si $t \in [t_0,t_1]$}\\
          \multicolumn{2}{c}{\vdots} &   \\
          b_{i(Q-1)} & & \text{si $t \in [t_{Q-1},t_Q]$}
        \end{array}
      \right.\]
    avec $b_{iq}=\frac{\int_{t_q}^{t_{q+1}} b_i(t) dt}{t_{q+1}-t_q}$.

    La solution $S'$ est alors construite de la manière suivante: 
    \begin{itemize}
    \item $st'_i=st_i$ 
    \item $b'_i(t)= \left\{ 
        \begin{array}{lll}
          \tilde{b}_i(t) &\quad& \text{si $t \in [st_i,et'_i]$}\\
          0 &\quad& \text{sinon}\\
        \end{array}
      \right.$
  \item $et'_i=\min(\tau | \int_{st_i}^{\tau} f_i(\tilde{b}_i(t))dt=W_i)$
  \end{itemize}

  Il est facile de voir que $S'$ satisfait les contrainte de fenêtre de
  temps~\eqref{tw_CECSP}, puisque, par le Lemme~\ref{lemmaEn},
  $et'_i\le et_i$. De plus, $S'$ vérifie la contrainte 
  d'énergie~\eqref{nrj_CECSP} puisqu'elle est définie de cette
  façon. Enfin, $S'$ vérifie aussi la contrainte de capacité de la
  ressource~\eqref{res_CECSP}. En effet, comme $S$ est une solution
  réalisable, nous avons $\forall q \in \{1,\dots,Q\}$ et $\forall t
  \in [t_q,t_{q+1}]$:  
  $\sum_{i\in \A}b_i(t) \le R \Rightarrow  
  \sum_{i\in \A} \int_{t_q}^{t_{q+1}} b_i(t)dt \le B(t_{q+1}-t_q)$.
 
  Donc, 
  \begin{align*}
    \sum_{i\in \A}b'_i(t) &\le 
                            \sum_{i\in \A} \tilde{b}_i(t)\\
                          &= 
                            \sum_{i\in \A} b_{iq}\\
                          &=
                            \sum_{i\in \A} \frac{\int_{t_q}^{t_{q+1}} b_i(t)dt}{t_{q+1}-t_q} \\
                          &\le B
  \end{align*}
  Nous pouvons montrer que $S'$ vérifie les contraintes de
  consommation minimale et maximale de la ressource d'une façon
  similaire. 
\end{proof}

Une remarque intéressante peut être faite à partir de la preuve du
théorème précédent. En effet, la nouvelle solution $S'$ possède la
propriété suivante: l'ensemble des points $t \in \H$ coïncidant avec
une variation de la consommation de ressource d'une activité $i$,
i.e. $\{t \in \H \ |\ \forall \epsilon>0,\ b_i(t) \neq b_i(t+\epsilon)\}$,
est contenue dans l'ensemble formé de toutes les dates de début et de
fin des activités. C'est ce qu'affirme le corollaire suivant: 

\begin{coro}
$\{t \in \H \ |\ \forall \epsilon>0,\ b_i(t) \neq b_i(t+\epsilon)\}
\subseteq \{st_i,et_i\ |\ i \in \A \}$.
\end{coro}

De plus, nous pouvons en déduire que le \CECSP~à date de début et de
fin fixées peut être résolu en temps polynomial. 

\begin{prop}
Soit $\Pi$ une instance du \CECSP~avec des dates de début, $st_i$, et
des dates de fin, $et_i$, fixées. On peut vérifier que $\Pi$ est réalisable en temps 
polynomial en la taille de l'instance. 
\end{prop}

En effet, dans ce cas là, il suffit de décider pour chaque intervalle
composé de deux dates de début/fin consécutifs, i.e. de la forme
$[st_i,st_j],\ [st_i,et_j],\ [et_i,et_j]$ ou $[et_i,st_j]$, la
quantité de ressource consommée par chaque activité à l'intérieur de
cet intervalle. Ce problème peut facilement être modélisé par un
programme linéaire.

Soit $(t_q)_{q=1..Q}$ la suite définie dans la preuve du
théorème~\ref{theo_LPM_CECSP} et $b_{iq}$ (respectivement $w_{iq}$),
$\forall (i,q) \in \A\times\{1,\dots,Q-1\}$, la
quantité de ressource consommée par (resp. la quantité d'énergie
apportée à) l'activité $i$ dans l'intervalle
$[t_q,t_{q+1}]$. Rappelons que $Q \le 2n$. Le programme linéaire
s'écrit alors de la manière suivante:
{\small
\begin{align}
&\sum_{\i \in A} b_{iq} \le R & \forall q\in
\{1..Q-1\} \label{poly2}\\ & b_{iq} \le \bmax & \forall i \in \A,\
\forall q \in \{1..Q-1\} |\ t_q \in [st_i,et_i[\label{poly3}\\ &
b_{iq} \ge \bmin& \forall i \in \A,\ \forall q \in \{1..Q-1\} |\ t_q
\in [st_i,et_i[\label{poly4}\\ & b_{iq}=0 & \forall i \in \A,\ \forall
q \in \{1..Q-1\} |\ t_q \not\in [st_i,et_i[\label{poly5}\\ &
\sum_{q=1}^{Q-1} w_{iq}(t_{q+1}-t_q) = W_i & \forall i \in
\A\label{poly6}\\ & w_{iq} \le a_{ip}b_{iq} + c_{ip} & \forall i \in
\A,\ \forall p \in \P_i,\ \forall q \in \{1..Q-1\}\label{poly7}\\ &
w_{iq} \le Mb_{iq} & \forall i \in \A,\ \forall q \in
\{1..Q-1\}\label{poly8}
\end{align} }
pour $M$ une constante suffisamment grande et $\P_i=\{1,\dots,P_i\}$
le nombre d'intervalles de définition de la fonction $f_i$. La
contrainte~\eqref{poly2} modélise la contrainte de capacité de la
ressource. Les contraintes~\eqref{poly3} et~\eqref{poly4} assurent les
contraintes de consommation minimale et maximale de la ressource
tandis que la contrainte~\eqref{poly5} fixe la consommation de la
ressource à $0$ si l'activité n'est pas en cours. La
contrainte~\eqref{poly6} stipule que chaque activité doit recevoir la
quantité d'énergie requise. Enfin, les contraintes~\eqref{poly7} et
\eqref{poly8} assure la conversion ressource/énergie. De plus la
contrainte~\eqref{poly8} fixe $w_{iq}$ à $0$ si $b_{iq}=0$,
i.e. modélise $f_i(0)=0$.

On peut remarquer que si $\forall i \in \A,\ \bmin=0$, alors le
problème devient polynomial. En effet, il suffit de prendre 
$(t_q)_{q=1..Q}$ la suite des différentes dates de début (resp. fin)
au plus tôt (resp. tard). Alors, le programme linéaire précédent nous
donne une solution réalisable. 

\begin{theo}
Le \CECSP~préemptif ($\forall i \in \A,\ \bmin=0$) peut être résolu en
temps polynomial.
\end{theo}

De ce fait, dans la suite, nous considérerons que $\exists i \in \A$
tel que $\bmin\neq 0$.


\section{Conclusion}

Dans ce chapitre, nous avons d'abord introduit les principales
caractéristiques des problèmes d'ordonnacment avant de nous intéresser
en particulier aux problèmes cumulatifs. Nous avons ensuite présenté
deux des principaux problèmes étudiés en ordonnancment sous
contrainte de ressource. Les limitations de ces problèmes ont ensuite
été démontré et une nouvelle modélisation de la consommation de
ressource nous a permis de définir un nouveau problème: le
\CECSP. 

Dans un premier temps, nous avons comparé ce problème avec les
problème existant dans la littérature, puis nous avons présenté un
ensemble de propriété qui va nous permettre, dans la suite de ce
manuscrit, de décrire des techniques pour sa résolution. La plupart de
ces techniques sont adaptées de techniques existantes et peuvent être
classées en deux catégories: 
\begin{itemize}
\item les techniques adaptées du \CUSP~et issues de la programmation
par contraintes. Ces techniques seront détaillées dans la
partie~\ref{part:PPC}.
\item les techniques adaptées du \RCPSP~et issues de la programmation
linéaire. Ces techniques seront détaillées dans la
partie~\ref{part:PLNE}.
\end{itemize}
\clearemptydoublepage%
\part{Programmation par contraintes}


\chapter[L'ordonnancement cumulatif en PPC]{L'ordonnancement cumulatif en programmation par contraintes}

 \section{La programmation par contraintes}
\label{sec:PPC}
Cette section s'intéresse à la présentation des concepts de base de la
programmation par contraintes (PPC). La programmation par contraintes
vise à résoudre des problèmes de satisfaction de contraintes (CSP)
mais aussi des problèmes d'optimisation (ce dernier cas ne sera pas
traité dans ce manuscrit). Pour cela, un problème est
modélisé à l'aide d'un réseau de contraintes et la recherche d'une
solution tend à trouver une affectation des variables satisfaisant
toutes les contraintes de ce réseau. Une présentation formelle des
problèmes de satisfaction de contraintes ainsi qu'un aperçu de
quelques méthodes permettant leur résolution sont présentés dans les
sous-sections suivantes. 

\subsection{Problème de satisfaction de contraintes}

Une instance d'un {\it problème de satisfaction de contraintes}, ou
CSP\footnote{en anglais, Constraint Satisfaction Problem.} est
la donnée d'un triplet $(\X,\D,\C)$ où: 
\begin{itemize}
\item $\X=\{x_1,x_2,\dots,x_n\}$ est l'ensemble des variables du problème;
\item $\D=\{D_1,D_2,\dots,D_n\}$ est l'ensemble des domaines de ces
  variables, i.e. $x_i \in D_i,\ i=1,\dots,n$;
\item $\C=\{c_1,c_2,\dots,c_m\}$ est l'ensemble des contraintes du
  problème où chaque $c_j$ définit un sous-ensemble du produit
  cartésien des domaines des variables sur lesquelles elle porte:

  $c_j(x_{j1},x_{j2},\dots,x_{jk}) \subseteq D_{j1} \times D_{j2}
  \times \dots \times D_{jk}$
\end{itemize}

La notion de domaine désigne l'ensemble des valeurs que peut prendre
une variable. La nature de ces domaines peut potentiellement être très
différente. Par exemple:
\begin{itemize}
\item un ou plusieurs intervalle d'entiers;
\item un ou plusieurs intervalle de réels;
\item un ensemble d'entiers non contigus: il est possible d'utiliser
  un ensemble d'entiers quelconque, e.g. $D=\{4,9,26\}$;
\item un ensemble de valeurs symboliques: on peut vouloir représenter
  des couleurs, ou encore des jours de la semaine... 
\end{itemize}
~

Une fois les variables définies, nous pouvons ajouter des contraintes
les liant entre elles. Formellement, une contrainte peut être définie
comme une relation portant sur un ensemble de variables.  Ici aussi,
plusieurs types de contraintes existent. Nous détaillons trois d'entre
elles.

Les premières contraintes présentées sont les contraintes en {\it
  extension}. Pour ces contraintes, étant donné un sous-ensemble de
variables, on définit explicitement la liste des tuples
autorisés. Suivant les cas, ces contraintes peuvent aussi être définies
comme une liste de tuples interdits. Les deuxièmes contraintes
détaillées sont les contraintes en {\it intention}. Dans ce cas,
chaque contrainte est décrite sous la forme d'une expression arithmétique
définissant  une relation entre les variables. Enfin, les dernières
contraintes décrites sont les contraintes globales. Ces contraintes
sont des relations prédéfinies, ayant une signification précise. Des
exemples de telles contraintes sont décrits dans
l'exemple~\ref{ex:contrainte}. Notons que si les variables sont à
valeurs dans $\mathbb{R}$ alors une définition en extension de la
contraintes peut comprendre un nombre infini de tuples.

\begin{ex}
\label{ex:contrainte}
  Soient trois variables $x_0,\ x_1$ et $x_2$ de domaines respectifs
  $D_0=[0,2],\ 
  D_1=[0,2]$ et $D_2=[1,2]$. 
  Nous considérons les contraintes $c_0,\ c_1$ et $c_2$ suivantes:
  \begin{itemize}
  \item $c_0$ porte sur les variables $x_0$ et $x_1$ et est décrite en
    intention: $x_0 \le x_1$; 
  \item $c_1$ porte sur $x_1$ et $x_2$ et est décrite en extension:
    $\{(0,1) , (0,2), (1,2), (2,2)\}$; 
  \item $c_2$ est une contrainte  globale portant sur les variable
    $x_0,x_1$ et $x_2$: allDifferent$(x_0,x_1,x_2)$. Cette contrainte
    stipule que les valeurs affectées à chaque variable sont
    différentes les unes des autres. 
  \end{itemize}
  Les domaines des variables étant continus, un nombre infini de
tuples aurait été nécessaire pour écrire $c_0$ en extension.
\end{ex}

Pour trouver une solution à un CSP, il faut instancier toutes les
variables du problème de telle sorte que toutes les contraintes
soient satisfaites. Une variable est dite {\it instanciée} quand on
lui assigne une valeur de son domaine. Pour trouver une telle
instanciation, un certain nombre de techniques ont été mises en
place. Ces techniques reposent principalement sur deux éléments
centraux: le filtrage des domaines et l'exploration de l'espace de
recherche. Ces deux concepts sont donc décrits dans les sous-sections 
suivantes. 



\subsection{Exploration de l'espace de recherche}
\label{sec:PPC_rech}

{\'E}tant donné un CSP, différentes techniques d'exploration de
l'espace de recherche peuvent être employées pour trouver des
solutions. Cette exploration se fait en général par {\it séparation et
  évaluation}, i.e. on sépare le problème difficile à résoudre en deux
sous-problèmes, plus petits, jusqu'à obtenir un problème que l'on sera
capable de résoudre en temps raisonnable.  

Une représentation usuelle du processus de recherche d'une solution
est l'{\it arbre de recherche}. La racine de cet arbre représente le
problème que l'on cherche à résoudre et les sommets sont des problèmes
réduits, obtenus en décomposant le domaine d'une des variables du
problème père. Les feuilles de cet arbre correspondent donc à des
instanciations de toutes les variables du problème.

Une première approche consiste alors à générer tous les
tuples de valeurs possibles et de tester s'ils sont solution du
problème, i.e. si cette instanciation satisfait bien toutes les
contraintes du problème. Cela revient à considérer toutes les
affectations possibles des variables et ce nombre, qui peut ne pas
être fini dans le cas des variables continues, est égal au produit
cartésien des cardinaux des domaines des variables impliquées dans le
CSP pour des variables dans $\mathbb{Z}$.

Classiquement, une méthode de séparation, ou de branchement, consiste
à fixer une variable à une valeur pour le premier sous-problème et de
retirer cette valeur du domaine de la variable dans le second
sous-problème. On va alors, à l'aide d'un parcours en profondeur, créer
un premier tuple, i.e. une solution candidate, qui pourra ainsi être
évalué. Si le tuple satisfait toutes les contraintes du problème,
alors c'est une solution et l'algorithme peut s'arrêter. Dans le cas 
contraire, on évaluera le second sous-problème créé lors de la
dernière séparation. Un tel parcours est décrit dans
l'exemple~\ref{ex:branchement}.

Dans le cas de CSP comprenant des variables continues, l'énumération
de tous les domaines des variables n'est pas possible. Une première
approche serait de considérer qu'une variable est instanciée quand son
domaine est réduit à un intervalle de taille suffisamment petite. Mais
même avec cette restriction supplémentaire, il est très coûteux
d'énumérer chacun de ces intervalles. L'idée est alors la suivante: au
lieu, à chaque étape de l'algorithme de séparation, d'instancier une
variable à une valeur de son domaine, nous séparons le domaine de
cette variable en deux (ou plusieurs) sous-domaines.

\begin{ex}
  \label{ex:branchement}
  Soient $3$ variables, $x_0,\ x_1$ et $x_2$, de domaines
  respectifs: $\{1,2\},\ \{1,2\},\ \{1,2,3\}$. Nous considérons les
  contraintes suivantes: $c_0: x_0<x_1$ et $c_1 : x_1 <x_2$.   
  \begin{figure}[!htb]
    \centering
    \begin{tikzpicture}
      [level 1/.style={sibling distance=7cm,level distance=1.8cm},
      level 2/.style={sibling distance=3.5cm,level distance=1.8cm},
      level 3/.style={sibling distance=2.5cm,level distance=1.8cm},
      level 4/.style={sibling distance=1.5cm,level distance=1.8cm},
      every node/.style={color=black},%
      dot/.style={circle,fill=black,minimum size=4pt,inner sep=0pt,%
        outer sep=-1pt},
      cross/.style={path picture={ 
          \draw
          (path picture bounding box.south east) -- (path picture bounding box.north west) (path picture bounding box.south west) -- (path picture bounding box.north east);
        }}]

      \node[tree] {} %racine
      child{ node [tree]{}  %l1
        child{ node[tree] {}%l2
          child{ node[leaf,cross] {} %l3
         edge from parent node[sloped,above]{$x_2=1$}}
          child{ node[tree] {} %l3
            child{ node[leaf,cross] {}
              edge from parent node[sloped,above]{$x_2=2$}}
            child{ node[leaf,cross] {}
              edge from parent node[sloped,above]{$x_2=3$}}
            edge from parent node[sloped,above]{$x_2\neq 1$}
          }
          edge from parent node[sloped,above]{$x_1=1$} }
        child{ node[tree] {} %l2
          child{ node[leaf,cross] {}%l3
            edge from parent node[sloped,above]{$x_2=1$}
          }
          child{ node[tree] {}%l3
            child{ node[leaf,cross] {}
              edge from parent node[sloped,above]{$x_2=2$}
            }
            child{ node[sol] {$\checkmark$}
              edge from parent node[sloped,above]{$x_2=3$}
            }          
            edge from parent node[sloped,above]{$x_2 \neq 1$}}
          edge from parent node[sloped,above]{$x_1=2$} 
        }
        % etc.
        edge from parent node[sloped,above] {$x_0=1$}
      }
      child{ node [tree]{}  %l1
        child{ node[tree] {}%l2
          child{ node[leaf,cross] {} %l3
         edge from parent node[sloped,above]{$x_2=1$}}
          child{ node[tree] {} %l3
            child{ node[leaf,cross] {}
              edge from parent node[sloped,above]{$x_2=2$}}
            child{ node[leaf,cross] {}
              edge from parent node[sloped,above]{$x_2=3$}}
            edge from parent node[sloped,above]{$x_2\neq 1$}
          }
          edge from parent node[sloped,above]{$x_1=1$} }
        child{ node[tree] {} %l2
          child{ node[leaf,cross] {}%l3
            edge from parent node[sloped,above]{$x_2=1$}
          }
          child{ node[tree] {}%l3
            child{ node[leaf,cross] {}
              edge from parent node[sloped,above]{$x_2=2$}
            }
            child{ node[leaf,cross] {}
              edge from parent node[sloped,above]{$x_2=3$}
            }          
            edge from parent node[sloped,above]{$x_2 \neq 1$}}
          edge from parent node[sloped,above]{$x_1=2$} 
        }
        edge from parent node[sloped,above]{$x_0=2$}
        % etc.
      };
    \end{tikzpicture}
    \caption{Exemple de parcours d'un arbre de recherche.}
    \label{fig:ex_tree}
  \end{figure}

La figure~\ref{fig:ex_tree} montre un parcours complet de l'espace de
recherche. Les noeuds internes de l'arbre (en gris) sont les noeuds
pour lesquels toutes les variables ne sont pas instanciées. Les
feuilles de l'arbre correspondent bien à des instanciations complètes
des variables. Parmi celles-ci, celles marquées d'une croix sont celles
qui ne satisfont pas la contrainte, tandis que celle représentée par
un noeud carré correspond à une solution réalisable du problème.
\end{ex}

Ces combinaisons peuvent être générées de différentes manières et donc
testées dans des ordres différents. Cet ordre peut avoir une influence
déterminante dans le processus de résolution d'un problème. Pour
définir cet ordre, on utilise des {\it heuristiques} de choix de
variable et de valeur. Une {\it heuristique de choix de
  variable} détermine la variable que l'on va instancier
prioritairement. Une {\it heuristique de choix de
  valeur} détermine la ou les valeurs à affecter en priorité à
cette variable. 

Nous donnons ci-après des exemples de telles heuristiques. 
\begin{description}
\item[Heuristiques de choix de variable]~

  \begin{itemize}
  \item instanciation des variables dans l'ordre lexicographique,
    c'est l'heuristique utilisée dans la figure~\ref{fig:ex_tree};
  \item instanciation des variables suivant un ordre aléatoire;
  \item instanciation de la variable de plus petit domaine~\cite{min_dom}...
  \end{itemize}
\item[Heuristiques de choix de valeur] ~

  \begin{itemize}
  \item sélectionner la valeur minimale (resp. maximale) du domaine de
    la variable courante, c'est l'heuristique utilisée dans la
    figure~\ref{fig:ex_tree}; 
  \item sélectionner une valeur aléatoire dans le domaine de la
    variable courante;
  \item dans le cas de variables continues, on peut choisir de séparer
    l'intervalle en deux ou plus de morceaux et le séparer au milieu,
    au tiers...
  \end{itemize}
\end{description}
Dans la suite du manuscrit, nous présenterons d'autres heuristiques
que nous appliquerons dans notre processus de recherche. 

Même si plusieurs heuristiques efficaces existent, on ne peut être sûr
de ne pas devoir explorer tout l'espace de recherche pour trouver une
solution. Comme il n'est pas concevable, en pratique, de devoir
explorer un tel espace, des techniques permettant de réduire l'arbre
de recherche ont donc été mises en place. Les techniques les plus
célèbres permettant une telle réduction sont les algorithmes de
filtrage. Ces algorithmes sont décrits dans la sous-section suivante.

\subsection{Détection d'incohérence et Filtrage}
\label{sec:PPC_propag}

Afin de réduire l'espace de recherche en programmation par contraintes,
il est indispensable de mettre en place des techniques permettant de
vérifier la validité d'une solution afin de ne pas considérer un tuple
comme étant cohérent s'il ne l'est pas. Ces {\it algorithmes de
détection d'incohérences}, aussi appelés checkers, permettent de
refuser une instanciation des variables si cette dernière n'est pas
cohérente, i.e. si elle viole un des contraintes. Un algorithme de
vérification est donc associé à chaque contrainte.

De plus, ces algorithmes peuvent aussi être étendus de manière à
détecter une incohérence en cours de recherche si aucune solution
n'est contenue dans le sous-arbre courant.
\begin{ex}
\label{ex:ex_checker}  Soit l'instance définie dans
l'exemple~\ref{ex:branchement}.    
  \begin{figure}[!htb]

    \centering
    \begin{tikzpicture}
      [level 1/.style={sibling distance=7cm,level distance=1.8cm},
      level 2/.style={sibling distance=3.5cm,level distance=1.8cm},
      level 3/.style={sibling distance=2.5cm,level distance=1.8cm},
      level 4/.style={sibling distance=1.5cm,level distance=1.8cm},
      every node/.style={color=black},%
      dot/.style={circle,fill=black,minimum size=4pt,inner sep=0pt,%
        outer sep=-1pt},
      cross/.style={path picture={ 
          \draw
          (path picture bounding box.south east) -- (path picture bounding box.north west) (path picture bounding box.south west) -- (path picture bounding box.north east);
        }}]

      \node[tree] {} %racine
      child{ node [tree]{}  %l1
        child{ node[tree,fill=black] {}%l2
          edge from parent node[sloped,above]{$x_1=1$}}
        child{ node[tree] {} %l2
          child{ node[leaf,cross] {}%l3
            edge from parent node[sloped,above]{$x_2=1$}
          }
          child{ node[tree] {}%l3
            child{ node[leaf,cross] {}
              edge from parent node[sloped,above]{$x_2=2$}
            }
            child{ node[sol] {$\checkmark$}
              edge from parent node[sloped,above]{$x_2=3$}
            }          
            edge from parent node[sloped,above]{$x_2 \neq 1$}}
          edge from parent node[sloped,above]{$x_1=2$} 
        }
        % etc.
        edge from parent node[sloped,above] {$x_0=1$}
      }
      child{ node [tree]{}  %l1
        child{ node[tree,fill=black] {}%l2
          edge from parent node[sloped,above]{$x_1=1$} }
        child{ node[tree,fill=black] {} %l2
          edge from parent node[sloped,above]{$x_1=2$} 
        }
        edge from parent node[sloped,above]{$x_0=2$}
        % etc.
      };
    \end{tikzpicture}
    \caption{Exemple de parcours d'un arbre de recherche avec
      détection d'incohérences.}
    \label{fig:ex_tree_check}
  \end{figure}

La figure~\ref{fig:ex_tree_check} montre un parcours de l'espace de
recherche si l'algorithme utilisé est muni d'un algorithme de
détection d'incohérence. Les noeuds noirs correspondent à des noeuds
pour lesquels aucune solution ne peut exister dans les sous-arbres
ayant un de ces noeuds pour racine. En effet, l'instanciation
partielle des variables faites à ce niveau est déjà incohérente avec
les contraintes $c_0$ ($x_0 < x_1$) et $c_1$ ($x_1 < x_2$).
\end{ex}


Une deuxième technique permettant la réduction de l'espace de
recherche est l'utilisation d'algortihme de {\it filtrage}. Un
algorithme de filtrage est aussi défini pour chaque contrainte et cet
algorithme consiste à détecter et à retirer des domaines des
variables, des valeurs qui ne sont pas cohérentes avec la contrainte.

 La mise en place de techniques de filtrage performantes,
i.e. qui retirent un maximum de valeurs du domaine des variables,
jouent un rôle important dans la résolution de problème en
programmation par contraintes. Un exemple simple de filtrage est
présenté dans l'exemple~\ref{ex:ex_filtrage}.

\begin{ex}
\label{ex:ex_filtrage}  Soit l'instance définie dans
l'exemple~\ref{ex:branchement}.       
  \begin{figure}[!htb]

    \centering
    \begin{tikzpicture}
      [level 1/.style={sibling distance=7cm,level distance=1.8cm},
      level 2/.style={sibling distance=3.5cm,level distance=1.8cm},
      level 3/.style={sibling distance=2.5cm,level distance=1.8cm},
      level 4/.style={sibling distance=1.5cm,level distance=1.8cm},
      every node/.style={color=black},%
      dot/.style={circle,fill=black,minimum size=4pt,inner sep=0pt,%
        outer sep=-1pt},
      cross/.style={path picture={ 
          \draw
          (path picture bounding box.south east) -- (path picture bounding box.north west) (path picture bounding box.south west) -- (path picture bounding box.north east);
        }}]

      \node[tree] {} %racine
      child{ node [tree]{}  %l1
        child{ node[] {}%l2
          edge from parent[white]
          node[rectangle, draw,above left=1cm,text width=2.6cm,align=center] (exp1) {la valeur $1$ est supprimée des
            domaines de $x_1$ et $x_2$} {};
          \draw[->] (exp1.east) -- (\tikzparentnode.west);}
        child{ node[tree] {} %l2
          child{ node[] {}%l3
            edge from parent[white] {}
          }
          child{ node[tree] {}%l3
            child{ node[] {}
              edge from parent[white] {} 
            }
            child{ node[sol] {$\checkmark$}
              edge from parent node[sloped,above]{$x_2=3$}
            }          
            edge from parent node[sloped,above]{$x_2 \neq 1$}}
          edge from parent node[sloped,above]{$x_1=2$} 
        }
        % etc.
        edge from parent node[sloped,above] {$x_0=1$} 
      }
      child{ node [tree]{}  %l1
        child{ node[] {}
          edge from parent[white]
          node[rectangle, draw,above right=0.75cm and 2.55cm,text
          width=2.6cm,align=center] (exp2) {les valeurs $1$ et $2$ sont supprimées du
            domaine de  $x_1$} {};
          \draw[->] (exp2.west) -- (\tikzparentnode.east);}
        child{ node[] {}
          edge from parent[white]}
        edge from parent node[sloped,above]{$x_0=2$}
        % etc.
      };
    \end{tikzpicture}
    \caption{Exemple de parcours d'un arbre de recherche avec fitrage
      des domaines.}
    \label{fig:ex_tree_filtrage}
  \end{figure}

La figure~\ref{fig:ex_tree_filtrage} montre un parcours de l'espace de
recherche si l'algorithme utilisé est muni d'un algorithme de filtrage
des domaines. Les contraintes sont considérées une par une.  A
la racine, on ne peut rien déduire, mais une fois que $x_0$ est
instanciée à $1$, alors nous pouvons retirer cette valeur du domaine
des autres variables car, pour satisfaire les contraintes $c_0$ et
$c_1$, les variables $x_1$ et $x_2$ doivent être strictement
supérieures à $1$. Puis, on instancie $x_1$ à $2$ et on supprime cette
valeur du domaine de $x_2$. La seule valeur possible pour $x_2$ est
alors $3$ et on obtient une solution.

L'autre cas correspond à l'instanciation de $x_0$ à $2$. Dans ce
cas-là, les valeurs $1$ et $2$ sont rétirées du domaines de $x_1$ qui
devient vide. Donc il n'existe pas de solution avec $x_0=2$.
\end{ex}

Dans l'exemple~\ref{ex:ex_filtrage}, nous pouvons remarquer que si on
avait considéré simultanément les $2$ contraintes, on
aurait pu déduire la solution au noeud racine. 

Quand les contraintes sont binaires, i.e. elles ne concernent que deux
variables à la fois, on peut représenter schématiquement le CSP par un
graphe. Dans ce graphe, chaque noeud correspond à une variable et
chaque arc à une contrainte. Si on vérifie la cohérence des
contraintes une par une, on parle alors de cohérence d'arcs. Cette
notion peut être généralisée à la considération simultanée de
plusieurs contraintes mais aussi à des contraintes contenant plus de
deux variables.

Enfin, pour certains problèmes, il n'est pas possible de retirer
toutes les valeurs incohérentes du domaines des variables en temps
raisonnables. En effet, il peut arriver que les problèmes sous-jacents
des contraintes que l'on est amené à résoudre soient NP-complets. Dans
ce cas-là, assurer de retirer les valeurs incohérentes signifie que
les valeurs restantes sont cohérentes, donc qu'il existe une
solution, ce qui ne peut être fait en temps raisonnable puisque le
problème est NP-complet.  Dans ce cas, il devient crucial de mettre
en place des algorithmes qui s'exécutent en temps raisonnable mais qui
n'assure pas un filtrage complet du domaine des variables. De tels
algorithmes seront présentés dans la sous-section~\ref{sec:cumu_propag}
dans le but de résoudre le \CUSP. La section suivante est quant à
elle dédiée à la modélisation des problèmes d'ordonnancement en PPC.
 \section{L'ordonnancement cumulative}
\label{sec:cumu}
\subsection{L'ordonnancement en programmation par contrainte}
\label{sec:cumu_ordo}

\subsection{La contrainte cumulative}
\label{sec:cumu_cume}

\subsection{Les filtrages de la contrainte cumulative}
\label{sec:cumu_propag}




\chapter{Algorithmes issus de la programmation par contraintes pour le
\CECSP}
\label{sec:PPC_CECSP}
Dans ce chapitre, nous décrivons des algorithmes et modèles issus de
la programmation par contrainte. Les deux premières sections sont
dédiées à la présentation d'algorithmes de filtrage pour le problème
de décision du \CECSP. En effet, comme dans le cas du \CUSP, la
NP-complétude de ce problème implique qu'un algorithme assurant la
cohérence des bornes de chaque variable - assurant l'existence d'une
solution où les valeurs de chaque variables sont comprises dans ces
bornes - ne peut s'exécuter en temps polynomial. Par conséquent,
plusieurs relaxations sont proposées afin de supprimer les valeurs
possibles de début et fin de tâches incohérentes en temps polynomial.

Dans un premier temps, nous montrons que une partie des raisonnements
pour le \CUSP~s'adapte directement au cas du \CECSP. En effet, dans le
\CECSP, une activité peut être vue comme deux sous-activités: 
\begin{itemize}
\item l'une correspondant à une activité rectangulaire, appelée {\it
partie fixe}, dont la date de début et de fin doivent être
déterminées et consommant une quantité $\bmin$ de la ressource;
\item la seconde, appelée {\it partie malléable}, correspondant à une
activité préemptive de même date de début que la partie minimale et
dont chaque partie consomme une quantité de ressource comprise entre
$0$ et $\bmax-\bmin$.
\end{itemize}
La présence d'activités rectangulaires permet alors l'adaptation direct
de certains raisonnement existant pour le \CUSP. Ici, nous
présenterons les adaptations du raisonnement Time-Table, disjonctif et
Time-Table disjonctif. 

Dans un second temps, nous présenterons un nouveau raisonnement
étendu, couplant Time-Table et problème de flots. 

Enfin, une adaptation complète du raisonnement énergétique est décrit
dans le paragraphe~\ref{sec:ER_CECSP}.

Le dernier paragraphe sera quant à lui consacré à la présentation d'un
modèle de programmation par contraintes permettant de résoudre le
\CECSP~ discret, i.e. où les variables ne peuvent prendre que des
valeurs discrètes.

%%% Local Variables:
%%% mode: latex
%%% TeX-master: "../main_file"
%%% End:

\section{Algorithmes de filtrage basés sur le Time-Table}

La section suivante présente plusieurs algorithmes de filtrage pour le
\CECSP. Tous ces algorithmes utilisant une adaptation du Time-Table
pour le \CUSP, nous commençons donc par présenter brièvement comment
ce raisonnement est modifié pour être adapté dans le cadre du \CECSP.
Puis, nous présentons deux autres algorithmes permettant
de réduire l'ensemble des valeurs possibles pouvant être prises par
chaque variable. Le premier est adapté du Time-Table disjonctif pour
le \CUSP~\cite{Gay2015} et le dernier utilise une combinaison entre
problème de flots et profil obligatoire.

\subsection{Le Time-Table}
\index{Time-Table!CECSP}
Comme pour le \CUSP, le Time-Table pour le \CECSP~se base sur la
notion de partie obligatoire des activités, i.e. l'intervalle pendant
lequel une activité est en cours d'exécution dans tous les
ordonnancements réalisables. Cependant, comme dans le cas du \CECSP,
nous ne connaissons pas la durée exacte d'une activité, nous utilisons
une borne inférieure sur sa durée pour calculer la date de début au
plus tard, $\LS$, et la date de fin au plus tôt, $\EE$. Pour calculer
cette borne, remarquons que la configuration permettant de finir une
activité le plus rapidement possible, est celle où l'activité est
exécutée à son rendement maximal $\bmax$. De ce fait, une borne
inférieure sur la durée de l'activité vaut $W_i/f_i(\bmax)$. Nous
pouvons donc calculer la date de début au plus tard de l'activité,
$\LS=d_i-W_i/f_i(\bmax)$, et sa date de fin au plus tôt,
$\EE=r_i+W_i/f_i(\bmax)$. La partie obligatoire d'une activité $i$ est
alors définie de a même manière que pour le \CUSP, i.e la partie
obligatoire de $i$ est l'intervalle $[\LS,\EE]$
(cf. figure~\ref{fig_mand_CECSP}).

Cependant, dans le cas où $\LS \le \EE$, i.e. où l'activité possède
une partie obligatoire, nous pouvons seulement déduire que l'activité
$i$ va consommer au moins une quantité $\bmin$ de la ressource durant
toute sa partie obligatoire, et ce, quelque soit le moment où
l'activité est ordonnancée. La notion de profil obligatoire de la
ressource est donc légèrement différente de celle définie pour le
\CUSP~(cf. définition~\ref{def:profil_oblig},
page~\pageref{def:profil_oblig}).

\begin{defi}
Le profil obligatoire d'une ressource $TT_{\A}$ dans le cas du
\CECSP~est définie de la façon suivante: 
\[TT_{\A}(t)=\sum_{\substack{i \in \A\\\LS \le t \le \EE}} \bmin\quad
  \forall t \in \H\]
Le problème est donc insatisfiable dans le cas où $\exists t \in \H\
:\ TT_{\A}(t) > R$
\end{defi}

\begin{ex}
Considérons l'activité suivante: 

\vspace{-0.5cm}
\begin{center}
  \begin{tabular}{|P{1cm}P{1cm}P{1cm}P{1cm}P{1cm}P{1cm}|}
    \hline
    \ES & \LE & W_i & \bmin & \bmax & f_i(b)\\
    \hline
    1 & 14 & 72 & 2 & 5 & b+3\\
    \hline
  \end{tabular}
\end{center}

Nous pouvons calculer sa date de début au plus tard,
$\LS=d_i-W_i/f_i(\bmax)=14 - 72/8 =5$, ainsi que sa date de fin au
plus tôt, $\EE=r_i+ W_i/f_i(\bmax)=1 + 72/8 =10$. Comme $\EE > \LS$,
l'activité possède une partie obligatoire qui est l'intervalle
$[5,10]$ (voir
figure~\ref{fig_mand_CECSP_a},~\ref{fig_mand_CECSP_b}). Cependant,
nous pouvons seulement en déduire que l'activité sera en cours dans
cet intervalle et ,grâce à la borne inférieure sur la quantité de
ressource que peut consommer l'activité durant son exécution, nous
pouvons déduire que, sur l'intervalle $[\LS,\EE]$, l'activité est au
moins exécutée à $\bmin$ (voir figure~\ref{fig_mand_CECSP_c}).
  
\begin{figure}[htb!]
\vspace{-0.8cm}
\subcaptionbox{Ordonnancement au plus tôt\label{fig_mand_CECSP_a}}[0.3\linewidth]{
    \begin{tikzpicture}
      [xscale=0.25, yscale= 0.4,node distance=0.5cm]
      \node (sil) at (1,0) {} ;
      \node (eil) at (10,0) {} ;
      \node [below of=eil,node distance=0.63cm]  {$\EE$};
      \draw (sil.center) node[below=0.2cm] {$\ES$};
      
      \draw (0,0) -- (14,0);
      \draw[line width=3pt] (1,0) -- (1,5);
      
      \draw[<->] (0,0.1) -- (0,4.4) node[midway,left] {$\bmax$};
      \draw (1,0) rectangle (10,4.4) node[midway] {$i$};

      \draw[dashed] (10,0) -- (10,5);

      \foreach \i in {0,...,14} {
        \draw (\i,0)  -- (\i,-0.2);
      }
    \end{tikzpicture}
}
\hfill
\subcaptionbox{Ordonnancement au plus tard\label{fig_mand_CECSP_b}}[0.3\linewidth]{
    \begin{tikzpicture}
      [xscale=0.25, yscale= 0.4,node distance=0.5cm]
      \node (sir) at (5,0) {} ;
      \node (eir) at (14,0) {} ;
      \node[below of= sir,node distance=0.63cm] {$\LS$};
      \draw (eir.center) node[below=0.2cm] {$\LE$};
      
      \draw (0,0) -- (14,0);
      \draw[line width=3pt] (14,0) -- (14,5);
      
      \draw[<->] (0,0.1) -- (0,4.4) node[midway,left] {$\bmax$};
      \draw (5,0) rectangle (14,4.4) node[midway] {$i$};

      \draw[dashed] (5,0) -- (5,5);

      \foreach \i in {0,...,14} {
        \draw (\i,0)  -- (\i,-0.2);
      }
    \end{tikzpicture}
}
\hfill
\subcaptionbox{Ordonnancement réalisable à $\bmin$ dans
  $[\LS,\EE]$ \label{fig_mand_CECSP_c}}[0.3\linewidth]{ 
    \begin{tikzpicture}
      [xscale=0.25, yscale= 0.4,node distance=0.5cm]
      \node (sir) at (5,0) {} ;
      \node (eil) at (10,0) {} ;
      \node [below of=eil,node distance=0.63cm]  {$\EE$};
      \node[below of= sir,node distance=0.63cm] {$\LS$};
      \draw[<->] (5,2.7) -- (10,2.7) node[midway,above,text width=1.4cm]
      {\begin{center} \scriptsize partie oblig. \end{center}};
      
      \draw[white] (5,0) rectangle (10,2.4) node[midway,color=black] {$i$};
           
      \draw (1,4.4) -- (5,4.4) -- (5,2.4) -- (10,2.4) -- (10,3.3) --
      (13,3.3) -- (13,0);
      
      \draw (0,0) -- (14,0);
      \draw[line width=3pt] (1,0) -- (1,5);
      \draw[line width=3pt] (14,0) -- (14,5);
      
      \draw[<->] (-0.1,0.1) -- (-0.1,2.4) node[midway,left] {$\bmin$};
    


      \draw[dashed] (5,0) -- (5,5);
      \draw[dashed] (10,0) -- (10,5);

      \foreach \i in {0,...,14} {
        \draw (\i,0)  -- (\i,-0.2);
      }
    \end{tikzpicture}
}
\caption{Partie obligatoire d'une activité $i$}
\label{fig_mand_CECSP}
\end{figure}
\end{ex}

Le profil obligatoire peut aussi être calculé en $O(n)$ à l'aide d'un
algorithme de balayage en triant au préalable les activités par date
de début au plus tard et date de fin au plus tôt.

Nous détaillons maintenant l'adaptation du Time-Table disjonctif au
\CECSP. 

\subsection{Le Time-Table disjonctif}

Le second algorithme de filtrage proposé repose sur un raisonnement
appelé Time-Table disjonctif et utilisé, en premier lieu, pour le
\CUSP (cf. paragraphe~\ref{sec:mix_CUSP}). Ce dernier repose sur le
raisonnement Time-Table décrit précédemment et sur le raisonnement
disjonctif.

Le raisonnement disjonctif dans le cadre du \CECSP, est très similaire
à celui défini pour le \CUSP. La différence repose sur la construction
des ensembles disjonctifs. Dans le cas du \CECSP, un couple
d'activités ($i,j)$ sera dit disjonctif si $\bmin+\bmin[j] >R$. Dans
ce cas, nous savons que:
\begin{itemize}
\item l'activité $i$ doit commencer après l'activité $j$, ou, 
\item l'activité $i$ doit finir avant l'activité $j$.  
\end{itemize}

Cette propriété permet, entre autre, d'ajuster la date de début au
plus tôt de $j$. En effet, $\ES[j] \le \EE$ et $\LS \le \EE[j]$
implique que $j$ doit commencer après la fin de l'activité $i$ (voir
figure~\ref{fig:disj_CECSP}). De ce fait, la début de $j$ ne peut arriver
avant la date de fin au plus tôt de $i$, et donc: $\ES[j] \ge \EE$. La
règle de filtrage est ensuite similaire à celle mise en place pour le
\CUSP. 

\begin{reg}
Soient $i, j \in \A, i \neq j$ telles que $\bmin+\bmin[j] < R$ et $\LS
< \EE[j]$. Alors la date de début au plus tôt de l’activité peut être
ajustée et on a : $\ES[j] \ge \EE$.
\end{reg}

\begin{ex}
Considérons les deux activités suivantes: 
\begin{center}
\begin{tabular}{|P{1cm}|P{1cm}P{1cm}P{1cm}P{1cm}P{1cm}P{2cm}|}
    \hline
    act & \ES & \LE & W_i & \bmin & \bmax & f_i(b_i(t))  \\
    \hline
   i & 2 & 11 & 28 & 2 & 3 & 2*b_i(t) +1\\
   j & 1 & 20 & 49 & 2 & 4 & voir fig~\ref{fig:fonct_CECSP}\\
    \hline
  \end{tabular}
\end{center}

La fonction $f_j(b)$ est définie par l'expression suivante: 
\[f_j(b)=\left\{
\begin{array}{lll}
2b & & b \in [2,3]\\
b+3 & & b \in [3,4]
\end{array}
\right.\] 
et décrite dans la figure~\ref{fig:fonct_CECSP}
\begin{figure}[!htb]
\centering
\begin{tikzpicture}
[xscale=0.8,yscale=0.56]
\node (O) at (1,2) {};
\draw[->] (1,2) -- (5.5,2);
\draw[->] (1,2) -- (1,8);

\path[draw] (2,4) -- (3,6) -- (4,7) ;

\draw[dotted] (2,2) node[below] {\footnotesize $2$} -- (2,8);
\draw[dotted,color=gray!70] (4,2) node[below,color=black] {\footnotesize $4$}
-- (4,8);
\draw[dotted] (3,2) node[below] {\footnotesize $3$} -- (3,8);

\draw (1,4) node[left] {\footnotesize $4$};
\draw (1,6) node[left] {\footnotesize $6$};
\draw (1,7) node[left] {\footnotesize $7$};
\end{tikzpicture}
\caption{Fonction $f_j(b_j(t))$}
\label{fig:fonct_CECSP}
\end{figure}

Dans cet exemple, comme $\bmin +\bmin[j] =4 > 3$, $i$ et $j$ ne
peuvent s'exécuter en parallèle. Si l'activité $i$ finit au temps
$\LE= 11$ alors, elle chevauche forcément l'activité $j $ (voir
figure~\ref{fig:disj_CECSPa} et~\ref{fig:disj_CECSPb}). Dans tous
ordonnancement réalisable, $i$ est donc exécuté avant $j$ et la date de
début au plus tôt de $j$ peut donc être ajustée, i.e. $j$ ne peut
commencer avant $\EE=6$ (voir
figure~\ref{fig:disj_CECSPc}).
  \begin{figure}[htb!] 
    \subcaptionbox{Si $i$ finit au temps $11$...\label{fig:disj_CECSPa}}[0.45\linewidth]{
    \centering
    \begin{tikzpicture} [yscale=0.4,xscale=0.4]   
        \node (O) at (0,0) {};
      \foreach \i in {0,5,...,10} {
        \draw (\i,0) -- (\i,-0.1) node[below] {\small $\i$};
      }
      \fill[gray!50] (2,0) rectangle (11,3.4);
      \fill[gray!50] (1,3.6) rectangle (14,8);
      
      \draw[fill=white] (6.4,0.2) -- (6.4,3.2)   -- (8.4,3.2) -- (8.4,2) --node[midway,below=0.2cm] {$i$}
      (11,2) -- (11,0.2) -- cycle;
      
      \draw[fill=white] (1,3.8) -- (1,7.8)  node[midway,right=0.6cm] {$j$} -- (6,7.8) -- (6,5.8) --
      (9.5,5.8) -- (9.5,3.8) -- cycle;
      \draw[white, pattern=north west lines] (6.4,0) rectangle (9.5,8);

      \draw[->] (0,0) -- (14,0);
      \draw[->] (0,0) -- (0,8) ;
      \draw (0,3) node[left] {$R=3$};
      \draw[densely dotted] (1,-0.1) -- (1,8) node[above] {$\ES[j]$};
      \draw[densely dotted] (8,-0.1) -- (8,8) node[above] {$\EE[j]$};
      \draw[densely dotted] (2,-0.1)  node[below] {$\ES$}-- (2,8);
      \draw[densely dotted] (6,-0.1 ) node[below=0.4cm] {$\EE$}-- (6,8) ;
      \draw[densely dotted] (7,-0.1) node[below] {$\LS$} -- (7,8) ;
      \draw[densely dotted] (11,-0.1) node[below right] {$\LE$} -- (11,8) ;
      \draw[<->] (14.5,0.2) -- (14.5,3.2) node[midway,right] {$\bmax$} ;
      \draw[<->] (14.5,3.8) -- (14.5,7.8) node[midway,right] {$\bmax[j]$} ;   
    \end{tikzpicture}
  }
    \subcaptionbox{...alors $i$ chevauche forcément l'activité $j$\label{fig:disj_CECSPb}}[0.45\linewidth]{
        \centering
        \begin{tikzpicture}[yscale=0.4,xscale=0.4]
      \node (O) at (0,0) {};
      \foreach \i in {0,5,...,10} {
        \draw (\i,0) -- (\i,-0.1) node[below] {\small $\i$};
      }
      \fill[gray!50] (2,0) rectangle (11,3.4);
      \fill[gray!50] (1,3.6) rectangle (14,8);
      
      \draw[fill=white] (7,0.2) rectangle (11,3.2) node[midway]
      {$i$};
      \draw[fill=white] (1,3.8) rectangle (8,7.8) node[midway]
      {$j$};
      \draw[white, pattern=north west lines] (7,0) rectangle (8,8);

      \draw[->] (0,0) -- (14,0);
      \draw[->] (0,0) -- (0,8) ;
      \draw (0,3) node[left] {$R=3$};
      \draw[densely dotted] (1,-0.1) -- (1,8) node[above] {$\ES[j]$};
      \draw[densely dotted] (8,-0.1) -- (8,8) node[above] {$\EE[j]$};
      \draw[densely dotted] (2,-0.1)  node[below] {$\ES$}-- (2,8);
      \draw[densely dotted] (6,-0.1 ) node[below=0.4cm] {$\EE$}-- (6,8) ;
      \draw[densely dotted] (7,-0.1) node[below] {$\LS$} -- (7,8) ;
      \draw[densely dotted] (11,-0.1) node[below right] {$\LE$} -- (11,8) ;
      % \draw[densely dotted] (6,-0.1) -- (6,8) node[above] {$\ES[j]^{'}$};
      % \draw[->] (1.8,5.8) -- (5.2,5.8);
      \draw[<->] (14.5,0.2) -- (14.5,3.2) node[midway,right] {$\bmax$} ;
      \draw[<->] (14.5,3.8) -- (14.5,7.8) node[midway,right]
      {$\bmax[j]$} ;
    \end{tikzpicture}
   }
    \subcaptionbox{$\ES[j]$ peut être ajusté\label{fig:disj_CECSPc}}[\linewidth]{    \centering
      \begin{tikzpicture} [yscale=0.4,xscale=0.4]
      \node (O) at (0,0) {};
      \foreach \i in {0,5,...,10} {
        \draw (\i,0) -- (\i,-0.1) node[below] {\small $\i$};
      }
      \fill[gray!50] (2,0) rectangle (11,3.4);
      \fill[gray!50] (6,3.6) rectangle (14,8);
      
      \draw[fill=white] (2,0.2) rectangle (6,3.2) node[midway]
      {$i$};
      \draw[fill=white] (6,3.8) rectangle (13,7.8) node[midway]
      {$j$};

      \draw[->] (0,0) -- (14,0);
      \draw[->] (0,0) -- (0,8) ;
      \draw (0,3) node[left] {$R=3$};
      \draw[densely dotted] (1,-0.1) -- (1,8) node[above] {$\ES[j]$};
      \draw[densely dotted] (8,-0.1) -- (8,8) node[above] {$\EE[j]$};
      \draw[densely dotted] (2,-0.1)  node[below] {$\ES$}-- (2,8);
      \draw[densely dotted] (6,-0.1 ) node[below=0.4cm] {$\EE$}-- (6,8) ;
      \draw[densely dotted] (7,-0.1) node[below] {$\LS$} -- (7,8) ;
      \draw[densely dotted] (11,-0.1) node[below right] {$\LE$} -- (11,8) ;
      \draw[<->] (14.5,0.2) -- (14.5,3.2) node[midway,right] {$\bmax$} ;
      \draw[<->] (14.5,3.8) -- (14.5,7.8) node[midway,right] {$\bmax[j]$} ;

  \end{tikzpicture}
}
  \caption{Raisonnement disjonctif}
  \label{fig:disj_CECSP}
\end{figure}
\end{ex}

Nous pouvons maintenant présenter l'adaptation du Time-Table
disjonctif au cas du \CECSP. Pour ce faire, nous commençons par
adapter la définition d'intervalle minimum de superposition, puis nous
présenterons les modifications apportées aux règles d'ajustement du
\CUSP~afin que ces dernières soient applicables dans le cas du
\CECSP. Enfin, l'extension de ces règles dans le cas où les activités
ne possèdent pas de partie obligatoire sera décrit. Cette extension
n'avait pas été présentée dans le chapitre sur le \CUSP~mais est
décrite dans l'article~\cite{Gay2015}. 

La notion d'intervalle minimum de superposition s'adapte naturellement
au cas du \CECSP. En effet, cette intervalle représente l'ensemble
minimal de points de temps tel que une activité $i$ est forcément en
cours durant un de ces points. 

\begin{defi}
\label{des:moi_CUSP} 
Soit $\EE^{-}$ le point le plus proche de $\EE$, i.e. $\forall \delta
>0 , |\EE^{-} -\EE| \le \delta$. L'intervalle minimum de
superposition d'une activité $i$,noté $moi_i$, est alors défini par
$moi_i=[\EE^{-},\LS{]}$ si $i$ ne possède pas de partie obligatoire et
$moi_i=\emptyset$ sinon.   

Il s'agit du plus petit intervalle de temps tel que $i$ s’exécute au
moins durant un point de temps de cet intervalle, et ce peu importe le
moment auquel l'activité $i$ est exécutée.
\end{defi}


\begin{ex}
La figure~\ref{fig:moi_CECSP} illustre l'intervalle minimum de
superposition d'une activité $i$ ne possédant pas de partie
obligatoire. En effet, quelque soit la position de l'activité $i$,
elle intersecte forcément $moi_i$.
\begin{figure}[!htb]
  \begin{center}
    \begin{tikzpicture}
      [xscale=0.5, yscale= 0.4,node distance=0.5cm][decoration={brace}]
      \node (sil) at (0,0) {} ;
      \node (sir) at (10,0) {}; 
      \draw (10,0) node[below] {$lst_i$};
      \node (eir) at (14,0) {} ;
      \node (eil) at (4,0) {} ;
      \draw (4,0) node [below]  {$eet_i$} ;
      \draw (sil.center) node[below] {$est_i$}--
      (eir.center) node[below] {$let_i$};
      
      \draw[<-] (3.95, 4.6) -- (4.3,5.1) node[above=0.2cm,right] {\scriptsize $\EE^{-}$};
      \draw[line width=3pt] (0,0.5) -- (0,2);
      \draw[line width=3pt] (14,0.5) -- (14,2);
      \draw[line width=3pt] (0,3) -- (0,4.5);
      \draw[line width=3pt] (14,3) -- (14,4.5);

      \fill[gray!80] (3.9,0.6) rectangle (10,1.9) node[midway,color=black] {$\mathbf{moi_i}$};
      \fill[gray!80] (3.9,3.1) rectangle (10,4.4) node[midway,color=black] {$\mathbf{moi_i}$};
      
      \draw (0,3.1) rectangle (4,4.4) node[midway] {$i$};
      \draw (10,0.6) rectangle (14,1.9) node[midway] {$i$};

      \draw[dashed] (3.9,0) -- (3.9,4.5);
      \draw[dashed] (10,0) -- (10,4.5);

      \foreach \i in {0,...,14} {
        \draw (\i,0)  -- (\i,-0.2);
      }
    \end{tikzpicture}
  \end{center}

  \caption{Intervalle minimum de superposition d'une activité}
  \label{fig:moi_CECSP}
\end{figure}
\end{ex}

La notion d'intervalle minimum de superposition permet, dans un
premier temps, d'améliorer le raisonnement disjonctif. En effet, dans
le cas où deux activités $i$ et $j$ ne peuvent être exécutée en
parallèle, la fenêtre de temps de $j$ ne peut contenir un point de
temps que $i$ devra forcément chevaucher, i.e. contenu dans
$moi_i$. Ceci nous permet de définir la règle d'ajustement suivante:  

\begin{reg}
\label{reg:RDR_CECSP}
  Soient deux activités $i$ et $j$ telles que $i$ ne possède pas de
  partie obligatoire et que $\bmin + \bmin[j] > R$. Si ordonnancer l'activité
 $j$ à sa date de début au plus tôt la fait se superposer complètement
 à l’intervalle minimal de superposition de $i$ ($moi_i \subseteq
 [\ES[j],\EE[j]{]}$), alors $\ES[j] \ge \EE$.
\end{reg}

\begin{ex}
Considérons les deux activités suivantes: 
\begin{center}
\begin{tabular}{|P{1cm}|P{1cm}P{1cm}P{1cm}P{1cm}P{1cm}P{2cm}|}
    \hline
    act & \ES & \LE & W_i & \bmin & \bmax & f_i(b_i(t))  \\
    \hline
   i & 0 & 14 & 28 & 2 & 3 & 2*b_i(t) +1\\
   j & 2 & 20 & 49 & 2 & 4 & voir fig~\ref{fig:fonct_ CECSP}\\
    \hline
  \end{tabular}
\end{center}


La règle~\ref{reg:RDR_CECSP} est illustrée par la
figure~\ref{fig:RDR_CECSP}. Dans les figures~\ref{fig:RDR_CECSPa} et~\ref{fig:RDR_CECSPb}, on
peut constater que si l'activité $j$
commence à $\ES[j]$, alors elle intersecterait complètement $moi_i$ et
il serait impossible d'ordonnancer $i$. Sur la figure~\ref{fig:RDR_CECSPc}, $\ES[j]$ a été ajusté et l'activité $j$ ne peut commencer
avant  $t > \min{moi_i}$.
  \begin{figure}[htb!] 
    \subcaptionbox{Si $j$ commence au temps $2$...\label{fig:RDR_CECSPa}}[0.45\linewidth]{
    \centering
    \begin{tikzpicture} [yscale=0.4,xscale=0.4]   
        \node (O) at (0,0) {};
      \foreach \i in {1,...,14} {
        \draw (\i,0) -- (\i,-0.1) ;
      }
      \fill[gray!50] (0,0) rectangle (14,3.4);
      \fill[gray!50] (2,3.6) rectangle (14,8);
      
      \draw[fill=white] (0,0.2) -- (0,3.2)   -- (2,3.2) -- (2,2) --node[midway,below=0.2cm] {$i$}
      (4.6,2) -- (4.6,0.2) -- cycle;
      
      \draw[fill=white] (2,3.8) -- (2,7.8)  node[midway,right=0.6cm] {$j$} -- (7,7.8) -- (7,5.8) --
      (10.5,5.8) -- (10.5,3.8) -- cycle;

        \draw[fill=black!70!] (3.9,3.2) rectangle
        (10,3.8) node[midway,white] {$\mathbf{moi_i}$};

      \draw[->] (0,0) -- (14,0);
      \draw[->] (0,0) -- (0,8) ;
      \draw (0,3) node[left] {$R=3$};
      \draw[densely dotted] (2,-0.1) -- (2,8) node[above] {$\ES[j]$};
      \draw[densely dotted] (9,-0.1) -- (9,8) node[above] {$\EE[j]$};
      \draw[densely dotted] (0,-0.1)  node[below] {$\ES$}-- (0,8);
      \draw[densely dotted] (4,-0.1 ) node[below=0.4cm] {$\EE$}-- (4,8) ;
      \draw[densely dotted] (10,-0.1) node[below] {$\LS$} -- (10,8) ;
      \draw[densely dotted] (14,-0.1) node[below right] {$\LE$} -- (14,8) ;
      \draw[<->] (14.5,0.2) -- (14.5,3.2) node[midway,right] {$\bmax$} ;
      \draw[<->] (14.5,3.8) -- (14.5,7.8) node[midway,right] {$\bmax[j]$} ;   
    \end{tikzpicture}
  }
    \subcaptionbox{...alors $j$ chevauche forcément $moi_i$\label{fig:RDR_CECSPb}}[0.45\linewidth]{
        \centering
        \begin{tikzpicture}[yscale=0.4,xscale=0.4]
      \node (O) at (0,0) {};
      \foreach \i in {1,...,14} {
        \draw (\i,0) -- (\i,-0.1) ;
      }
      \fill[gray!50] (0,0) rectangle (14,3.4);
      \fill[gray!50] (2,3.6) rectangle (14,8);
      
      \draw[fill=white] (0,0.2) rectangle (4,3.2) node[midway]
      {$i$};
      \draw[fill=white] (2,3.8) rectangle (9,7.8) node[midway]
      {$j$}; 
      \draw[fill=black!70!] (3.9,3.2) rectangle
        (10,3.8) node[midway,white] {$\mathbf{moi_i}$};

      
      \draw[->] (0,0) -- (14,0);
      \draw[->] (0,0) -- (0,8) ;
      \draw (0,3) node[left] {$R=3$};
      \draw[densely dotted] (2,-0.1) -- (2,8) node[above] {$\ES[j]$};
      \draw[densely dotted] (9,-0.1) -- (9,8) node[above] {$\EE[j]$};
      \draw[densely dotted] (0,-0.1)  node[below] {$\ES$}-- (0,8);
      \draw[densely dotted] (4,-0.1 ) node[below=0.4cm] {$\EE$}-- (4,8) ;
      \draw[densely dotted] (10,-0.1) node[below] {$\LS$} -- (10,8) ;
      \draw[densely dotted] (14,-0.1) node[below right] {$\LE$} -- (14,8) ;
      % \draw[densely dotted] (6,-0.1) -- (6,8) node[above] {$\ES[j]^{'}$};
      % \draw[->] (1.8,5.8) -- (5.2,5.8);
      \draw[<->] (14.5,0.2) -- (14.5,3.2) node[midway,right] {$\bmax$} ;
      \draw[<->] (14.5,3.8) -- (14.5,7.8) node[midway,right]
      {$\bmax[j]$} ;
    \end{tikzpicture}
   }
    \subcaptionbox{$\ES[j]$ peut être ajusté\label{fig:disj_CECSPc}}[\linewidth]{    \centering
      \begin{tikzpicture} [yscale=0.4,xscale=0.4]
      \node (O) at (0,0) {};
      \foreach \i in {1,...,14} {
        \draw (\i,0) -- (\i,-0.1);
      }
      \fill[gray!50] (0,0) rectangle (14,3.4);
      \fill[gray!50] (4,3.6) rectangle (14,8);
      
      \draw[fill=white] (0,0.2) rectangle (4,3.2) node[midway]
      {$i$};
      \draw[fill=white] (4,3.8) rectangle (11,7.8) node[midway]
      {$j$};
      \draw[fill=black!70!] (3.9,3.2) rectangle
        (10,3.8) node[midway,white] {$\mathbf{moi_i}$};

      \draw[->] (0,0) -- (14,0);
      \draw[->] (0,0) -- (0,8) ;
      \draw (0,3) node[left] {$R=3$};
      \draw[densely dotted] (4,-0.1) -- (4,8) node[above] {$\ES[j]$};
      \draw[densely dotted] (11,-0.1) -- (11,8) node[above] {$\EE[j]$};
      \draw[densely dotted] (0,-0.1)  node[below] {$\ES$}-- (0,8);
      \draw[densely dotted] (4,-0.1 ) node[below=0.4cm] {$\EE$}-- (4,8) ;
      \draw[densely dotted] (10,-0.1) node[below] {$\LS$} -- (10,8) ;
      \draw[densely dotted] (14,-0.1) node[below right] {$\LE$} --
      (14,8) ;
      \draw[<->] (14.5,0.2) -- (14.5,3.2) node[midway,right] {$\bmax$} ;
      \draw[<->] (14.5,3.8) -- (14.5,7.8) node[midway,right] {$\bmax[j]$} ;

  \end{tikzpicture}
}
  \caption{Raisonnement disjonctif restreint pour le \CECSP.}
  \label{fig:disj_CECSP}
\end{figure}
\end{ex}

La règle~\ref{reg:RDR_CUSP} compare seulement les consommations de $i$
et de $j$ avec la capacité de la ressource $R$. Cependant, les
consommations obligatoires des autres activités peuvent ne pas laisser
$R$ unités de ressource durant l'intersection de $i$ et de $j$. La
règle suivante prend donc en considération le profil obligatoire de la
ressource. 

\begin{reg}
\label{reg:TTDR_CUSP}
Soient donc $i$ et $j$ deux activités qui ne possèdent pas de partie
obligatoire et telles que $ \bmin +\bmin[j] + \min_{t \in moi_i} TT_{\A}(t) >
R$.   Si ordonnancer l'activité
 $j$ à sa date de début au plus tôt la fait se superposer complètement
 à l’intervalle minimal de superposition de $i$ ($moi_i \subseteq
 [\ES[j],\EE[j]{]}$), alors $\ES[j] \ge \EE$.
\end{reg}

\begin{ex}
Considérons les deux activités suivantes: 
\begin{center}
\begin{tabular}{|P{1cm}|P{1cm}P{1cm}P{1cm}P{1cm}P{1cm}P{2cm}|}
    \hline
    act & \ES & \LE & W_i & \bmin & \bmax & f_i(b_i(t))  \\
    \hline
   i & 2 & 11 & 21 & 1 & 2 & 2*b_i(t) +1\\
   j & 1 & 20 & 14 & 1 & 2 & b_i(t)\\
   k & 2 & 11 & 18 & 2 & 2 & \frac{1}{3}b_i(t) + \frac{4}{3}\\
    \hline
  \end{tabular}
\end{center}


  Les activités $1$ et $2$ ne possèdent pas de partie obligatoire tandis
  que l'activité $3$ est forcément en cours d'exécution durant
  l'intervalle $[2,11[$. 


  \begin{figure}[!htb]
    \centering
    \begin{tikzpicture}
      [yscale=0.45,xscale=0.6]
      \node (O) at (0,0) {};
      \foreach \i in {0,1,...,20} {
        \draw (\i,0) -- (\i,-0.15) node[below] {\small $\i$};
      }
      
      \draw (2,0) rectangle (11,2);
      \fill[gray!30] (5,3.8) rectangle (20,6.2);
      \fill[gray!30] (2,6.8) rectangle (11,9.2);
      
      \draw [->] (1.2,4.5) -- (4.8,4.5);
      \draw [->] (1.8,-1.7) -- (4.3,-1.7);
      \draw[densely dotted] (1,-0.1) node[below=0.3cm] {$\ES[j]$}-- (1,8.2) ;
      \draw[densely dotted] (5,-0.1 ) node[below=0.3cm] {$\EE$}--
      (5,8.2);
      \draw[densely dotted] (10,-0.1) node[below=0.3cm] {$\EE[j]$} -- (10,8.2) ;
      \draw[densely dotted] (8,-0.1) node[below=0.3cm] {$\LS$} -- (8,8.2) ;
      
      \draw[fill=black!70!] (4,6) rectangle
      (8,7) node[midway,white] {$\mathbf{moi_i}$};
      
      \draw[fill=white] (2,7) rectangle (5,9) node[midway]
      {$i$};
      \draw[fill=white] (5,4) rectangle (14,6) node[midway]
      {$j$};
      
      \draw[->] (0,0) -- (21,0);
      \draw[->] (0,0) -- (0,5.5) ;
       \draw[thick] (0,3)  node[left] {$R=3$} -- (20,3);
       % \draw[densely dotted] (6,-0.1) -- (6,5.5) node[above] {$\ES[j]^{'}$};
      % \draw[->] (1.8,5.8) -- (5.2,5.8);  
    \end{tikzpicture}
    \caption{Illustration du Time-Tabe disjonctif}
    \label{fig:TTDR_CUSP}
  \end{figure}
L'intervalle $moi_i=[4,8]$ est complètement inclus dans l'intervalle
formé par $\ES[j]$ et $\EE[j]$, i.e. $[1,10[$. De plus, le minimum du
profil de consommation de la ressource dans $moi_i$ est de $2$. Donc,
les activités $i$ et $j$, consommant  au minimum  $1$ unité
de ressource, ne peuvent se chevaucher l'intervalle $[4,8]$. Donc
$\ES[j]$ peut être ajuster à $5$. 
\end{ex}

\subsection{Le Time-Table basé sur les flots}




\section{Algorithme de filtrage du raisonnement énergétique}
\label{sec:ER_CECSP}
\index{raisonnement énergétique!pour le CECSP}

Le paragraphe ci-dessous décrit l'adaptation du raisonnement
énergétique, introduit par~\cite{RELopez} pour la contrainte cumulative,
et décrit dans le paragraphe~\ref{sec:cumu_propag}. Nous commençons,
dans un premier temps par décrire l'algorithme de vérification de ce
raisonnement, puis nous présenterons les règles d'ajustements qui
peuvent être mises en place pour filtrer les domaines des
variables. Enfin, la dernière partie de ce paragraphe sera consacrée à
la caractérisation des intervalles d'intérêt pour l'algorithme de
vérification et pour les règles d'ajustement.

\subsection{Algorithme de vérification}

\subsubsection{Condition nécessaire d'existence de solution}
Pour décrire l'algorithme de vérification, nous rappelons d'abord
l'idée principale sur laquelle repose le raisonnement énergétique. Le
principe est donc, étant donné un intervalle $[t_1,t_2[$, de calculer
les consommations minimales de ressource des activités dans cette
intervalle et de les comparer à la quantité de ressource disponible
dans ce même intervalle. Si la ressource disponible n'est pas
suffisante pour ordonnancer les consommations minimales de toutes les
activités, une incohérence est détectée.

Dans le cas du \CUSP, la quantité de ressource requise par une
activité pouvait être calculée de manière directe. Ici, ce calcul sera
fait en deux fois: nous calculons d'abord la quantité d'énergie
requise par une activité à l'intérieur de l'intervalle $[t_1,t_2{[}$,
notée $\wb$, puis nous traduisons cette énergie en une quantité de
ressource, notée $\bb$. Ceci nous permettra ensuite de la comparer
avec la ressource disponible dans $[t_1,t_2{[}$.

Formellement ces quantités sont représentées par les expressions
suivantes: 
\begin{align}
  \wb= \min \int_{t_1}^{t_2} f_i(b_i(t))dt & & \text{sous 
\eqref{tw_CECSP}-\eqref{nrj_CECSP}}\\
  \bb= \min \int_{t_1}^{t_2} b_i(t)dt & & \text{sous 
\eqref{tw_CECSP}-\eqref{nrj_CECSP}}
\end{align}

Comme pour le cas du \CUSP, la fonction de marge, notée $SL(t_1,t_2)$,
permet de mesurer l'écart entre la quantité de ressource disponible et
les consommations minimales de toutes les tâches dans l'intervalle
${[}t_1,t_2{]}$. Cette fonction est définie de la manière suivante:
\[ SL(t_1,t_2)=B(t_2-t_1)-\sum\limits_{i \in A} \bb \]

Et ceci nous permet d'énoncer la condition nécessaire d'existence
d'une solution qui est à la base de l'algorithme de vérification du
raisonnement énergétique:

\begin{theo}
  \label{th:ER_CECSP}
  Soit $\I$ une instance du \CECSP. S'il existe $t_1 < t_2 \in
  \mathbb{R}^2$ tel que $SL(t_1,t_2) <0$ alors $\I$ ne peut pas avoir
  de solution.
\end{theo}

\begin{proof}
Par l'absurde, supposons qu'il existe $t_1 < t_2 \in \mathbb{R}^2$ tel
que $SL(t_1,t_2) > 0$ et que l'instance $\I$ soit satisfiable. Par
définition, $\bb$ est la quantité de ressource minimale que doit
consommer l'activité $i$ dans l'intervalle $[t_1,t_2{[}$. 

Donc, dans toute solution réalisable, nous avons: 
\begin{align*}
 & \int_{t_1}^{t_2} b_i(t)dt \ge \bb\\
\Rightarrow  & \sum_{i \in \A} \int_{t_1}^{t_2} b_i(t)dt \ge \sum_{i
               \in \A}  \bb > R(t_2-t_1)
\end{align*}
Et ceci contredit le fait que $\sum_{i \in \A} b_i(t) \le
R(t_2-t_1)$. 
\end{proof}

Dans un premier temps, nous allons nous intéressé au calcul de $\wb$,
le calcul de $\bb$ sera détaillé dans un second temps. 


\subsubsection{{\'E}nergie minimale dans un intervalle}

Pour calculer $\wb$, nous analysons les différentes configurations de
la consommation minimale d'une tâche. Remarquons que les
configurations conduisant à une consommation minimale dans
l'intervalle $[t_1,t_2{[}$ sont celles où l'activité est ordonnancée à
$\bmax$ à l'intérieur de cet intervalle. Ces configurations, décrites
dans la figure~\ref{fig:conso_CECSP}, peuvent être regroupée en trois
catégories:
\begin{itemize}
\item l'activité est {\it calée à
gauche} (figure~\ref{fig:conso_CECSPa}, \ref{fig:conso_CECSPb},
\ref{fig:conso_CECSPd} et \ref{fig:conso_CECSPg}): l'activité démarre
à $\ES$ et est ordonnancée à $\bmax$ pendant l'intervalle
$[\ES,t_1{[}$;
\item l'activité est {\it calée à
droite} (figure~\ref{fig:conso_CECSPb}, \ref{fig:conso_CECSPc},
\ref{fig:conso_CECSPf} et \ref{fig:conso_CECSPi}): l'activité finit à
$\LE$ et est ordonnancée à $\bmax$ pendant l'intervalle $[t_2,\LE{[}$;
\item l'activité est {\it centrée} (figure~\ref{fig:conso_CECSPe} et
\ref{fig:conso_CECSPh}): l'activité occupe tout l'intervalle
$[t_1,t_2[$, soit en étant ordonnancée à $\bmax$ pendant l'intervalle
$[\ES,t_1{[} \cup [t_2,\LE{[}$, soit en étant ordonnancée à $\bmin$
durant tout l'intervalle $[t_1,t_2{[}$.
\end{itemize}
En effet, lorsque l'activité est ordonnancée à $\bmax$ pendant
l'intervalle $[\ES,t_1{[} \cup [t_2,\LE{[}$, il peut arriver que la
quantité d'énergie restant à apporter à l'activité durant l'intervalle
$[t_1,t_2[$ ne soit pas suffisante pour assurer la satisfaction de la
contrainte de consommation minimale~\eqref{req_CECSP}. Le cas où
l'activité est ordonnancée à $\bmin$ durant tout l'intervalle
$[t_1,t_2{[}$ doit donc être considéré.

\begin{figure}[!htb]  
\centering
\subcaptionbox{\label{fig:conso_CECSPa}}[0.3\linewidth]{
  \begin{tikzpicture}
    [xscale=0.37,yscale=0.3]
    \node[] (O) at (0,0) {};
    \node[label={[shift={(-0.4,-0.4)}]$\bmin$}] (bmin) at (0,1) {};
    \node[label={[shift={(-0.4,-0.4)}]$\bmax$}] (bmax) at (0,4) {};
    \node (t1) at (6,0) {}; 
    \node[label={[shift={(0,-0.8)}]$\ES$}] (ri) at (1,0) {};
    \node (t2) at (7,0) {};

    \draw[->] (O.center) -- (8,0)node[below] {$t$};
    \draw (O.south) -- (bmax.north);
    \draw (bmin.center) -- (8,1);
    \draw (bmax.center) -- (8,4);
    \draw(ri.south) -- (ri.center);
    \draw[fill=white] (ri.center) rectangle (5,4);
    \draw[pattern=north west lines] (ri.center) rectangle (5,4);
    \draw[thick] (t1.south) -- (6,4.1) node[above] {$t_1$};
    \draw[thick] (t2.south) -- (7,4.1) node[above] {$t_2$};
  \end{tikzpicture}}
\hfill
\subcaptionbox{\label{fig:conso_CECSPb}}[0.3\linewidth]{
  \begin{tikzpicture}
    [xscale=0.37,yscale=0.3]
    \node[] (O) at (0,0) {};
    \node[label={[shift={(-0.4,-0.4)}]$\bmin$}] (bmin) at (0,1) {};
    \node[label={[shift={(-0.4,-0.4)}]$\bmax$}] (bmax) at (0,4) {};
    \node (t1) at (1,0) {}; 
    \node[label={[shift={(0,-0.8)}]$\ES$}] (ri) at (2,0) {};
    \node (t2) at (7,0) {};
    \node[label={[shift={(0,-0.8)}]$\LE$}] (di) at (6,0) {};

    \draw[->] (O.center) -- (8,0)node[below] {$t$};
    \draw (O.south) -- (bmax.north);
    \draw (bmin.center) -- (8,1);
    \draw (bmax.center) -- (8,4);
    \draw(ri.south) -- (ri.center);
    \draw(di.south) -- (di.center);
    \draw[fill=white] (2.5,0) rectangle (5.5,4);
    \draw[pattern=north west lines] (2.5,0) rectangle (5.5,4);
    \draw[thick] (t1.south) -- (1,4.1) node[above] {$t_1$};
    \draw[thick] (t2.south) -- (7,4.1) node[above] {$t_2$};
  \end{tikzpicture}}
\hfill
\subcaptionbox{\label{fig:conso_CECSPc}}[0.3\linewidth]{
  \begin{tikzpicture}
    [xscale=0.37,yscale=0.3]
    \node[] (O) at (0,0) {};
    \node[label={[shift={(-0.4,-0.4)}]$\bmin$}] (bmin) at (0,1) {};
    \node[label={[shift={(-0.4,-0.4)}]$\bmax$}] (bmax) at (0,4) {};
    \node (t1) at (0.5,0) {};
    \node (t2) at (1.5,0) {};
    \node[label={[shift={(0,-0.8)}]$\LE$}] (di) at (7,0) {};
    
    \draw[->] (O.center) -- (8,0)node[below] {$t$};
    \draw (O.south) -- (bmax.north);
    \draw (bmin.center) -- (8,1);
    \draw (bmax.center) -- (8,4);
    \draw[fill=white] (3,0) rectangle (7,4);
    \draw[pattern=north west lines] (3,0) rectangle (7,4);
    \draw(di.south) -- (di.center);
    \draw[thick] (t1.south) -- (0.5,4.1) node[above] {$t_1$};
    \draw[thick] (t2.south) -- (1.5,4.1) node[above] {$t_2$};
  \end{tikzpicture}}


\subcaptionbox{\label{fig:conso_CECSPd}}[0.3\linewidth]{
\begin{tikzpicture}
  [xscale=0.37,yscale=0.3]
    \node[] (O) at (0,0) {};
    \node[label={[shift={(-0.4,-0.4)}]$\bmin$}] (bmin) at (0,1) {};
    \node[label={[shift={(-0.4,-0.4)}]$\bmax$}] (bmax) at (0,4) {};
    \node (t1) at (4,0) {};
    \node (t2) at (7,0) {};
    \node[label={[shift={(0,-0.8)}]$\LE$}] (di) at (6,0) {};
    \node[label={[shift={(0,-0.8)}]$\ES$}] (ri) at (1,0) {};
    
    \draw[->] (O.center) -- (8,0)node[below] {$t$};
    \draw (O.south) -- (bmax.north);
    \draw (bmin.center) -- (8,1);
    \draw (bmax.center) -- (8,4);
    \draw[fill=white] (ri.center) rectangle (4,4);
    \draw[pattern=north west lines] (ri.center) rectangle (4,4);
    \draw[fill=white] (t1.center) rectangle (5,1.5);
    \draw[pattern=north west lines] (t1.center) rectangle (5,1.5);
    \draw(di.south) -- (di.center);
    \draw(ri.south) -- (ri.center);
    \draw[thick] (t1.south) -- (4,4.1) node[above] {$t_1$};
    \draw[thick] (t2.south) -- (7,4.1) node[above] {$t_2$};
  \end{tikzpicture}}
\hfill
\subcaptionbox{\label{fig:conso_CECSPe}}[0.3\linewidth]{
\begin{tikzpicture}
 [xscale=0.37,yscale=0.3]
 \node[] (O) at (0,0) {};
 \node[label={[shift={(-0.4,-0.4)}]$\bmin$}] (bmin) at (0,1) {};
 \node[label={[shift={(-0.4,-0.4)}]$\bmax$}] (bmax) at (0,4) {};
 \node (t1) at (2.5,0) {}; 
 \node[label={[shift={(0,-0.8)}]$\ES$}] (ri) at (1.5,0) {};
 \node (t2) at (6,0) {};
 \node[label={[shift={(0,-0.8)}]$\LE$}] (di) at (7,0) {};
 
  \draw[->] (O.center) -- (8,0)node[below] {$t$};
  \draw (O.south) -- (bmax.north);
  \draw (bmin.center) -- (8,1);
  \draw (bmax.center) -- (8,4);
  \draw(ri.south) -- (ri.center);
  \draw(di.south) -- (di.center);
  \draw[fill=white] (2.5,0) rectangle (1.5,4);
  \draw[pattern=north west lines] (2.5,0) rectangle (1.5,4);
  \draw[fill=white] (2.5,0) rectangle (6,2);
  \draw[pattern=north west lines] (2.5,0) rectangle (6,2);
  \draw[fill=white] (6,0) rectangle (7,4);
  \draw[pattern=north west lines] (6,0) rectangle (7,4);
  \draw[thick] (t1.south) -- (2.5,4.1) node[above] {$t_1$};
  \draw[thick] (t2.south) -- (6,4.1) node[above] {$t_2$};
 \end{tikzpicture}}
\hfill
\subcaptionbox{\label{fig:conso_CECSPf}}[0.3\linewidth]{
  \begin{tikzpicture}
 [xscale=0.37,yscale=0.3]
 \node[] (O) at (0,0) {};
 \node[label={[shift={(-0.4,-0.4)}]$\bmin$}] (bmin) at (0,1) {};
 \node[label={[shift={(-0.4,-0.4)}]$\bmax$}] (bmax) at (0,4) {};
 \node[label={[shift={(0,-0.8)}]$\LE$}] (di) at (7,0) {};
 \node (t1) at (1,0) {}; 
 \node[label={[shift={(0,-0.8)}]$\ES$}] (ri) at (2.5,0) {};
 \node (t2) at (4,0) {};
 
 \draw[->] (O.center) -- (8,0)node[below] {$t$};
 \draw (O.south) -- (bmax.north);
 \draw (bmin.center) -- (8,1);
 \draw (bmax.center) -- (8,4);
 \draw(di.south) -- (di.center);
 \draw(ri.south) -- (ri.center);
 \draw[fill=white] (t2.center) rectangle (7,4);
 \draw[pattern=north west lines] (t2.center) rectangle (7,4);
 \draw[fill=white] (t2.center) rectangle (3.2,2);
 \draw[pattern=north west lines] (t2.center) rectangle (3.2,2);
 \draw[thick] (t1.south) -- (1,4.1) node[above] {$t_1$};
 \draw[thick] (t2.south) -- (4,4.1) node[above] {$t_2$};
 \end{tikzpicture}}



\subcaptionbox{\label{fig:conso_CECSPg}}[0.3\linewidth]{
  \begin{tikzpicture}
  [xscale=0.37,yscale=0.3]
    \node[] (O) at (0,0) {};
    \node[label={[shift={(-0.4,-0.4)}]$\bmin$}] (bmin) at (0,1) {};
    \node[label={[shift={(-0.4,-0.4)}]$\bmax$}] (bmax) at (0,4) {};
    \node (t1) at (3,0) {};
    \node (t2) at (5,0) {};
    \node[label={[shift={(0,-0.8)}]$\LE$}] (di) at (7,0) {};
    \node[label={[shift={(0,-0.8)}]$\ES$}] (ri) at (1,0) {};
    
    \draw[->] (O.center) -- (8,0)node[below] {$t$};
    \draw (O.south) -- (bmax.north);
    \draw (bmin.center) -- (8,1);
    \draw (bmax.center) -- (8,4);
    \draw[fill=white] (t1.center) rectangle (1,4);
    \draw[pattern=north west lines] (t1.center) rectangle (1,4);
    \draw[fill=white] (t1.center) rectangle (3.7,1);
    \draw[pattern=north west lines] (t1.center) rectangle (3.7,1);
    \draw(di.south) -- (di.center);
    \draw(ri.south) -- (ri.center);
    \draw[thick] (t1.south) -- (3,4.1) node[above] {$t_1$};
    \draw[thick] (t2.south) -- (5,4.1) node[above] {$t_2$};
  \end{tikzpicture}}
\hfill
\subcaptionbox{\label{fig:conso_CECSPh}}[0.3\linewidth]{
  \begin{tikzpicture}
  [xscale=0.37,yscale=0.3]
   \node[] (O) at (0,0) {};
    \node[label={[shift={(-0.4,-0.4)}]$\bmin$}] (bmin) at (0,1) {};
    \node[label={[shift={(-0.4,-0.4)}]$\bmax$}] (bmax) at (0,4) {};
    \node (t1) at (2.5,0) {}; 
    \node[label={[shift={(0,-0.8)}]$\ES$}] (ri) at (1.5,0) {};
    \node (t2) at (6,0) {};
    \node[label={[shift={(0,-0.8)}]$\LE$}] (di) at (7,0) {};

    \draw[->] (O.center) -- (8,0)node[below] {$t$};
    \draw (O.south) -- (bmax.north);
    \draw (bmin.center) -- (8,1);
    \draw (bmax.center) -- (8,4);
    \draw(ri.south) -- (ri.center);
    \draw(di.south) -- (di.center);
    \draw[fill=white] (2.5,0) rectangle (2,2.7);
    \draw[pattern=north west lines] (2.5,0) rectangle (2,2.7);
    \draw[fill=white] (2.5,0) rectangle (6,1);
    \draw[pattern=north west lines] (2.5,0) rectangle (6,1);
    \draw[fill=white] (6,0) rectangle (7,4);
    \draw[pattern=north west lines] (6,0) rectangle (7,3);
    \draw[thick] (t1.south) -- (2.5,4.1) node[above] {$t_1$};
    \draw[thick] (t2.south) -- (6,4.1) node[above] {$t_2$};
  \end{tikzpicture}}
\hfill
\subcaptionbox{\label{fig:conso_CECSPi}}[0.3\linewidth]{
\begin{tikzpicture}
 [xscale=0.37,yscale=0.3]
 \node[] (O) at (0,0) {};
 \node[label={[shift={(-0.4,-0.4)}]$\bmin$}] (bmin) at (0,1) {};
 \node[label={[shift={(-0.4,-0.4)}]$\bmax$}] (bmax) at (0,4) {};
 \node[label={[shift={(0,-0.8)}]$\LE$}] (di) at (7,0) {};
 \node (t1) at (2.5,0) {}; 
 \node (t2) at (4,0) {};
 \node[label={[shift={(0,-0.8)}]$\ES$}] (ri) at (1,0) {};
 
 \draw[->] (O.center) -- (8,0)node[below] {$t$};
 \draw (O.south) -- (bmax.north);
 \draw (bmin.center) -- (8,1);
 \draw (bmax.center) -- (8,4);
 \draw(di.south) -- (di.center);
 \draw(ri.south) -- (ri.center);
 \draw[fill=white] (t2.center) rectangle (7,4);
 \draw[pattern=north west lines] (t2.center) rectangle (7,4);
 \draw[fill=white] (t2.center) rectangle (3.2,1);
 \draw[pattern=north west lines] (t2.center) rectangle (3.2,1);
 \draw[thick] (t1.south) -- (2.5,4.1) node[above] {$t_1$};
 \draw[thick] (t2.south) -- (4,4.1) node[above] {$t_2$}; 
 \end{tikzpicture}}
\caption{Les différentes configurations menant à une consommation minimale}
\label{fig:conso_CECSP}
\end{figure}
	
Il est facile de calculer l'expression de la consommation minimale
d'énergie dans un intervalle pour une fonction $f_i$ croissante. En
effet, les différentes configurations possibles étant toujours celle
où la tâche est exécutée à son rendement maximum en dehors de
l'intervalle ${[}t_1,t_2{[}$, il suffit de retrancher à $W_i$
l'énergie produite par l'exécution de la tâche en dehors de
${[}t_1,t_2{]}$. Il existe une exception à cette règle, produite par
la contrainte du rendement minimal, mais ce cas est facilement traiter
puisque l'énergie minimale correspond alors à la configuration où la
tâche est exécutée à son rendement minimal durant ${[}t_1,t_2{[}$. 

Pour donner l'expression mathématique de $\wb$ , nous introduisons
trois notations. $\wbLS$ (respectivement $\wbRS$ et $\wbCS$)
correspond à la quantité d'énergie apportée à l'activité $i$ dans
l'intervalle $[t_1,t_2[$ quand l'activité est calée à gauche
(respectivement calée à droite et centrée). Formellement, ces trois
quantités peuvent être exprimer de la manière suivante:
\begin{align}
\wbLS&= \max\left(0\, ,\, W_i- f_i(\bmax)\max(0,t_1 -
\ES)\right) \label{eq:LSnrj_CECSP}\\
\wbRS&= \max\left(0\, ,\, W_i- f_i(\bmax)\max(0,
\LE-t_2)\right)\label{eq:RSnrj_CECSP}\\
\wbCS&=\max\left( f_i(\bmin) * (t_2-t_1)\, ,\, W_i - f_i(\bmax)\max(0,
d_i - r_ i - t_2 + t_1) \right)\label{eq:CSnrj_CECSP}
\end{align}
Alors, l'expression de l'énergie minimale est le minimum de ces trois
quantités, i.e.
\begin{equation}
\wb=\min\left(\, \wbLS\, ,\,\wbRS\, ,\,\wbCS \, \right)
\end{equation} 

Nous allons maintenant utiliser l'expression de $\wb$ et la fonction
$f_i$ pour calculer $\bb$.

\subsubsection{Consommation minimale de la ressource}
	
Pour calculer l'expression de $\bb$, nous allons utiliser les
propriétés de la fonction $f_i$. En effet, une fois que nous avons
calculer $\wb$, nous voulons savoir quelle est la quantité minimale de
ressource que nous devons fournir à la tâche pour obtenir cette
quantité d'énergie dans l'intervalle ${[}t_1,t_2{[}$. Dans le cas où
la fonction $f_i$ est l'identité, nous avons que: $\bb=\wb$. Dans les
deux autres cas, i.e. $f_i$ est affine et $f_i$ est concave et affine
par morceaux, soit $I=[t_1,t_2[ \cap [\ES,\LE{[}$, alors trouver $\bb$
revient à résoudre le programme suivant:
\begin{align}
  \text{minimiser }   & \int_{I} b_i(t)dt \label{eq:conv_obj} \\
  \text{sous } & \int_{I} f_i(b_i(t))dt \ge \wb \label{eq:conv_nrj}\\
                     & \bmin \le b_i(t) \le \bmax \label{eq:conv_min}
\end{align}
En effet, l'objectif de ce programme est de minimiser la quantité de
ressource consommée dans l'intervalle $[t_1,t_2[$
(équation~\eqref{eq:conv_obj}), tout en s'assurant que l'énergie
requise, i.e. $\wb$, est bien apportée à l'activité
(équation~\eqref{eq:conv_nrj}). 

Nous utilisons ensuite le lemme~\ref{lemmaEn} pour simplifier ce
programme. En effet, le lemme affirme que, si $f_i$ est affine ou
concave et affine par morceaux, alors si une solution optimale
$b_i(t)$ pour le programme ci-dessus, cette solution peut être
transformée en une autre solution optimale vérifiant la propriété que
$b_i(t)$ soit constante et inférieure ou égale à
$\frac{\int_{I}b_i(t)dt}{|I|}$. Nous noterons $b_i$ cette
constante. Le programme simplifié s'écrit de la manière suivante:
\begin{align}
  \text{minimiser }   & b_i|I| \label{eq:convSimp_obj} \\
  \text{sous } & f_i(b_i)|I| \ge \wb \label{eq:convSimp_nrj}\\
                     &\bmin \le  b_i \le \bmax \label{eq:convSimp_nrj}
\end{align}


Ce programme peut être réécrit de la façon suivante: 
\[ b_i=\min\left( b \in [\bmin,\bmax]\ |\ f_i(b) \ge
    \frac{\wb}{|I|}\right)
\]

Nous pouvons remarquer que, si l'on suppose que $f_i(\bmax) (\LE - \ES
) \ge W_i$, nous sommes sûrs que la solution optimale vérifie $b_i \le
\bmax$. En effet, ceci est dû au fait qu'exécuter l'activité à une
faible consommation de ressource a un meilleur rendement que son
exécution à un rendement plus élevé, i.e. la fonction
$\frac{f_i(b_i(t))}{b_i(t)}$ est décroissante. 

De ce fait, seule la borne inférieure sur $b_i$, $\bmin$, doit être
considérée. Or, pour apporter l'énergie $\wb$ à l'activité dans
l'intervalle $I$, sa consommation $b_i$ doit vérifiée $f_i(b_i) \ge
\frac{\wb}{|I|}$ et donc $b_i \ge
f_i^{-1}\left(\frac{\wb}{|I|}\right)$. En couplant cette contrainte
avec la contrainte de consommation minimale, nous obtenons, si $\bmin
\neq 0$:
\[
 b_i= \max\left(\, \bmin \, ,\, 
    f_i^{-1}\left(\frac{\wb}{|I|}\right)\, \right)
\]
Et si $\bmin = 0$, nous avons:
\[
 b_i=  f_i^{-1}\left(\frac{\wb}{|I|}\right)
\]
Nous allons maintenant donner l'expression de $\bb$  en fonction des
coefficients $a_{ip}$  et $c_{ip}$ de la fonction $f_i$. Pour
simplifier la compréhension, nous commençons par détailler cette
expression dans le cas où $f_i$ est affine puis nous la généraliserons
dans le cas où $f_i$ est concave et affine par morceaux. 

\paragraph{Fonctions affines}

Dans ce paragraphe, nous allons décrire l'expression de $\bb$ dans le
cas où la fonction $f_i$ est affine, i.e. de la forme $a_ib+c_i$. Dans
ce cas-là, nous avons:  
\[f_i^{-1}\left( \frac{\wb}{|I|}\right)= \frac{ \wb- c_i|I|}{a_i|I|}
\]
Et donc, si $\bmin\neq 0$: 
\begin{equation}
\bb= \max\left(b_i^{min} \frac{\wb}{f_i(b_i^{min})} \, ,\,  
  \frac{1}{a_i}\left(\wb-|I|c_{i})\right)\right)
\end{equation}
Et si $\bmin=0$: 
\[
\bb=  \frac{1}{a_i}\left(\wb-|I|c_{i})\right)
\]
Notons que, le premier cas correspond au cas où l'intervalle $I$ est
suffisamment grand pour ordonnancer l'activité $i$ à $\bmin$ et lui
apporter l'énergie requise, i.e. $\wb$. En effet, comme ordonnancer
$i$ à $\bmin$ a la meilleur rendement, si l'on peut, i.e. l'intervalle
est assez grand, c'est cette valeur que l'on va choisir pour
$b_i$. Si, au contraire, l'intervalle n'est pas suffisamment grand,
c'est $f_i^{-1}(\wb/|I|) \times |I|$ que l'on va choisir. Ceci est
détaillé dans l'exemple~\ref{ex:convLin_CECSP}.

\begin{ex}
\label{ex:convLin_CECSP}

\end{ex}

Ce raisonnement peut être étendu dans le cas où $f_i$ est concave et
affine par morceaux. 

\paragraph{Fonctions concave et affine par morceaux}

Dans le cas des fonctions concaves et affines par morceaux, le
raisonnement décrit au paragraphe précédent peut être généralisé. En
effet, comme dans le cas des fonctions affines, la fonction de
rendement, $f_i(b_i(t))/b_i(t)$, est décroissante. Donc, il est
toujours préférable d'exécuter l'activité avec une consommation de
ressource aussi basse que possible. La condition selon laquelle on
peut ou non exécuter l'activité $i$ avec une consommation $b_i$ dépend
donc de la taille de l'intervalle $I$. 

Si $\bmin\neq 0$ et si l'intervalle est suffisamment grand, on va donc
exécuter l'activité à $\bmin$. Sinon, et si l'intervalle est
suffisamment grand, nous essayons d'exécuter l'activité avec une
consommation $b_i \in ]\bmin,x_2^i]$, etc. Dans le cas où $\bmin \neq
0$, l'expression de $\bb$ est formalisée ci-dessous:
\[f_i^{-1}\left(\frac{\wb}{|I|}\right)=\left\{
\begin{aligned}
\bmin & \qquad & \text{si } |I| \ge \frac{\wb}{f_i(\bmin)}\\
\frac{\wb -c_{i1}|I|}{a_{i1}|I|} & \qquad & \text{si } \frac{\wb}{f_i(\bmin)}
> |I| > \frac{\wb}{f_i(x^i_2)} \\
\frac{\wb -c_{i2}|I|}{a_{i2}|I|} & \qquad & \text{si }  \frac{\wb}{f_i(x^i_2)}
\ge |I| > \frac{\wb}{f_i(x^i_3)}\\
 & \qquad \vdots& \\
\frac{\wb -c_{iP_i}|I|}{a_{iP_i}|I|} & \qquad & \text{si }  \frac{\wb}{f_i(x^i_{P_i}}
\ge |I| > \frac{\wb}{f_i(\bmax)}\end{aligned}
\right.
\]
Et ceci, nous donne l'expression de $\bb$ suivante:
\begin{equation}
\bb= 
  \max \left\{b_i^{min} \frac{\wb}{f_i(b_i^{min})},
  \max_{ p \in \P_i} \left(\frac{1}{a_{ip}}(\wb-|I|c_{ip})\right)\right\}
\end{equation}
Dans le cas où $\bmin=0$, nous avons: 
\[
\bb= 
  \max_{ p \in \P_i} \left(\frac{1}{a_{ip}}(\wb-|I|c_{ip})\right)
\]
Un exemple du calcul de $\bb$ dans le cas d'une fonction $f_i$ concave
et linéaire par morceaux est décrit dans
l'exemple~\ref{ex:convPWL_CECSP}.

\begin{ex}
\label{ex:convPWL_CECSP}

\end{ex}

\subsection{Les ajustements de borne}
\label{sec:adjustment_tw}
 
Dans ce paragraphe nous décrivons les ajustements qui peuvent être
faits sur les fenêtre de temps des activités. Tout d'abord, nous
introduisons les notations suivantes: $\bbLS$ (respectivement $\bbRS$
et $\bbCS$) correspond à la quantité de ressource consommée par
l'activité $i$ dans l'intervalle $[t_1,t_2[$ quand l'activité est
calée à gauche (respectivement calée à droite et
centrée). Formellement, ces trois quantités peuvent être exprimer de
la manière suivante:
\begin{align}
\bbLS&=   \max \left\{b_i^{min} \frac{\wbLS}{f_i(b_i^{min})},
  \max_{ p \in \P_i} \left(\frac{1}{a_{ip}}(\wbLS-|I|c_{ip})\right)\right\}\\
\bbRS&=   \max \left\{b_i^{min} \frac{\wbRS}{f_i(b_i^{min})},
  \max_{ p \in \P_i} \left(\frac{1}{a_{ip}}(\wbRS-|I|c_{ip})\right)\right\}\\
\bbCS&=  \max \left\{b_i^{min} \frac{\wbCS}{f_i(b_i^{min})},
  \max_{ p \in \P_i} \left(\frac{1}{a_{ip}}(\wbCS-|I|c_{ip})\right)\right\}
\end{align}

Nous allons décrire deux règles d'ajustements pour les fenêtres du
temps du \CECSP. La première permet d'ajuster $\LS$ et la seconde
$\ES$ et nous avons des ajustements symétriques pour $\EE$ et $\LE$. 

Pour les premier ajustements décrits, nous allons essayer de faire
démarrer l'activité après $t_1$et si la quantité de ressource
disponible n’est pas suffisante pour ordonnancer les consommations
minimales de toutes les activités – excepté celle de l’activité i qui
est remplacée par $\bbRS$ – alors, nous pouvons déduire que l’activité
doit commencer avant $t_1$.

\begin{reg}
\label{reg:ajust_CECSP}
S’il existe un intervalle $[t_1 , t_2 [$ avec $t_1 > \ES$ et une
activité $i$ pour lesquels:
\[ \sum_{\substack{j \in \A\\ j\neq i}} \bb[j] + \bbRS > R(t_2-t_1)
\]
alors, on a:
\[ \LS \le t_1 - \frac{1}{\bmax}\left(\sum_{\substack{j \in \A\\ j\neq i}} \bb[j] + \bbRS - R(t_2-t_1)\right)
\]
\end{reg}

\begin{proof}
Soit $i$ et $[t_1,t_2[$ vérifiant la condition de la
règle~\ref{reg:ajust_CECSP}. Supposons que l'activité $i$ démarre à
$t_1$. 

La consommation minimale de $i$ dans $[t_1,t_2[$ est donc $\bbRS$. On
a donc que $\sum_{\substack{j \in \A\\ j\neq i}} \bb[j] + \bbRS $ est
la consommation minimale de toutes les activités dans $[t_1,t_2[$
quand l'activité $i$ commence à $t_1$. Donc, si cette quantité est
plus grande que la quantité de ressource disponible dans l'intervalle
$[t_1,t_2[$, i.e. $R(t_2-t_1)$, on obtient une contradiction avec le
fait que l'instance soit réalisable et $i$ doit commencer avant
$t_1$. 

Pour calculer la nouvelle date de début au plus tard de $i$,
remarquons que la quantité de ressource devant être consommée avant
$t_1$ est d'au moins $\sum_{\substack{j \in \A\\ j\neq i}} \bb[j] +
\bbRS - R(t_2-t_1)$, i.e. ce qui ne ``rentrait'' pas dans
$[t_1,t_2[$. Comme nous cherchons une borne supérieure sur la date de
début de $i$, nous cherchons donc à ordonnancer cette partie de
l'activité le plus rapidement possible. 

Le temps minimal requis pour ordonnancer cette
consommation étant obtenu en exécutant l'activité à $\bmax$, nous
obtenons comme borne supérieure sur la date de début de $i$:
$1/\bmax \times \left(\sum_{\substack{j \in \A\\ j\neq
i}} \bb[j] + \bbRS - R(t_2-t_1)\right)$
\end{proof}

De manière similaire, nous avons les ajustements suivants sur la date
de début au plus tôt d'une activité. 
\begin{reg}
\label{ajustRi_CECSP}
S'il existe un intervalle $[t_1,t_2[$ avec $ t_2 > \LE$ et une
activité $i$ telle que $\bmin \neq 0$ pour lesquels:
  \[ \sum_{\substack{j \in \A \\ j \neq i}} \bb[j] +
    \min(\bbCS,\bbLS) > R (t_2-t_1)\] 
  alors
  \[ \ES \ge t_2 - \frac{1}{\bmin} \left(R (t_2-t_1) -\sum_{\substack{j
\in \A \\ j \neq i}} \bb[j] \right) \]
\end{reg}

\begin{proof}
Soit $i$ et $[t_1,t_2[$ vérifiant la condition de la
règle~\ref{reg:ajust_CECSP}. Nous allons décider si $i$ peut commencer
avant $t_1$. Les configurations où l'activité $i$ démarre avant $t_1$
et consomme le moins de ressource possible dans $[t_1,t_2[$ sont les
configurations où $i$ est soit calée à gauche, soit centrée.

La consommation minimale de $i$ dans l'intervalle $[t_1,t_2[$ quand
$i$ commence avant $t_1$ est donc $\min( \bbLS, \bbCS)$. On a donc que
$\sum_{\substack{j \in \A\\ j\neq i}} \bb[j] + \min( \bbLS, \bbCS) $
est la consommation minimale de toutes les activités dans $[t_1,t_2[$
quand l'activité $i$ commence avant $t_1$. Donc, si cette quantité est
plus grande que la quantité de ressource disponible dans l'intervalle
$[t_1,t_2[$, i.e. $R(t_2-t_1)$, on obtient une contradiction avec le
fait que l'instance soit réalisable et $i$ doit commencer après $t_1$.

Pour calculer la nouvelle date de début au plus tard de $i$,
remarquons que la quantité de ressource disponible dans $[t_1,t_2[$
pour exécuter $i$ est de $R(t_2-t_1) -\sum_{\substack{j \in \A\\
j\neq i}} \bb[j]$. Comme nous cherchons une borne inférieure sur la date de
début de $i$, nous cherchons donc à ordonnancer cette partie de
l'activité le moins rapidement possible. 

Le temps maximal requis pour ordonnancer cette consommation étant
obtenu en exécutant l'activité à $\bmin$, et, comme $\bmin \neq 0$,
l'activité ne peut être préemptée, nous savons que l'activité doit
être en cours à $t_2$. Nous obtenons alors comme borne inférieure sur
la date de début de $i$:
$t_2 - 1/\bmin \left(R (t_2-t_1) -\sum_{\substack{j
\in \A \\ j \neq i}} \bb[j] \right) $
\end{proof}

\begin{ex}
 Considérons l'instance à $3$ activités et avec $R=5$ suivante:

  \begin{center}
    \begin{tabular}{|P{1cm}|P{1cm}P{1cm}P{1cm}P{1cm}P{1cm}P{2cm}|} 
      \hline 
      act. & \ES & \LE & W_i & \bmin& \bmax & f_i(b_i(t))\\ 
      \hline 
      1 & 0 & 6 & 28 & 1 & 5 & 2b_i(t)+1\\ 
      2 & 2 & 6 & 32 & 2 & 5 & b_i(t)+5\\ 
      3 & 2 & 5 & 6 & 2 & 2 & b_i(t)\\ 
      \hline
    \end{tabular}
  \end{center}
Une solution réalisable est décrit par la figure~\ref{exSol}.
\begin{figure}[!htb]
  \begin{center} 
\begin{tikzpicture}
[scale=0.7]
\node (O) at (0,0) {};
\node (2) at (4,2) {\LARGE $2$};
\node (1) at (1,2) {\LARGE $1$};
\node (3) at (3.5,4) {\LARGE $3$};
\node[label={[shift={(-0.4,0)}]$B=5$}] (B) at (0,5) {};


\node (r1) at (0,-0.5) {$\ES[1]$}; 
\node (r2) at (2,-0.5) {$\ES[2]$};
\node (r3) at (2,-0.9) {$\ES[3]$};
\node (d1) at (6,-0.5) {$\LE[1]$};
\node (d2) at (6,-0.9) {$\LE[2]$};
\node (d3) at (5,-0.5) {$\LE[3]$};


\draw[->,>=latex] (6,0) -- (6.5,0) node[below] {$t$};

%\draw (0,0) rectangle (6,5);
\draw (4,0) -- (6,0) -- (6,5) -- (5,5);

   \draw (2,5) -- (2,1) -- (4,1) -- (4,0) -- (0,0) -- (0,5) -- cycle;

\draw (2,3) -- (5,3) -- (5,5) -- (2,5) ;


\foreach \i in {0,...,5}
{
  \draw (\i,-0.3) -- (\i,0);
  \draw (-0.3,\i) -- (0,\i);
}

 \draw (6,-0.3) -- (6,0);

\end{tikzpicture}
    \caption{Une solution réalisable du \CECSP}
    \label{exSol}
  \end{center}
\end{figure}

Nous allons ajuster la fenêtre de temps de l'activité $1$. Pour cela,
considérons l'intervalle $[t_1,t_2[=[2,5[$. On a:
\begin{itemize}
  \item $\bb[2][2][5]=7$
  \item $\bb[3][2][5]=6$
  \item $\bbRS[1][2][5]=7$ et $\bbCS[1][2][5]=3$
  \end{itemize}
Nous avons donc: $\sum_{j\in A;\ j\neq i} \bb[j][2][5] +
\bbRS[1][2][5]= 7+7+6 =20 > 5 (5-2) =15$. Donc $\LS[1]$ peut être ajusté
à $ 2 -\frac{1}{5} (20-15) = 1$ .
En effet, la quantité de ressource disponible dans l'intervalle
$[2,5[$ pour ordonnancer l'activité $1$ est de $15-7-6=2$. Donc, si
l'activité $1$ commence après $t_1$ , elle ne peut consommer que $2$
unités de ressource dans $[2,5[$.  Or, il faudrait $\bbRS[1][2][5]= 7$
unités de ressource disponible dans $[2,5[$ pour que $1$ puisse
démarrer avant $t_1$. Donc l'activité $1$ ne peut commencer après $t_1$
et, au moins $5= 7-2$ unités de ressource doivent être exécutées avant
$t_1$. Donc $\LS[1]$ peut être ajusté à $2 - \frac{1}{5}(20-15) =1 $.

De plus, $\LE[1]$ peut être ajusté à $t_1 + 1/ \bmin \times (R(t_2-t_1) -
\sum_{j\in A;\ j\neq i} \bb[j] )= 2+(15-13)=4$. En effet, si l'activité
$1$ finit après $t_2$, alors elle doit consommer au moins
$\min(\bbRS[1][2][5],\bbCS[1][2][5])=\min(11,6)=6$ unités de ressource
dans l'intervalle $[2,5[$. Or, seulement $2$ unités sont
disponibles. L'activité $1$ ne peut donc pas finir après $t_2$ et nous
pouvons ajuster $\LE[1]$ à $4$.
\end{ex}

Nous avons montrer qu'il était possible de calculer, étant donné un
intervalle et une activité, sa consommation minimale et les
ajustements pouvant être faits sur ses fenêtres de temps en
$O(1)$. {\'E}tant donné un intervalle, la fonction de marge ainsi que
tout les ajustements peuvent donc être calculés en $O(n)$. Dans le
paragraphe suivant, nous montrons qu'il suffit d'exécuter l'algorithme
de vérification, ainsi que les ajustements sur un nombre polynomial
d'intervalles $[t_1,t_2[$.

\subsection{Caractérisation des intervalles d'intérêt}

\subsubsection{Intervalles d'intérêt pour l'algorithme de
  vérification}

Dans un premier temps, nous prouvons que nous pouvons seulement
considérer un nombre polynomial d'intervalles pour détecter une
incohérence. En effet, à cause de la nature continue du problème, on
aurait pu être amené à considérer un nombre potentiellement infini
d'intervalles. 

\begin{theo}
L'algorithme de vérification du raisonnement énergétique a seulement
besoin d'être appliqué sur un nombre polynomial d'intervalles $[t_1,t_2[$.
\end{theo}

\begin{proof}
La fonction de marge étant la différence entre une fonction affine,
$R(t_2-t_1)$, et la somme de fonction affine par morceaux,
$\bb$, c'est aussi une fonction affine par morceaux à deux
dimensions. Le minimum de cette fonction est donc atteint en un point
extrême d'une des polygones convexes dans lequel elle est
affine. 

Les segments de définitions, i.e. les segments où l'expression de la
fonction est la même, de la fonction de marge sont les mêmes que ceux
de la somme des consommations individuelles de chaque activité. Donc,
d'un point de vue géométrique, un point extrême de la fonction de marge
est l'intersection de deux segments chacun correspondant à un segment
de définition de la fonction de consommation d'une activité. 

Donc, pour calculer la fonction de marge, il suffit d'énumérer tous
ces points d'intersections et, pour chacun d'entre eux, d'appliquer le
test de satisfiabilité décrit par le
théorème~\ref{th:ER_CECSP}. Comme, pour chaque activité, il existe un
nombre constant de segments de définition, le nombre de points
d'intersection est de l'ordre de $O(n^2)$. 
\end{proof}


\subsubsection{Intervalles d'intérêt pour les ajustements}


\section{Modèle de programmation par contraintes pour le cas discret}
%
\clearemptydoublepage%
%TODO justification modèle RCPSP discret + phrases de transition
%entre modele discret et modele à evenement + justification
%présentation des modèle

%!TeX root =../main_file.tex 

\newpage
\begin{minipage}{0.95\linewidth}
\part{Programmation linéaire en nombres entiers}
\vspace{15mm} % l'espacement souhaité
\parttoc 
\end{minipage}


\chapter{Programmation linéaire et ordonnancement
  de projet}
\label{sec:PLNE_RCPSP}
%!TeX root =../main_file.tex

\section{La programmation linéaire en nombres entiers}
\label{sec:PLNE}
Dans cette section, nous présentons les concepts de base de la
programmation linéaire ainsi que les notions et outils qui seront
utilisés dans la suite de cette partie. Cette présentation n'étant que
partielle, nous renvoyons le lecteur à~\cite{LP} pour une description
plus détaillée des différentes techniques existantes.

\subsection{La programmation linéaire}

La programmation linéaire vise à résoudre des problèmes
d'optimisation ayant la particularité de pouvoir s'exprimer à
l'aide de contraintes, se présentant sous la forme d'inégalités
et/ou d'égalités, linéaires en fonction des variables du problème.
De plus, la fonction objectif du problème, i.e. ce que l'on
cherche à maximiser/minimiser, s'écrit aussi sous la forme d'une
fonction linéaire.

Sous forme canonique, un programme linéaire (PL) s'écrit de la manière
suivante: 
 \[ \begin{array}{lcl}
\text{maximiser } & & \displaystyle cx\\ 
\text{tel que }& & \displaystyle Ax \le b\\
 & & \displaystyle x \in \mathbb{R}^n_+
 \end{array}
\]
avec $c \in \mathbb{R}^n$, $b \in \mathbb{R}^m$ et $A \in
\mathbb{R}^{m,n}$. 

Nous pouvons supposer, sans perte de généralité, que les variables du
problème sont positives et que la fonction objectif $x\rightarrow cx$
doit être maximisée. Les vecteurs $x$ de $\mathbb{R}^n$ qui vérifient
$Ax \le b$ sont appelés les solutions réalisables du problème et les
vecteurs $x^*$ qui, en plus, maximisent le critère $cx^*$ sont appelés
solutions optimales.    

La force de ces formulations repose sur le fait que l'ensemble des
contraintes du problème définit alors un polyèdre convexe nommé
polyèdre des solutions réalisables. De plus, si ce polyèdre est non
vide, alors toute solution optimale se trouve forcément sur un sommet,
appelé point extrême, de celui-ci. Les solutions optimales peuvent aussi
se trouver sur une face du polyèdre, i.e. l'intersection de celui-ci
avec un plan défini par une des contraintes du programme linéaire $\{x
\in \mathbb{R}^n\ | \ A_jx=b_j\}$. 

Ceci a permis la mise en place de plusieurs algorithmes pour résoudre
ces programmes linéaires. Un des plus efficaces en pratique et donc
des plus utilisés est l'algorithme du Simplexe. Le principe de cet
algorithme est le parcours "intelligent" des points extrêmes du
polyèdre des solutions admissibles: à chaque itération, l'algorithme
se "déplace" vers un sommet adjacent de meilleur (ou même)
coût. L'algorithme se termine donc lorsque tous les sommets adjacents
possèdent un coût moins bon que le sommet courant.

Un des inconvénients de cet algorithme est qu'il a, dans le pire des
cas, une complexité exponentielle. D'autres algorithmes permettent de
résoudre un programme linéaire en temps polynomial. Cependant, en
pratique, l'algorithme du simplexe reste plus efficace et c'est
souvent ce dernier qui est implémenté dans les solveurs d'optimisation
linéaire. 

\subsection{Contrainte d'intégrité}

De nombreux problèmes d'optimisation combinatoire ne peuvent s'écrire
sous la forme d'un programme linéaire. En effet, pour des problèmes
tels que les problèmes d'ordonnancement, tout ou partie des
variables doivent être restreintes à prendre des valeurs entières.  

Quand toutes les variables du problème sont contraintes à prendre des
valeurs entières, on parle de programme linéaire en nombres
entiers (PLNE). Lorsque seulement une partie des variables sont
autorisées à prendre des valeurs réelles, on parle de programme
linéaire mixte (PLM). 

De manière générale, un programme linéaire mixte 
peut s'écrire de la façon suivante: 

\[ \begin{array}{lcl}
\text{maximiser } & & \displaystyle \sum_{i=1}^p c_ix_i +
\sum_{j=1}^{n-p} f_jy_j\\ \text{tel que }& & \displaystyle
\sum_{i=1}^p a_{ki}x_i + \sum_{j=1}^{n-p} d_{kj}y_j \le b_k, \quad
k=1,\dots,m\\ & & \displaystyle x_i \ge0,\quad i=1,\dots,p\\ & &
\displaystyle y_j \in \mathbb{N},\quad j=1,\dots,n-p 
\end{array}
\] $y_j,\ j=1,\dots,n-p$ sont les variables de décisions ne
pouvant prendre que des valeurs entières tandis que les variables
$x_i,\ i=1,\dots,p$ peuvent prendre n'importe quelles valeurs
réelles positives.

La difficulté de la résolution de ces programmes repose sur le fait
que l'ensemble des solutions réalisables du problème ne forme plus un
polyèdre convexe et les algorithmes traditionnels ne peuvent être
appliqués directement dans ce cas. Le problème de décision associé à
un PLNE, ou à un PLM, est NP-complet~\cite{NP_bible}. De ce fait,
plusieurs techniques permettant de trouver la solution optimale en
temps raisonnable ont été développées.

\subsection{Techniques de résolution}

Parmi les techniques les plus utilisées, on retrouve les techniques
utilisant des algorithmes de recherche arborescente par séparation et
évaluation ainsi que des méthodes de coupes.

Le principe de la {\it recherche arborescente par séparation et
évaluation} repose sur deux idées principales: 
\begin{itemize}
\item {\bf la séparation} qui consiste à décomposer selon une
partition l'ensemble des solutions en plusieurs sous-ensembles;
\item {\bf l'évaluation} qui consiste à examiner la qualité des
solutions d'un sous-ensemble de façon optimiste, i.e. trouver une
borne supérieure (resp. inférieure) de la meilleure solution de ce
sous-ensemble lorsqu'il s'agit d'un problème de maximisation
(resp. minimisation).
\end{itemize} 
L’algorithme propose de parcourir l’arborescence des solutions
possibles en évaluant chaque sous-ensemble de solutions de façon
optimiste. Lors de ce parcours, il maintient la valeur de la meilleure
solution trouvée jusqu’à présent. Quand l’évaluation d’un
sous-ensemble donne une valeur plus faible (plus forte pour un
problème de minimisation) ou égale, il est inutile d’explorer plus loin ce
sous-ensemble. L’ensemble des solutions admissibles est ainsi
représenté par une arborescence dans laquelle un grand nombre de nœuds
peuvent être éliminés. L’évaluation des sous-problèmes se fait
typiquement par {\it relaxation linéaire} (on ignore la contrainte
d’intégrité) en utilisant le Simplexe.

Les {\it méthodes de coupes} cherchent à caractériser le plus petit
polyèdre contenant l'ensemble de toutes les solutions
admissibles. Pour cela, un ensemble de coupes, i.e. des inégalités
permettant de supprimer une partie du polyèdre ne contenant que des
solutions non entières, est généré. Si ces coupes sont assez fortes,
le polyèdre des solutions admissibles du modèle devient alors
exactement l'enveloppe convexe des solutions admissibles
entières. Dans ce cas, la résolution de la relaxation du programme
linéaire  réduit à cet ensemble donne une solution optimale
entière.



La sous-section suivante est dédiée aux modèles de programmation linéaire
mis en place pour le problème d'ordonnancement de projet sous
contraintes de ressource. 

%!TeX root =../main_file.tex 
\section[Application au RCPSP]{Application à l'ordonnancement de projet sous contraintes de
  ressources}   
\label{sec:PLNE_ordo_res}

La facilité de modélisation des problèmes d'ordonnancement sous
contraintes de ressources à l'aide de la programmation linéaire en a
fait une des principales méthodes de résolution pour ces derniers. De
ce fait, de nombreuses techniques de modélisation ont été développées
et la littérature sur ce sujet est très vaste. Dans ce paragraphe,
nous nous intéressons principalement aux modèles développés pour le
problème d'ordonnancement de projet sous contraintes de ressources, le
\RCPSP. Ces modèles sont regroupés en plusieurs familles:
\begin{itemize}
\item les formulations indexées par le temps (ou à temps discret)
sont, en général, des formulations purement discrètes. Ces
formulations ont la particularité d'avoir les meilleures relaxations
linéaires au détriment du nombre de
variables~\cite{CAVT,ex_RCPSP_discret};
\item les formulations avec variables de séquencement sont des
formulations qui ont l'avantage d'être plus compactes que celles
indexées par le temps mais possèdent de moins bonnes relaxations
linéaires que ces dernières~\cite{AVT,AMR};
\item les formulations à événements, plus récentes, possèdent aussi de
faibles relaxations linéaires mais ont un nombre polynomial de
variables, moins élevé que  les formulations à temps
discret~\cite{modele_RCPSP}.
\end{itemize}

Dans ce paragraphe, nous présentons quelques modèles de
programmation linéaire en nombres entiers ou mixte développés pour les
problèmes d'ordonnancement sous contraintes de ressources. Nous nous
intéressons particulièrement aux modèles décrits
dans~\cite{modele_RCPSP} pour le \RCPSP. En effet, ce sont ces modèles
qui seront adaptés dans le cas du \CECSP. Nous commençons donc, dans
un premier temps, par présenter les modèles indexés par le temps puis,
dans un second temps, nous présentons les modèles à événements. Les
modèles avec variables de séquencement ne sont pas traités ici. Des
exemples de telles formulations dans le cadre du \RCPSP~sont décrites
dans~\cite{ADN}.

\subsection{Modèles indexés par le temps}
\label{sec:time_RCPSP}

Une des méthodes permettant la modélisation des problèmes
d'ordonnancement de projet consiste à discrétiser l'horizon de
temps. Ces modèles de programmation linéaire en nombres entiers
indexés par le temps, aussi appelés modèles à temps discret, ont été
largement étudiés dans le cadre des problèmes d'ordonnancement en
général. Ceci étant principalement dû à leur relativement forte
relaxation linéaire et à la facilité d’extension à de nombreux
objectifs et contraintes.

Dans ces formulations, l'horizon de temps est discrétisé en périodes de
temps, i.e en intervalles, généralement unitaires, et un
ensemble de variables est défini pour modéliser le statut d'une
activité $i$ en période $t$, en cours, commencé ou terminé. Pour
chaque activité $i$ et pour chaque période $t$ (discret), une variable
$x_{it} \in \{0,1\}$ est donc définie.

Dans la formulation présentée pour le \RCPSP, cette variable prendra
la valeur $1$ si et seulement si l'activité $i$ commence au début de
la période $t$. Le nombre de ces variables dépend donc de la taille de
l'horizon de temps du problème, i.e. du nombre de périodes, qui peut
être, pour certains problèmes, aussi grand que la somme de toutes les
durées des activités. Le calcul d'une borne supérieure raisonnable
pour la durée du projet $T$ est donc indispensable.  Dans la suite,
nous noterons ${\cal H}=\{0,\dots,T-1\}$ l'ensemble des périodes du
problème.

En pratique, pour chaque activité, nous calculons l'ensemble des
périodes de temps pendant lesquels l'activité peut commencer. Pour
cela, nous ajoutons deux activités fictives au problème: $0$ et
$n+1$. Ces activités ont les caractéristiques suivantes: leur
durée est égale à 0, elles ne consomment aucune ressource durant
leur exécution et l'activité $0$ (resp. $n+1$) doit être
ordonnancée avant (resp. après) toutes les autres activités. Ces
deux activités fictives vont nous permettre d'associer à chaque
activité $i$, une date de début au plus tôt $\ES$ et une date de
début au plus tard $\LS$. Donc, l'intervalle $[\ES,\LS]$ est la
fenêtre de temps pendant laquelle l'activité $i$ peut commencer.

Pour calculer ces fenêtres de temps, remarquons que, si chaque
arc $(i,j)$ du graphe des précédences $G$ est pondéré par
$p_i$, la date de début au plus tôt de $i$, $\ES$ peut prendre la
valeur du plus long chemin entre l'activité $0$ et l'activité $i$
et la date de début au plus tard $\LS$, $T$ moins la valeur du
plus long chemin entre $i$ et $n+1$.

Dans la suite, nous notons ${\cal A}$ l'ensemble des
activités réelles du projet et par $V={\cal A} \cup \{0,n+1\}$
l'ensemble des activités réelles et fictives du projet.

Nous pouvons maintenant exhiber la formulation à temps discret
suivante~\cite{ex_RCPSP_discret}:
{\small \begin{align} &\text{min }
\sum_{t \in {\cal H}} tx_{n+1,t} \label{obj_RCPSP_discret}\\
&\sum_{t \in \H} tx_{jt} - \sum_{t \in {\cal H}} tx_{it} \ge p_i &
&\forall (i,j) \in E \label{prec_RCPSP_discret}\\ &\sum_{i \in
V}\,\sum_{\tau=\max(0,t-p_i+1)}^t r_{ik}x_{i\tau} \le R_k& &\forall t\in
{\cal H},\ \forall k \in {\cal R} \label{res_RCPSP_discret}\\
&\sum_{t\in {\cal H}} x_{it}=1& & \forall i \in V
\label{preem_RCPSP_discret}\\ & x_{it} \in \{0,1\} & &\forall i
\in V,\ \forall t \in {\cal H} \end{align}
 } 
Dans cette formulation, l'objectif est donné par
l'expression~\eqref{obj_RCPSP_discret}. Cette expression signifie que
l'on cherche à minimiser la date de début de l'activité $n+1$. Or
comme cette activité est forcément placée à la fin de
l'ordonnancement du fait des relations de précédences, ceci
revient bien à minimiser la durée totale du projet.

La contrainte~\eqref{prec_RCPSP_discret} modélise les relations de
précédences entre les activités. En effet, toute paire d'activités
$(i,j)$ ayant la propriété que $i$ doit précéder $j$ vérifie la
relation suivante: la date de début de $j$ est supérieure ou égale
à la date de début de $i$ plus sa durée, i.e. la date de fin de
$i$.

La contrainte~\eqref{res_RCPSP_discret} formalise les contraintes de
ressource. En effet, la contrainte s'assure que, pour chaque ressource
$k \in {\cal R}$, la somme des consommations instantanées sur $k$ des
activités en cours durant la période $t$ est bien inférieure ou égale à la
capacité $R_k$ de cette ressource.

La contrainte~\eqref{preem_RCPSP_discret} impose que chaque activité
n'ait qu'une et une seule date de début. Ceci peut aussi être vu comme
une contrainte de non-préemption puisque ceci revient à empêcher que
l'activité soit interrompue et redémarrée plus tard dans
l'ordonnancement final.

Le modèle possède donc $(n+2)T$ variables binaires et
$|E|+T*m+n$ contraintes.

Il existe d'autres formulations à temps discret comme la formulation
désagrégée de Christofides et al.~\cite{CAVT} ou la formulation basée
sur les ensembles réalisables de Mingozzi et al.~\cite{MMRB}. Ces
formulations ne seront pas détaillées dans ce manuscrit. Une
description des modèles de PLNE existants pour le \RCPSP~peut être
trouvée dans~\cite{ADN}.

\subsection{Modèles à événements}
\label{sec:event_RCPSP}

Nous allons présenter deux formulations à événements: une formulation
dite start/end et une formulation dite on/off.  Ce paragraphe montre,
dans un premier temps, l'intérêt de l'utilisation des modèles à
événements à la place des modèles à temps discret, en comparant
notamment leur nombre respectif de variables et de contraintes.
Dans un second temps, les modèles seront détaillés.

\subsubsection{Motivations des modèles à événements}

Parmi les formulations existantes pour le \RCPSP, celles basées sur des
modèles indexés par le temps sont les plus utilisées dû au fait de la
qualité (relative) de leur relaxation.  Cependant, comme la taille de
ces modèles, et donc leur complexité, dépend grandement de la taille
de l'horizon de temps du problème, ces modèles peuvent s’avérer moins
efficaces sur certains types d'instances~\cite{modele_RCPSP}. Pour
pallier ce problème, des formulations basées sur des {\it événements}
ont été mises en place.

La notion d'événement permet de caractériser seulement les dates
"importantes" de l'ordonnancement, i.e. les dates de début et de
fin de chaque activité. Ainsi, la considération de chaque date de
l'horizon de temps n'est plus nécessaire. Cela permet de réduire
le nombre de variables qui ne dépend alors que du nombre
d'activités à ordonnancer. De plus, pour le \RCPSP, seuls les
événements correspondant au début d'une activité ont besoin d'être
considérés. En effet, si l'on considère que les activités sont
ordonnancées au plus tôt, i.e. dès que les ressources requises sont
disponibles, alors la date de début de chaque activité correspond soit
à l'instant $0$, soit à la date de fin d'une tâche.

Les formulations à événements possèdent aussi la caractéristique
suivante: elles permettent de résoudre des instances pouvant
contenir des valeurs (de paramètre et/ou de solution) non
entières.

Dans~\cite{modele_RCPSP}, les auteurs montrent les limitations des
modèles indexés par le temps pour le \RCPSP. Les instances considérées
sont des instances pour lesquelles l'horizon de temps, i.e. $T$, est
très grand. Dans ce cas-là, le nombre de variables et de contraintes
des modèles à temps discret est nettement supérieur à ceux des modèles
à événements. 

Par exemple, la famille d'instances KSD15\_d~\cite{theseOumar} possède
480 instances de $15$ activités et avec $4$ ressources dont l'horizon
de temps peut varier entre $187$ et $999$ et possédant entre $77$ et
$144$ relations de précédence. Pour le \RCPSP, le modèle à temps
discret peut alors posséder jusqu'à $17000$ variables binaires et
 $4200$ contraintes. Les modèles à événements possèdent quant à
eux jusqu'à $500$ variables binaires, $16$ variables continues et
$4200$ contraintes pour le modèle start/end et jusqu'à $250$ variables
binaires, $16$ variables continues et $4000$ contraintes.

Considérant des instances de taille similaire pour le \CECSP~ avec
fonction de rendement affine, le modèle à temps discret peut alors
posséder jusqu'à $30000$ variables binaires, le même nombre de
variables continues et $90000$ contraintes.  Les modèles à événements
possèdent quant à eux jusqu'à $900$ variables binaires, le même nombre
de variables continues et $7200$ contraintes pour le modèle start/end
et jusqu'à $450$ variables binaires, $900$ variables continues et
$7100$ contraintes.

Dans ce cas-là, la faiblesse de la relaxation linéaire des modèles à
événements est largement compensée par le grand nombre de variable et
de contraintes des modèles indexés par le temps.

Dans ce contexte, l'amélioration des modèles à événements s'avère être
une direction de recherche nécessaire dans l'élaboration de méthodes de
résolution efficaces pour le \RCPSP. 



\subsubsection{Modèle start/end}

Dans cette formulation, la date de chaque événement est représentée
par une variable continue $t_e$, pour tout $e \in {\cal E}$, avec
${\cal E}$ l'ensemble des indices des événements. Afin d'associer
ces dates aux débuts et fins des activités, nous définissons, pour
toute activité $i \in \A$ et pour tout événement $e \in \E$, deux
variables binaires, $x_{ie}$ et $y_{ie}$, ayant les propriétés
suivantes: 
\begin{itemize}
 \item $x_{ie}=1$ si et seulement si
l'activité $i$ commence à l'événement $e$, c'est-à-dire commence à
la date $t_e$. 
\item $y_{ie}=1$ si et seulement si l'activité $i$
finit à l'événement $e$, c'est-à-dire finit à la date $t_e$.
\end{itemize} 
Notons que les événements correspondant à la fin
d'une activité correspondent aussi au début d'une autre activité
(à l'exception de l'événement correspondant à la date de fin de la
dernière activité), le nombre d'événements à considérer est $n+1$. De
ce fait, nous avons comme ensemble d'événements ${\cal
  E}=\{1,\dots,n+1\}$. 

Nous définissons aussi une variable continue $b_{ek}$, pour chaque
événement $e \in \E$ et pour chaque ressource $k \in \R$, modélisant
la consommation totale de la ressource $k$ à l'événement $e$ (et
donc durant tout l'intervalle compris entre $t_e$ et $t_{e+1}$).
Ceci nous permet de présenter la formulation
suivante issue de~\cite{modele_RCPSP} et corrigées dans~\cite{ABKKLM}: 
{\small
\begin{align} 
& \text{min } t_{n+1}
\label{obj_RCPSP_SE}\\ 
& t_1 =0 & & \label{t0_RCPSP_SE}\\ 
&t_e \le
t_{e+1} & & \forall e \in {\cal E} \label{ordre_RCPSP_SE}\\
&\sum_{e\in {\cal E}} x_{ie} =1 & & \forall i \in \A
\label{start_RCPSP_SE}\\ 
&\sum_{e\in {\cal E}} y_{ie} =1 & &
\forall i \in \A\label{end_RCPSP_SE}\\ 
&\ES x_{ie} \le t_e \le \LS
x_{ie}+\LS[n+1]\left(1-x_{ie}\right) & & \forall i \in \A,\ \forall
e \in {\cal E} \label{twx_RCPSP_SE}\\ 
&\left(\LS+p_i\right)y_{ie}
+\left(1-y_{ie}\right)\LS[n+1] \ge t_e & & \forall i \in \A ,\
\forall e \in {\cal E} \label{twy1_RCPSP_SE}\\ 
& t_e \ge
y_{ie}\left(\ES+p_i\right) & & \forall i \in \A ,\ \forall e \in
{\cal E} \label{twy2_RCPSP_SE}\\ 
&b_{1k} - \sum_{i \in \A}
r_{ik}x_{i1}=0 & & \forall k \in {\cal R} \label{res0_RCPSP_SE}\\
& b_{ek} - b_{e-1,k} + \sum_{i\in \A}r_{ik}
\left(y_{ie}-x_{ie}\right)=0 & & \forall e \in {\cal E}
\setminus\{1\},\ \forall k \in {\cal R} \label{resCons_RCPSP_SE}\\ 
&
b_{ek} \le R_k & & \forall e \in {\cal E},\ \forall k\in {\cal R}
\label{res_RCPSP_SE}\\
 &t_f \ge t_e + (x_{ie} + y_{if} -1) p_i & &
\forall i \in \A,\ \forall (e,f) \in {\cal E}^2,\ f>e
\label{dur_RCPSP_SE}\\ 
&\sum_{f=1}^e y_{if} +\sum_{f=e}^n x_{if}
\le 1 & & \forall i \in \A,\ \forall e \in {\cal E}
\label{xby_RCPSP_SE}\\
 &\sum_{f=e}^n y_{if} - \sum_{f=1}^{e-1}
x_{jf} \le 1 & & \forall (i,j) \in E,\ \forall e \in {\cal E}
\label{prec_RCPSP_SE}\\ 
& \ES[n+1] \le t_n \le \LS[n+1]&
&\label{Btn_RCPSP_SE}\\ 
&t_e \ge 0 & & \forall e \in {\cal E}
\label{Bte_RCPSP_SE}\\ 
& b_{ek} \ge 0 & & \forall k \in {\cal R},\
\forall e \in {\cal E} \label{Bbek_RCPSP_SE}\\ 
&x_{ie} \in
\{0,1\},\ y_{ie} \in \{0,1\} & & \forall i \in \A,\ \forall e \in
{\cal E} \label{Bxy_RCPSP_SE} \end{align}
 }
Dans cette formulation, l'objectif est donné par
l'équation~\eqref{obj_RCPSP_SE}. En effet, comme pour la
formulation à temps discret, on cherche à minimiser la date de fin
du projet. Or, l'événement $n+1$ représente, par définition, le
dernier événement, c'est à dire le début de l'activité fictive $n+1$
et donc la fin de l'ordonnancment.

La contrainte~\eqref{t0_RCPSP_SE} fixe le premier événement à la date
0, tandis que la contrainte~\eqref{ordre_RCPSP_SE} ordonne les dates
des événements par ordre croissant.

La contrainte~\eqref{start_RCPSP_SE} (resp.~\eqref{end_RCPSP_SE})
stipule que chaque activité ne peut commencer (resp. finir) qu'une
et une seule fois. En effet, chaque début (resp. fin) d'activité
ne peut être associé qu'à un et un seul événement.
 
La contrainte~\eqref{twx_RCPSP_SE} garantit que la date d'un
événement correspondant au début d'une activité soit bien comprise
dans sa fenêtre de temps, i.e. entre sa date de début au plus tôt
$\ES$ et sa date de début au plus tard $\LS$. En effet, si
l'événement $e$ correspond au début de l'activité $i$, i.e.
$x_{ie}=1$, alors l'inégalité devient: $\ES \le t_e \le \LS$. Au
contraire, si $x_{ie}=0$, i.e. $e$ ne correspond pas au début de
$i$, alors l'inégalité devient: $0 \le t_e \le \LS[n+1]$ et, grâce
aux contraintes~\eqref{ordre_RCPSP_SE} et~\eqref{Btn_RCPSP_SE},
ceci est vrai pour tout $e \in {\cal E}$. De même, les
contraintes~\eqref{twy1_RCPSP_SE} et~\eqref{twy2_RCPSP_SE}
s'assurent que, si un événement $f$ correspond à la fin d'une
activité $i$, alors $t_f$ est bien compris entre $\ES+p_i$ et
$\LS+p_i$. 

Les contraintes~\eqref{res0_RCPSP_SE}
et~\eqref{resCons_RCPSP_SE} représentent les contraintes de
conservation des ressources. La contrainte~\eqref{res0_RCPSP_SE}
modélise la consommation totale de chaque ressource $k$ à
l'événement $1$. Pour une ressource $k$, cette quantité est égale
à la somme des consommations sur $k$ de chaque activité commençant
à l'événement $1$. La contrainte \eqref{resCons_RCPSP_SE} donne
la consommation totale de chaque ressource $k$ à l'événement $e$,
$b_{ek}$. En effet, pour chaque événement $e$, cette quantité est
égale à la consommation totale à l'événement $e-1$, i.e.
$b_{e-1,k}$, à laquelle on retranche les consommations des
activités finissant à l'événement $e$ et on ajoute les
consommations de chaque activité commençant à l'événement $e$.
Enfin, la contrainte~\eqref{res_RCPSP_SE} limite la demande totale
sur chaque ressource à sa capacité.

La contrainte~\eqref{dur_RCPSP_SE} assure que si l'activité $i$
commence à l'événement $e$ et se termine à l'événement $f$, alors
les dates correspondant à ces deux événements sont séparées par
au moins la durée de cette tâche, i.e. $p_i$. Dans ce cas,
l'inégalité s'écrit $t_f \ge t_e+p_i$, et pour toute autre
combinaison de $x_{ie}$ et $y_{if}$, on obtient soit $t_f \ge
t_e$ ou $t_f\ge t_e- p_i$ ce qui est vérifié par la
contrainte~\eqref{ordre_RCPSP_SE}.

La contrainte~\eqref{xby_RCPSP_SE} garantit qu'une activité ne peut se
terminer avant d'avoir commencé. Si l'activité $i$ finit entre
l'événement $1$ et $e$, i.e. $\sum_{f=1}^{e} y_{if}=1$, alors elle ne
peut pas commencer entre l'événement $e$ et $n$, i.e. $\sum_{f=e}^{n}
x_{if}=0$, et inversement.

Enfin, la contrainte~\eqref{prec_RCPSP_SE} modélise les relations
de précédences entre les activités: si une activité $i$, devant
précéder une activité $j$, finit à l'événement $e$ ou après, alors
l'activité $j$ ne peut commencer avant l'événement $e$, i.e.
$\sum_{f=e}^n x_{if}=1 \Rightarrow \sum_{f=1}^{e-1} y_{jf}=0$.

Le modèle possède $2n^2+n$ variables binaires, $n+1$ variables
continues et $(n+1)(m+n^2/2+|E|)+3n$ contraintes. Le nombre de
variables et de contraintes du modèle est donc bien polynomial en
fonction du nombre d'activités et de ressources.
 
\subsubsection{Modèle on/off} 

Dans le modèle on/off, comme pour la
formulation start/end, la variable $t_e$ représente la date
de chaque événement, pour tout $e \in {\cal E}$. Pour associer ces
dates aux débuts et fins de chaque activité, nous définissons, pour
chaque activité $i \in \A$ et pour chaque événement $e \in \E$, la
variable $z_{ie} \in \{0,1\}$. Cette variable prendra la valeur
$1$ si et seulement si l'activité $i$ est en cours durant
l'intervalle $[t_e,t_{e+1}]$, et 0 sinon. Ainsi, une activité
commencera (resp. finira) à l'événement $e$ si
$z_{ie}-z_{i,e-1}=1$ (resp.$z_{i,e-1}-z_{ie}=1$).

Notons que, comme dans le cas du modèle start/end, nous pouvons
limiter le nombre d'événements au début de chaque activité. Ainsi,
${\cal E}=\{1,\dots,n\}$; l'utilisation d'activités fictives
n'est, ici, plus nécessaire pour modéliser l'événement de fin de
la dernière activité. De plus, la variable $z_{ie}$ modélisant le fait
qu'un variable soit en cours entre $t_e$ et $t_{e+1}$, nous pouvons
limiter le nombre de ces variables en ne les définissant que pour les
événements $e \in \Em[n]$. Ceci conduit à la formulation
suivante~\cite{modele_RCPSP}: {\small \begin{align} & \text{min } C_{max}
\label{obj_RCPSP_OO}\\ &C_{max} \ge t_e+ (z_{ie}-z_{i,e-1})p_i & &
\forall e \in {\cal E}\setminus\{1\},\ \forall i \in {\cal A}
\label{Cmax_RCPSP_OO}\\ &t_1=0 & & \label{t0_RCPSP_OO}\\ &t_e \le
t_{e+1} & &\forall e \in {\cal E}\setminus\{n\}
\label{ordre_RCPSP_OO}\\ &\sum_{e \in \Em[n]} z_{ie} \ge 1 & &
\forall i \in {\cal A} \label{start_RCPSP_OO}\\ & \ES z_{ie}\le
t_e & & \forall e \in \Em[n],\ \forall i \in {\cal A}
\label{ES_RCPSP_OO} \\ & t_e \le
\LS(z_{ie}-z_{i,e-1})+(1-(z_{ie}-z_{i,e-1}))\LS[n+1] & & \forall e
\in \Em[n]\setminus\{1\},\ \forall i \in {\cal A}
\label{LS_RCPSP_OO}\\ &t_f\ge t_e
+\left((z_{ie}-z_{i,e-1})-(z_{if}-z_{i,f-1})-1\right)p_i & & \forall
e>f \in (\Em[1,n])^2,\ \forall i \in {\cal
A}\label{dur_RCPSP_OO}\\ &\sum_{f=1}^{e} z_{if} \le
e(1-(z_{ie}-z_{i,e-1})) & & \forall e \in \Em[1,n],\
\forall i \in {\cal A}\label{preem1_RCPSP_OO}\\ &\sum_{f=e}^{n-1}
z_{if} \le (n-e)(1+(z_{ie}-z_{i,e-1})) & & \forall e \in \Em[1,n],\ 
\forall i \in {\cal A}\label{preem2_RCPSP_OO}\\
&\sum_{i \in {\cal A}} r_{ik}z_{ie} \le R_k & &\forall e \in \Em[n]
,\ \forall k \in {\cal R} \label{res_RCPSP_OO}\\
&z_{ie}+\sum_{f=1}^e z_{jf} \le 1 + (1-z_{ie})e & & \forall e \in
\Em[1,n],\ \forall (i,j) \in E \label{prec_RCPSP_OO}
\end{align}
}
L'objectif, donné par l'équation~\eqref{obj_RCPSP_OO}, 
consiste à minimiser la date de fin du projet, ici
représentée par la variable $C_{max}$. La
contrainte~\eqref{Cmax_RCPSP_OO} s'assure que $C_{max}$ prend bien
la valeur de la date de fin du projet.

Les contraintes~\eqref{t0_RCPSP_OO} et~\eqref{ordre_RCPSP_OO} jouent
le même rôle que dans la formulation start/end, à savoir, ordonner
les événements.

La contrainte~\eqref{start_RCPSP_OO} stipule que chaque activité
doit être ordonnancée sur au moins un événement dans toute la durée du projet.
 
Les contraintes~\eqref{ES_RCPSP_OO} et~\eqref{LS_RCPSP_OO}
garantissent que la date d'un événement correspondant au début
d'une activité soit bien comprise dans sa fenêtre de temps, i.e.
entre sa date de début au plus tôt $\ES$ et sa date de début au
plus tard $\LS$. En effet, si l'événement $e$ correspond au début
de l'activité $i$, i.e. $z_{ie}=1$ et $z_{i,e-1}=0$ , l'inégalité
devient alors: $\ES \le t_e \le \LS$. Nous distinguons 3 autres
cas: \begin{itemize} \item {\it $z_{ie}=z_{i,e-1}=1$ :} l'activité
est en cours entre les événements $t_{e-1}$ ($z_{i,e-1}=1$) et
$t_{e+1}$ ($z_{ie}=1$). L'inégalité devient: $\ES\le t_e \le
\LS[n+1]$. Comme l'activité $i$ a déjà commencé à un événement $f
<e$, on a: $\ES \le t_f \le t_e$. L'autre inégalité est triviale.
\item {\it$z_{ie}=0$ et $z_{i,e-1}=1$:} l'activité se termine à
l'événement $e$. L'inégalité donne alors $0 \le t_e \le
2*\LS[n+1]-\LS$. Or, $2*\LS[n+1] -\LS \ge \LS[n+1]$. Donc
l'inégalité est vérifiée. \item {\it $z_{ie}=z_{i,e-1}=0$:}
l'activité n'est pas en cours entre les événements $t_{e-1}$ et
$t_{e+1}$. L'inégalité devient $0 \le t_e \le \LS[n+1]$ et est
trivialement vérifiée. \end{itemize}

La contrainte~\eqref{dur_RCPSP_OO} assure une séparation
suffisante, i.e. la durée de l'activité, entre un événement
correspondant au début d'une activité et un événement
correspondant à la fin de cette même activité. La validité de
cette contrainte suit la même logique que pour la
contrainte~\eqref{dur_RCPSP_SE} du modèle précédent.

Les contraintes~\eqref{preem1_RCPSP_OO}
et~\eqref{preem2_RCPSP_OO}, appelées {\it contraintes de
contiguïté}, assure la non-préemption des activités. La validité
de cette contrainte n'est pas détaillée ici mais une preuve
formelle peut être trouvée dans~\cite{modele_RCPSP}.

La contrainte~\eqref{res_RCPSP_OO} représente les limites de
capacité de chaque ressource. En effet, une activité $i$ consomme
une quantité $r_{ik}$ de la ressource $k$ entre $t_e$ et $t_{e+1}$
si et seulement si elle est en cours entre ces deux dates, i.e.
$z_{ie}=1$. On pourra remarquer que dans ce modèle le nombre de
contraintes nécessaires pour modéliser les contraintes de capacité
des ressources est nettement inférieur à celui du modèle précédent.

Enfin, la contrainte~\eqref{prec_RCPSP_OO} modélise les relations
de précédences entre les activités: si une activité $i$, devant
précéder une activité $j$, est en cours entre les événements $e$
et $e+1$, i.e. $z_{ie}=1$, alors l'activité $j$ ne peut être en
cours avant l'événement $e$, i.e. $z_{ie}=1 \Rightarrow
\sum_{f=0}^e z_{jf}=0 $.

Le modèle possède $n^2$ variables binaires, deux fois moins que
pour le modèle start/end, $n+1$ variables continues et
$(n-1)(3+|E|+m+n^2/2)+n^2+n$ contraintes. Le nombre de variables
et de contraintes du modèle est donc bien polynomial en fonction
du nombre d'activités et de ressources.

\section*{Conclusion}

Dans ce chapitre, nous avons rappelé les principales notions de la
programmation linéaire, puis de la programmation linéaire mixte et en
nombres entiers. Nous avons ensuite vu plusieurs méthodes de
modélisation pour un problème d'ordonnancement avec contraintes de
ressource: le \RCPSP. Pour ce problème, nous avons vu deux types de
modèles. Le premier fait partie des modèles indexés par le temps. Ces
derniers possèdent de relativement bonnes relaxation, ce qui en fait
un des types de modèles les plus efficaces pour le \RCPSP~et en
particulier sur les célèbres instances de la PSPLIB~\cite{PSPLIB}. 

Cependant, ces modèles souffrent de limitations dues à leur taille,
dépendante de l'horizon de temps du projet. De ce fait, quand
l'horizon de temps des instances devient grand, la tailles des modèles
associés augmente et leurs performances décroissent. Pour pallier ce
problème, un autre type de modèle est mis en place: les modèles à
événements. L'avantage de ces modèles est leur taille, polynomiale en
fonction de la taille de l'instance et surtout indépendante de
l'horizon de temps. Dans~\cite{modele_RCPSP}, Kone {\it et al.}
comparent les performances de ces modèles avec ceux indexés par le
temps sur des instances ayant de grand horizon de temps. De plus, ces
modèles permettent la modélisation de problèmes continus. De ce fait,
nous allons adapter ces modèles dans le cadre du \CECSP~dans le
chapitre suivant. De plus, un modèle indexés par le temps est aussi
présenté, permettant d'avoir des solutions plus rapidement (mais pas
forcément optimales).

 
\chapter[Modèles pour le \CECSP~et renforcement des modèles]{Programmation linéaire pour le \CECSP~et renforcement des modèles}
\label{sec:PLNE_CECSP}




%!TeX root =../main_file.tex
\section{Modèles de programmation linéaire mixte pour le
\CECSP}

Le \CECSP ~et le \RCPSP~étant deux problèmes relativement proches --
utilisation d'une ou plusieurs ressources cumulatives, de fenêtres de
temps -- la modélisation du \CECSP~à l'aide de la programmation
linéaire mixte est une approche naturelle pour
sa résolution. 

De plus, le théorème~\ref{theo_LPM_CECSP} nous assurant que toute solution
$S$ du \CECSP~peut être transformée en une solution $S'$ ayant la
propriété que $\forall i \in \A,\ b_i(t)$ soit constante par morceaux,
le problème peut être modélisé à l'aide de la programmation linéaire
mixte. Nous pouvons aussi remarquer que, comme la transformation
proposée par le théorème n'augmente ni la date de début des activités,
ni leur date de fin, ni leur consommation totale de ressource, tout modèle
ayant un objectif impliquant seulement la minimisation de ces trois
quantités sera valide pour le \CECSP.

Dans ce paragraphe, nous commençons par décrire un modèle indexé par
le temps, puis, nous présentons deux modèles à événements. Ces trois
modèles sont adaptés des modèles pour le \RCPSP~présentés dans le
paragraphe~\ref{sec:PLNE_ordo_res} et sont présentés
dans~\cite{Nattaf_ORSpectrum}.   


\subsection{Modèle indexé par le temps}

La première formulation proposée est une formulation indexée par le
temps. Comme pour le modèle à temps discret du \RCPSP l'horizon de
temps est ici divisé en périodes de taille unitaire. Le calcul d'une
borne supérieure $T$ sur la durée totale du projet est trivial: il
suffit de prendre la plus grande date échue, i.e. $T=\max_{i \in \A}
\LE$. L'ensemble des périodes, noté $\H$, peut donc être défini par:
$\H=\{0,\dots,T-1\}$.  Notons que, par translation, nous pouvons
toujours supposer que $\min_{i\in \A} \ES=0$.

Une des principales différences entre le modèle à temps discret du
\RCPSP~et celui du \CECSP~ repose sur le fait que, dans le
second, la durée des activités n'est pas connue à l'avance et doit
être déterminée par le modèle. De ce fait, nous devons définir,
pour chaque activité $i \in \A$ et pour chaque instant $t
\in \H$, deux variables binaires $x_{it}$ et $y_{it}$ pour
modéliser le début et la fin des activités. La variable $x_{it}$
(resp. $y_{it}$) prendra la valeur $1$ si et seulement si
l'activité $i$ commence (finit) à l'instant $t$.

Une seconde différence entre les deux modèles est le calcul des 
fenêtres de temps, effectué de manière triviale pour le \CECSP. 
En effet, pour une activité $i$,
la fenêtre correspondante à sa date de début est $[\ES,\LS]$, avec
$\LS=\LE-W_i/f_i(\bmax)$, et celle correspondante à sa date de fin
$[\EE,\LE]$, avec $\EE=\ES+W_i/f_i(\bmax)$. Enfin, l'ajout des
activités fictives marquant le début et la fin du projet n'est 
pas nécessaire ici.

\paragraph{Fonction de rendement identité}

Dans le cas où la fonction de rendement de chaque activité est la
fonction identité, i.e. $f_i(b_i(t))=b_i(t),\ \forall i \in \A$, nous
définissons une variable $b_{it}$, pour chaque activité $i \in \A$ et
pour chaque période de temps $t \in \H$, qui représentera la quantité
de ressource consommée par l'activité $i$ dans la période de temps
$t$, i.e. dans l'intervalle $[t,t+1]$.

Ceci conduit à la formulation suivante:
{\small
 \begin{align} &\text{min }
\sum_{i\in \A}\sum_{t \in \H} b_{it}
\label{obj_CECSP_TI}\\ &\sum_{t=\ES}^{\LS} x_{it} = 1 & &\forall i
\in \A \label{start_CECSP_TI}\\ &\sum_{t=\EE}^{\LE} y_{it} =
1 & & \forall i \in \A
\label{end_CECSP_TI}\\&\left(\sum_{\tau=\ES}^{t} x_{i\tau}
-\sum_{\tau=\ES+1}^{t} y_{i\tau}\right)\bmin \le b_{it} & &
\forall t \in \{\ES,\dots,\LE-1\},\ \forall i \in \A
\label{bmin_CECSP_TI}\\ &\left(\sum_{\tau=\ES}^{t} x_{i\tau} -
\sum_{\tau=\ES+1}^{t} y_{i\tau}\right)\bmax\ge b_{it}& & \forall t
\in \H ,\ \forall i \in \A \label{bmax_CECSP_TI}\\
&\sum_{t=\ES}^{\LE} b_{it} \ge W_i & & \forall i \in \A
\label{nrj_CECSP_TI}\\ &\sum_{i \in \A} b_{it} \le R & &
\forall t \in \H \label{res_CECSP_TI}\\ &b_{it} = 0 & &
\forall t \not\in \{\ES,\dots,\LE-1\},\ \forall i \in \A
\label{consNul_CECSP_TI}\\ &x_{it} = 0 & & \forall t \not\in
\{\ES,\dots,\LS\},\ \forall i \in \A \label{twx_CECSP_TI}\\
&y_{it} = 0 & & \forall t \not\in \{\EE,\dots,\LE\},\ \forall i
\in \A \label{twy_CECSP_TI}\\ & b_{it} \ge 0 & & \forall t
\in \H; \forall i \in \A \label{Bb_CECSP_TI}\\
&x_{it}\in \{0,1\},\ y_{it} \in \{0,1\} & & \forall t \in {\cal
H},\ \forall i \in \A \label{Bxy_CECSP_TI} \end{align}
} 
Ce modèle s'inspire grandement du modèle de~\cite{ALR}, la principale
différence résidant dans la considération de fenêtres de temps plus
fine pour les variables, i.e. $[\ES,\LS]$ pour $x_{it}$ au lieu de
$[\ES,\LE-1]$ et $[\EE,\LE]$ pour $y_{it}$ au lieu de
$[\ES-1,\LE]$. Ces fenêtres de temps pouvant être raffinées à l'aide
d'heuristiques, de la programmation par contraintes (cf.
paragraphe~\ref{sec:adjustment_tw}) ou par le biais d'autres méthodes,
ces modifications peuvent grandement améliorer les performances du
modèle.

Dans cette formulation, l'objectif est décrit par
l'équation~\eqref{obj_CECSP_TI}. Ici, l'objectif est de minimiser la
consommation totale de la ressource durant tout le projet. Cet
objectif a moins d'impact dans le cas où la fonction de rendement de
chaque activité est la fonction identité mais se révèle très pertinent
pour d'autres types de fonctions de rendement.  Cependant, comme la
formulation à temps discret ne nous permet pas de s'assurer que la
quantité de ressource consommée par une activité soit exactement égale
à la quantité nécessaire pour apporter l'énergie requise,
i.e. $\sum_{t=\ES}^{\LE} b_{it} \ge W_i$, cet objectif reste valide
dans le cas des fonctions identités.

Si l'on souhaite modifier l'objectif pour minimiser la date de fin
du projet, il suffit d'introduire une variable $C_{max}$ ainsi que
la contrainte suivante: 
\begin{equation} 
C_{max} \ge \sum_{i \in \A} ty_{it} \quad \forall t \in \H \notag
\end{equation} 
et alors l'objectif s'écrit facilement comme: 
\begin{equation}
\text{min } C_{max} \notag
\end{equation} 
Les contraintes~\eqref{start_CECSP_TI} et~\eqref{end_CECSP_TI}
stipulent que chaque activité est exécutée une et une seule fois
durant la durée du projet. En effet, grâce à ces contraintes, chaque
activité n'a qu'une et une seule date de début et une et une seule
date de fin.

Les contraintes~\eqref{bmin_CECSP_TI} et~\eqref{bmax_CECSP_TI}
permettent de s'assurer que la consommation instantanée de chaque
activité est bien comprise entre $\bmin$ et $\bmax$ durant toute sa
durée d'exécution. Pour s'en assurer, il suffit de remarquer que
$\sum_{\tau=\ES}^{t} x_{i\tau} -\sum_{\tau=\ES+1}^{t}y_{i\tau}=1$ si
et seulement si l'activité $i$ est en cours à l'instant $t$. Les
autres configurations possibles sont $\sum_{\tau=\ES}^{t} x_{i\tau}
-\sum_{\tau=\ES+1}^{t} y_{i\tau}=$0 et $\sum_{\tau=\ES}^{t} x_{i\tau}
-\sum_{\tau=\ES+1}^{t} y_{i\tau}=-1$. Dans le premier cas, ceci
signifie que l'activité n'a pas encore débuté et les inégalités
imposent que $b_{it} \le 0$ et donc $b_{it}=0$. Dans le second cas,
l'équation devient $-\bmax \ge b_{it}$, ce qui est impossible. Notons
que ce dernier cas nous assure ainsi qu'une activité ne peut finir avant
d'avoir commencé. Cependant, afin de renforcer la relaxation linéaire
du modèle, des contraintes spécifiques, empêchant que le début d'une
activité ne se produise avant sa fin, peuvent être ajoutées. Ces
inégalités sont de la forme: 
\begin{equation}
 \sum_{\tau=\ES}^{t} x_{i\tau}
-\sum_{\tau=\ES+1}^{t} y_{i\tau} \le 0 \quad \forall t \in \{\ES,\dots,\LE-1\}
\end{equation}
L'intérêt de l'ajout de ces équations sera discuté dans le
chapitre~\ref{sec:expe}.  

La contrainte~\eqref{nrj_CECSP_TI} modélise le fait qu'une activité
doit recevoir au moins la quantité d'énergie requise par la donnée du
problème.

La contrainte~\eqref{res_CECSP_TI} limite la quantité de ressource
utilisée simultanément à la capacité de cette dernière.

Enfin, les contraintes~\eqref{consNul_CECSP_TI},~\eqref{twx_CECSP_TI}
et~\eqref{twy_CECSP_TI} fixent la valeur des variables à zéro en
dehors de leurs fenêtres de temps, i.e. $[\ES,\LE]$ pour $b_{it}$,
$[\ES,\LS]$ pour $x_{it}$ et $[\EE,\LE]$ pour $y_{it}$.

Cette formulation possède $2nT$ variables binaires, $nT$ variables
continues et au plus $3n+T*(5n+1)$ contraintes.

\paragraph{Fonction de rendement affine}

Dans le cas où les fonctions de rendement sont toutes la fonction
identité, l'énergie apportée à une activité durant une certaine
période est égale à la quantité de ressource consommée par celle-ci
durant la même période de temps. Or, lorsque les fonctions de
rendement deviennent affines, ce n'est plus le cas.  Pour s'assurer
qu'une activité $i$ reçoive bien l'énergie requise, nous devons
déclarer une nouvelle variable qui permettra de mesurer l'énergie
apportée à cette dernière durant la période $t$, $w_{it},\ \forall
(i,t) \in \A \times \H$. Dans le cas
précédent, nous avions $w_{it}=b_{it}$.

Nous devons donc vérifier, pour chacune des contraintes du modèle
précédent impliquant la variable $b_{it}$, si cette dernière
modélisait une quantité de ressource ou d'énergie, i.e. si nous devons
la remplacer par $w_{it}$. De plus, une contrainte liant les variables
$b_{it}$ et $w_{it}$ devra être rajoutée pour assurer la bonne
conversion entre les deux quantités.

La contrainte~\eqref{bmin_CECSP_TI} (resp.~\eqref{bmax_CECSP_TI})
représentant la quantité minimale (resp. maximale) de ressource qu'une
activité doit consommer à chaque instant de son exécution, reste donc
inchangée. De même, la contrainte~\eqref{res_CECSP_TI}, modélisant la
contrainte sur la capacité de la ressource, n'a pas besoin d’être
modifiée.

La contrainte~\eqref{nrj_CECSP_TI}, qui s'assure qu'une activité
reçoive bien la quantité d'énergie requise doit être réécrite en
utilisant la variable $w_{it}$. Ce qui nous donne l'inégalité
suivante:
\begin{equation} 
\sum_{t=\ES}^{\LE} w_{it} \ge W_i\quad \forall i \in \A \tag{\ref{nrj_CECSP_TI}'} \label{nrj2_CECSP_TI}
\end{equation} 
De plus, nous devons ajouter une contrainte permettant de lier la
quantité de ressource consommée par une activité et la quantité
d'énergie apportée à celle-ci. Nous ajoutons donc la contrainte
suivante au modèle:
\begin{equation}
w_{it}=a_ib_{it}+c_i\left(\sum_{\tau=\ES}^t
x_{i\tau}-\sum_{\tau=\ES+1}^t y_{i\tau}\right) \quad \forall t\in
\H,\ \forall i \in \A \label{conv_CECSP_TI}
\end{equation} 

Cette contrainte nous permet de modéliser la fonction de rendement
$f_i,\ \forall i \in \A$. En effet, $\left(\sum_{\tau=\ES}^t
x_{i\tau}-\sum_{\tau=\ES+1}^t y_{i\tau}\right) $ est égale à $1$ si et
seulement si l'activité $i$ est en cours à l'instant $t$.  Dans ce
cas-là, la valeur de l'énergie apportée à $i$ est bien
$w_{it}=a_ib_{it}+c_i$. Le second cas se produit quand l'activité $i$
n'est pas en cours à $t$. Dans ce cas, la
contrainte~\eqref{bmax_CECSP_TI} nous dit que $b_{it}=0$ et donc
$w_{it}=0$.

Le modèle possède donc $2nT$ variables binaires, $2nT$ variables
continues et au plus $3n+T*(6n+1)$ contraintes.

\paragraph{Fonction de rendement concave affine par morceaux}

Lorsque les fonctions de rendements sont des fonctions concaves
affines par morceaux, nous utilisons aussi la variable $w_{it}$ pour
représenter l'énergie reçue par l'activité $i$ dans la période $t$. La
contrainte~\eqref{nrj2_CECSP_TI} est utilisée pour modéliser le fait
que cette activité doive recevoir une quantité d'énergie $W_i$ durant
son exécution. 

Pour s'assurer de la bonne conversion entre la quantité de ressource
consommée par une activité $i$ dans une période $t$ et la quantité
d'énergie reçue par cette dernière dans la même période,
l'égalité~\eqref{conv_CECSP_TI} est remplacée par l'inégalité
suivante: 
\begin{equation}
w_{it} \le a_{ip}b_{it} + c_{ip}\left(\sum_{\tau=\ES}^t
x_{i\tau}-\sum_{\tau=\ES+1}^t y_{i\tau}\right) \quad  \forall i \in
\A,\ \forall p \in \P_i,\ \forall t \in \H 
\label{conv_CECSP_TI2}
\end{equation}
avec $\P_i$ l'ensemble des intervalles de définition de la fonction
$f_i$.

Notons que l'utilisation d'une inégalité peut impliquer que la
variable $w_{it}$ ne représente pas exactement la quantité d'énergie
apportée à $i$ dans la période $t$. Cependant, la contrainte nous
assure que $w_{it} \le f_i(b_{it})$. De ce fait, à une solution
optimale donnée par le modèle correspond toujours une solution
optimale du \CECSP~à dates de début et de fin entières. En effet,
supposons que dans une solution optimale renvoyée par le PLNE, il
existe $(i,t)$ tel que $w_{it} < f_i(b_{it})$, alors, deux cas sont
possibles:
\begin{itemize}
\item $\bmin \ge f_i^{-1}(w_{it})$ et dans ce cas-là, $b_{it}$ peut
prendre la valeur $f_i^{-1}(w_{it})$ et cette solution aura un coût,
i.e. une consommation totale de ressource, plus faible que la solution
précédente et ceci contredit l'optimalité de la solution.
\item $\bmin < f_i^{-1}(w_{it})$ et dans ce cas-là, $b_i(t)$ prend la
valeur $\bmin$ et $et_i=\min \{t \in\mathbb{N}\ |\ \\
\int_{\tau=st_i}^{t} f_i(b_i(t))dt \le W_i\}$. Ici, deux cas sont
possibles: le premier correspond au cas où $et_i=\{t\ | \ y_{it}=1\}$
et la solution renvoyée par le PLNE est optimale; dans le second cas
$et_i<\{t\ |\ y_{it}=1\}$ mais alors, $y_{it-1}$ peut prendre la
valeur $1$ et $y_{it}$ la valeur $0$ et cette solution est de plus
faible coût que la solution précédente et ceci contredit l'optimalité
de la solution renvoyée par le PLNE.
\end{itemize}


Le modèle ainsi défini possède alors $2nT$ variables binaires, $2nT$
variables continues et au plus $3n + (5n +nP+1)T$ contraintes, avec
$P=\max_{i\in A}|\P_i|$.  

\subsection{Modèles à événements}

Dans ce paragraphe, nous proposons deux formulations à événements pour
le \CECSP. Ces deux formulations sont grandement inspirées des
formulations start/end et on/off du \RCPSP, présentées dans le
paragraphe~\ref{sec:PLNE_ordo_res} et issues
de~\cite{modele_RCPSP}. Comme dans le cas du \RCPSP, ces modèles se
justifient par leur nombres polynomial de variables et de contraintes
qui peut s'avérer très pertinent dans la cas de grands horizons de
temps. Pour le \CECSP, un argument supplémentaire vient s'ajouter pour
le \CECSP. En effet, il peut arriver qu'une instance de ce problème ne
possède que des solutions à valeurs non entières et ce, même si toutes
les données sont entières (voir exemple~\ref{exemple_NE},
page~\pageref{exemple_NE}).


Comme dans le cas de la formulation à temps discret, ici, les dates de
fin des activités ne sont plus totalement définies par leur date de
début. De ce fait, dans les formulations à événements du \CECSP, nous
avons besoin de modéliser deux types d'événements, les débuts et les
fins des activités, soit, au plus, $2n$ événements. Ces événements,
indexés par l'ensemble $\E=\{1,\dots,2n\}$, sont représentés par un
ensemble de variables continues, notées $t_e$.

Nous rappelons aussi les notations suivantes: $T=\max_{i \in \A}
\LE$ est une borne supérieure sur la date de fin du projet;
$[\ES,\LS]$, avec $\LS= \LE-W_i/f_i(\bmax)$, est la fenêtre de temps
dans laquelle l'activité $i$ peut débuter; de même, $[\EE,\LE]$,
avec $\EE=\ES + W_i/f_i(\bmax)$, la fenêtre de temps durant laquelle
l'activité $i$ peut finir.

Nous allons présenter, en premier lieu, la formulation start/end du
\CECSP~avec fonctions de rendement identité dont nous dériverons 
les cas de fonctions de rendement affines et affines par morceaux.
Ensuite, nous présenterons la formulation on/off du même problème et
de ses dérivés.

\subsubsection{Modèle start/end}

Dans le modèle start/end, deux variables binaires $x_{ie}$ et
$y_{ie}$, $\forall (i,e) \in \A\times \E$, 
servent à affecter les dates des différents événements, modélisés par
les variables $t_e$, aux débuts et fins des activités. En effet, la
variable $x_{ie}$ prendra la valeur $1$ si et seulement si l'événement
$e$ correspond au début de l'activité $i$, i.e. l'activité $i$
commence à la date $t_e$, et sera égale à $0$ sinon. De même, la
variable $y_{ie}$ sera égale à $1$ si et seulement si l'événement $e$
correspond à la fin de l'activité $i$, et vaudra $0$ sinon.

\paragraph{Fonction de rendement identité}

Dans le cas où toutes les fonctions de rendement sont la fonction
identité, un seul ensemble de variables est nécessaire pour modéliser
la consommation de la ressource. En effet, comme la quantité de
ressource consommée par une activité $i$ est égale à la quantité
d'énergie apportée à cette même activité, il n'est pas utile de
définir une variable représentant cette quantité d'énergie.

Un ensemble de variables supplémentaires, $b_{ie}$, $\forall (i,e) \in
\A\times \Em$, est donc défini. La variable $b_{ie}$ représente la
quantité de ressource consommée par une activité $i$ entre les dates
$t_e$ et $t_{e+1}$.  Ceci nous permet de modéliser le problème de la
façon suivante:
{\small
\begin{align}
& \text{min } \sum_{i\in A}\ \sum_{e\in \E\setminus\{2n\}} b_{ie} 
\label{obj_CECSP_SE}\\
&t_e \le t_{e+1} & & \forall e \in
\E\setminus\{2n\} \label{ordre_CECSP_SE}\\
 &\sum_{e\in \E} x_{ie} =1 & & \forall i \in
A \label{start_CECSP_SE}\\
 &\sum_{e\in \E} y_{ie} =1 & & \forall i \in A \label{end_CECSP_SE}\\
 &x_{ie}\ES \le t_e & & \forall i \in A,\ \forall e \in
\E \label{twx1_CECSP_SE}\\
 &t_e \le x_{ie}\LS+ \left(1-x_{ie}\right)T & & \forall i \in A,\
\forall e \in \E \label{twx2_CECSP_SE}\\
 &t_e \ge y_{ie}\EE & & \forall i \in A ,\ \forall e \in \E 
 \label{twy1_CECSP_SE}\\
 &\LE y_{ie} + \left(1-y_{ie}\right)T \ge t_e & & \forall i \in A,\ \forall e \in \E 
 \label{twy2_CECSP_SE}\\
 &\sum_{i \in A} b_{ie} \le R \left(t_{e+1}- t_e\right) & & 
 \forall e \in \E\setminus\{2n\} \label{res_CECSP_SE}\\
 &t_f \ge t_e +  \left(x_{ie} + y_{if} -1\right) W_i/f_i(\bmax) & & \forall i \in A,\ 
 \forall e,f \in \E\ ; f\ge e \label{dur_CECSP_SE}\\
 &\sum_{e\in \E\setminus\{2n\}} b_{ie} = W_i  & & \forall i \in A 
 \label{nrj_CECSP_SE}\\
&b_{ie} \ge \bmin \left(t_{e+1}-t_e\right) 
- M \left(1 - \sum_{f=0}^e x_{if} +\sum_{f=0}^e y_{if}\right) 
& &  \forall i \in A,\ \forall e \in \E\setminus\{2n\} \label{bmin_CECSP_SE}\\
&b_{ie} \le \bmax  \left(t_{e+1} - t_e\right) & &
\forall i \in A,\ \forall e \in \E\setminus\{2n\} \label{bmax_CECSP_SE}\\
& \left(\sum_{f=0}^{e} x_{if} - \sum_{f=0}^e y_{if}\right)M\ge b_{ie} & &
 \forall i \in A,\ \forall e \in \E\setminus\{2n\} \label{res0_CECSP_SE}\\
&t_e \ge 0 & & \forall e \in \E \label{eq36}\\
& b_{ie} \ge 0 & & \forall i \in A,\ \forall e \in \E\setminus\{2n\} 
\label{Bb_CECSP_SE}\\
&x_{ie} \in \{0,1\},\ y_{ie} \in \{0,1\} & & 
\forall i \in A,\ \forall e \in \E \label{eq39}
\end{align}
}

L'objectif, décrit par l'équation~\eqref{obj_CECSP_SE}, est de
minimiser la consommation totale de la ressource, i.e. la somme des
consommations de toutes les tâches. On peut facilement modifier cet
objectif afin de minimiser la date de fin du projet en remplaçant
l'objectif par:
\begin{equation}
\text{min } t_{|\E|} \notag
\end{equation}
La contrainte~\eqref{ordre_CECSP_SE} ordonne les événements. La
contrainte~\eqref{start_CECSP_SE} (resp.~\eqref{end_CECSP_SE})
s'assure que chaque activité ne commence (resp. finisse) qu'une et une
seule fois. En effet, chaque début (resp. fin) d'activité ne peut être
associé qu'à un et un seul événement.
 
Les contraintes~\eqref{twx1_CECSP_SE} et~\eqref{twx2_CECSP_SE}
garantissent que la date d'un événement correspondant au début d'une
activité soit bien comprise dans sa fenêtre de temps, i.e. dans
l'intervalle $[\ES,\LS]$. En effet, si l'événement $e$ correspond au
début de l'activité $i$, i.e. $x_{ie}=1$, alors la première inégalité
devient $\ES \le t_e$ et la seconde $t_e \le \LS$. Les autres
configurations donnent: $0 \le t_e \le T$ et ceci est trivialement
vérifié. De même, les contraintes~\eqref{twy1_CECSP_SE}
et~\eqref{twy2_CECSP_SE} s'assurent que, si un événement $e$
correspond à la fin d'une activité $i$, alors $t_e$ est bien compris
entre $\EE$ et $\LE$.

La contrainte~\eqref{res_CECSP_SE} s'assure que, dans la période de
temps comprise entre $t_e$ et $t_{e+1}$, la quantité de ressource
consommée n’excède pas la capacité de cette dernière.

La contrainte~\eqref{dur_CECSP_SE} modélise le fait que si l'activité
$i$ commence à l'événement $e$ et se termine à l'événement $f$, alors
les dates correspondant à ces deux événements sont au moins séparées
par la durée de cette tâche. Or, comme nous ne connaissons pas cette
durée, nous utilisons une borne inférieure: $W_i/f_i(\bmax)$. 

La contrainte~\eqref{nrj_CECSP_SE} modélise le fait qu'une activité
doit recevoir au moins la quantité d'énergie requise par la donnée du
problème.

Les contraintes~\eqref{bmin_CECSP_SE} et~\eqref{bmax_CECSP_SE}
permettent de s'assurer que la consommation instantanée de chaque
activité est bien comprise entre $\bmin$ et $\bmax$ durant toute sa
durée d'exécution. Pour s'en assurer, il suffit de remarquer que, dans
la première inégalité, $\sum_{f=0}^{e} x_{if}-\sum_{f=0}^{e}y_{if}=1$
si et seulement si l'activité $i$ est en cours entre les événements
$e$ et $e+1$.  L'inégalité devient donc $b_{ie} \ge
\bmin(t_{e+1}-t_e)$. Les autres configurations possibles donnent
$b_{ie} \ge \bmin(t_{e+1}-t_e)-M$, avec $M$ une constante
suffisamment grande pour dépasser $\bmin(t_{e+1}-t_e)$.

La contrainte~\eqref{res0_CECSP_SE} garantit que si l'activité $i$
n'est pas en cours entre les événements $e$ et $e+1$, alors
$b_{ie}=0$. En effet, si l'activité $i$ n'est pas en cours,
i.e. $\sum_{f=0}^{e} x_{if}-\sum_{f=0}^{e}y_{if}=0$, la contrainte
s'écrit $0\ge b_{ie}$ et donc on a bien $b_{ie} =0$. Dans le cas où
l'activité $i$ est en cours, i.e. $\sum_{f=0}^{e} x_{if}
-\sum_{f=0}^{e}y_{if}=1$, la contrainte s'écrit $M\ge b_{ie}$ et, si
$M$ est une constante suffisamment grande pour servir de borne
supérieure à $b_{ie}$, alors la contrainte est valide. Notons aussi
que ces contraintes empêchent qu'une activité ne puisse commencer
avant d'avoir fini. De plus, pour renforcer la formulation, nous
pouvons ajouter les contraintes~\eqref{xby_RCPSP_SE} modélisant
explicitement le fait qu'un événement ``début d'activité'' se produit
forcément avant l'événement ``fin d'activité'' lui
correspondant. L'intérêt de l'ajout de telles contraintes sera discuté
dans le chapitre~\ref{sec:expe}.

Cette formulation possède $4n^2$ variables binaires, $2n^2+n$
variables continues et $2n^3+13n^2+4n-2$ contraintes.

\paragraph{Fonction de rendement affine}

Pour adapter cette formulation au cas des fonctions de rendement
affines, nous avons, dans un premier temps, besoin de définir une
variable $w_{ie},\ \forall i \in \A$ et $\forall e \in \E
\setminus\{2n\}$. Cette variable sert à modéliser la quantité
d'énergie apportée à une activité $i$ durant l'intervalle de temps
$[t_e,t_{e+1}]$. De plus, nous devons identifier les contraintes
impliquant la variable $b_{ie}$ pour lesquelles cette variable doit
être remplacée par la variable $w_{ie}$. En d'autres termes, nous
devons identifier les contraintes traitant de la ressource et les
contraintes liées à l'énergie.

La contrainte~\eqref{res_CECSP_SE}, modélisant la contrainte de
capacité de la ressource, n'est pas modifiée. De même, les
contraintes~\eqref{bmin_CECSP_SE} et~\eqref{bmax_CECSP_SE},
garantissant la contrainte de consommation maximale et minimale de la
ressource, restent inchangées. Enfin, la
contrainte~\eqref{res0_CECSP_SE}, s'assurant de la non consommation de
ressource des activités en dehors de leur exécution, ne nécessite
aucun changement.

La contrainte~\eqref{nrj_CECSP_SE} cependant, étant en charge du
respect de l'apport requis en énergie de chaque activité, doit être
réécrite en utilisant la variable adéquate, $w_{ie}$. Ceci donne
l'inégalité suivante:
\begin{equation} 
\sum_{e\in \E\setminus\{2n\}} w_{ie} = W_i\quad \forall i \in \A
\tag{\ref{nrj_CECSP_SE}'}
\label{nrj2_CECSP_SE}
\end{equation} 

Il nous reste à écrire les contraintes permettant de lier les deux
variables, i.e.  de s'assurer de la bonne conversion
ressource/énergie. Nous ajoutons donc les contraintes suivantes au
modèle:
\begin{align}
  &w_{ie} \le a_ib_{ie}+c_i(t_{e+1}-t_e) & & \forall i \in A,\ \forall
  e \in \E \setminus\{2n\} \label{conv1_CECSP_SE}\\
     &w_{ie} \le W_i (\sum_{f=0}^{e} x_{if} -\sum_{f=0}^{e} y_{if}) &
    & \forall i \in A,\ \forall e \in {\cal
      E}\setminus\{2n\} \label{conv2_CECSP_SE}\\
& w_{ie} \ge 0 & & \forall i \in A,\ \forall e \in {\cal
      E}\setminus\{2n\}
\label{Bw_CECSP_SE}
\end{align}

La contrainte~\eqref{conv1_CECSP_SE} permet de modéliser l'énergie
apportée à l'activité $i$ en fonction de la quantité de ressource
consommée par cette même activité dans l'intervalle
$[t_e,t_{e+1}]$. Notons d'abord l'utilisation d'une inégalité et non
d'une égalité. Ceci est dû au fait que, lorsque l'activité n'est pas
en cours entre $t_e$ et $t_{e+1}$, i.e. $b_{ie}=0$, la contrainte
devient $w_{ie} \le c_i(t_{e+1}-t_e)$. Dans le cas d'une égalité, ceci
aurait été faux puisqu'on devrait avoir $w_{ie}=0$. Le rôle de la
seconde contrainte est donc de s'assurer que lorsque l'activité n'est
pas en cours, on ait bien $w_{ie}=0$. En effet, lorsque l'activité
n'est pas en cours, la contrainte devient $w_{ie}\le 0$ et dans le cas
contraire, l'inégalité s'écrit $w_{ie} \le W_i$.

La seconde remarque qui peut être faite est que, puisque la
contrainte~\eqref{conv1_CECSP_SE} utilise une inégalité, $w_{ie}$ peut
ne pas être réellement égale à la quantité d'énergie apportée à $i$
entre $t_e$ et $t_{e+1}$. En fait, grâce à
l'objectif~\eqref{obj_CECSP_SE}, ceci ne peut arriver.

\begin{theo}
  \label{th:conv}
  Soit $S$ une solution optimale du modèle décrit
  par~\eqref{obj_CECSP_SE}--\eqref{dur_CECSP_SE},
  \eqref{nrj2_CECSP_SE} et
  \eqref{bmin_CECSP_SE}--\eqref{Bw_CECSP_SE}. Alors $S$ vérifie la
  condition suivante: \[\forall i \in \A,\ \forall e \in \E\setminus\{2n\},\
  w_{ie}=f_i\left(\frac{b_{ie}}{t_{e+1}-t_e}\right)(t_{e+1}-t_e)\]
\end{theo}

\begin{proof}
Pour prouver le théorème, commençons par remarquer que la
contrainte~\eqref{conv1_CECSP_SE} implique
$w_{ie} \le f_i\left(\frac{b_{ie}}{t_{e+1}-t_e}\right)(t_{e+1}-t_e)$. En effet, 
\begin{align*}
  f_i\left(\frac{b_{ie}}{t_{e+1}-t_e}\right)(t_{e+1}-t_e)&
  =\left(a_i\frac{b_{ie}}{t_{e+1}-t_e}+c_i\right)(t_{e+1}-t_e)\\  
   &=a_ib_{ie}+c_i(t_{e+1}-t_e)
\end{align*}
Il nous reste donc à montrer que $w_{ie}\ge
f_i\left(\frac{b_{ie}}{t_{e+1}-t_e}\right)(t_{e+1}-t_e)$. Par l'absurde,
supposons que $\exists i \in \A,\ \exists e \in \E\setminus\{2n\}$
tel que $w_{ie} < 
f_i\left(\frac{b_{ie}}{t_{e+1}-t_e}\right)(t_{e+1}-t_e)$. Soit $\S_C$
l'ensemble des solutions du \CECSP~et soit  $c :\S_C
\rightarrow \mathbb{R}$ la fonction qui associe à une solution $S$ sa
consommation de ressource $c(S)$. Nous allons créer une nouvelle
solution $S'$ qui vérifie que $c(S)>c(S')$, ce qui invalidera
l'optimalité de $S$.

Pour cela, nous considérons la solution du \CECSP~associée à
$S$. Notons $\widehat{S}$ cette solution. $\widehat{S}$ est obtenue à
partir de $S$ de la manière suivante, $\forall (i,e)
\in \A \times \E$: 
\begin{itemize}
\item si $x_{ie}=1$ alors $st_i$ est fixé à $t_e$,
\item si $y_{ie}=1$ alors $et_i$ est fixé à $t_e$,
\item $b_i(t)=\frac{b_{ie}}{t_{e+1}-t_e}, \forall t \in [t_e,t_{e+1}]$. 
\end{itemize}

$S'$ est définie de la façon suivante: 
\begin{itemize}
\item $st_i'=st_i$,
\item $et_i'=\min(t \in \H | \int_{st_i}^t f_i(b_i(t)) = W_i)$,
\item $b_i'(t)=\left\{ \begin{array}{ll}
                         b_i(t)& \text{si } st_i' \le t \le et_i'\\ 
                         0 & \text{sinon}
                       \end{array}
                     \right.$
\end{itemize}
Clairement, $S'$ est une solution pour le \CECSP. Nous montrons
maintenant que $c(S) = c(\widehat{S})> c(S')$. Pour cela, nous allons montrer qu'il
existe une tâche $i$ telle que $et_i > et_i'$ et donc
$\int_{st_i}^{et_i} b_i(t)dt > \int_{st_i'}^{et_i'} b_i(t)dt$. De ce
fait, nous aurons bien que $\sum_{i \in \A} \int_{st_i}^{et_i}
b_i(t)dt > \sum_{i \in \A}\int_{st_i'}^{et_i'} b_i(t)dt$. 

On sait que $\exists (i,e) \in \A \times \E$ tel que:
\begin{align*}
  &w_{ie} < f_i\left(\frac{b_{ie}}{t_{e+1}-t_e}\right)(t_{e+1}-t_e)\\  
  \Rightarrow &\sum_{e \in  \E} w_{ie} < \sum_{e \in \E} f_i\left(
                \frac{b_{ie}}{t_{e+1}-t_e}\right)(t_{e+1}-t_e)\\ 
  \Rightarrow & W_i=\sum_{e \in  \E} w_{ie} < \sum_{e \in \E}
                f_i(b_i(t))(t_{e+1}-t_e)=\int_{st_i}^{et_i}
                f_i(b_i(t))dt \\
  \Rightarrow & \min\left(t \in \H |
                \int_{st_i}^{t}f_i(b_i(t))dt\right) = et_i' < et_i                 
\end{align*}

Donc nous avons bien $et_i > et_i'$ et ceci achève la démonstration.
\end{proof}

Notons que pour d'autres objectifs cela peut ne pas être valide. Mais, la
solution obtenue par le modèle peut toujours être transformée en une
solution du \CECSP~en temps polynomial. Pour cela, il suffit de suivre
la transformation proposée dans la preuve du théorème~\ref{th:conv}.

Le modèle ainsi défini possède $4n^2$ variables binaires, $4n^2$
variables continues et $2n^3+17n^2+2n-2$ contraintes.

\paragraph{Fonction de rendement affine par morceaux}

Lorsque les fonctions de rendement sont des fonctions concaves
affines par morceaux, nous utilisons aussi la variable $w_{ie}$ pour
représenter l'énergie reçue par l'activité $i$ dans la période
$[t_e,t_{e+1}]$. La contrainte~\eqref{nrj2_CECSP_SE} est utilisée pour
modéliser le fait que cette activité doive recevoir une quantité
d'énergie $W_i$ durant son exécution.

Pour s'assurer de la bonne conversion entre la quantité de ressource
consommée par une activité $i$ dans une période $[t_e,t_{e+1}]$ et la
quantité d'énergie reçue par cette dernière dans la même période,
l'inégalité~\eqref{conv1_CECSP_SE} est remplacée par l'inégalité
suivante:
\begin{equation}
w_{ie} \le a_{ip}b_{ie} + c_{ip}\left( t_{e+1} - t_e\right) \quad  
\forall i \in \A,\ \forall p \in \P_i,\ \forall e \in \E\setminus\{2n\}  
\label{conv2_CECSP_SE}
\end{equation}

Notons que la même argumentation que celle de la preuve du
théorème~\ref{th:conv} permet de nous assurer que la conversion faite
par le modèle entre la quantité d'énergie reçue par l'activité et la
quantité de ressource consommée par cette dernière est valide.

Le modèle ainsi défini possède $4n^2$ variables binaires, $4n^2$
variables continues et $2n^3+ (15 + 2P)n^2+\left(1-P\right)n-2$
contraintes, avec $P=\max_{i\in A}|\P_i|$.

\subsubsection{Modèle on/off}
\label{sssection:OO_CECSP}
Dans le modèle on/off, une variable binaire $z_{ie}$ se charge
d'assigner à chaque événement les dates de début et de fin des
activités. La variable $z_{ie}$ prendra la valeur $1$ si et seulement
si l'activité $i$ est en cours d'exécution dans la période
$[t_e,t_{e+1}]$ et sera égale à $0$ sinon. Ceci permet de diviser par
deux le nombre de variables binaires utilisées par rapport à la
formulation start/end.

\paragraph{Fonction de rendement identité}

Dans le cas où les fonctions de rendement sont la fonction identité,
nous définissons la variable $b_{ie}$ qui modélise la quantité de
ressource consommée par l'activité $i$ entre les dates $t_e$ et
$t_{e+1}$. Comme pour le modèle start/end, nous avons seulement besoin
de définir la variable $b_{ie},\ \forall i \in \A$ et $\forall e \in
\Em$. Ceci conduit à la formulation suivante: 
{\small
\begin{align}
& \text{min } \sum_{i\in A}\ \sum_{e\in \E\setminus
    \{2n\}}b_{ie}
 \label{obj_CECSP_OO}\\ 
&t_e \le t_{e+1} & &\forall e \in \E\setminus
 \{2n\} \label{ordre_CECSP_OO}\\
&\sum_{e \in \E} z_{ie} \ge 1 & & \forall i \in {\cal
   A}\label{start_CECSP_OO}\\
& \ES z_{ie}\le t_e \le \LS(z_{ie}-z_{i,e-1})+(1-(z_{ie}-z_{i,e-1}))T
 & & \forall e \in \E\setminus \{1\},\ \forall i \in {\cal
   A}\label{twx_CECSP_OO}\\
&\EE(z_{i,e-1}-z_{ie})\le t_e & & \forall e \in \E\setminus
 \{1\},\ \forall i \in \A\label{twy1_CECSP_OO}\\
&t_e \le \LE(z_{i,e-1}-z_{ie})+(1-(z_{i,e-1}-z_{ie}))T & & \forall e
 \in \E\setminus \{1\},\ \forall i \in {\cal
   A}\label{twy2_CECSP_OO}\\
&t_f\ge t_e +((z_{ie}-z_{i,e-1})-(z_{if}-z_{i,f-1})-1)W_i/f_i(\bmax) &
 & \forall e,f \in \E,\ f>e\neq 1,\ \forall i \in {\cal
   A} \label{dur_CECSP_OO}\\
&\sum_{e'=1}^{e} z_{ie'} \le e(1-(z_{ie}-z_{i,e-1})) & & \forall e \in
       \E\setminus \{1\},\ \forall i \in \A
\label{preem1_CECSP_OO}\\
&\sum_{e'=e}^{2n} z_{ie'} \le (2n-e)(1+(z_{ie}-z_{i,e-1})) & & \forall
e \in \E\setminus \{1\},\ \forall i \in \A
\label{preem2_CECSP_OO}\\
&\sum_{i\in \A} b_{ie} \le R(t_{e+1}-t_e) & & \forall e \in
      \E\setminus \{2n\}\label{res_CECSP_OO}\\
&\sum_{e \in \E\setminus \{2n\}} b_{ie} = W_i & & \forall i \in
        \A\label{nrj_CECSP_OO}\\
&b_{ie} \ge \bmin(t_{e+1}-t_e) - M(1-z_{ie}) & & \forall e \in {\cal
  E}\setminus \{2n\}, \ \forall i \in \A\label{bmin_CECSP_OO}\\
&b_{ie} \le \bmax(t_{e+1}-t_e) & &\forall e \in \E\setminus
\{2n\},\ \forall i \in \A\label{bmax_CECSP_OO}\\
&z_{ie}M\ge b_{ie} & & \forall e\in \E\setminus \{2n\},
\ \forall i \in \A\label{res0_CECSP_OO}\\
&t_e \ge 0 & & \forall e \in \E \label{eq54}\\
& b_{ie} \ge 0 & & \forall i \in A,\ \forall e \in \E\setminus
\{2n\}
 \label{Bb_CECSP_OO}\\
&z_{ie} \in \{0,1\} & & \forall i \in A,\ \forall e \in {\cal
   E} \label{eq57}
\end{align}
}
Dans cette formulation, l'objectif est décrit
par~\eqref{obj_CECSP_OO}. Ici, l'objectif est de minimiser la
consommation totale de la ressource durant tout le projet. Si 
l'on souhaite minimiser la date de fin du projet, l'objectif s'écrit 
facilement comme: 
\begin{equation}
\text{min } t_{|\E|} \notag
\end{equation} 

La contrainte~\eqref{ordre_CECSP_OO} joue le même rôle que dans la
formulation start/end, à savoir, ordonner les événements.

La contrainte~\eqref{start_CECSP_OO} stipule que chaque activité doit
être ordonnancée une fois dans toute la durée du projet.
 
La contrainte~\eqref{twx_CECSP_OO} s'assure que la date d'un événement
correspondant au début d'une activité soit bien comprise dans sa
fenêtre de temps.  En effet, si l'événement $e$ correspond au début de
l'activité $i$, i.e. $z_{ie}=1$ et $z_{i,e-1}=0$ , l'inégalité devient
alors: $\ES \le t_e \le \LS$. Notons que pour le cas $e=1$,
$z_{i,e-1}-z_{ie}$ est remplacé par $z_{ie}$. De même, les
contraintes~\eqref{twy1_CECSP_OO} et~\eqref{twy2_CECSP_OO}
garantissent qu'un événement correspondant à la fin d'une activité
soit bien compris entre $\EE$ et $\LE$. Ici, il n'est pas
nécessaire de considérer le cas $e=1$ car cet événement ne peut pas
correspondre à la fin d'une activité. 

La contrainte~\eqref{dur_CECSP_OO} assure une séparation suffisante
entre un événement correspondant au début d'une activité et un
événement correspondant à la fin de cette même activité. Ici, comme
nous ne connaissons pas la durée d'une activité, nous utilisons une
borne inférieure sur cette dernière. Pour le cas où $e=1$, il suffit
de remplacer $z_{i,e-1}-z_{ie}$ par $z_{ie}$.

Les contraintes~\eqref{preem1_CECSP_OO} et~\eqref{preem2_CECSP_OO},
appelées {\it contraintes de contiguïté}, assure la non-préemption des
activités. Une preuve formelle de la validité de ces contraintes est
décrite dans~\cite{modele_RCPSP}. Ici aussi, le cas $e=1$, est 
traité en remplaçant $z_{i,e-1}-z_{ie}$ par $z_{ie}$.

La contrainte~\eqref{res_CECSP_OO} représente la capacité de
la ressource tandis que la contrainte~\eqref{nrj_CECSP_OO} s'assure
que chaque activité reçoit bien la quantité d'énergie requise par la
donnée du problème.

Les contraintes~\eqref{bmin_CECSP_OO},~\eqref{bmax_CECSP_OO} et
\eqref{res0_CECSP_OO} -- permettant respectivement de modéliser les
contraintes de consommation minimale, de consommation maximale et de
consommation nulle en dehors de l'intervalle d'exécution de $i$ -- se
déduisent directement des
contraintes~\eqref{bmin_CECSP_SE},~\eqref{bmax_CECSP_SE} et
\eqref{res0_CECSP_SE}. Dans un premier temps, remarquons que nous
avons la relation suivante:
\begin{equation}
\label{SEversOO}
z_{ie} = \sum_{f=1}^e x_{if} -  \sum_{f=1}^e y_{if} \quad \forall i
\in \A,\ \forall e\in \Em
\end{equation}
En effet, $z_{ie}$ vaut $1$ si et seulement si l'activité $i$ est en
cours entre $t_e$ et $t_{e+1}$, c'est-à-dire si l'activité a commencé à
l'événement $e$ ou avant, i.e. $\sum_{f=1}^ex_{if}=1$, et si elle
ne finit pas à l'événement $e$ ou avant, i.e. $\sum_{f=1}^ey_{if}=
0$. Dans ce cas, nous avons bien $\sum_{f=1}^ex_{if} -
\sum_{f=1}^ey_{if}= 1 = z_{ie}$. Les
contraintes~\eqref{bmin_CECSP_OO},~\eqref{bmax_CECSP_OO} et
\eqref{res0_CECSP_OO} s'obtiennent donc facilement à partir
de~\eqref{bmin_CECSP_SE},~\eqref{bmax_CECSP_SE} et
\eqref{res0_CECSP_SE} en remplaçant $\sum_{f=0}^{e}
x_{if}-\sum_{f=0}^{e}y_{if}$ par $z_{ie}$.


Le modèle ainsi défini possède $2n^2-n$ variables binaires, $2n^2+n$
variables continues et $2n^3+13n^2-n-2$ contraintes.    

\paragraph{Fonction de rendement affine}

Pour modéliser la conversion de la ressource en énergie, nous
définissons une variable $w_{ie}$, $\forall (i,e) \in \A \times \Em$,
représentant la quantité d'énergie 
apportée à l'activité $i$ dans l'intervalle $[t_e,t_{e+1}]$. Puis,
nous identifions les contraintes du précédent modèle qui ont besoin
d'être réécrites à l'aide de cette variable. Ici, seule la
contrainte~\eqref{nrj_CECSP_OO} est concernée. La réécriture donne la
contrainte suivante:
\begin{equation}
\sum_{e\in \E} w_{ie}=W_i \quad \forall i \in \A
\tag{\ref{nrj_CECSP_OO}'}
\end{equation}

Dans un second temps, nous écrivons les contraintes liant les variables
$b_{ie}$ et $w_{ie}$:
\begin{align}
  &w_{ie} \le a_ib_{ie}+c_i(t_{e+1}-t_e) & & \forall i \in A,\ \forall
  e \in \E \setminus\{2n\} \label{conv1_CECSP_OO}\\
  &w_{ie} \le W_i z_{ie} & & \forall i \in A,\ \forall e \in
                             \E\setminus\{2n\} \label{conv2_CECSP_OO}\\ 
  & w_{ie} \ge 0 & & \forall i \in A,\ \forall e \in \E\setminus\{2n\}
  \label{Bw_CECSP_OO}
\end{align}

Notons que la même argumentation que celle de la preuve du
théorème~\ref{th:conv} permet de nous assurer de la bonne conversion
entre énergie et ressource du modèle. 

Le modèle ainsi défini possède $2n^2-n$ variables binaires, $4n^2$
variables continues et $2n^3+17n^2-3n-2$ contraintes.

\paragraph{Fonction de rendement affine par morceaux}

Lorsque les fonctions de rendement sont des fonctions concaves
affines par morceaux, nous utilisons aussi la variable $w_{ie}$ pour
représenter l'énergie reçue par l'activité $i$ dans la période
$[t_e,t_{e+1}]$. La contrainte~\eqref{nrj2_CECSP_SE} est utilisée pour
modéliser le fait que cette activité doive recevoir une quantité
d'énergie $W_i$ durant son exécution.

Pour s'assurer de la bonne conversion entre la quantité de ressource
consommée par une activité $i$ dans une période $[t_e,t_{e+1}]$ et la
quantité d'énergie reçue par cette dernière dans la même période,
l'inégalité~\eqref{conv1_CECSP_SE} est remplacée par l'inégalité
suivante:
\begin{equation}
w_{ie} \le a_{ip}b_{ie} + c_{ip}\left( t_{e+1} - t_e\right) \quad  
\forall i \in \A,\ \forall p \in \P_i,\ \forall e \in \Em
\label{conv2_CECSP_SE}
\end{equation}

Le modèle ainsi défini possède $2n^2-n$ variables binaires, $4n^2$
variables continues et $2n^3+(15 + 2P)n^2+\left(2-P\right)n-2$
contraintes, avec $P=\max_{i\in A}|\P_i|$.    


Dans la partie suivante, nous allons montrer comment améliorer ces
modèles par le biais d'une étude polyédrale et la proposition
d'inégalités valides.

\section{Renforcement des modèles}

Dans cette section, nous présentons des améliorations mises en place
pour les modèles indexés par le temps et les modèles à événements du
\RCPSP~et du \CECSP. 

Dans un premier temps, nous nous sommes intéressés aux modèles à temps
discrets pour lesquels nous proposons des inégalités valides déduites
directement du raisonnement énergétique présenté à la
section~\ref{section:RE}. En effet, ces modèles restent les modèles
les plus efficaces pour résoudre le \CECSP~ou le \RCPSP, ceci étant
principalement dû à la qualité relative de leur relaxation. 

Cependant, les limitations de ces modèles nous ont poussé à nous
intéresser plus particulièrement à l'amélioration des modèles à
événement. En effet, le nombre de variables et de contraintes de ces
modèles dépendent de la taille de l'horizon de temps du problème. Leur
taille peut donc s'avérer très importante si l'horizon de temps est grand.
De plus, dans les modèles indexés par le temps du \CECSP~, la
discrétisation de l'horizon de temps conduit à une réduction de
l'espace des solutions et peut donc mener à des solutions
sous-optimales (voir exemple~\ref{exemple_NE},
page~\pageref{exemple_NE}). 

Pour améliorer les relaxations linéaires des modèles à événement, nous
avons étudié le polyèdre défini par l'ensemble de toutes les
affectations possibles des variables binaires pour une seule activité,
pour lequel nous exhibons un ensemble minimal d'inégalités permettant
de le décrire. De plus, plusieurs ensembles d'inégalités valides sont
proposées pour contribuer à l'amélioration des performances des
modèles à événement.
 
\subsection{Amélioration des modèles indexés par le temps}

Dans cette section, nous montrons comment nous utilisons le
raisonnement énergétique décrit à la section~\ref{section:RE} pour
exhiber des inégalités valides pour les modèles de programmation
linéaires mixtes indexés par le temps du \RCPSP~et du \CECSP. Nous
commençons par détailler le cas du \CECSP~avant de montrer comment
appliquer le même raisonnement pour le \RCPSP.

Pour intégrer le raisonnement énergétique dans les modèles indexés par
le temps, nous utilisons l'équation qui se trouve au centre de ce
raisonnement (cf. équation~\ref{centreRE}). Soit $\I$ l'ensemble des
intervalles d'intérêt du raisonnement énergétique, alors l'équation
s'écrit:  
\begin{align} 
  SL(t_1,t_2) &\ge 0 \quad \forall [t_1,t_2] \in \I \nonumber\\
  \Rightarrow \bb + \sum_{\substack{j\in \A\\j\neq i}}\bb[j] &\le
  R(t_2-t_1)\quad \forall [t_1,t_2] \in \I \nonumber
\end{align}  

Soit $CR: \mathbb{R} \times \H \times \H \rightarrow \mathbb{R}$ qui
associe à une quantité d'énergie $w$ et à un intervalle $[t_1,t_2]$,
la quantité de ressource minimale à consommer dans $[t_1,t_2]$ pour
apporter à une activité une énergie $w$ (voir
équation~\eqref{eq:conv}). En considérant toutes les
expressions possibles pour $\bb$ et en utilisant les variables
$x_{it}$ et $y_{it}$ pour activer ou non l'équation correspondant à la
consommation minimale de ressource dans $[t_1,t_2]$, nous obtenons les
inégalités suivantes: 
\begin{align}
&\left(1-\sum_{t\le t_1}x_{it} - \sum_{t\ge t_2}y_{it}\right) \, \conv[W_i] +
\sum_{j\neq i} \bb[j] \le R(t_2-t_1) \nonumber\\
& \hspace{9.55cm} \forall i \in {\cal A},\ \forall [t_1,t_2] \in \I
\label{total}
\end{align}

L'inégalité~\eqref{total} représente le cas où l'intervalle
$[\ES,\LE]$ est complètement inclus dans $[t_1,t_2]$. En effet,
les différents cas possibles sont: 
\begin{itemize}
\item l'activité commence après $t_1$ et finit avant $t_2$. Dans ce
  cas là, l'inégalité devient: $\conv[W_i] +\sum_{j\neq i} \bb[j] \le
  R(t_2-t_1)$. Ceci correspond au cas où la contrainte est activée. 
\item l'activité commence avant $t_1$ et finit avant $t_2$. Ici,
  l'inégalité devient: $\sum_{j\neq i} \bb[j] \le R(t_2-t_1)$ et ceci
  est vérifié quelque soit $i$ et $[t_1,t_2]$. Ce cas correspond au
  cas où la contrainte n'est pas activée. 
\item l'activité commence après $t_1$ et finit après $t_2$. Ce cas est
  similaire au précédent. 
\item l'activité commence avant $t_1$ et finit après
  $t_2$. L'inégalité s'écrit alors: $\sum_{j\neq i} \bb[j]-\conv[W_i]
  \le R(t_2-t_1)$. Ce cas est aussi similaire au second. 
\end{itemize}

\begin{align}
&\left(x_{i,\ES} + y_{i,\LE} -1 \right) \, \conv[W_i-f_i(\bmax)\left(t_1-\ES + \LE
-t_2\right)] \nonumber \\
&+ \sum_{j\neq i} \bb[j] \le R\left(t_2-t_1\right) \hspace{4.35cm} \forall i \in
{\cal A},\ \forall [t_1,t_2] \in \I  \label{both}
\end{align}

L'inégalité~\eqref{both} correspond au cas où l'activité est centrée,
i.e. $\ES\le t_1<t_2 \le \LE$ et $W_i-\left(t_1-\ES + \LE -t_2\right)
\ge f_i(\bmin)(t_2-t_1)$. Dans ce cas, l'inégalité sera activée si et
seulement si l'activité commence à $\ES$ ($x_{i,\ES}=1$) et finit à
$\LE$ ($y_{i,\LE}=1$). 

\begin{align}
&\left(x_{i,\ES} + \sum_{t=t_1}^{t_2}y_{it} -1\right) \,
\conv[W_i-f_i\left(\bmax\right)\left(t_1-\ES\right)]+ \nonumber\\
&\sum_{j\neq i} \bb[j] \le R\left(t_2-t_1\right) \hspace{4.75cm} \forall i \in
  {\cal A},\ \forall [t_1,t_2] \in \I
\label{left}
\end{align}

L'inégalité~\eqref{left} correspond au cas où l'activité est calée à
gauche, i.e. exécutée à $\bmax$ pendant l'intervalle
$[\ES,t_1]$. Cette inégalité est activée quand l'activité commence à
$\ES$ et finit dans l'intervalle $[t_1,t_2]$. 

\begin{align}
&\left(\sum_{t=t_1}^{t_2}x_{it} + y_{i,\LE}-1\right) \,
\conv[W_i-f_i\left(\bmax\right)\left(t_2-\LE\right)]\nonumber\\
& + \sum_{j\neq i} \bb[j] \le R\left(t_2-t_1\right) \hspace{4.29cm} \forall i \in
{\cal A},\ \forall [t_1,t_2] \in \I
\label{right}
\end{align} 

L'inégalité~\eqref{right} correspond au cas où l'activité est calée à
droite, i.e. exécutée à $\bmax$ pendant l'intervalle
$[t_2,\LE]$. Cette inégalité est activée quand l'activité commence
dans l'intervalle $[t_1,t_2]$ et finit a $\LE$.

\begin{align}
&\left(\sum_{t\le t_1}x_{it} + \sum_{t\ge t_2}y_{it} -1 \right) \,
  \bmin\left(t_2-t_1\right) + \sum_{j\neq i} \bb[j] \nonumber\\
& \le R\left(t_2-t_1\right)\hspace{7cm} \forall i \in {\cal A},\ \forall [t_1,t_2]
\in \I
\label{min}
\end{align}

Enfin, l'inégalité~\eqref{min} correspond au cas où l'activité est
exécutée à $\bmin$ durant l'intervalle $[t_1,t_2]$. Cette inégalité
assure que si l'activité commence à $t_1$ ou avant et finit à $t_2$ ou
après alors la quantité de ressource disponible dans $[t_1,t_2]$ est
suffisante pour exécuter l'activité à $\bmin$ durant tout cet
intervalle.


Notons que ces inégalités peuvent aussi être définies dans le cas du
\RCPSP. Dans ce cas là, nous devons appliquer le même raisonnement
pour chaque ressource. Cependant, il y aura seulement trois cas à
considérer: l'activité est calée à gauche, l'activité est calée à
droite et l'activité est en cours d'exécution durant l'intervalle
$[t_1,t_2]$. Ceci donne les inégalités suivantes:
\begin{multline} \left(x_{i,\ES} + \sum_{t=t_1}^{t_2}y_{it} -1\right)
\, r_{ik}p_i^+(t_1)+ \sum_{j\neq i} \bb[j] \le R\left(t_2-t_1\right)
\qquad \\ \forall i \in {\cal A},\ \forall k \in \R ,\ \forall
[t_1,t_2] \in \I
    \label{RCPSP_left}
\end{multline} \vspace{-1.3cm}
\begin{multline} \left(\sum_{t=t_1}^{t_2}x_{it} + y_{i,\LE}-1\right)
\, b_ip_i^-(t_2) + \sum_{j\neq i} \bb[j] \le R\left(t_2-t_1\right)
\qquad \\ \forall i \in {\cal A},\ \forall k \in \R ,\ \forall
[t_1,t_2] \in \I
    \label{RCPSP_right}
\end{multline} \vspace{-1.3cm}
\begin{multline} \left(\sum_{t\le t_1}x_{it} + \sum_{t\ge t_2}y_{it}
-1 \right) \, b_i\left(t_2-t_1\right) + \sum_{j\neq i} \bb[j] \le
R\left(t_2-t_1\right) \qquad \\ \forall i \in {\cal A},\ \forall k \in
\R ,\ \forall [t_1,t_2] \in \I
    \label{RCPSP_min}
\end{multline}
avec $p_i^+(t_1)$ et $p_i^-(t_2)$ comme définis dans la
section~\ref{section:ERCuSP} pour le \CUSP.

De telles inégalités ont été proposées dans le cadre du modèle
préemptif sur les ensembles admissibles~\cite{BD}. Ces inégalités sont
ajoutées au modèle indexé par le temps du \CECSP~et du \RCPSP~et les
performances de ces modèles avec ou sans ces inégalités sont évaluées
dans le chapitre~\ref{sec:expe} afin de montrer leur intérêt.
 
Les modèles à temps discret sont particulièrement efficaces pour le
\RCPSP~et le \CECSP. Cependant, ces modèles possèdent des limitations
qui sont décrites dans la section suivante. 



\subsection{Amélioration des modèles à événement}

Dans cette section, nous présentons les améliorations possibles des
modèles à événement. Dans un premier temps, nous montrons que le
modèle start/end possède de meilleures relaxations que le modèle
on/off, puis nous présentons un ensemble d'inégalités que nous pouvons
rajouter au modèle on/off pour renforcer sa relaxation. Cet ensemble
d'inégalités est ensuite utilisé pour donner une description minimale
du polyèdre défini par l'ensemble de toutes les affectations possibles
des variables binaires $z_{ie}$ pour une seule activité. 

D'autres ensembles d'inégalités sont ensuite présentées permettant
l'ajout de nouvelles contraintes au modèle, la suppression des
coefficients big-$M$ dans les contraintes existantes ou l'amélioration
des contraintes existantes. Dans cette section, nous présentons les
résultats pour le \CECSP~car, dans la plupart des cas, le remplacement
de $\Em=\{1,\dots,2n-1\}$ par $\{1,\dots,n\}$ suffit à l'adaptation
de la preuve au \RCPSP. Cependant, lorsque ce n'est pas le cas ou que
le résultat n'a pas été prouvé pour le \RCPSP, nous le précisons. 

\subsubsection{Comparaison des relaxations linéaires}

Pour montrer que le modèle start/end possède de meilleures relaxations
linéaires que le modèle on/off, nous commençons par montrer que le
modèle start/end possède des relaxations au moins aussi bonne que
celle du modèle on/off. Pour cela, il suffit de  montrer que toute solution du 
modèle start/end, entière ou non, est solution du modèle on/off. Ceci
peut être fait en écrivant les variables $z_{ie}$ en fonction des
variables $x_{ie}$ et $y_{ie}$ et nous avons la relation suivante,
décrite le paragraphe~\ref{sssection:OO_CECSP}: 
\begin{equation}
\label{SEversOO}
z_{ie} = \sum_{f=1}^e x_{if} -  \sum_{f=1}^e y_{if} \quad \forall i
\in \A,\ \forall e\in \Em
\end{equation}

Il nous reste maintenant à montrer que les relaxations du modèle
on/off ne sont pas aussi bonnes que celles du modèle start/end. Pour
cela, nous relâchons les contraintes d'intégrité des deux modèles et
nous trouvons une solution pour la relaxation du modèle on/off qui
n'en est pas une pour la relaxation du modèle start/end. 

Considérons le sous-modèle formé des
équations~\eqref{start_CECSP_OO}, \eqref{preem1_CECSP_OO} et
\eqref{preem2_CECSP_OO} et le sous-modèle formé des
équations~\eqref{start_CECSP_SE}, \eqref{end_CECSP_SE},
\eqref{SEversOO} et de l'ensemble d'équations suivant: $\sum_{f=1}^e
y_{if} +\sum_{f=e}^n x_{if}\le 1,\ \forall i \in \A$. Clairement, si
nous trouvons une solution pour le premier sous-modèle qui n'en est
pas une pour le second, ceci montrera bien que les relaxations du modèle
on/off ne sont pas aussi bonnes que celles de smodèles start/end. En
effet, toute solution d'un des sous-modèles est solution du modèle
entier (start/end ou on/off) en considérant l'instance suivante:
$\forall i \in \A,\ \ES=0,\ \LE=T,\ \bmin=0,\ \bmax=1,\ W_i=1,\
f_i(b_i(t))=b_i(t)$ et $B=n$. 

Nous nous plaçons dans le cas où le nombre d'activité est de 2. Alors,
nous avons $|\E|=4$.  Dans la suite, les tâches sont dénotées par les
indices $a,\ b$ et $c$ et les évênements par $0,\ 1,\ 2$ et $3$.

Nous constatons tout d'abord que $z_a=(0,6;0;0,7;0)$ est bien une
solution du sous-modèle on/off. 

Nous devons maintenant montrer que le sous modèle start/end ne possède
pas de solution avec $z_a=(0,6;0;0,7;0)$. Pour cela, étudions les contraintes du
modèle qui n'impliquent que l'activité $a$:

\begin{align} &x_{a0}+x_{a1}+x_{a2}+x_{a3}= 1 \label{c9}\\
              &y_{a0}+y_{a1}+y_{a2}+y_{a3}= 1 \label{c10}\\
              &y_{a0}+x_{a0}+x_{a1}+x_{a2}+x_{a3}\le1 \label{c11}\\
              &y_{a0}+y_{a1}+x_{a1}+x_{a2}+x_{a3}\le1 \label{c12}\\
              &y_{a0}+y_{a1}+y_{a2}+x_{a2}+x_{a3}\le1 \label{c13}\\
              &y_{a0}+y_{a1}+y_{a2}+y_{a3}+x_{a3}\le1 \label{c14}\\
              &0.6=x_{a0}-y_{a0}\label{c17}\\
              &0=x_{a0}-y_{a0}+x_{a1}-y_{a1}\label{c18}\\
              &0.7=x_{a0}-y_{a0}+x_{a1}-y_{a1}+x_{a2}-y_{a2}\label{c19}\\
              &0=x_{a0}-y_{a0}+x_{a1}-y_{a1}+x_{a2}-y_{a2}+x_{a3}-y_{a3}\label{c20}
\end{align}

Clairement, dans le modèle start/end, nous avons les contraintes
suivantes: $x_{a3}=0$ et $y_{a0}=0$. Donc, les contraintes~(\ref{c17})
et~(\ref{c20}) impliquent que $x_{a0}=0.6$ et $y_{a3}=0.7$. 

Or, la contrainte~(\ref{c18}) implique $y_{a1}=0.6+x_{a1}$
et la contrainte~(\ref{c9}) implique $0.6+x_{a1}+x_{a2}=1 \Rightarrow
x_{a2}=0.4-x_{a1}$. 

La contrainte~(\ref{c12}) peut donc s'écrire
$0.6+2x_{a1}+x_{a2}\le 1 \Rightarrow 1+x_{a1}\le 1 \Rightarrow
x_{a1}=0$. 

Ceci implique $x_{a2}=0.4$ et $y_{a1}=0.6$, et l'on obtient
une contradiction avec la contrainte~(\ref{c10}) puisque $0+0.6+y_{a2}+0.7>1$.

Donc, nous avons bien montré que le modèle start/end possède en
théorie de meilleures relaxations que le modèle on/off. Cependant, si
nous ajoutons un ensemble particulier d'inégalités à ce modèle, ceci
n'est plus le cas. Ces inégalités sont décrites dans le paragraphe
suivant.


\subsubsection{Inégalités de non-préemption}

Dans ce paragraphe, nous présentons un ensemble d'inégalités que nous
utilisons pour donner une description minimale du polyèdre défini par
l'ensemble de toutes les affectations possibles des variables binaires
$z_{ie}$ pour une seule activité. 

L'ensemble d'inégalités, appelées inégalités de non-préemption, est
défini comme suit. Puisque, dans tout ordonnancement réalisable,
chaque activité doit être exécutée sans préemption, $z_{ie}$ doit
satisfaire:
\begin{equation}
  \sum_{e_u \in {\cal F}} (-1)^{u} z_{i,e_u} \le 1
\label{non_preem_ineg}
\end{equation}
où ${\cal F}=\{e_0,e_1,\dots,e_{2v}\}$ est un sous-ensemble ordonné de
cardinalité impaire de $\E^*=\Em$.  

Soit le polyèdre $ZP_i=\{z_i \in [0,1]^{\E^*}\ | \ z_i\text{ satisfait
\eqref{start_CECSP_OO} et \eqref{non_preem_ineg}}\}$ et soit
$ZQ_i=\mathrm{conv}\{z_i \in \{0,1\}^{\E^*}\ | \ z_i\text{ satisfait
\eqref{start_CECSP_OO},\eqref{preem1_CECSP_OO} et
\eqref{preem2_CECSP_OO}}\}$. Nous allons montrer le théorème suivant: 

\begin{theo}
$ZP_i=ZQ_i$
\end{theo}

\begin{proof}

Dans un premier temps, nous rappelons le lemme de Farkas que nous
utiliserons plus tard dans la preuve. 

\begin{lemma}[Lemme de Farkas]
\label{Farkas_Lemma}
Soit $A$ une matrice de taille $m \times n$ et $b$ un vecteur de
$\mathbb{R}^m$. Il existe un vecteur $x \in \mathbb{R}^n$ vérifiant
$Ax = b$ si et seulement si pour tout vecteur $y \in \mathbb{R}^m$
tel que $ y^TA\le 0$ on a $y^Tb \le 0$. 
\end{lemma}

Nous allons exhiber un ensemble d'inégalités linéaires décrivant
$ZQ_i$. Pour cela, commençons par remarquer que les sommets de $ZQ_i$
sont exactement les vecteurs de dimension $|\E^*|$ suivants:
\[
z_{i,e}^{uv}=\left\{
\begin{array}{ll}
1 & \text{ si }u \le e \le v\\
0 & \text{ sinon}
\end{array}
\right. \qquad \forall u,\ v \in \E^*,\ u \le v
\]
Donc, un point $\overline{z}_i \in ZQ_i$ si et seulement si le système
linéaire suivant admet une solution réalisable:
\begin{align}
& \sum_{u\le v} z_{i,e}^{uv} \lambda_{uv} = \overline{z}_{i,e} & &
\forall e \in \E^* \label{kis1}\\
&\sum_{u \le v} \lambda_{uv} =1 & & \label{kis2}\\
&\lambda \ge 0 & & \label{kis3}
\end{align}

Or, d'après le lemme de Farkas, le
système~\eqref{kis1}--\eqref{kis3} admet une solution réalisable si et
seulement si $\forall \mu \in \mathbb{R}^{|\E^*|}$ tel que: 
\begin{equation}
\sum_{e=u}^{v} \mu_e + \mu_0 \le 0, \quad u \le v 
\label{kis4} 
\end{equation}
$\mu$ satisfait aussi la condition suivante: 
\begin{equation}
\sum_{e \in \E^*} \mu_e \overline{z}_{i,e}+ \mu_0 \le 0
\label{kis5}
\end{equation}
Donc, puisque les rayons extrêmes du cône caractérisé par~\eqref{kis4}
définissent toutes les inégalités linéaires requises pour décrire
$ZQ_i$, il suffit, pour prouver le théorème, de trouver tous ces
rayons extrêmes. 

Nous allons ensuite montrer qu'il existe une bijection entre les
rayons extrêmes du cône~\eqref{kis4} vérifiant~\eqref{kis5} et les
inégalités de $ZP_i$.

Pour trouver les rayons extrêmes du cône~\eqref{kis4}, il suffit de
considérer les trois cas suivant: 
\paragraph{\boldmath $\mu_0 > 0$.} Quitte à normaliser, nous pouvons suppposer que
  $\mu_0=1$. Dans ce cas là, pour tout $e \in \E^*$, nous avons
  $\mu_e \le -1$. En effet, ceci est déduit de~\eqref{kis4} en
  considérant l'équation pour $u=v=e$. Alors, en prenant
  $\mu_e=-1,\forall e \in \E^*$, ~\eqref{kis5} implique:
\[ \sum_{e \in \E^*} - \overline{z}_{i,e} \le -1 \]
D'après le lemme de Farkas, ceci est une inégalité valide pour
$ZQ_i$. On peut remarquer que ces inégalités sont équivalentes
à~\eqref{start_CECSP_OO}. 
\paragraph{\boldmath $\mu_0 = 0$.} Alors, nous avons toujours un cône dont les rayons
  extrêmes sont les vecteurs unité négatifs de $\mathbb{R}^{\E^*}$. Ces
  rayons extrêmes nous donnent les inégalités $z_{i,e} \ge 0$ qui sont
  les contraintes de non-negativité valides pour $ZQ_i$. 
\paragraph{\boldmath $\mu_0 < 0$.} Quitte à normaliser, nous pouvons suppposer que
  $\mu_0=-1$. Nous allons prouver que dans ce cas, il existe une
  bijection entre les points extrêmes du polyèdre $H$ défini par:
  \begin{equation} \sum_{e=u}^{v} \mu_e \le 1 \quad u \le
    v \label{kis6}
  \end{equation} et les
  inégalités~\eqref{non_preem_ineg}. Comme~\eqref{kis6}
  implique~\eqref{kis5}, ceci montrera bien la bijection
  entre les rayons extrêmes de~\eqref{kis4} vérifiant~\eqref{kis5} et
  les inégalités~\eqref{non_preem_ineg}.

  Dans un premier temps, nous montrons que le vecteur formé des
  coefficients de la partie gauche de chaque inégalités
  de~\eqref{non_preem_ineg} est une solution de~\eqref{kis6}
  correspondant à un point extrême de $H$. 

  Soit ${\cal F}=\{e_0,e_1,\dots,e_{2v}\}$ un ensemble d'événements
  vérifiant $e_i<e_{i+1}$ pour $i=0,\dots,2v-1$. Le vecteur
  $\overline{\mu}$ formé des coefficients de la partie gauche de chaque
  inégalité de~\eqref{non_preem_ineg} est défini par:
  \[ \overline{\mu}_e=\left\{ 
      \begin{array}{ll}
        (-1)^u & \text{ si } e=e_u \in \cal F \\
        0 & \text{ si } e \in \E^* \setminus \cal F
      \end{array}
    \right.
  \]
  Pour prouver que $\overline{\mu}$ est une solution de~\eqref{kis6}
  correspondant à un point extrême de $H$, nous exhibons un
  sous-système $L$ de~\eqref{kis6} 
  contenant $|\E^*|$ inégalités linéairement indépendantes telles que
  chaque inégalité de $L$ soit vérifiée à l'égalité par
  $\overline{\mu}$. 

  Le sous-système $L$ est formé des inégalités suivantes: 
  \begin{align*}
    & \sum_{e=u}^{e_0} \mu_e \le 1 & & u=1,\dots, e_0-1\\
    & \sum_{e=e_{2v}}^u \mu_e \le 1& & u=e_{2v},\dots,|\E^*|
  \end{align*}
  De plus, $L$ contient l'ensemble d'inégalités suivant, défini pour
  tout ensemble formé de 3 événements consécutifs
  $e_{2u},e_{2u+1},e_{2u+2} \in \cal F$:   
  \begin{align*}
    & \sum_{e=e_{2u}}^{e_{2u+2}} \mu_e \le 1 & & \\
    & \sum_{e=e_{2u}}^t \mu_e \le 1& & t=e_{2u},\dots,e_{2u+2}-1\\
    & \sum_{e=t}^{e_{2u+2}} \mu_e \le 1& & t=e_{2u+1}+1,\dots,e_{2u+2}-1
  \end{align*} 
  On peut facilement vérifier que le système ci-dessus est formé de
  $|\E^*|$ inégalités linéairement indépendantes et que
  $\overline{\mu}$ vérifie chacune d'entre elle à l'égalité et ceci
  prouve notre affirmation. 

  Nous montrons maintenant que toute solution $\overline{\mu}$
  correspondant à un point extrême de $H$ équivaut à une inégalité
  de~\eqref{non_preem_ineg}. Dans un premier temps, remarquons que la
  matrice formée par les coefficients de la partie gauche
  de~\eqref{kis6} est totalement unimodulaire. En effet, les colonnes
  de cette matrice peuvent être réordonner de telle sorte que chaque
  ligne ne contienne que des $1$ consécutifs. Donc, tout sommet de
  ce polyèdre est un vecteur à valeurs entières. Nous remarquons aussi
  que $\overline{\mu}_e\le 1,\ \forall e \in \E$, puisque $\mu_e \le 1$
  est une inégalité de~\eqref{kis6} ($u=v$) pour tout $e \in \E^*$.
  
  Soit $u_1$ le premier indice tel que $\overline{\mu}_{u_1} \neq 0$. Nous
  affirmons que $\overline{\mu}_{u_1}=1$. Supposons que ce ne soit pas
  le cas, i.e. $\overline{\mu}_{u_1} \le -1$ (les coordonnées de
  $\overline{\mu}$ sont entières). Puisque $\overline{\mu}$ est un point
  extrême de $H$, il existe un sous-ensemble $L$ formé de $|\E^*|$
  inégalités linéairement indépendantes de~\eqref{kis6} qui sont
  satisfaites à l'égalité par $\overline{\mu}$. 

  Remarquons que $L$ doit contenir une inégalité impliquant la
  variable $\mu_{u_1}$. En effet, si ce n'est pas le cas, cette variable
  peut être arbitrairement fixée à une valeur négative et toujours
  satisfaire toutes les inégalités de $L$. Et donc $\overline{\mu}$ ne
  serait pas un point extrême de $H$ ce qui est une contradiction. Comme
  $\overline{\mu}_e=0$ pour $e< u_1$, une telle inégalité doit être de
  la forme $ \sum_{e=u_1}^{v_1} \mu _e \le 1$. Comme elle doit être
  satisfaite par $\overline{\mu}$ à l'égalité et que $\overline{\mu}_e
  \le 1$, nous avons que $\overline{\mu}$ doit contenir au moins deux
  coordonnées $q_1$ et $q_2$ tels que $ u_1 \le q_1 \le q_2 \le v_1$
  avec $\overline{\mu}_{q_1}=\overline{\mu}_{q_2}=1$ et
  $\overline{\mu}_{e}=0$ pour $q_1 < e < q_2$. Mais, dans ce cas là,
  $\overline{\mu}$ violerait l'inégalité $\sum_{e=q_1}^{q_2} \mu_e \le
  1$ et ceci est une contradiction. Donc, la première coordonnée non
  nulle de $\overline{\mu}$ doit 
  prendre la valeur $1$. 
  
  La seconde coordonnée non nulle, disons
  $u_2$, ne peut prendre la valeur $1$ par le même raisonnement que
  précédemment. Donc, cette valeur doit être négative et
  entière. Mais, dans ce cas là, nous pouvons suivre la même
  argumentation que précédemment pour montrer que $u_2$ doit
  apparaître dans une des inégalités de $\ell \in L$ et aboutir à la même
  contradiction que précédemment. Donc, la seconde coordonnée non nulle
  de $\overline{\mu}$ doit être égale à $-1$. De plus, $\ell$ doit
  contenir une inégalité impliquant une variable $\mu_{u_3}$ de valeur
  $1$ dans $\overline{\mu}$. En effet, dans le cas contraire,
  $\overline{\mu}$ ne peut satisfaire $\ell$ à l'égalité. Si nous
  poursuivons cette argumentation tant que $\mu$ possède des
  coefficients non nuls après $u_3$, nous reconnaissons la suite de
coefficients $1/-1$ présente dans~\eqref{non_preem_ineg}.
\end{proof}

Nous avons donc définit un ensemble d'inégalités permettant une
description complète du polyèdre formé par les vecteurs $z_i$ solution
du modèle on/off. Cependant, nous ne pouvons ajouter directement ces
inégalités au modèle car leur nombre est exponentiel. Dans le
chapitre~\ref{sec:expe}, nous présenterons un algorithme de 
séparation polynomial qui nous permettra de définir un algorithme de
branch-and-cut pour le \CECSP~et le \RCPSP~(rappelons que les
résultats si dessus sont valides dans le cas du \RCPSP~en posant
$\E^*=\{1,\dots,n\}$). 

Le paragraphe suivant présente d'autres inégalités valides pour le
\CECSP~et le \RCPSP. 

\subsubsection{Autres inégalités valides}

Dans ce paragraphe nous décrivons plusieurs ensembles d'inégalités
valides pour le \RCPSP~ et le \CECSP. Dans la suite, nous considérons
qu'un événement correspond à une et une seule date de début/fin,
i.e. $\forall e \in \Em[1]$, il existe une et une seule activité $i$
vérifiant $(z_{i,e-1}-z_{ie}=1) \vee (z_{ie}-z_{i,e-1}=1)$. Cela est
toujours possible puisque $|\E|=2n$. De plus, l'ajout de cette
supposition ne change pas la véracité de ce qui précède. 

\paragraph{Séparation maximale entre deux événements} 

Les inégalités définies dans cette section sont des inégalités bornant
supérieurement la valeur de $t_{e+1}- t_e, \forall e\in \Em$. Pour
définir ces inégalités, nous étudions les fenêtres de temps de chaque
date de début et de fin d'une activité, i.e. $[\ES,\LS]$ et
$[\EE,\LE],\ \forall i \in \A$. L'idée principale repose sur le fait
que, dans chacun de ces intervalles, un événement doit forcément avoir
lieu. De ce fait, nous savons qu'il y a au moins deux événements
consécutifs dans l'union de deux fenêtres de temps consécutives. 

Soit ${\cal D}$ l'ensemble de toutes les fenêtre de temps, i.e. ${\cal
D}=\{[\ES,\LS] , [\EE,\LE],\ \forall i \in \A\}$. Nous commençons par
trier les intervalles de $\cal D$ suivant la règle suivante: 
$[a,b] \le [c,d]
\Leftrightarrow a<c \lor \left( a=c \land b\le d\right)$

Alors, soit $\underline{{\cal D}_e}$ (resp. $\overline{{\cal D}_e}$)
la borne inférieure (resp. supérieure) de l'intervalle ${\cal D}_e$,
nous avons la propriété suivante:
\begin{equation} \label{sep_CECSP_OO} 
t_{e+1}-t_e \le \max(\overline{{\cal D}_e},\overline{{\cal D}_{e+1}}) -
\min(\underline{{\cal D}_e},\underline{{\cal D}_{e+1}}) \qquad \forall e \in {\cal E}\setminus\{2n\}
\end{equation}

\begin{ex}
\label{ex:evt_sep} Considérons l'ensemble d'intervalles suivant:
\begin{center}
\begin{tikzpicture} [yscale=0.8,xscale=0.6] \node (O) at (0,0) {};

\draw[->] (0,0) -- (23,0);
 \draw (1,0) node[red] {$[$} node[above=0.2cm,red]
{$\ES[1]$}-- +(0:3cm)node[red] {$]$} node[right=0.1cm,above=0.2cm,red]{$\LS[1]$};

\draw(7.05,0) node[red] {$[$} node[above=0.2cm,red] {$\EE[1]$}-- +(0:3cm) node[red]
{$]$} node[above=0.2cm,red] {$\LE[1]$};
 \draw (3,0) node[Green] {$[$} node[below=0.3cm,Green]
{$\ES[2]$}-- +(0:4cm) node[Green] {$]$} node[below=0.2cm,Green] {$\LS[2]$};
 \draw
(16,0) node[Green] {$[$} node[above=0.2cm,Green] {$\EE[2]$}-- +(0:3cm) node[Green] {$]$}
node[right=0.1cm,above=0.2cm,Green] {$\LE[2]$};
 \draw(10.1,0) node[blue] {$[$}
node[below=0.3cm,blue] {$\ES[3]$}-- +(0:3cm) node[blue] {$]$} node[below=0.2cm,blue]
{$\LS[3]$};
 \draw (15,0) node[blue] {$[$} node[below=0.2cm,blue] {$\EE[3]$}--
+(0:7cm) node[blue] {$]$} node[below=0.25cm,blue] {$\LE[3]$};

\end{tikzpicture}
\end{center}

Après avoir trié les intervalles, nous obtenons:
$[\ES[1],\LS[1]] \le [\ES[2],\LS[2]] \le
[\EE[1],\LE[1]] \le [\ES[3],\LS[3]] \le 
[\EE[3],\LE[3]] \le [\EE[2],\LE[2]] $.

Alors, nous avons l'ensemble de contraintes suivantes:

\begin{itemize}
\item $t_2-t_1 \le \LS[2]-\ES[1]$
\item $t_3-t_2 \le \LE[1]-\ES[2]$
\item $t_4-t_3 \le \LS[3]-\EE[1]$
\item $t_5-t_4 \le \LE[3]-\ES[3]$
\item $t_6-t_5 \le \LE[3]-\EE[3]$
\end{itemize}

Notons aussi que l'ensemble d'inégalités
$\underline{{\cal D}_e} \le t_e \le \overline{{\cal D}_e}$ peut ne pas
être valides. En effet, ici $t_6$ peut correspondre à la fin de
l'activité $3$ et $t_5$ à la fin de l'activité $2$, alors que
${\cal D}_5=[\EE[3],\LE[3]] \le [\EE[2],\LE[2]]={\cal D}_6$. On aurait
alors $\underline{{\cal D}_6} \le t_5 \le \overline{{\cal D}_6}$.
\end{ex}


Ces contraintes peuvent être ajoutées au modèle on/off ou utilisées
comme borne supérieure sur la valeur de $\bmin(t_{e+}-t_e)$
dans~\eqref{bmin_CECSP_OO}. L'inégalité se réécrit donc comme:
\[ b_{ie} \ge \bmin(t_{e+}-t_e) - \bmin\left(\max(\overline{{\cal
D}_e},\overline{{\cal D}_{e+1}}) - \min(\underline{{\cal
D}_e},\underline{{\cal D}_{e+1}})\right)(1-z_{ie})\qquad \forall (i,e)
\in {\cal A}\times{\cal E}
\]

Ces inégalités peuvent être généralisées à tout sous-ensemble de $k$
intervalles ordonnés $\{{\cal D}_{e_1},\dots,{\cal D}_{e_k}\}$ avec
$t_{e_k}-t_{e_1} \le \max(\overline{{\cal D}_{e_1}},\overline{{\cal
D}_{e_k}}) - \min(\underline{{\cal D}_{e_k}},\underline{{\cal
D}_{e_1}}) $.

\paragraph{Date maximale d'un événement}


Une idée similaire à celle décrite dans le paragraphe précédent peut
être utilisée pour ordonner les événements et calculer des bornes
supérieures sur leur date. 

Pour faire cela, nous commençons par trier les bornes supérieures des
fenêtres de temps de chaque activité, i.e. $\LS$ et $\ES,\ \forall i
\in \A$, par ordre croissant. Alors, puisqu'un événement doit avoir
lieu dans chaque fenêtre de temps, i.e. avant chaque borne supérieure de
chaque fenêtre, nous pouvons déduire une borne supérieure sur la date
de chaque événement.

En effet, soit ${\cal UP}$ l'ensemble formé de toutes les bornes
supérieures de toutes les fenêtres de temps. Alors, nous avons la
propriété suivante: 
\begin{equation} \label {Bte_CECSP_OO} t_e \le {\cal UP}_e \qquad
\forall e \in {\cal E}
\end{equation}

\begin{ex} 
Considérons les intervalles définie dans
l'exemple~\ref{ex:evt_sep}. Alors, nous pouvons déduire l'ensemble de
contraintes suivantes:

\begin{itemize}
\item $t_1 \le \LS[1]$
\item $t_2 \le \LS[2]$
\item $t_3 \le \LE[1]$
\item $t_4 \le \LS[3]$
\item $t_5 \le \LE[2]$
\item $t_6 \le \LE[3]$
\end{itemize}
\end{ex}

Comme précédemment, nous pouvons utiliser ces inégalités comme
contraintes additionnelles des modèles à événements ou les utiliser
à la place de  $T$ dans les contraintes \eqref{twx_CECSP_OO}
et\eqref{twy2_CECSP_OO}. Les contraintes s'écrivent alors: 
\begin{align*}
& \ES z_{ie}\le t_e \le \LS(z_{ie}-z_{i,e-1})+(1-(z_{ie}-z_{i,e-1})){\cal UP}_e 
 & & \forall e \in \E\setminus \{1\},\ \forall i \in {\cal
   A}\\
&t_e \le \LE(z_{i,e-1}-z_{ie})+(1-(z_{i,e-1}-z_{ie})){\cal UP}_e  & & \forall e
 \in \E\setminus \{1\},\ \forall i \in {\cal
   A}
\end{align*}

Pour le \RCPSP, $t_n$ correspond comme borne supérieure sur la durée
totale du projet, i.e. sur $T$.

\paragraph{Inégalités valides dérivées du problème de sac à dos}

Le rendement minimal de chaque activité pouvant être positif, nous
pouvons considérer les contraintes de type sac-à-dos suivantes pour
tout $e \in \Em$ et les transformer facilement en inégalités valides: 
\begin{equation}
\sum_{i\in \A^+} \bmin z_{ie} \leq B  \qquad  \forall e \in \E
\end{equation}
où $\A^+$ est le sous-ensemble d'activités avec $\bmin > 0$. 


\paragraph{Inégalités de cliques}

Les inégalités de clique permettent de modéliser le fait que plusieurs
variables binaires $z_{ie}$ ne peuvent prendre la valeur $1$
simultanément. Ces inégalités, déjà été établies dans le cas du
\RCPSP~\cite{CAVT_clique}, correspondent aux sous-ensembles
disjonctifs d'activités. Elles sont facilement adaptable au cas du
\CECSP et sont définies de la manière suivante. Soit $C$ un ensemble
minimal d'activité ne pouvant s'exécuter en parallèle, i.e. telles que
$\sum_{i \in C} \bmin > B$, alors l'ensemble d'inégalités suivantes
est valide pour le \CECSP: 
\begin{equation} 
\sum_{i\in C} z_{ie}  \le |C| -1 \qquad \forall C,\ \forall e \in \E
\end{equation}


Différentes techniques permettant l'intégration des inégalités
ci-dessus seront présentées et comparées dans le Chapitre~\ref{sec:expe}.

\section*{Conclusion}

Dans ce chapitre, nous avons présenté des modèles de programmation
linéaire en nombres entiers pour le \CECSP. Pour ce problème, trois
modèles sont présentés,  
un modèle utilisant une discrétisation de l'horizon de temps et deux
modèles basés sur une représentation des événements pertinents du
problème. 

Enfin, des améliorations de ces modèles sont proposées dans la
dernière partie du chapitre. Ces améliorations sont basées sur le
raisonnement énergétique, la mise en place d'inégalités valides et des
études polyédrales. De plus, les avantages et inconvénients de chacun
des modèles sont décrits ce qui permet de justifier l'intérêt de ces
améliorations. 

Des résultats numériques évaluant les performances de ces formulations
ainsi que l'intérêt de chaque amélioration sur diverses instances du
\CECSP~et du \RCPSP~feront l'objet d'un paragraphe dans le chapitre
portant sur les expérimentations (cf. Chapitre~\ref{sec:expe}). De
plus, un algorithme permettant de séparer les inégalités de
non-préemption en temps polynomial sera également détaillé dans ce
chapitre. 
%
\clearemptydoublepage%
\part{Implémentations et Expérimentations}

\chapter{Implémentations}
\label{sec:expe}
%
\clearemptydoublepage%
\chapter*{Conclusions et Perspectives\markboth{CONCLUSIONS ET PERSPECTIVES}{}}

Beaucoup de problèmes d'ordonnancement cumulatifs définis dans la
littérature souffrent d'une limitation imposant que chaque activité
aille une durée et une consommation de ressource fixe durant son 
exécution. Il arrive cependant, dans de nombreux cas pratique, que
cette limitation empêche la modélisation correcte du problème. De ce
fait, de nouveaux problèmes permettant de modéliser la malléabilité
des activités, i.e. ayant une durée et une consommation de ressource
variable, ont été introduits~\cite{DDH,NK,FT,Kis,BLN}. Cette thèse
introduit un nouveau problème appartenant à cette classe, le \CECSP. 
Une comparaison des différents problèmes existant permettant de
modéliser des activités malléables a été réalisée et a permis de
montrer que le \CECSP~est très différents de ces derniers. 

La principale difficulté du \CECSP~repose sur la combinaison entre
ressource continue et malléabilité des activités. En effet, ces deux
caractéristiques impliquent que les activités peuvent prendre des
formes quasi quelconques. De ce fait, un des premiers travail effectué
dans cette thèse a été une étude détaillée du problème. Cette étude a
permis de décrire des cas particuliers du \CECSP~pouvant être résolus
en temps polynomial mais aussi a permis, dans certains cas, la mise en
place d'une propriété permettant de décomplexifier le problème en
caractérisant les différentes formes que peut prendre une activité. 
Cette simplification du problème a ensuite été utilisée pour mettre en
place des méthodes de résolution pour le \CECSP, souvent adaptées de
méthodes existantes dans le cadre des problèmes d'ordonnancement
cumulatifs. 

Les méthodes de résolution dédiées au \CECSP~et décrites dans ce
manuscrit sont regroupées en deux catégories: les méthodes issues de
la programmation par contraintes et adaptées des méthodes définies
pour la contrainte cumulative et les méthodes issues de la
programmation linéaire mixte et en nombres entiers adaptées des
méthodes définies pour le \RCPSP. 

Plusieurs des techniques de programmation par contraintes utilisées dans le
cadre de la résolution de la contrainte cumulative ont décrites et, en
particulier, les principaux algorithmes de filtrage qui lui sont
dédiés. Une partie de ces algorithmes a été adaptés au cas du
\CECSP. C'est le cas, par exemple, du raisonnement énergétique qui
prend une part importante de ce manuscrit. Ce raisonnement comptant
parmi les plus forts dans le cas de la contrainte cumulative a été le
premier à avoir été adapté. De plus, les récents travaux de Derrien
{\it et al.} permettant l'accélération de ce raisonnement ont aussi pu
être adaptés. D'autres raisonnements comme le Time-Table, le
raisonnement disjonctif et le raisonnement Time-Table disjonctif ont
aussi pu être transformés afin de s'appliquer dans le cas du
\CECSP. Enfin, un nouvel algorithme de détection d'incohérence
utilisant un programme linéaire basé sur un problème de flot couplé
avec le raisonnement Time-Table a été présenté. Les expérimentations
conduites sur ces raisonnements -- inclus à l'intérieur d'une méthode
de branchement hybride -- ont permis de montrer que ce nouvel
algorithme de vérification permet la détection d'un plus grand nombre
d'incohérences que le raisonnement énergétique à lui seul.

Différents modèles de programmation linéaire en nombres entiers pour le
\RCPSP~ont été décrits dans le chapitre~\ref{sec:PLNE_RCPSP}. Parmi
ces modèles, le premier repose sur une formulation indexée par le temps
et les deux autres sur des formulations basées sur les événements. Ces
dernières ont montré leur efficacité dans le cas où l'horizon de temps
des modèles devient très grand. De plus, contrairement aux modèles
indexés par le temps, ils permettent de modéliser des dates de début
et de fin d'activités continue. De ce fait, nous avons adapté ces
modèles au cas du \CECSP. Un modèle indexé par le temps est aussi
détaillé. En effet, ces modèles ont prouvé leur efficacité dans le
cadre du \RCPSP.

Pour chacun des modèles présentés, des inégalités valides et/ou des
techniques de coupes ont été présenté afin de renforcer ces
derniers. Des inégalités directement déduites du raisonnement
énergétique sont introduites pour les modèles indexés par le temps du
\CECSP~et du \RCPSP. Pour les modèles à événements, une comparaison
des relaxations linéaires des deux modèles présentés est
effectuée. Des inégalités permettant de renforcer le modèle ayant les
moins bonnes relaxations, le modèle On/Off, sont présentées. Ces
inégalités, appelées inégalités de non-préemption, ont été montré
comme appartenant à l'enveloppe convexe du polyèdre formé par
l’ensemble de toutes les affectations possibles pour les variables
binaires correspondant à une seule activité. Enfin, plusieurs autres
ensembles d'inégalités pour les modèles à événements ont été présentés
et les performances relatives à l'ajout de différentes combinaisons de
ces inégalités au modèle ont été détaillées. 

Malgré tout, le \CECSP~reste un problème difficile, en particulier
dans sa forme générale. En effet, es résultats présentés dans cette
thèse nous ont permis de résoudre des instances allant jusqu'à $60$
activités pour la version décisionnelle de ce problème et seulement
jusqu'à $30$ pour la version ayant pour objectif la minimisation de la
consommation de la ressource. 

Le \CECSP~est un nouveau problème pour lequel de nombreux travaux
restent à faire. Parmi les perspectives directes des résultats
présentés dans ce manuscrit, on trouve:
\paragraph{La considération de fonction de rendement plus générale.} Même si les fonctions concaves permettent une plus grande
  liberté d'expression du problème, elles restent insuffisante pour
  modéliser certains problèmes réels. Dans un premier temps, la
  considération de fonctions de rendement convexes est une
  continuation naturelle de ce travail. Cependant, ce n'est toujours
  pas suffisant dans certains cas. En effet, beaucoup de fonctions 
  de rendement ne sont ni totalement concave, ni totalement convexe
  mais sont convexes sur certains intervalles et concaves sur les
  autres intervalles. 

\paragraph{La mise en place d'algorithme de filtrage plus
  performants.} Qu'il s'agisse d'algorithme plus performant en termes
de temps de calcul ou plus performant en termes de filtrage, cette
direction de recherche est une perspective importante. L'adaptation
d'autres algorithmes de filtrage mis en place pour la contrainte
cumulative semble une piste intéressante mais l'accélération du temps
de calcul de ces derniers est fait au prix d'un filtrage de moins
bonne qualité. Ceci peut s'avérer critique dans le cas du \CECSP~où
les algorithmes de filtrage pour la contrainte cumulative sont
beaucoup plus faibles. A l'inverse, la mise en place d'algorithmes
dédiés au problème ayant un pouvoir de filtrage plus grand sera
probablement effectuée au prix d'une augmentation importante du temps 
de calcul. 

\paragraph{Amélioration des modèles de programmation linéaire mixte.}
Dans un premier temps, les modèles décrits dans ce manuscrit
pourraient encore être renforcés. En effet, les inégalités
énergétiques décrites pour le modèle indexé par le temps sont un moyen
intéressant de coupler des techniques issues de la programmation par
contraintes à la programmation linéaire mixte. D'autres inégalités de
ce type pourraient être définies ou de meilleures techniques
d'intégration de ces inégalités dans le processus de résolution
pourraient être mises en place. 

L'amélioration des modèles à événement est aussi une piste de
recherche importante puisque, contrairement aux modèles indexés par le
temps, ces derniers permettent d'obtenir des solutions exactes pour le
\CECSP~mais au prix d'un temps de calcul beaucoup plus important. Pour
renforcer ces modèles, d'autres jeux d'inégalités, plus fortes,
peuvent être mises en place. Par exemple, des inégalités décrivant des
facettes du polyèdre formé par l'ensemble des solutions de chaque
modèle peuvent être exhibées.

\paragraph{L'utilisation d'autres méthodes de résolution.}
Les techniques décrites dans ce manuscrit utilisent principalement des
techniques de programmation linéaire mixte et de programmation par
contraintes. Cependant, d'autres paradigmes existent et des techniques
issues de ces derniers pourraient être considérées. De même, des
techniques appliquées sur des problèmes connexes pourraient aussi être
adaptées, quitte à discrétiser le problème. Parmi ces techniques, on
pourra trouver les techniques de génération de colonnes, de
branch-and-cut, la programmation dynamique ou des techniques utilisant
des règles de priorités, etc. 









%
\clearemptydoublepage%
\bibliographystyle{alpha}
\bibliography{main_file,JFPC,IESM}

\begin{bibunit}[alpha]
\nocite{main_file}
\nocite*
\putbib[main_file]
\end{bibunit}
%
\clearemptydoublepage%
\appendix
\part*{Annexes}

\chapter{Programmation linéaire mixte et programmation par contraintes
  pour un problème d'ordonnancement à contraintes énergétiques}  
\chaptermark{PLNE et PPC
  pour un problème à contraintes énergétiques}
\label{ann:JFPC}
Nous considérerons un problème d'ordonnancement cumulatif dans
lequel les tâches ont une durée et un profil de consommation de
ressource variable. Ce profil, qui peut varier en fonction du temps, est
une variable de décision du problème dont dépend la durée de la tâche
associée. 
Pour ce problème NP-difficle, nous présentons un modèle de programmation
par contraintes et un modèle de programmation linéaire en nombres
entiers (PLNE). De plus, des inégalités valides déduites de la
programmation par contraintes  viennent renforcer le PLNE. 
Ces modèles sont ensuite comparés par le biais d'expérimentations.


\section{Introduction}

Nous étudions un problème d’ordonnancement avec ressource continue et
contraintes énergétiques, le Continuous Energy-Constrained Scheduling
Problem (CECSP). Dans ce problème, un ensemble de tâches ${\cal
A}=\{1,\dots,n\}$ utilisant une ressource continue et cumulative de
capacité limitée $B$ doit être ordonnancé.  La quantité de ressource
nécessaire à l'exécution d'une tâche n'est pas fixée mais - le profil
de consommation de cette dernière est une fonction $b_i(t)$ définie
pour tout  $t \in \mathbb{R}$\footnote{Le domaine de définition de la
fonction peut être réduit mais, pour faciliter les notations, nous
supposons qu'elle est définie pour tout $t \in \mathbb{R}$.} - doit
être déterminé. Une fois la tâche commencée et jusqu'à sa date de fin,
la fonction $b_i(t)$ doit être comprise entre une valeur maximale,
$\bmax$, et minimale, $\bmin$.   

De plus, la consommation, à un instant $t$, d'une partie de
la ressource permet la production d'une certaine quantité d'énergie
et, une tâche finit lorqu'elle a reçu une énergie $W_i$. Cette
énergie est calculée par le biais d'une fonction de rendement $f_i$, 
propre à chaque tâche. Dans cet article, ces fonctions sont supposées
continues, croissantes, affines et peuvent être exprimées de la
manière suivante: 
 
\noindent
  $f_i(b)=\left\{
    \begin{array}{ll} 0 & \quad \text{si }b=0\\ a_i*b+c_i &\quad
                                                            \text{si }\bmin=0\text{ et }b \in ]\bmin,\bmax] \\ a_i*b+c_i &\quad
                                                                                                                           \text{si }\bmin\neq 0 \text{ et }b \in [\bmin,\bmax]
    \end{array} \right.$ 
  
  \noindent
  avec $a_i>0$ et $c_i \geq -a_i*\bmin $ pour
  s'assurer que $f_i(b) \geq 0,\ \forall b \in \inter[\bmin][\bmax]$.

Dans la suite, nous dénotons par $\ES$ et $\LS$
la date de début au plus tôt et au plus tard de $i$ et par
$\EE$ et $\LE$ la date de fin au plus tôt et au
plus tard de $i$.

Pour trouver une solution pour le CECSP, nous devons déterminer, pour
chaque tâche $i \in {\cal A}$, sa date de début $st_i$, sa date de fin
$et_i$ et sa fonction d'allocation de ressource $b_i(t)$, $\forall t
\in {\cal T}=[\min_{i\in{\cal A}} \ES, \max_{i\in {\cal
    A}} \LE]$. De plus, ces variables doivent satisfaire
les contraintes suivantes:
\begin{eqnarray} 
  \ES\le st_i < et_i \le \LE & & \forall i \in
{\cal A} \label {tw_CECSP}\\
  \bmin \le b_i(t) \le \bmax & & \forall i \in {\cal A},\
\forall t \in [st_i,et_i] \label {bminmax_CECSP}\\
  b_i(t)=0 & & \forall i \in {\cal A},\ \forall t \not\in
[st_i,et_i] \label {b0_CECSP}\\
  \int_{st_i}^{et_i} f_i(b_i(t))dt =W_i & & \forall i \in {\cal
A} \label{nrj_CECSP}\\
  \sum_{i \in {\cal A}} b_i(t) \le B & & \forall t \in {\cal
T} \label{res_CECSP}
\end{eqnarray}
L'objectif auquel nous nous sommes intéressés est la minimisation de
la consommation totale de la ressource. Dans~\cite{Nattaf2015}, les
auteurs montrent que trouver une solution admissible pour le CECSP est
déjà un problème NP-complet. 

De plus, une instance ayant des données seulement entières peut
n'avoir que des solutions à valeurs dans
$\mathbb{R}$~\cite{Nattaf2015}. Cependant, une dilatation de
l'instance, i.e. multiplier les données par un certain coefficient
$\alpha$, permet de palier à ce problème. 
De ce fait et dans le but de résoudre des instances entières, nous
nous sommes intéressés, dans un premier temps, à la version discrète
du CECSP, le DECSP (Discrete Energy Constrained Scheduling
Problem). Dans ce problème, toutes les données sont supposées entières
et les domaines de chaque variable ne contiennent
que des valeurs entières, i.e. $st_i,\ et_i,\ b_i(t) \in \mathbb{N}$ et
$b_i(t)$ est défini $\forall t \in {\cal T}_{\cal D}=\{\min_{i\in{\cal
A}} \ES,\dots, \max_{i\in {\cal A}} \LE\}$. 

Pour ce problème, nous présentons un modèle de
programmation par contraintes (PPC) permettant l'utilisation des
algorithmes de propagation mis en place pour la contrainte
cumulative, notamment~\cite{Gay2015}.  Un modèle de
programmation linéaire en nombres entiers (PLNE) est aussi
présenté. Ce modèle est ensuite renforcé à l'aide d'inégalités
valides déduites du raisonnement énergétique~\cite{Lopez1990}. 
Ces deux modèles sont ensuite testés sur des instances à données
entières, avec et sans dilatation.

\section{Modèle de programmation par contraintes}

Pour modéliser le DECSP à l'aide de la PPC, nous divisons chaque tâche
$i$ en deux sous-tâches $i_{min}$ et $i_{preem}$. La première,
$i_{min}$, est une tâche ayant une consommation de ressource fixe,
égale à $\bmin$, et une durée variable $p_i$. Cette tâche
représente la quantité de ressource obligatoirement consommée par une
activité durant son exécution, i.e. $\bmin$.  La seconde, $i_{preem}$
est une tâche préemptive optionnelle, consommant une quantité variable
de ressource comprise entre $0$ et $\bmax-\bmin$ et devant s'exécuter
en même temps que $i_{min}$. Cette tâche est elle-même divisée en
sous-tâches $i_{preem}^\ell,\ \ell \in \{1,\dots, et_i-st_i\}={\cal L}_i$. Notons
que $|{\cal L}_i|=et_i - st_i \le \lceil \frac{W_i}{f_i(\bmin)}\rceil$. 
\begin{ex}
Considérons la tâche possédant les attributs suivant:  $\ES= 0$,
$\LE=6$, $W_i=28$, $r^{min}=1$,
$r^{max}=5$ et $f_i(b)=2b+1$. La Figure~\ref{fig:ex:PPC} présente un
ordonnancement de cette tâche (à droite) et l'ordonnancement
correspondant donné par le modèle (à gauche) avec $r_{i^2_{preem}}=0$. 
\begin{figure}[!htb]
  \centering
  \begin{tikzpicture}
[xscale=0.8,yscale=0.75]
\node (O) at (0,0) {};
\draw (0,0) rectangle (4,1) node[midway] {$i_{min}$};
\draw (0,1) rectangle (1,5) node[midway,rotate=-90] {$i_{preem}^1$};
\draw (2,1) rectangle (3,3) node[midway,rotate=-90] {$i_{preem}^3$};
\draw (3,1) rectangle (4,3) node[midway,rotate=-90] {$i_{preem}^4$};

\draw[->,>=latex] (0,0) -- (4.5,0);

\foreach \i in {0,...,4}
{
  \draw (\i,-0.3) -- (\i,0);
  \draw (-0.3,\i) -- (0,\i)node[left=0.2cm] {\footnotesize \i};
}

\draw (7,5) -- (7,1) -- (8,1) -- (8,3)  -- node[below=0.5cm,midway] {\Large $i$}(10,3)  --  (10,1)
-- (10,0)--  (6,0) -- (6,5) -- cycle;
\draw[->,>=latex] (6,0) -- (10.5,0);

\foreach \i in {0,...,4}
{
  \draw (\i+6,-0.3) -- (\i+6,0);
  \draw (5.7,\i) -- (6,\i) node[left=0.2cm] {\footnotesize \i}; 
}    
  \end{tikzpicture}
  \caption{Exemple de solution du modèle PPC.}
  \label{fig:ex:PPC}
\end{figure}
\end{ex}
Le problème du DECSP peut alors être formulé à l'aide des variables:
\begin{itemize}
\item $i_{min}=\{s_{i_{min}}, e_{i_{min}}, b_{i_{min}},p_{i_{min}}\},\ 
  \forall i \in {\cal A}$ 
\item $i_{preem}^\ell=\{s_{i_{preem}^\ell},e_{i_{preem}^\ell},
  b_{i_{preem}^\ell},p_{i_{preem}^\ell}\}$, $\forall i \in {\cal A},\
  \ell \in {\cal L}_i$.
\end{itemize}
et des contraintes:
\begin{enumerate}  
\item $\forall (i,l) \in {\cal A} \times {\cal L}_i:\
  e_{i_{preem}^\ell} = s_{i_{preem}^{\ell+1}}$ 
\vspace{0.2cm}
\item $\forall i \in {\cal A}:\  s_{i_{min}} = s_{i_{preem}^1}$ et $ e_{i_{preem}^{|{\cal L}_i|}} = e_{i_{min}}$   
\vspace{0.2cm}
\item $\forall t \in  {\cal T}:\\ \sum\limits_{\substack{i \in {\cal A} \\ t \in [s_{i_{min}},e_{i_{min}}[}} b_{i_{min}}
  + \sum\limits_{\substack{(i,l) \in {\cal A}\times {\cal L}_i\\ t \in
      [s_{i_{preem}^\ell},e_{i_{preem}^\ell}[}}
  b_{i_{preem}^\ell} \le B$ 
\vspace{0.2cm}
\item $\forall i \in {\cal A}:\ \sum_{l \in {\cal L}_i} \left(f_i(b_{i_{preem}^\ell})
    (e_{i_{preem}^\ell}-s_{i_{preem}^\ell})\right) +
  f_i(b_{i_{min}}) (e_{i_{min}}-s_{i_{min}}) \ge W_i $
\end{enumerate}
La première contrainte permet d'ordonner les sous-tâches de
$i_{preem}$. Ceci dans le but de faciliter la modélisation des autres
contraintes. La seconde contrainte modélise le fait que $i_{preem}$
commence et finit en même temps que $i_{min}$. La troisième contrainte
assure que la capacité de la ressource n'est pas excédée en sommant, à
un instant $t$, les consommations minimales des tâches en cours ainsi
que les consommations des sous-tâches préemptives en cours. Enfin, la
quatrième contrainte permet de s'assurer que chaque tâche reçoit au
moins l'énergie requise $W_i$. 

Un des avantages de cette formulation est qu'elle permet l'utilisation
des algorithmes de propagation mis en places pour la contrainte
cumulative tels que le time-table classique~\cite{Baptiste2001},
disjonctif~\cite{Gay2015} ou associé au edge-finding~\cite{Vilim2011},
le raisonnement disjonctif~\cite{Baptiste2001}, ou encore le
raisonnement énergétique~\cite{Lopez1990}. Cependant, certains de ces
raisonnements peuvent être adaptés pour 
prendre en compte l'ensemble du problème. C'est le cas, par exemple,
du raisonnement énergétique détaillé ci-dessous.

\subsection{Raisonnement énergétique}


Ce paragraphe présente un algorithme de propagation pour le
DECSP basé sur le raisonnement énergétique défini pour le
CECSP~\cite{Nattaf2015}.  L'adaptation de ce raisonnement au cas
discret est quasi-directe. Cependant, nous rappelons les bases de
celui-ci car nous l'utiliserons dans la suite pour déduire des
inégalités valides pour le PLNE.

Le principe du raisonnement énergétique est de comparer la quantité de
ressource disponible dans un intervalle avec la quantité minimale de
ressource consommée par toutes les tâches dans cet intervalle.

Les configurations pour lesquelles la quantité de ressource requise
par une tâche $i$ dans l'intervalle $[t_1,t_2[$ est minimale
correspondent toujours à une configuration où la tâche reçoit le
maximum d'énergie possible, i.e. est ordonnancée à $\bmax$, en dehors
de $[t_1,t_2[$, tout en respectant les contraintes
\eqref{tw_CECSP}--\eqref{res_CECSP}. Ceci correspond donc à une des
configurations suivantes:
\begin{itemize}
\item la tâche est calée à gauche: ordonnancée à $\bmax$ durant
$[ \ES , t_1 [$;
\item la tâche est calée à droite: ordonnancée à $\bmax$ durant
$[t_2,\LE{[}$;
\item la tâche est centrée: ordonnancée à $\bmax$ durant
$[\ES,t_1[ \cup [t_2,\LE{[}$ ou ordonnancée à
$\bmin$ durant $[t_1,t_2[$.
\end{itemize} En effet, dans le dernier cas, il peut arriver
qu'ordonnancer la tâche à $\bmax$ dans $[\ES,t_1[ \cup
[t_2,\LE{[}$ implique que la quantité d'énergie restant à
apporter à la tâche dans $[t_1,t_2[$ ne soit pas suffisante pour
ordonnancer la tâche à $\bmin$ durant $[t_1,t_2[$. Or, ceci
impliquerait une violation de la
contrainte~\eqref{bminmax_CECSP}. Dans ce cas, la tâche est donc
ordonnancée à $\bmin$ durant l'intervalle $[t_1,t_2[$. Alors la
quantité de ressource requise par la tâche $i$ dans $[t_1,t_2[$ est la
quantité minimale requise par ces configurations.  

Les intervalles $[t_1,t_2[$ sur lesquels appliquer ce test pour le
CECSP sont décrits dans~\cite{Nattaf2015}. Pour le DECSP, nous devons
considérer les projections de ces intervalles sur les entiers,
i.e. $[a,b[ \rightarrow [\lfloor a \rfloor, \lceil b \rceil[$.  Les
ajustements pour le CECSP s'adaptent aussi naturellement au DECSP à
l'aide de cette même projection.

\section{Modèle de programmation linéaire en nombres entiers}
\label{MIP}
\subsection{Modèle}
La formulation proposée dans cet article est une formulation indexée par
le temps. Elle est adaptée de la formulation décrite
dans~\cite{Nattaf2015}. 
Dans ces formulations, l'horizon de temps est divisé en
intervalles de taille 1 et est défini par: ${\cal T}_{\cal D}$. 
Pour chaque activité $i \in {\cal A}$ et pour chaque instant $t \in
{\cal T}_{\cal D}$, nous définissons deux variables binaires $x_{it}$
et $y_{it}$ pour modéliser le début et la fin des activités. La
variable $x_{it}$ (resp. $y_{it}$) prendra la valeur $1$ si et
seulement si l'activité $i$ commence (finit) à l'instant $t$.
Pour modéliser la consommation de ressource et l'apport en énergie,
nous introduisons deux variables, $b_{it}$ et $w_{it}$ qui
représentent respectivement la quantité de
ressource consommée par l'activité $i$ dans la période de temps
$t$ et l'énergie reçue par cette même activité durant cette période. 

Par manque de place, ce modèle n'est pas entièrement décrit ici mais
nous décrivons les contraintes permettant de lier les variables
$b_{it}$ et $w_{it}$, i.e. permettant de calculer l'énergie apportée à
$i$ dans la période $t$, $w_{it}$ , en fonction de la consommation de
ressource $b_{it}$. Nous donnons aussi le nombre de variables et de
contraintes du modèle.

Les contraintes liant $b_{it}$ et $w_{it}$, $\quad \forall t\in {\cal T}_{\cal D},\
\forall i \in {\cal A}$ sont les suivantes:
\begin{equation}  w_{it}=a_ib_{it}+c_i\left(\sum_{\tau=\ES}^t
x_{i\tau}-\sum_{\tau=\ES+1}^t y_{i\tau}\right)  \label{conv_CECSP_TI}
\end{equation} Cette contrainte nous permet de modéliser la fonction
de rendement $f_i,\ \forall i \in {\cal A}$. En effet,
$\left(\sum_{\tau=\ES}^t x_{i\tau}\right.$ $\left.-\sum_{\tau=\ES+1}^t
y_{i\tau}\right) $ est égale à $1$ si et seulement si l'activité $i$
est en cours à l'instant $t$.  Dans ce cas là, la valeur de l'énergie
apportée à $i$ est bien $w_{it}=a_ib_{it}+c_i$. Le second cas se
produit quand l'activité $i$ n'est pas en cours à $t$. Dans ce cas,
$b_{it}=0$ implique $w_{it}=0$.

Le modèle possède donc $2n|{\cal T}_{\cal D}|$ variables binaires,
$2n|{\cal T}_{\cal D}|$ variables continues et au plus $3n+|{\cal
T}_{\cal D}|*(6n+1)$ contraintes.

\section{Inégalités valides basées sur le raisonnement énergétique}
Ce paragraphe décrit des inégalités valides déduites du raisonnement
énergétique pour le PLNE. Soit ${\cal R}$ l'ensemble des intervalles
d'intérêt pour le raisonnement énergétique.
\begin{align}
&(x_{i\ES} + y_{i\LE} -1 ) \, \bb + \sum_{j\neq i}
  \bb[j] \le \nonumber\\ 
&  B(t_2-t_1) \quad \forall i \in {\cal A},\ \forall
  [t_1,t_2] \in {\cal R}
\label{both}\\[2mm] 
&(x_{i\ES} + \sum_{t=t_1}^{t_2}y_{it} -1) \, \bb + \sum_{j\neq i}
  \bb[j] \le \nonumber \\
&  B(t_2-t_1) \quad \forall i \in {\cal A},\ \forall
  [t_1,t_2] \in {\cal R}
\label{left}\\[2mm] 
&(\sum_{t=t_1}^{t_2}x_{it} + y_{i\LE}-1) \, \bb + \sum_{j\neq i}
  \bb[j] \le \nonumber\\
&  B(t_2-t_1) \quad \forall i \in {\cal A},\ \forall
  [t_1,t_2] \in {\cal R}
\label{right}\\[2mm] 
&(1-\sum_{t<t_1}x_{it} - \sum_{t>t_2}y_{it}) \, \bb + \sum_{j\neq i}
  \bb[j] \le \nonumber \\
&  B(t_2-t_1) \quad \forall i \in {\cal A},\ \forall
  [t_1,t_2] \in {\cal R}
\label{total}\\[2mm] 
&(\sum_{t\le t_1}x_{it} + \sum_{t\ge t_2}y_{it} -1 ) \, \bb  \le
B(t_2-t_1) \nonumber\\
&\forall i \in {\cal A},\ \forall
[t_1,t_2] \in {\cal R}
\label{min}
\end{align}

L'inégalité~\eqref{both} correspond au cas où la tâche est centrée et est
ordonnancée à $\bmax$ durant $[\ES,t_1] \cup
[t_2,\LE{]}$. En effet, cette inégalité n'est active que dans
le cas où $(x_{i\ES} + y_{i\LE} -1 )= 1 \Rightarrow
\left[x_{i\ES}=1\land y_{i\LE}=1\right]$. Or,
ceci implique que la tâche commence à $\ES$ et finit
$\LE$. Donc, la ressource disponible dans $[t_1,t_2[$ doit
être suffisante pour donner la quantité de ressource minimale requise
par $i$ dans $[t_1,t_2[$ dans cette configuration.  Dans tous les
autres cas, l'inégalité devient $\sum_{j\neq i} \bb[j] \le B(t_2-t_1)$
ou $\sum_{j\neq i} \bb[j] - \bb \le B(t_2-t_1)$.

Les inégalités \eqref{left}, \eqref{right}, \eqref{total}, \eqref{min}
correspondent respectivement au cas où $i$ est calée à gauche, $i$ est
calée à droite, $i$ est complètement incluse dans $[t_1,t_2]$, $i$ est
exécutée à $\bmin$ durant l'intervalle $[t_1,t_2]$ et sont déduites de
la même façon que~\eqref{both}.  Ces inégalités seront ajoutées au
modèle indexé par le temps décrit à la section~\ref{MIP} pour
renforcer ce dernier. 
 
\section{Résultats expérimentaux}

Nous avons testé les différentes méthodes de résolution proposées dans
cet article sur les instances de~\cite{Nattaf2015}. Les
expérimentations ont été conduites sous le système d'exploitation
Ubuntu 64-bit 12.04 et les résultats sont calculés au moyen d'un
processeur 4-core, 8 thread Core (TM) i7-4770 CPU et de 8GB de mémoire
RAM. 

Le modèle de PLNE est résolu à l'aide de IBM Cplex 12.6 avec 2 threads
et une limite de temps de 100 secondes. Les inégalités déduites du
raisonnement énergétique sont calculées avant la résolution du PLNE et
ajoutées statiquement au modèle. Ceci augmente la taille du modèle de
$5|{\cal R}|n$ contraintes (avec $|{\cal R}| \in O(n^2)$). 

Le tableau~\ref{table1} décrit les résultats du PLNE. 

L'ajout des inégalités du raisonnement énergétique permet de résoudre
les instances à $25$ tâches de manières plus efficace. Cependant, elles
ralentissent le modèle pour les instances à $20$ ou $30$ tâches mais
la perte de rapidité dans ce cas là est beaucoup moins élevée que le
gain fait sur les instances à $25$. Une poursuite de recherche
intéressante serait d'essayer d'ajouter ces contraintes pendant la
résolution du PLNE en tant que coupes. 

Le modèle de PPC est résolu avec IBM CP Optimizer 12.6. Le
tableau~\ref{table2} décrit les résultats du modèle de PPC.  Le modèle 
de PPC est testé sans ajout du raisonnement énergétique présentés dans
cet article mais le modèle utilise les propagateurs du solveur. Des
résultats expérimentaux plus détaillées seront proposés lors de la
conférence.
\begin{table}
\centering
    \begin{tabular}{|c|c|cc|cc|}
      \hline
       & & \multicolumn{2}{c|}{$1^{ère}$ sol.} & \multicolumn{2}{c|}{fin algo.}\\
      \hline
    &  \#tâches & temps(s) &  écart & temps  & \%opt. \\
      \hline
      DEF	&	20	&	5.37	&	7.85	&	75.4	&	0.25\\
      ER	&	20	&	8.4	&	10.6	&	78.9	&	0.22\\
      \hline
      DEF	&	25	&	4.6	&	4.4	&	83.8	&	0.17\\
      ER	&	25	&	0.06	&	3.86	&	60.1	&	0.4\\
      \hline
      DEF	&	30	&	0.99	&	7.18	&	75.19	&	0.25\\
      ER	&	30	&	5.66	&	7.53	&	75.8	&	0.25\\
      \hline
    \end{tabular}
  \caption{Résultats du PLNE avec et sans inégalités valides: ER et
    DEF resp. (TL 1000s)}
  \label{table1}
\end{table}


\begin{table}
\centering
    \begin{tabular}{|c|cc|cc|}
      \hline
      & \multicolumn{2}{c|}{$1^{ère}$ sol.} & \multicolumn{2}{c|}{fin algo.}\\
      \hline
      \#tasks & time & deviation & time lim. &  \%solved \\
      \hline
20	&0.19&	34.1&	100&	95\\
25	&0.3	&47.9&	100	&91\\
30	&0.42	&43.1	&100	&95\\
      \hline
    \end{tabular}
  \caption{Résultats du modèle PPC}
\label{table2}
\end{table}

Les résultats montrent l'intérêt des inégalités valides ajoutées au
PLNE. Le modèle de PPC ne permet pas de prouver l'optimalité des solutions
trouvées mais a des résultats similaires au PLNE. En effet, dans
presque tous les cas, le modèle de PPC trouve une solution aussi bonne
que le PLNE, sans toutefois prouver son optimalité.

Les méthodes présentées ont aussi été testées sur des instances
dilatées dans le but de garantir l'existence de solutions entières.  
La dilatation est effectuée de la manière suivante. Soit $\alpha$ le
plus petit commun 
multiple à tous les $\bmin$ et $\bmax$. Alors, la dilatation
consiste à multiplier $\LE,\ \EE,\
\LS,\ \ES$ et $W_i$ par $\alpha$. Ces
expérimentations n'ont pas donné de résultats dû à la grande taille
de ces modèles. Les modèles continus pourraient donc rester  la
seule alternative pour obtenir des solutions optimales dans le cas où
la solution est réelle.

Parmi les poursuites de recherche possibles, on trouve l'amélioration
des modèles avec la réduction du nombre de variable et/ou de contraintes et la
mise en place d'algorithmes de propagation dédiés.
\section*{Bibliographie\markboth{}{BIBLIOGRAPHIE}}
\begin{bibunit}[alpha]
\nocite{JFPC}
\nocite*
\putbib[JFPC]
\end{bibunit}









\newpage%
\thispagestyle{empty}

\vspace{2cm}
\noindent\hrulefill

{\bf Auteur: }Margaux NATTAF

{\bf Titre:} Ordonnancement sous contraintes d'énergie

{\bf Directeur de thèse: }Christian ARTIGUES et Pierre LOPEZ

Soutenue le 18/10/2016 au LAAS-CNRS (Toulouse) 

\noindent\hrulefill

{\bf Résumé: }Les problèmes d’ordonnancement à contraintes de
ressource ont été largement étudiés dans la littérature. Cependant,
dans la plupart des cas, il est supposé que les activités ont une
durée fixe et nécessitent une quantité constante de la ressource
durant toute leur exécution.  Dans cette thèse, nous nous proposons de
traiter un problème d'ordonnancement dans lequel les tâches ont une
durée et un profil de consommation de ressource variables. Ce profil,
qui peut varier en fonction du temps, est une variable de décision du
problème dont dépend la durée de la tâche associée. Par ailleurs, la
considération de fonctions de rendement linéaires et non linéaires
pour la représentation de l’utilisation des ressources complexifie le
problème et permet de modéliser de manière réaliste les transferts de
ressources énergétiques. Pour ce problème NP-complet, nous présentons
plusieurs propriétés permettant de dériver des modèles et méthodes de
résolution. Ces méthodes de résolution sont divisées en deux
parties. La première partie visualise ce problème du point de vue de
la Programmation Par Contraintes et plusieurs méthodes dérivées de ce
paradigme sont détaillées dont le développement du raisonnement
énergétique sur le problème étudié. La seconde partie de la thèse est
dédiée à des approches de Programmation Linéaire Mixte et plusieurs
modèles, notamment un modèle à temps continu basé sur les événements,
ainsi que des analyses théoriques et des techniques d’amélioration de
ces modèles sont présentés. Enfin, des expérimentations viennent
appuyer les résultats présentés dans ce manuscrit.

\noindent\hrulefill

{\bf Mots-clés: } ordonnancemment, énergie, recherche arborescente et
locale, programmation mathématique, propagation de
contraintes, complexité



\noindent\hrulefill

{\bf Discipline administrative: }Informatique

\noindent\hrulefill

{\bf Adresse du laboratoire :} 

Laboratoire d'Analyse et d'Architecture des Systèmes

7, avenue du Colonel Roche 

BP 54200 

31031 Toulouse cedex 4, France

\end{document}