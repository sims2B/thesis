\documentclass[11pt,a4paper,svgnames]{book}

\usepackage[hmargin=2.5cm,vmargin=3.5cm]{geometry}
\usepackage[ED=EDSYS - Info, Ets=UT3]{tlsflyleaf}
\usepackage{algpseudocode}
\usepackage[french,ruled,vlined]{algorithm2e}
\usepackage[utf8]{inputenc}
\usepackage[T1]{fontenc}
\usepackage[french]{babel}
\usepackage[utf8]{inputenc}
\usepackage{xcolor}
\usepackage{amsmath}
\usepackage{amssymb}
\usepackage{amsthm}
\usepackage{tikz}
\usetikzlibrary{patterns,shapes,positioning,shapes.misc}
\usepackage{array}
\usepackage{xargs}
\usepackage{multirow}
\usepackage{setspace}
\usepackage{tabularx}
\usepackage{caption}
\usepackage{float}
\usepackage[small,compact]{titlesec} 
\usepackage{makeidx}
\usepackage{minitoc} 
\setcounter{minitocdepth}{1}
\usepackage{subcaption} 
\captionsetup[subfigure]{font+=footnotesize}
\allowdisplaybreaks[4]
 \renewcommand{\baselinestretch}{1.2} 
\usepackage{fancyhdr}
\fancyhead{}
\fancyhead[LO]{\rightmark}
\fancyhead[RE]{\leftmark}
\pagestyle{fancy}
\fancypagestyle{plain}{
\fancyhead{}
\fancyhead[LO]{\rightmark}
\fancyhead[RE]{\leftmark}
}

\mtcsettitle{parttoc}{}

\newcommand\RCPSP{RCPSP}
\newcommand\CECSP{CECSP}
\newcommand\CUSP{CuSP}
\newcommand\CuSP{probl{\`e}me cumulatif}
\renewcommand\H{{\cal H}}
\newcommand\A{{\cal A}}
\renewcommand\P{{\cal P}}
\renewcommand\S{{\cal S}}
\newcommand\R{{\cal R}}
\newcommand\I{{\cal I}}
\renewcommand\O{{\cal O}}
\newcommand\Q{{\cal Q}}
\newcommand\X{{\cal X}}
\newcommand\C{{\cal C}}
\newcommand\D{{\cal D}}
\newcommandx{\bmin}[1][1=i]{r_{#1}^{min}}
\newcommandx{\bmax}[1][1=i]{r_{#1}^{max}}
\newcommandx{\emin}[1][1=i]{e_{#1}^{min}}
\newcommandx{\smax}[1][1=i]{s_{#1}^{max}}

\newcommandx{\ES}[1][1=i]{est_{#1}}
\newcommandx{\LS}[1][1=i]{lst_{#1}}
\newcommandx{\EE}[1][1=i]{eet_{#1}}
\newcommandx{\LE}[1][1=i]{let_{#1}}

\newcommandx{\inter}[2][1=t_1,2=t_2]{{[}#1,#2{[}}
\newcommandx{\htd}{G_{t_2}} 
\newcommandx{\bdp}[1]{\Delta'(#1)}
\newcommandx{\bup}[1]{\Gamma'(#1)} 
\newcommandx{\bd}[1]{\Delta(#1)}
\newcommandx{\bu}[1]{\Gamma(#1)} 
\newcommandx{\htu}{G_{t_1}}
\newcommandx{\itd}{D_{t_2}} 
\newcommandx{\itu}{D'_{t_1}} 

\newcommandx{\bb}[3][1=i,2=t_1,3=t_2]{\underline{b}(#1,#2,#3)}
\newcommandx{\wb}[3][1=i,2=t_1,3=t_2]{\underline{w}(#1,#2,#3)}
\newcommandx{\conv}[1][1=W_i]{CR\left(#1,t_1,t_2\right)}
\newcommandx{\wbLS}[3][1=i,2=t_1,3=t_2]{\underline{w}_{LS}(#1,#2,#3)}
\newcommandx{\wbRS}[3][1=i,2=t_1,3=t_2]{\underline{w}_{RS}(#1,#2,#3)}
\newcommandx{\wbCS}[3][1=i,2=t_1,3=t_2]{\underline{w}_{CS}(#1,#2,#3)}
\newcommandx{\bbLS}[3][1=i,2=t_1,3=t_2]{\underline{b}_{LS}(#1,#2,#3)}
\newcommandx{\bbRS}[3][1=i,2=t_1,3=t_2]{\underline{b}_{RS}(#1,#2,#3)}
\newcommandx{\bbCS}[3][1=i,2=t_1,3=t_2]{\underline{b}_{CS}(#1,#2,#3)}

\newcommandx{\Em}[1][1=2n]{\E\setminus\{#1\}}
\newcommand\E{{\cal E}}
%%%%%%%%%% INDEX %%%%%%%%%%%%%%%

\newcommand\CuSPidx{CuSP}
\newcommand\RCPSPidx{RCPSP}
%%%%%%%%%%%%% THEOREME %%%%%%%%%%%%%%%%

\newtheorem{theo}{Théorème}[chapter]
\newtheorem{defi}{Définition}[chapter]
\newtheorem{coro}{Corollaire}[theo]
\newtheorem{ex}{Exemple}[section]
\newtheorem{lemma}{Lemme}[chapter]
\newtheorem{prop}{Proposition}[chapter]
\newtheorem{reg}{Règle}[chapter]
%%%%%%%%%% TIKZ %%%%%%%%%%%%%

\tikzset{every node/.style={circle,minimum size=0pt,node
distance=3cm,inner sep=2pt}}

%%%%%%%%%%%%%%%% ARRAY %%%%%%%%%%%

\newcolumntype{M}[1]{>{\centering\arraybackslash}m{#1}}
\newcolumntype{P}[1]{>{$}M{#1}<{$}}

%%%%%%%%%%%%%%%%%TIKZ STYLE%%%%%%%%%%%
    \tikzstyle{tree}=[circle,draw=gray!70!,minimum width=0.7cm]
    \tikzstyle{leaf}=[circle,draw,minimum width=0.7cm]
    \tikzstyle{sol}=[rectangle,draw,very thick,minimum width=0.6cm,minimum height=0.6cm]
 \parskip=7pt
 \setcounter{topnumber}{1}
 \setcounter{bottomnumber}{1}

\hyphenation{ordon-nan-ce-ment}
\hyphenation{con-train-tes}
\title{\textbf{\large Ordonnancement sous contraintes d'énergie}}
\author{Margaux NATTAF}
\defencedate{18/10/2016}
\lab{Laboratoire d'Analyse et d'Architecture des Systèmes (LAAS)}


\nboss{2}
\makesomeone{boss}{2}{Christian ARTIGUES}{}{}  % Sera affiche en second
\makesomeone{boss}{1}{Pierre LOPEZ}{}{} % Sera afiche en premier

%% Referee
\nreferee{2}
\makesomeone{referee}{1}{Philippe BAPTISTE}{}{}
\makesomeone{referee}{2}{Claude-Guy QUIMPER}{}{}

%% Judges
\njudge{8}
\makesomeone{judge}{1}{Christian ARTIGUES}{Directeur de recherche au
  CNRS, LAAS-CNRS\\
\vspace{0.3cm}}{Directeur de thèse}
\makesomeone{judge}{2}{Philippe BAPTISTE}{Directeur de recherche au CNRS\\

\vspace{0.3cm}}{Rapporteur}
\makesomeone{judge}{3}{Cyril BRIAND}{Professeur,
  Université Toulouse 3-Paul Sabatier\\
\vspace{0.3cm}}{Examinateur}
\makesomeone{judge}{4}{Tam{\'a}s KIS}{Research fellow, MTA SZTAKI, Budapest, Hongrie\\
\vspace{0.3cm}}{Examinateur}
\makesomeone{judge}{5}{Philippe LABORIE}{Principal Scientist, IBM, Paris\\
\vspace{0.3cm}}{Examinateur}
\makesomeone{judge}{6}{Pierre LOPEZ}{Directeur de recherche au CNRS, LAAS-CNRS\\
\vspace{0.3cm}}{Directeur
  de thèse} 
\makesomeone{judge}{7}{Alain QUILLIOT}{Professeur, Université Blaise Pascal, Clermont-Ferrand\\
\vspace{0.3cm}}{Examinateur}
\makesomeone{judge}{8}{Claude-Guy QUIMPER}{Professeur agrégé,
  Université Laval, Québec, Canada}{Rapporteur}

% ============================================================
% DOCUMENT
\begin{document}
\makeflyleaf
\doparttoc
\chapter*{Remerciements}

Je tiens d'abord à remercier mes deux directeurs de thèse, Pierre
Lopez et Christian Artigues, pour m'avoir donné la chance de faire
cette thèse avec eux. Son bon déroulement est grandement dû à leurs
qualités scientifiques et humaines. Merci de m'avoir si bien encadrée
et supportée pendant ces trois dernières années.

Je voudrais aussi remercier Alain Quillot pour l'honneur qu'il m'a
fait en présidant le jury de ma soutenance. 

Je remercie aussi Claude Guy Quimper et Philippe Baptiste d'avoir
accepté d'être rapporteur de cette thèse et pour leur conseil
d'experts sur mes travaux. 

Je tiens aussi à remercier Philippe Laborie, Tam{\'a}s Kis et Cyril
Briand pour avoir accepter de participer à mon jury en temps
qu'examinateur mais aussi pour l'intérêt qu'ils ont montré dans ce
travail.

J'adresse également mes sincères remerciements à tous les membres de
l'équipe ROC et à son personnel administratif grâce à qui ces trois
années de thèse ont pu se dérouler dans de si bonnes conditions.

Merci à tous les gens qui m'ont permis de décompresser dans les
moments difficiles (et les moments faciles aussi). Grâce à eux, ces
trois ans de vie toulousaine ont été un plaisir (tellement que je
reste encore un peu). 

Merci aussi aux doctorants du Laboratoire l'Informatique et de
Robotique et de Micro-électronique de Montpellier qui m'ont aidé sur
les problèmes que j'ai rencontré lors de cette thèse et qui m'ont
toujours gardé une petite place pour mes visite. 

Enfin, merci à tous ceux qui, de près ou de loin, ont contribué à
l'aboutissement de cette thèse. {\'A} tous ceux qui m'ont supporté et
à tous ceux qui n'ont jamais douté que j'y arriverai. %
\clearemptydoublepage%

\tableofcontents
%
\clearemptydoublepage%

\thispagestyle{empty}

\listoffigures
%
\clearemptydoublepage%
\listoftables
%
\clearemptydoublepage%
%\listofalgorithms
e\chapter*{Introduction\markboth{INTRODUCTION}{}}

De nos jours, de nombreuses tâches, auparavant fastidieuses, ont vu
leur complexité largement diminuer grâce à la mise en place d'outils
informatiques permettant leur traitement. Ces outils sont constamment
améliorés et l'augmentation des performances des ordinateurs les
rendent de plus en plus efficaces. Cependant, nombre de ces outils
reposent sur la résolution de problèmes complexes demandant un
grand nombre d'opérations. Dans le cas où la variante décisionnelle du
problème est NP-complète, ce nombre d'opérations est
exponentiel en fonction de la taille du problème. Dans ce cas, la
résolution de tels problèmes peut prendre plusieurs centaines
d'années. 

La mise en place de techniques dédiées permettant de prendre en
considération les propriétés intrinsèques du problème est donc un
axe de recherche majeur dans le domaine de l'informatique. Parmi les
classes de problèmes les plus étudiés, on retrouve les problèmes
d'ordonnancement, les problèmes de tournées de véhicules ou les
problèmes d'affectation. 

Dans cette thèse, nous nous sommes intéressé aux problèmes
d'ordonnancement et, plus particulièrement, aux problèmes
d'ordonnancement avec contraintes de ressource. Parmi les problèmes
les plus étudiés dans la littérature, nous retrouvons le problème
d'ordonnancement de projet avec contraintes de ressource et le problème
cumulatif.

Dans le problème d'ordonnancement de projet avec contraintes de
ressource, nous devons ordonnancer un ensemble d'activités, chacune
d'entre elles consommant une partie d'une ou plusieurs ressources (de
capacité limitée) et étant liées par des relations de précédence. Le
plus souvent, ces activités doivent être ordonnancées de manière à
minimiser la date de fin du projet mais de nombreux autres objectifs,
tels que la minimisation du coût ou des retards peuvent être trouvés 
dans la littérature.

Dans le problème cumulatif, les activités consomment une quantité
d'une et une seule ressource. Il s'agit de la même ressource pour
toutes les activités. Dans ce problème, il n'y a pas de relation de
précédence entre les activités mais chaque activité dispose d'une
fenêtre de temps dans laquelle elle doit s'exécuter. Ce problème
correspond à une relaxation de la variante décisionnelle du problème d'ordonnancement de projet avec contraintes de
ressource.

La plus grande limites de ces problème est que les activités sont
supposées de durée fixes et consommant une quantité de ressource
constante au cours du temps. Cependant, il existe de nombreuses
applications pratiques dans lesquelles ces contraintes ne sont pas
respectées~\cite{HaitArtiguesLopez,Blaz,W80}. En effet, la
possibilité, pour une activité, de consommer plus de ressource
(respectivement moins) afin de finir plus rapidement (resp. lentement)
n'est pas modélisée dans ces problèmes. Les activités satisfaisant
cette propriété sont dites {\it malléables} dans le sens où leur
forme, définie par leur durée et consommation de ressource pendant
leur exécution, doit être décidée pendant le processus de
résolution. Des exemples de telles activités sont variés lorsque les
activités doivent consommer, par exemple, une ressource de type
énergétique comme 
l'électricité.  La quantité de ressource 
allouée à une activité peut alors être modulée à tout instant pour
accélérer l'activité 
ou au contraire la ralentir et diminuer sa consommation (souvent
pour réduire les coûts 
énergétiques). Un autre exemple intervient dans l'ordonnancement de
projet où la durée d'une activité dépend de la quantité de ressource
qui lui est attribuée (p.e. personnel). 

Cette thèse étudie une nouvelle modélisation des activités malléables
représentée par un problème appelé le problème d'ordonnancement
continu avec contraintes énergétiques. Dans ce problème, un ensemble
d'activités,  utilisant une ressource continue et cumulative de
capacité limitée, doit être ordonnancé. La quantité de ressource
nécessaire à l'exécution d'une activité n'est pas fixée mais doit
être déterminée à chaque instant. Une fois la tâche commencée et
jusqu'à sa date de fin, la quantité de ressource consommée par
l'activité doit être comprise entre une valeur maximale et une valeur
minimale. De telles bornes dans des problèmes pratiques
peuvent représentées, par exemple, la quantité d'énergie
minimale requise pour qu'une réaction chimique ou thermodynamique
puisse se faire dans les conditions prescrites ou encore le nombre
minimum et 
maximum d'employés pouvant être affectés à une activité.
De plus, la consommation, à un instant donné, d'une partie
de la ressource permet à l'activité de recevoir une certaine quantité
d'énergie, calculée par le biais d'une fonction de rendement. Ces
fonctions de rendement peuvent, par exemple, représenter les pertes
dues à la conversion de la ressource en énergie (p.e. AC/DC) ou les
coûts de communication dans des architectures multiprocesseurs. 
La connaissance de l'énergie reçue par une activité nous permet de
savoir quand 
l'activité est terminée, i.e. quand elle a reçue une quantité
suffisante d'énergie. 

Pour le problème cumulatif et le problème d'ordonnancement de projet
avec contraintes de ressource, différentes techniques permettant de
trouver des solutions ont été mises en place dans la littérature. Ces
techniques utilisent des concepts et théories pouvant être très
variés. Cependant, deux de celles figurant parmi les plus utilisées
demeurent les techniques issues de la programmation par contraintes et
de la programmation linéaire mixte (ou en nombres entiers). En effet,
ces techniques se sont révélées très efficace dans le processus de
résolution de ces deux problèmes. Ce sont quelques unes de ces
méthodes que nous nous proposons d'étendre au problème considéré dans
cette thèse.

Le plan de la thèse est le suivant: 
\begin{itemize}
\item le chapitre~\ref{sec:chapter1} commence par détailler les
principales caractéristiques des problèmes d'ordonnancement
(paragraphe~\ref{sec:ordo_def}). Ensuite, une définition formelle des
problèmes d'ordonnancement cumulatif et de projet
avec contraintes de ressource ainsi qu'une description des principales
limitations de ces problèmes en termes de modélisation de certaines
ressources sont données
(paragraphe~\ref{sec:ordo_res}).  Enfin, le
paragraphe~\ref{sec:ordo_nrj} décrit le problème étudié dans ce
manuscrit: le problème d'ordonnancement continu avec contraintes
énergétiques. De plus, ce paragraphe décrit les modélisations
préexistantes des activités malléables tout en expliquant en quoi ces
modélisations ne suffisaient pas à la modélisation de certains problèmes
réels. La dernière partie du chapitre est consacrée à la présentation
d'un certain nombre de propriétés remarquables satisfaites par ce problème.
\item les chapitres~\ref{sec:PPC_CUSP} et~\ref{sec:PPC_CECSP} sont
  dédiés aux méthodes de programmation par
  contraintes. Après une brève introduction à la programmation par
  contraintes (paragraphe~\ref{sec:PPC}) et à l'ordonnancement en
  programmation par contraintes, nous
  présentons les principaux algorithmes de filtrage mis en place pour
  le problème cumulatif (paragraphe~\ref{sec:cumu}). Dans le
  chapitre~\ref{sec:PPC_CECSP}, nous adaptons une partie de ces
  algorithmes au problème d'ordonnancement continu avec contraintes
  énergétiques. Dans un premier temps, nous montrons qu'une partie de
  ces algorithmes peut facilement être adaptée en considérant
  l'ordonnancement des activités dans le pire des cas en termes de
  durée ou de consommation de ressource (voir
  paragraphe~\ref{sec:time_CECSP}). Le paragraphe~\ref{sec:ER_CECSP} est
  consacré à l'adaptation du raisonnement énergétique. Pour ce
  raisonnement, nous présentons plusieurs méthodes permettant de
  caractériser les intervalles sur lesquels appliquer ce
  raisonnement. Nous attirons ici l'attention du lecteur sur
  l'utilisation du terme {\it énergie}. En effet, dans ce manuscrit
  nous utiliserons à la fois ce terme pour le raisonnement
  énergétique (algorithme de filtrage pour la contrainte cumulative)
  mais aussi dans un problème qui modélise des ressources énergétiques
  telles que l'électricité.
\item les chapitres~\ref{sec:PLNE_RCPSP} et~\ref{sec:PLNE_CECSP}
  présentent les techniques de résolution issues de la programmation
  linéaire mixte. Dans un premier temps, nous décrivons les concepts généraux
  de la programmation linéaire mixte (paragraphe~\ref{sec:PLNE}). Nous
  présentons ensuite trois modèles mis en place pour
  résoudre le problème d'ordonnancement de projet avec contraintes de
  ressource (paragraphe~\ref{sec:PLNE_ordo_res}). Le premier est un
  modèle indexé par le temps et les deux autres utilisent des
  formulations basées sur les événements. Ces modèles sont
  ensuite adaptés au problème d'ordonnancement continu
avec contraintes énergétiques (paragraphe~\ref{sec:modele_CECSP}). Le
paragraphe~\ref{sec:amelioration_modele} présente plusieurs
ensembles d'inégalités permettant de renforcer les modèles
présentés. Pour le modèle indexé par le temps, des inégalités
directement déduites du raisonnement énergétique sont exhibées. Pour
les modèles à événements, cinq jeux d'inégalités sont présentés et
l'un d'entre eux est utilisée pour donner une description minimale de
l'enveloppe convexe du polyèdre formé par toutes les affectations
possibles des variables binaires correspondant à une seule activité.
\item le chapitre~\ref{sec:expe} présente les résultats expérimentaux
  conduits pour valider les notions théoriques décrites dans les
  chapitres précédents. Dans un premier temps, nous présentons les
  instances sur lesquelles les algorithmes ont été appliqués (voir
  paragraphe~\ref{sec:instance}). Le paragraphe~\ref{sec:expe_PLNE}
  présente les performances des différents modèles de programmation
  linéaire mixte définis pour le \CECSP. De plus, l'influence des 
  inégalités définies pour renforcer les modèles est évaluée. Le
  paragraphe suivant (\ref{sec:expe_PPC}) présente les résultats obtenus
  lors des expérimentations portant sur la programmation par
  contraintes. Les raisonnements présentés dans le manuscrit sont
  intégrés dans une méthode de branchement hybride utilisant un modèle
  de programmation linéaire.  
\item les annexes~\ref{ann:JFPC} et~\ref{ann:IESM} présentent des
  travaux réalisés pendant la thèse mais pas assez aboutis ou trop
  éloignés du sujet de ce manuscrit pour y figurer à part entière.
  L'annexe~\ref{ann:JFPC} présente l'étude du cas discret du problème
  (article publié aux Journées Francophones de la Programmation par
  Contraintes~\cite{Nattaf_JFPC}). L'annexe~\ref{ann:IESM} porte sur
  la mise en place d'une matheuristique pour un problème industriel
  d'ordonnancement avec contraintes et objectifs énergétiques. Ce
  travail s'inscrit dans le cadre d'une collaboration pour un projet
  franco-chilien et a été publié dans la conférence IESM (International
  Conference on Industrial Engineering and Systems
  Management~\cite{Nattaf_IESM}). 
\end{itemize}











%
\clearemptydoublepage%

\part{Introduction}

\chapter{Ordonnancement, ressource et énergie}

\section{Ordonnancement et contraintes de ressources}
\label{sec:ordo}
\subsection{L'ordonnancement}
\label{sec:ordo_def}
La théorie de l'ordonnancement s'intéresse au calcul de dates
d'exécution d'un ensemble d'activités. Dans cette optique,
l'utilisation d'une ou plusieurs ressources peut être nécessaire et
l'exécution d'une activité implique souvent une telle consommation. Un
problème d'ordonnancement peut alors être vu comme l'organisation dans
le temps de la réalisation d'activités soumises à des contraintes de
temps et de ressource. Dans la plupart des cas, un ou plusieurs
objectifs sont définis et une solution au problème d'ordonnancement 
vise à optimiser ces objectifs.
 
\subsubsection{Les activités}

Une activité peut être définie par une date de début $st_i$ et une date
de fin $et_i$ et une durée $p_i$ vérifiant $et_i=st_i+p_i$. Si
l'activité utilise une ou plusieurs ressources durant son exécution, il
est nécessaire d'ajouter à cette définition une fonction d'allocation
de ressource propre à chaque activité $i$ et à chaque
ressource $k$. Si cette fonction est donnée dans les paramètres du
problème, on la note $r_{ik}(t)$. Si, au contraire, elle fait partie
des variables de décision du problème, on la note $b_{ik}(t)$. Enfin,
si cette fonction ne dépend pas du temps, i.e. est constante, le
paramètre $t$ pourra être omis. 

Selon les problèmes, une activité peut être contrainte à s'exécuter en
un seul morceau. On parle alors d'activité non préemptive et dans le
cas contraire, i.e. les activités peuvent être exécutées en plusieurs
morceaux, on parle d'activité préemptive.

Une activité peut être utilisée pour représenter, par exemple, une
opération dans un processus de production, le décollage/atterrissage
d'un avion ou encore une étape d'un projet de construction.

\subsubsection{Les ressources}

Une ressource $k$ est un moyen technique ou humain requis pour la
réalisation d'une activité et est disponible en quantité
limitée. Cette quantité, appelée {\it disponibilité de la ressource ou
  capacité}, peut être soit constante ou varier au cours du
temps. Dans ce manuscrit, nous considérons des ressources à capacité
constante et cette capacité est notée $R_k$. Les ressources utilisées
par les activités peuvent être de nature diverse. Parmi elles, on peut
distinguer:
\begin{itemize}
\item les ressources renouvelables: ces ressources peuvent être
réutilisées dès lors qu'elles sont libérées. Il s'agit, en fait, de
ressources qui, après avoir été utilisées par une ou plusieurs
activités, sont de nouveau disponible en même quantité. Ces ressources
peuvent, par exemple, représenter la main d'\oe uvre d'une entreprise,
des machines, de l'électricité ou des équipements.
\item à l'inverse, les ressources consommables sont des ressources
dont la consommation globale est limitée au cours du temps. Il peut
s'agir, par exemple, de matières premières ou d'un budget.
\end{itemize}

Parmi les ressources renouvelables, on distingue par ailleurs, les
ressources disjonctives qui ne peuvent exécuter qu'une activité à la
fois -- e.g. pistes de décollage, salles -- et les ressources cumulatives
qui peuvent, elles,  être utilisées par plusieurs activités en
parallèle mais sont disponibles en quantité limitée -- e.g. main d'\oe
uvre,
processeurs.


Du point de vue de leur divisibilité, les ressources peuvent aussi
être divisées selon deux catégories: 
\begin{itemize}
\item les ressources continues, i.e. divisibles en temps ou en
  quantité continu: dans le premier cas, il s'agit de ressources
  pouvant être ré-allouées à tout instant $t \in [0,T]$, où $T$ est une
  borne supérieure sur la date de fin de l'ordonnancement; dans le
  second cas, il s'agit de ressources pouvant être allouées en quantité
  continue, i.e. non discrète. Ce type de ressource permet, par
  exemple, de modéliser l'électricité, l'essence, l'énergie hydraulique...
\item les ressources discrètes, i.e. divisibles en temps ou en quantité
  discret: à l'inverse, le premier cas décrit une ressource où la
  ré-allocation de cette dernière ne peut être 
  exécutée qu'à des temps discrets $t \in \{0,\dots,T\}$; le second
  cas correspond aux ressources ne pouvant être attribuées aux
  activités qu'en quantité discrète,  e.g. employés, machines...
\end{itemize}

\subsubsection{Les contraintes}

Une contrainte permet d'exprimer des restrictions sur les valeurs que
peuvent prendre une ou plusieurs variables du problème. Parmi les
principales, on distingue:
\begin{itemize}
\item les contraintes de temps: elles intègrent les contraintes de
  temps alloué, issues généralement d'impératifs de gestion et
  relatives aux dates limites des activités (e.g. dates
    de livraison) ou à la durée totale d'un projet mais aussi les
    contraintes d'enchaînement qui décrivent des
    positionnements relatifs devant être respectés entre les
    activités. Ces contraintes peuvent, par exemple, modéliser des
    contraintes de précédence entre les activités, i.e. une activité
    ne peut commencer avant qu'une autre n'ait été achevée, ou des temps
    de transition à respecter entre les activités.

    {\'E}tant donné une activité $i$, la date à partir de laquelle
    l'activité $i$ peut être exécutée est appelée {\it date de début au
      plus tôt} et est notée $\ES$ (earliest start time en anglais). De
    même, la date avant laquelle l'activité $i$ doit avoir été
    complètement exécutée sera appelée {\it date de fin au plus tard} et
    notée $\LE$ (latest end time en anglais).

\item les contraintes de ressources:  ce sont des contraintes d'utilisation
  des ressources qui expriment la nature et la quantité de moyens
  utilisés par les activités, ainsi que les caractéristiques
  d'utilisation de ces moyens. Ces contraintes peuvent aussi représenter
  des contraintes de disponibilité des ressources qui précisent la
  nature et la quantité de moyens disponibles au cours du temps.
\end{itemize}

\subsubsection{Les objectifs}

Lors de la résolution d'un problème d'ordonnancement, deux buts
différents peuvent être poursuivis. Le premier vise à trouver une solution
réalisable pour le problème tandis que le second cherche à trouver une
solution optimisant un ou plusieurs critères ou objectifs.

Ces objectifs peuvent être liés à différents aspect de la solution. On
distingue par exemple:
\begin{itemize}
\item les objectifs liés au temps: le temps total d'exécution ou le temps moyen
  d'achèvement d'un ensemble d'activités peuvent être minimisés, mais
  aussi  les retards (maximum, moyen, somme...) par rapport
  aux dates de fin au plus tard fixées par le problème.
\item les objectifs liés aux ressources: la quantité (maximale,
  moyenne, pondérée...) de ressources nécessaires pour réaliser un
  ensemble d'activités peut, par exemple, être minimisée.
\item les objectifs liés aux coûts de lancement, de production, de
  transport, de stockage ou liés aux revenus, aux retours
  d'investissements... 
\item les objectifs liés à une énergie, un débit...
\end{itemize}

Deux exemples de problèmes d'ordonnancement sont présentés dans le
paragraphe suivant: le \RCPSP~et le \CUSP.

\section{Contraintes de ressource et contraintes }

Dans ce manuscrit, nous sommes principalement intéressé aux problèmes
d'ordonnancement cumulatifs. Parmi ces derni

\subsection{Problèmes cumulatifs}
\subsubsection{Le problème d'ordonnancement  de projet à contraintes de
  ressources}
\index{\RCPSPidx}
Le problème d'ordonnancement de projet à contraintes de ressources
(\RCPSP) est un problème d'ordonnancement très général, utilisé pour
modéliser de nombreux problèmes pratiques. L'objectif est
d'ordonnancer un ensemble d'activités de telle sorte que la capacité
de la ressource ne soit pas excédée et qu'une certaine fonction
objectif soit minimisée. Parmi les ressources modélisées, on trouve
des ressources telles que des machines, des personnes, des salles, de
l'argent ou encore de l'énergie. Pour les fonctions objectifs, des
quantités telles que la durée totale du projet, le retard ou les coûts
peuvent être minimiser.

Formellement, le \RCPSP~est défini de la manière suivante: nous
considérons un ensemble d'activités non-préemptives
$\A=\{1,\dots,n\}$ à ordonnancer et un ensemble $\R$ de ressources
renouvelables. Chacune de ces ressources $k \in \R$ est disponible
tout au long du projet en quantité $R_k$ et, durant son exécution, une
activité consomme une quantité $r_{ik}$ (pouvant être nulle) de cette
ressource. Dans ce problème, une activité $i \in \A$ a une durée fixe
$p_i$ et des relations de précédence lient les activités entre
elles. Ces relations sont souvent modélisées à l'aide d'un graphe
$G=(V,E)$, appelé graphe de précédence, dans lequel l'ensemble des
arcs $(i,j) \in E$ représente les relations de précédence, i.e. $(i,j)
\in E \Leftrightarrow i $ doit être ordonnancer avant $j$ dans toute
solution. Dans ce graphe, l'ensemble des sommets, dénoté par
$V=\{0,\dots,n+1\}$, correspond aux $n$ activités auxquelles on ajoute
deux activités fictives $0$ et $n+1$ qui représentent respectivement
le début et la fin du projet. Ces activités fictives ne consomment pas
de ressource et ont une durée d'exécution nulle. De plus, $E$ contient
les arcs suivants: 
\begin{itemize}
\item $(0,i),\ \forall i \in \A$,
\item $(i,n+1),\ \forall i \in \A$.
\end{itemize}

Pour ce problème, la fonction objective la
plus rencontrée dans la littérature étant la minimisation de la date
de fin du projet, i.e. $C_{max}$, nous considérons principalement cet
objectif dans la suite de ce manuscrit. Si un objectif différent est
considéré, nous le précisons. 

L'objectif du problème est donc de déterminer la date de début $st_i$
de chaque activité $i\in \A$ de telle sorte que:
\begin{itemize}
\item à chaque instant $t$, la somme, pour chaque activité, des
  consommations d'une même ressource $k \in \R$ ne doit pas dépasser la
  capacité $R_{k}$ de cette dernière, i.e.
  \begin{equation}\forall t \in \H, \forall k \in \R,\sum_{\substack{i\in \A\\ t \in
        [st_i,st_i+p_i]}} r_{ik} \le R_k\end{equation} 
  avec $\H=\{0,\dots,T\}$ défini l'horizon de temps du projet, où $T$
  est une borne supérieure sur la date de fin du projet.
\item les contraintes de précédences sont satisfaites, i.e. 
  \begin{equation} \forall (i,j) \in E,\ s_i+p_i \le p_j \end{equation}
\item la date de fin du projet $C_{max}= \max_{i \in \A} s_i+p_i$
  soit minimale. 
\end{itemize}

\begin{ex}
  \label{ex_RCPSP}
  Considérons l'instance à quatre activités et deux ressources suivante:
  \begin{itemize}
  \item $R_1=5$ et $R_2=7$
  \item cf. figure~\ref{instance_ex_RCPSP}
  \end{itemize}
  \begin{figure}[!htb]
    \centering
    \subcaptionbox{Données de l'instance de l'exemple du \RCPSP.}{
      \begin{tabular}{|P{1cm}|P{1cm}P{1cm}P{1cm}|}
        \hline
        i & p_i & r_{i1} & r_{i2}\\
        \hline
        1 & 4 & 2 & 3 \\
        2 & 3 & 1 & 5 \\
        3 & 5 & 2 & 2 \\
        4 & 8 & 2 & 4 \\
        \hline
      \end{tabular}}
    \hfill
    \subcaptionbox{Graphe de précédence de l'instance de
      l'exemple du \RCPSP.}{
      \begin{tikzpicture}
        [xscale=1.3]
        \node[draw,circle] (O) at (0,0) {\small $0$};
        \node[right of=O,draw,node distance=1.5cm] (D) {\small $2$}; 
        \node[above right of=O,draw] (U) {\small $1$}; 
        \node[below right of=D,draw,node distance =1.5cm] (T) {\small $3$}; 
        \node[below right of=O,draw] (Q) {\small $4$};
        \node[right of=D,draw,node distance=4.5cm] (C) {\small $5$};  

        \draw[->,thick] (O) -- (U.south west) node[midway, left=0.5pt] {\small $0$};
        \draw[->,thick] (O) -- (D.west) node[midway, above=0.5pt] {\small $0$};
        \draw[->,thick] (O) -- (T.south west) node[midway, below=0.5pt] {\small $0$};
        \draw[->,thick] (O) -- (Q.west) node[midway, left=0.5pt] {\small $0$};
        \draw[->,thick] (U) -- (C.north west) node[midway, above=0.5pt] {\small $0$};
        \draw[->,thick] (D) -- (C.west) node[midway, above=0.5pt] {\small $0$};
        \draw[->,thick] (D) -- (T.west) node[midway, right=0.5pt] {\small $3$};
        \draw[->,thick] (T) -- (C.south west) node[midway, above=0.5pt] {\small $0$};
        \draw[->,thick] (Q) -- (C.south) node[midway, below=0.5pt] {\small $0$};
      \end{tikzpicture}}
    \caption{Instance de l'exemple du \RCPSP.} 
    \label{instance_ex_RCPSP}
  \end{figure}

  La figure~\ref{solution_ex_RCPSP_feas} présente un ordonnancement
  réalisable avec $C_{max}=15$. Cet ordonnancement n'est pas optimal
  puisque si l'activité $1$ est décalée à droite de manière à commencer
  au temps $t=8$, on obtient un ordonnancement ayant une date de fin
  inférieure à celle de l'ordonnancement précédent, i.e. $C_{max}=12$
  (cf. figure~\ref{solution_ex_RCPSP_opt}).

  \begin{figure}[!htb]
    \centering
    \subcaptionbox{Une solution réalisable\label{solution_ex_RCPSP_feas}}[0.4\linewidth]{
      \begin{tikzpicture}
        [xscale=0.4,yscale=0.3]
        \node (O) at (0,0) {};
        \node (bmax) at (0,5) {};

        \draw[->] (O.center) -- (16,0);
        \draw (O.south) -- (bmax.north);

        \draw[dashed] (bmax.center) node[left=0.5pt] {\small $R_1=5$} -- (16,5);
        \draw[fill=white] (0,0) rectangle (4,2) node[midway] {\small $1$};
        \draw[fill=white] (4,0) rectangle (7,1) node[midway] {\small $2$};
        \draw[fill=white] (7,0) rectangle (12,2) node[midway] {\small $3$};
        \draw[fill=white] (7,2) rectangle (15,4) node[midway] {\small $4$};

        \draw (0,0) -- (0,-0.1) node[below=0.5pt] {\small $0$};
        \draw (4,0) -- (4,-0.1) node[below=0.5pt] {\small $4$};
        \draw (7,0) -- (7,-0.1) node[below=0.5pt] {\small $7$};
        \draw (12,0) -- (12,-0.1) node[below=0.5pt] {\small $12$};
        \draw (15,0) -- (15,-0.1) node[below=0.5pt] {\small $15$};

        \foreach \i in {0,...,5}
        {\draw (0,\i) -- (-0.1,\i);}
      \end{tikzpicture}

      \begin{tikzpicture}
        [xscale=0.4,yscale=0.3]
        \node (O) at (0,0) {};
        \node (bmax) at (0,7) {};

        \draw[->] (O.center) -- (16,0);
        \draw (O.south) -- (bmax.north);

        \draw[dashed] (bmax.center) node[left=0.5pt] {\small $R_2=7$} -- (16,7);
        \draw[fill=white] (0,0) rectangle (4,3) node[midway] {\small $1$};
        \draw[fill=white] (4,0) rectangle (7,5) node[midway] {\small $2$};
        \draw[fill=white] (7,0) rectangle (12,2) node[midway] {\small $3$};
        \draw[fill=white] (7,2) rectangle (15,6) node[midway] {\small $4$};

        \draw (0,0) -- (0,-0.1) node[below=0.5pt] {\small $0$};
        \draw (4,0) -- (4,-0.1) node[below=0.5pt] {\small $4$};
        \draw (7,0) -- (7,-0.1) node[below=0.5pt] {\small $7$};
        \draw (12,0) -- (12,-0.1) node[below=0.5pt] {\small $12$};
        \draw (15,0) -- (15,-0.1) node[below=0.5pt] {\small $15$};

        \foreach \i in {0,...,7}
        {\draw (0,\i) -- (-0.1,\i);}
      \end{tikzpicture}}
    \hfill
    \subcaptionbox{La solution optimale\label{solution_ex_RCPSP_opt}}[0.4\linewidth]{
      \begin{tikzpicture}
        [xscale=0.4,yscale=0.3]
        \node (O) at (0,0) {};
        \node (bmax) at (0,5) {};
        \node at (16,0) {};
        \draw[->] (O.center) -- (13,0);
        \draw (O.south) -- (bmax.north);

        \draw[dashed] (bmax.center) node[left=0.5pt] {\small $R_1=5$} -- (13,5);
        \draw[fill=white] (0,0) rectangle (3,1) node[midway] {\small $2$};
        \draw[fill=white] (3,0) rectangle (8,2) node[midway] {\small $3$};
        \draw[fill=white] (3,2) rectangle (11,4) node[midway] {\small $4$};
        \draw[fill=white] (8,0) rectangle (12,2) node[midway] {\small $1$};

        \draw (0,0) -- (0,-0.1) node[below=0.5pt] {\small $0$};
        \draw (3,0) -- (3,-0.1) node[below=0.5pt] {\small $3$};
        \draw (8,0) -- (8,-0.1) node[below=0.5pt] {\small $8$};
        \draw (12,0) -- (12,-0.1) node[below =0.5pt] {\small $12$};

        \foreach \i in {0,...,5}
        {\draw (0,\i) -- (-0.1,\i);}
      \end{tikzpicture}

      \begin{tikzpicture}
        [xscale=0.4,yscale=0.3]
        \node (O) at (0,0) {};
        \node (bmax) at (0,7) {};
        \node at (16,0) {};

        \draw[->] (O.center) -- (13,0);
        \draw (O.south) -- (bmax.north);

        \draw[dashed] (bmax.center) node[left=0.5pt] {\small $R_2=7$} -- (13,7);
        \draw[fill=white] (0,0) rectangle (3,5) node[midway] {\small $2$};
        \draw[fill=white] (3,3) rectangle (11,7) node[midway] {\small $4$};
        \draw[fill=white] (3,0) rectangle (8,2) node[midway] {\small $3$};
        \draw[fill=white] (8,0) rectangle (12,3) node[midway] {\small $1$};

        \draw (0,0) -- (0,-0.1) node[below=0.5pt] {\small $0$};
        \draw (3,0) -- (3,-0.1) node[below=0.5pt] {\small $3$};
        \draw (8,0) -- (8,-0.1) node[below=0.5pt] {\small $8$};    
        \draw (12,0) -- (12,-0.1) node[below=0.5pt] {\small $12$};

        \foreach \i in {0,...,7}
        {\draw (0,\i) -- (-0.1,\i);}
      \end{tikzpicture}}
    \caption{Deux solutions réalisables pour l'exemple du \RCPSP.} 
    \label{solution_ex_RCPSP}
  \end{figure}
\end{ex}

Le \RCPSP~est un problème qui a été prouvé NP-complet au sens
fort~\cite{NP_RCPSP}. Ce problème a donc été très étudié dans la
littérature, notamment pour trouver des méthodes efficaces pour sa
résolution. Dans la section~\ref{sec:PLNE_ordo_res}, nous présentons des
modèles de programmation linéaire permettant de trouver la solution
optimale à ce problème. 

\subsubsection{Le problème cumulatif}

En ordonnancement avec contraintes de ressources, on distingue deux
types de ressources:
\begin{itemize}
\item les ressources disjonctives qui ont la particularité que, à un
  instant $t$, une et une seule activité peut utiliser la ressource.
\item à l'inverse, les ressources cumulatives se distinguent par le
  fait que plusieurs activités peuvent utiliser la ressource
  simultanément à la condition que la consommation totale n'excède pas
  la capacité de la ressource.
\end{itemize}
Le problème d'ordonnancement cumulatif (CuSP)\index{\CuSPidx} permet
de caractériser le fait qu'une ressource (ou un sous-ensemble) soit de
type cumulative dans un projet. Il s'agit en fait d'un cas particulier
du \RCPSP.

Formellement, le \CUSP~prend en entrée un ensemble $
\A=\{1,\dots,n\}$ d'activités non-préemptives à ordonnancer. Pour s'exécuter, une
activité doit consommer une partie de la ressource $r_i$ et ce jusqu'à
l'arrêt de celle-ci, i.e. après un temps $p_i$ correspondant à la
durée de l'activité $i$. Cette ressource est de type cumulative et
renouvelable, disponible en quantité $R$. 

De plus, chaque activité dispose d'une fenêtre de temps $[\ES,\LE]$
dans laquelle l'activité doit obligatoirement s'exécuter. On appelle $\ES$ 
la date de début au plus tôt de $i$ et $\LE$ sa date de
fin au plus tard. 

L'objectif du \CUSP~est donc de déterminer la date de début $st_i$ de
chaque activité $i \in \A$ telle que:
\begin{itemize}
\item la capacité de la ressource n'est excédée à aucun moment du
  projet, i.e.
  \begin{equation} \forall t \in \H,\sum_{\substack{i\in \A\\ t \in
        [st_i,st_i+p_i]}} r_{i} \le  R\end{equation}
  où $\H=\{0,\dots,T\}$ est l'horizon de temps du projet et $T=\max_{i
    \in \A} \LE$.
\item la fenêtre de temps de chaque activité est respectée, i.e. 
  \begin{equation} \forall i \in \A,\ \ES \le st_i < st_i+p_i \le \LE \end{equation}
\end{itemize}

Trouver une solution réalisable pour ce problème - étant
une extension de la variante de décision du problème à une machine
($R=1$ et $r_i=1$) et du problème à $m$ machines ($R=m$ et $r_i=1$) - 
est NP-complet au sens fort~\cite{NP_bible}. De ce fait, dans la
littérature, ce problème est souvent étudié sans fonction
objectif. Sauf précision du contraire, nous ferons de même dans la
suite du manuscrit. 

\begin{ex}
  \label{CUSP_ex}
  Considérons l'instance à quatre activités suivante:
  \begin{itemize}
  \item $R=4$
  \item cf. table~\ref{instance_CUSP_ex}\begin{table}[!htb]
      \centering
      \begin{tabular}{|P{1cm}|P{1cm}P{1cm}P{1cm}P{1cm}|}
        \hline
        i & p_i & \ES & \LE & r_i\\
        \hline
        1 & 2 & 1 & 5 & 2 \\
        2 & 1 & 3 & 5 & 2\\
        3 & 1 & 3 & 5 & 3\\
        4 & 4 & 1 & 10 & 1 \\
        \hline
      \end{tabular}
      \caption{Données de l'instance de l'exemple du \CUSP.}
      \label{instance_CUSP_ex}
    \end{table}
  \end{itemize}
  La figure~\ref{solution_CUSP_ex} présente plusieurs solutions
  réalisables pour cette instance. 
  \begin{figure}
    \begin{minipage}{0.45\linewidth}
      \begin{tikzpicture}
        [xscale=0.6,yscale=0.7]
        \node (O) at (0,0) {};
        \node (bmax) at (0,3) {};
        \node at (9,0) {};

        \draw[->] (O.center) -- (9,0);
        \draw (O.south) -- (bmax.north);

        \draw[dashed] (bmax.center) node[left=0.5pt] {\small $R=3$} -- (9,3);
        \draw[fill=white] (1,0) rectangle (3,2) node[midway] {\small $1$};
        \draw[fill=white] (3,0) rectangle (4,3) node[midway] {\small $3$};
        \draw[fill=white] (4,2) rectangle (8,3) node[midway] {\small $4$};
        \draw[fill=white] (4,0) rectangle (5,2) node[midway] {\small $2$};

        \draw (1,0) -- (1,-0.1) node[below=0.5pt] {\small $1$};
        \draw (3,0) -- (3,-0.1) node[below=0.5pt] {\small $3$};
        \draw (5,0) -- (5,-0.1) node[below=0.5pt] {\small $5$};    
        \draw (4,0) -- (4,-0.1) node[below=0.5pt] {\small $4$}; 
        \draw (8,0) -- (8,-0.1) node[below=0.5pt] {\small $8$};

        \node at (10,0) {};
        \foreach \i in {0,...,3}
        {\draw (0,\i) -- (-0.1,\i);}
      \end{tikzpicture}
    \end{minipage}
    \hfill
    \begin{minipage}{0.45\linewidth}
      \begin{tikzpicture}
        [xscale=0.6,yscale=0.7]
        \node (O) at (0,0) {};
        \node (bmax) at (0,3) {};
        \node at (10,0) {};

        \draw[->] (O.center) -- (10,0);
        \draw (O.south) -- (bmax.north);

        \draw[dashed] (bmax.center) node[left=0.5pt] {\small $R=3$} -- (10,3);
        \draw[fill=white] (1,0) rectangle (3,2) node[midway] {\small $1$};
        \draw[fill=white] (4,0) rectangle (5,3) node[midway] {\small $3$};
        \draw[fill=white] (5,2) rectangle (9,3) node[midway] {\small $4$};
        \draw[fill=white] (3,0) rectangle (4,2) node[midway] {\small $2$};

        \draw (1,0) -- (1,-0.1) node[below=0.5pt] {\small $1$};
        \draw (3,0) -- (3,-0.1) node[below=0.5pt] {\small $3$};
        \draw (5,0) -- (5,-0.1) node[below=0.5pt] {\small $5$};    
        \draw (4,0) -- (4,-0.1) node[below=0.5pt] {\small $4$}; 
        \draw (8,0) -- (8,-0.1) node[below=0.5pt] {\small $8$};

        \foreach \i in {0,...,3}
        {\draw (0,\i) -- (-0.1,\i);}
      \end{tikzpicture}
    \end{minipage}
    \caption{Deux solutions réalisables pour l'exemple du \CUSP.}
    \label{solution_CUSP_ex}
  \end{figure}
\end{ex}

Dans la section~\ref{sec:etat_CUSP}, nous présenterons des
méthodes de résolution pour ce problème utilisant la programmation par
contrainte. 

\section{L'ordonnancement sous contraintes énergétiques}
\label{sec:ordo_nrj} 
La modélisation présentée dans cette section repose sur le problème
d'ordonnancement continu à contraintes
énergétiques~\cite{ArtiguesLopez}, le \CECSP.

\subsection{Définition du problème}

Dans ce problème, un ensemble d'activités non préemptives
$\A=\{1,\dots,n\}$ utilisant une ressource continue, cumulative et
renouvelable, de capacité $R$ doit être ordonnancé. Durant son
exécution, une activité consomme une quantité variable $b_i(t)$ de la
ressource qui doit être comprise entre une valeur minimale, $\bmin \in
[0,R]$, et une valeur maximale, $\bmax \in [\bmin,R]$. De plus, la fin
d'une activité correspond au moment où cette dernière a reçu une
certaine quantité d'énergie $W_i$. Cette énergie est reçue via la
ressource et calculée à l'aide d'une fonction $f_i: \{0\} \cup
[\bmin,\bmax]\longrightarrow\{0\} \cup [f(\bmin),f(\bmax)]$
($f_i(0)=0$ est imposé). Ces fonctions, appelées fonctions de
rendement, font partie de la donnée du problème et sont supposées
continues et strictement croissantes. La quantité d'énergie reçue par
$i$ à l'instant $t$ est donc $\int_{0}^t f_i(b_i(s))ds$. La dernière
contrainte du problème précise que chaque activité doit être exécutée
dans sa fenêtre de temps $[\ES,\LE]$. 

L'objectif du \CECSP~est donc de déterminer la date de début $st_i$ et
de fin $et_i$ de chaque activité $i \in \A$, ainsi que la fonction
d'allocation de ressource, $b_i(t)$, associée à
cette activité telle que: 
\begin{itemize}
\item la fenêtre de temps de chaque activité est respectée, i.e. 
  \begin{equation} 
    \forall i \in \A,\ \ES \le st_i < et_i \le \LE \label{tw_CECSP}
  \end{equation}
  Les activités de durée nulle ne sont pas considérées. 
\item la capacité de la ressource n'est excédée à aucun moment du
  projet, i.e.
  \begin{equation} 
    \forall t \in \H,\sum_{\substack{i\in \A\\ t \in
        [st_i,et_i[}} b_i(t) \le  R \label{res_CECSP}
  \end{equation}
  où $\H=\{0,\dots,T\}$ est l'horizon de temps du projet et $T=\max_{i
    \in \A} \LE$.
\item si une activité est en cours à l'instant $t$ alors les
  contraintes de consommation minimale et maximale doivent être
  respectées, i.e.  
  \begin{equation}
    \forall i \in \A,\ \forall t \in [st_i,et_i[,\ \bmin \le b_i(t) \le
    \bmax \label{req_CECSP}
  \end{equation}
\item si l'activité n'est pas en cours, alors elle ne consomme pas de
  ressource, i.e.
  \begin{equation}
    \label{nulleConso_CECSP}
    \forall i \in \A,\ \forall t \not\in [st_i,et_i[,\  b_i(t)=0 
  \end{equation}
\item l'énergie requise doit être apportée à chaque activité, i.e. 
  \begin{equation}
    \forall i \in \A,\ \int_{st_i}^{et_i}f_i(b_i(t))dt=W_i \label{nrj_CECSP}
  \end{equation}
  Dans certains cas, nous pourrons remplacer cette contrainte par la
  contrainte suivante:
  \begin{equation}
    \forall i \in \A,\ \int_{st_i}^{et_i}f_i(b_i(t))dt \ge W_i \tag{\ref{nrj_CECSP}a}
  \end{equation}
  C'est par exemple le cas quand $b_i(t)$ ou $st_i$ et $et_i$ sont
  contraints à prendre des valeurs entières.
\end{itemize}

Dans ce manuscrit, nous considérons les cas où les fonctions $f_i$
sont continues, croissantes et:
\begin{itemize}
\item égales à la fonction identité, $\forall i \in \A$,
\item affines, $\forall i \in \A$,
\item concaves et affines par morceaux, $\forall
  i \in \A$.
\end{itemize}

L'intérêt de considérer de telles fonctions de rendement est
double. Premièrement, un certain nombre de fonctions de rendement
réelles ont une forme concave~\cite{Ex1,Ex2} due au fait, qu'à partir
d'un certain seuil, il n'est plus aussi ``rentable'' d'allouer plus de
ressource à une activité. Deuxièmement, les fonctions affines et
concaves affines par morceaux nous permettent d'approcher un grand
nombre de fonctions de rendement réelles non linéaires. Un exemple
présentant de telles approximations sera présenté
(cf. exemple~\ref{ex_approx_CECSP}).

Soit $P_i$ le nombre d'intervalles de définition de la fonction $f_i$,
i.e. le nombre d'intervalles où la fonction $f_i$ a une expression
différente, et $\P_i=\{1,\dots,P_i\}$ l'ensemble des indices de ces
intervalles. Les points de cassures de la fonctions $f_i$ sont notés
$x_\ell^i$, $\forall \ell \in \P_i$. La fonction $f_i$ peut alors
s'écrire de la manière suivante: 
\[f_i(b)=\left\{
    \begin{array}{ll}
      0 & \quad \text{si }b=0\\
      a_{i1}*b+c_{i1} &\quad \text{si }\bmin=0\text{ et }b \in ]\bmin,x_{2}^i] \\
      a_{i1}*b+c_{i1} &\quad \text{si }\bmin\neq 0 \text{ et }b \in
                            [\bmin,x_{2}^i] \\
      a_{i\ell}*b+c_{i\ell} &\quad \text{si } b \in
                                  ]x_{\ell}^i,x_{\ell+1}^i], \ell \in \P_i\setminus\{1\}
    \end{array}
  \right.\]
De plus, nous considérons que la fonction $f_i$ satisfait les
propriétés suivantes: 
\begin{itemize}
\item $a_{i1} >a_{i2} > \dots > a_{iP_i}>0$ et $c_{i1}
  <c_{i2} < \dots < c_{iP_i}$ pour assurer la croissance et la
  concavité de la fonction; 
\item $-a_{i1}*\bmin \ge c_{i1}$  afin de s'assurer que $f_i(b) \ge
  0,\ \forall b \in [\bmin,\bmax]$;
\item $a_{i\ell}*b+c_{i\ell}=a_{i\ell+1}*b+c_{i\ell+1}$
  pour assurer la continuité de la fonction.
\end{itemize}
Dans le cas où $f_i(b)$ est une fonction affine, on pourra noter:
$f_i$:
\[f_i(b)=\left\{
\begin{array}{ll}
  0 & \quad \text{if }b=0\\
  a_i*b+c_i &\quad \text{if }\bmin=0\text{ and }b \in ]\bmin,\bmax] \\
  a_i*b+c_i &\quad \text{if }\bmin\neq 0 \text{ and }b \in [\bmin,\bmax]
\end{array}
\right.
\]



\begin{ex}
  \label{ex_approx_CECSP}
  Considérons l'instance à  quatre activités suivante:
  \begin{itemize}
  \item $B=2$
  \item $est=(0,2,0,5)$
  \item $let=(6,10,9,13)$
  \item $r^{min}=(0,0.25,2,1)$
  \item $r^{max}=(1,1,2,1.5)$
  \item $W=(1,5,7,8)$
  \item $f(b)=(b,\sqrt{b},b,\sqrt{b})$
  \end{itemize}
  Le but de cet exemple étant d'illustrer l'approximation d'une
  fonction non affine ou concave affine par morceaux par une fonction
  vérifiant cette propriété, aucune fonction objectif n'est définie pour
  cette instance.

  Nous devons approcher les fonctions $f_2(b)$ et $f_4(b)$. Commençons
  par approcher l'activité $2$ par une fonction affine. Pour
  cela, nous calculons le coefficient directeur de la tangente en
  $\bmin + (\bmax-\bmin)/ 2 =0.625$. Ce coefficient directeur est égal à
  $\frac{1}{2\sqrt{0.625}}$, et donc,
  $f'_2(b)=\frac{1}{2\sqrt{0.625}}*b+ \frac{\sqrt{0.625}}{2}$
  (cf. figure~\ref{approx_aff}).  

  \begin{figure}[!htb]
    \centering
    \subcaptionbox{Approximation par une fonction affine
      \label{approx_aff}}[0.45\linewidth]{
      \begin{tikzpicture}
        [xscale=4.5,yscale=2.5]
        \node (O) at (0,0) {};

        \draw[->] (0,0) -- (1.28,0) node[below] {$b$};
        \draw[->] (0,0) -- (0,1.3) node[left] {$f_i(b)$};

        \draw (0,1) -- (-0.02,1) node[left] {$1$};
        \draw (0,0.5) -- (-0.02,0.5) node[left] {$\sqrt{0.25}$};

        \draw (1,0) -- (1,-0.02) node[below] {$1$};
        \draw (0.25,0) -- (0.25,-0.02) node[below] {$0.25$};

        \draw[color=gray,domain=0:1.28,samples=50] plot ({\x},{sqrt(\x)});
        \draw[dashed,thick,domain=0.25:1,samples=50] plot
        ({\x},{\x/(2*sqrt(0.625))+sqrt(0.625)/2});


        \draw[dotted] (0.25,0)-- (0.25,1.3);
        \draw[dotted] (1,0) -- (1,1.3);
      \end{tikzpicture}}
    \hfill
    \subcaptionbox{Approximation par une fonction concave, affine par
      morceaux
      \label{approx_affparmorceau}}[0.45\linewidth]{
      \begin{tikzpicture}
        [xscale=4.5,yscale=2.5]
        \node (O) at (0,0) {};

        \draw[->] (0,0) -- (1.28,0) node[below] {$b$};
        \draw[->] (0,0) -- (0,1.3) node[left] {$f_i(b)$};

        \draw (0,1) -- (-0.02,1) node[left] {$1$};
        \draw (0,0.5) -- (-0.02,0.5) node[left] {$\sqrt{0.25}$};

        \draw (1,0) -- (1,-0.02) node[below] {$1$};
        \draw (0.25,0) -- (0.25,-0.02) node[below] {$0.25$};

        \draw[color=gray,domain=0:1.28,samples=50] plot ({\x},{sqrt(\x)});
        
        \draw[ dashed,thick,domain=0.25:0.5,samples=50] plot
        ({\x},{\x/(2*sqrt(0.375))+sqrt(0.375)/2});
        \draw[dashed, thick,domain=0.5:0.75,samples=50] plot
        ({\x},{\x/(2*sqrt(0.625))+sqrt(0.625)/2});
        \draw[dashed, thick,domain=0.75:1,samples=50] plot
        ({\x},{\x/(2*sqrt(0.875))+sqrt(0.875)/2});

        \draw[dotted] (0.5,0)-- (0.5,1.3);
        \draw[dotted] (0.25,0) -- (0.25,1.3);
        \draw[dotted] (0.75,0) -- (0.75,1.3);
        \draw[dotted] (1,0) -- (1,1.3);
      \end{tikzpicture}}
    \caption{Exemple d'approximation d'une fonction de rendement non
      linéaire par une fonction affine et par une fonction concave et affine par morceaux.}
    \label{approx}
  \end{figure}

  De même, pour l'activité $4$, nous avons
  $f'_4(b)=\frac{1}{2\sqrt{1.25}}*b+ \frac{\sqrt{1.25}}{2}$.

  Approchons maintenant la fonction $f_2(b)$  par une
  fonction concave et affine par morceaux. Dans un premier temps,
  nous devons choisir le pas d'approximation $\epsilon$, i.e. la taille
  des intervalles pour lesquels la fonction $f_i$ a une
  expression différente. Dans cet exemple, nous choisissons
  $\epsilon=1/4$. Le nombre d'intervalles de définition de la fonction
  $f_2$ est alors $(\bmax-\bmin)/\epsilon=3$. Pour chacun de ces
  intervalles, nous appliquons la procédure utilisée pour
  l'approximation de $f_2$ par une fonction affine. Nous obtenons donc
  l'approximation suivante (cf. figure~\ref{approx_affparmorceau}):
  \[f_2=\left\{ 
      \begin{array}{lll}
        \frac{1}{2\sqrt{3/8}}*b + \frac{\sqrt{3/8}}{2}& & \text{si } b \in
                                                          [0.25,0.5]\\
        \frac{1}{2\sqrt{5/8}}*b + \frac{\sqrt{5/8}}{2}& & \text{si } b \in [0.5,0.75]\\
        \frac{1}{2\sqrt{7/8}}*b + \frac{\sqrt{7/8}}{2}& & \text{si } b \in [0.75,1]
      \end{array}
    \right.\]
  
\end{ex}

Avec une telle approche, il peut arriver que la fonction de rendement
$f_i$ ne vérifie pas $\bmin=0 \Rightarrow f_i(\bmin)=0$. Dans ce cas,
la valeur de $f_i(0)$ est mise à $0$. La fonction n'est donc plus
continue sur tout son intervalle de définition. La
contrainte~\eqref{nrj_CECSP} est donc remplacée par:
\begin{equation}
  \int_{st_i}^{et_i}{\bf 1}_{NZ}(t)f_i(b_i(t))dt = W_i \tag{\ref{nrj_CECSP}b}
\end{equation}
\noindent 
où ${\bf 1}_{NZ}(t):=\left\{
  \begin{array}{ll}
    1 & \text{si }t \in NZ:=\{t|b_i(t)\neq0\}\\
    0 & \text{sinon}
  \end{array}
\right.$ est la fonction caractéristique de l'ensemble $\mathbb{R}^+$.

La difficulté du \CECSP~repose, entre autres choses, sur le fait que la
fonction d'allocation de ressource $b_i(t)$ peut n'être ni constante, ni
constante par morceaux. De ce fait, la représentation temps/ressource
d'une activité peut prendre n'importe quelle forme
(cf. figure~\ref{figure_forme_conso}). 

\begin{figure}[!htb]
  \centering
  \subcaptionbox{$b_i(t)$ constante}[0.45\linewidth]{
    \begin{tikzpicture}
      [xscale=0.75,yscale=0.5]
      \node[] (O) at (0,0) {};
      
      
      \draw (0.5,0) node[below] {$\ES$};
      \draw (6,0) node[below] {$\LE$};
      \draw  (0,1) node[left] {$\bmin$};
      \draw (0,4) node[left] {$\bmax$};

      \draw[dotted] (0,1) -- (6.5,1);
      \draw[dotted] (0,4) -- (6.5,4);
      \draw[dotted] (0.5,0) -- (0.5,4);
      \draw[dotted] (6,0) -- (6,4);

      \draw[->] (O.center) -- (0,4.5) node[above] {$b_i(t)$};
      \draw[->] (O.center) -- (6.5,0) node[right] {$t$};
      
      \draw (5.7,0) -- (5.7,2) -- (1,2) -- (1,0);
    \end{tikzpicture}
  }
  \hfill
  \subcaptionbox{$b_i(t)$ constante par morceaux}[0.45\linewidth]{
    \begin{tikzpicture}
      [xscale=0.75,yscale=0.5]
      \node[] (O) at (0,0) {};
      
      
      \draw (0.5,0) node[below] {$\ES$};
      \draw (6,0) node[below] {$\LE$};
      \draw  (0,1) node[left] {$\bmin$};
      \draw (0,4) node[left] {$\bmax$};

      \draw[dotted] (0,1) -- (6.5,1);
      \draw[dotted] (0,4) -- (6.5,4);
      \draw[dotted] (0.5,0) -- (0.5,4);
      \draw[dotted] (6,0) -- (6,4);

      \draw[->] (O.center) -- (0,4.5) node[above] {$b_i(t)$};
      \draw[->] (O.center) -- (6.5,0) node[right] {$t$};
      
      \draw (5.7,0) -- (5.7,2) -- (4,2) -- (4,4) -- (2,4) -- (2,1) -- (1,1) --  (1,0);
    \end{tikzpicture}
  }\\
  \vspace{0.2cm}
  \subcaptionbox{$b_i(t)$ quelconque}[0.9\linewidth]{
    \begin{tikzpicture}
      [xscale=0.75,yscale=0.5]
      \node (O) at (0,0) {};

      \path[draw] (1,0) -- (1,4) parabola [bend at end] (5.7,1) -- (5.7,0); 

      \draw (0.5,0) node[below] {$\ES$};
      \draw (6,0) node[below] {$\LE$};
      \draw  (0,1) node[left] {$\bmin$};
      \draw (0,4) node[left] {$\bmax$};

      \draw[dotted] (0,1) -- (6.5,1);
      \draw[dotted] (0,4) -- (6.5,4);
      \draw[dotted] (0.5,0) -- (0.5,4);
      \draw[dotted] (6,0) -- (6,4);
      
      
      \draw[->] (O.center) -- (0,4.5) node[above] {$b_i(t)$};
      \draw[->] (O.center) -- (6.5,0) node[right] {$t$};
      
      \draw (1,4) -- (1,0);
      
      
      \path[draw] (1,4) parabola [bend at end] (5.7,1); 
      
    \end{tikzpicture}
    \hfill
    \begin{tikzpicture}
      [xscale=0.75,yscale=0.5]
      \node (O) at (0,0) {};
      
      \draw (0.5,0) node[below] {$\ES$};
      \draw (6,0) node[below] {$\LE$};
      \draw  (0,1) node[left] {$\bmin$};
      \draw (0,4) node[left] {$\bmax$};

      \draw[dotted] (0,1) -- (6.5,1);
      \draw[dotted] (0,4) -- (6.5,4);
      \draw[dotted] (0.5,0) -- (0.5,4);
      \draw[dotted] (6,0) -- (6,4);
      
      
      \draw[->] (O.center) -- (0,4.5) node[above] {$b_i(t)$};
      \draw[->] (O.center) -- (6.5,0) node[right] {$t$};

      \draw (1,1) parabola bend (1.8,3)(2.5,2); 
      \draw (2.5,2) parabola bend (3.2,1) (4.5,3);
      \draw (4.5,3) parabola bend (5.2,4) (5.7,2);
      \draw (1,0) -- (1,1);
      \draw (5.7,0) -- (5.7,2);
    \end{tikzpicture}
  }
\caption{Différentes formes de fonction d'allocation de ressource pour
le \CECSP.}
\label{figure_forme_conso}
\end{figure} 

Nous allons maintenant décrire un exemple d'instance et de solution
pour le \CECSP. Cependant, par souci de clarté, nous présentons un
exemple où il existe une solution dans laquelle toutes les fonctions
$b_i(t)$ sont constantes par morceaux.

\begin{ex}
Considérons l'instance à trois activités du \CECSP~suivante:
\begin{itemize}
\item $R=5$
\item cf. figure~\ref{ex_CECSP}
\item la fonction $f_2(b)$ est définie par l'expression suivante: 
\[f_2(b)=\left\{
\begin{array}{lll}
2b & & b \in [3,4]\\
b+4 & & b \in \ ]4,5]
\end{array}
\right.\]
\end{itemize}
\begin{figure}[!htb]
\centering
\subcaptionbox{Fonction $f_2(b)$ \label{fonction_ex_CECSP}}[0.4\linewidth]{
\begin{tikzpicture}
[xscale=1.65,yscale=0.56]
\node (O) at (2,5) {};
\draw[->] (2,4) -- (5.5,4) node[below] {$b$}; 
\draw[dashed] (2,4) -- (2,5.5);
\draw[->] (2,5.5) -- (2,10) node[left] {$f_2(b)$};


\path[draw] (3,6) -- (4,8) -- (5,9) ;

\draw[dotted] (3,4) node[below] {\footnotesize $3$} -- (3,10);
\draw[dotted,color=gray!70] (4,4) node[below,color=black] {\footnotesize $4$}
-- (4,10);
\draw[dotted] (5,4) node[below] {\footnotesize $5$} -- (5,10);

\draw (2,6) node[left] {\footnotesize $6$};
\draw (2,8) node[left] {\footnotesize $8$};
\draw (2,9) node[left] {\footnotesize $9$};
\end{tikzpicture}}
\hfill
\subcaptionbox{Données de l'instance}[0.55\linewidth]{
  \begin{tabular}{|M{0.6cm}|M{0.6cm}M{0.6cm}M{0.6cm}M{0.6cm}M{0.6cm}M{1.2cm}|}
    \hline
    $i$ & $r_i$ & $d_i$ & $W_i$ & $\bmin$ & $\bmax$ & $f_i(b)$\\[2mm]
\hline
    1 & 0 & 2 & 6 & 3 & 3 & $b$\\[2mm]
    2 & 1 & 5 & 22 & 3 & 4 & fig.~\ref{fonction_ex_CECSP}\\[2mm]
    3 & 0 & 6 & 39 & 1 & 5 & $3b$\\[2mm]
    \hline
    \multicolumn{7}{c}{}
  \end{tabular}} 
\caption{Exemple d'une instance pour le \CECSP.}
\label{ex_CECSP}
\end{figure}
La figure~\ref{sol_ex_CECSP} présente une solution réalisable pour le
\CECSP. Dans cette figure, nous pouvons voir que l'énergie reçue par une
activité n'est, a priori, pas égale à la quantité de ressource
consommée par cette dernière. En effet, regardons l'activité $2$. Sa
consommation de ressource sur l'intervalle $[2,3[$ est de $3$. Sur
cet intervalle, l'énergie reçue par l'activité est alors de
$f_2(3)=6$. Sur les intervalles $[3,4[$ et $ [4,5[$, l'activité
consomme $4$ unités de ressource et reçoit une énergie de
$f_2(4)=8$. Au total, l'activité $2$ consomme $3 + 4 + 4 = 11$ unités
de ressource et reçoit une quantité d'énergie égale à $f_2(3)+ f_2(4)+
f_2(4)= 6 + 8 + 8 =22$. 

\begin{figure}[!htb]
\centering
\begin{tikzpicture}
[xscale=0.75,yscale=0.56]
\node (O) at (0,0) {};
\draw[->] (0,0) -- (6.5,0) node[below] {$t$};
\draw[->] (0,0) -- (0,5.5) node[above] {$b(t)$};

\draw (0,2) rectangle (2,5) node[midway] {$1$};
\path[draw] (0,2) -- (3,2) -- (3,1) node[left=0.4cm] {$3$} -- (5,1) -- (5,5) -- (6,5)  -- (6,0);
\draw (2,5) -- (5,5) node[midway,below=0.5cm] {$2$};

\draw (0,1) node[left] {\footnotesize $1$};
\draw (0,2) node[left] {\footnotesize $2$};
\draw (0,5) node[left] {\footnotesize $5$};


\draw (2,0) node[below] {\footnotesize $2$};
\draw (3,0) node[below] {\footnotesize $3$};
\draw (5,0) node[below] {\footnotesize $5$};

\foreach \i in {1,...,5}{
\draw (\i,-0.1) -- (\i,0);
\draw (-0.1,\i) -- (0,\i);
}
\end{tikzpicture}
\caption{Exemple de solution pour le \CECSP.}
\label{sol_ex_CECSP}
\end{figure}
\end{ex}

La sous-section suivante présente les différentes modélisations des
activités à profil variable présentes dans la
littérature.

\subsection{Autres modélisations des activités à profil variable}

Dans un premier temps, nous nous intéressons aux extensions du
\RCPSP. Une des extensions les plus célèbres est le problème
d'ordonnancement de projet multimode (MRCPSP). Dans ce problème, un
choix de différents modes est disponible pour chaque activité et une
activité doit être exécutée selon un de ces modes. Un mode correspond
à une combinaison formée d'un temps d'exécution constant et d'une
consommation de ressource qui permet d'apporter à l'activité au moins
la quantité d'énergie requise. Même si de nombreux problèmes basés sur
ce concept de mode existent~\cite{DDH,RK,RDK,DD} et que des méthodes
de résolution efficaces ont été mises en place pour résoudre le
MRCPSP~\cite{PV}, cette modélisation peut amener à une mauvaise
allocation de la ressource.

Si nous reprenons l'exemple de la peinture d'un bateau,
décrit à la sous-section~\ref{sec:limit_CUSP}, l'activité avait besoin de
$3$ unités d'énergie pour s'exécuter. Dans le contexte du MRCPSP,
seulement $3$ modes seraient décrits: $(3,1),\ (2,2)$ et $(1,3)$. Or,
dans le second cas, on donne une unité de trop à l'activité et la
possibilité d'allouer $2$ unités de ressource pendant une période de
temps et $1$ unité pendant la seconde n'est pas représentée ici. 

La principale limitation du MRCPSP est donc que les activités sont
contraintes à être rectangulaires, i.e. avec une consommation de ressource
constante. 

D'autres extensions du \RCPSP~existent. C'est le cas par exemple des
problèmes d'ordonnancement de projet avec une ressource de type {\it
  work-content}~\cite{FT} ou du problème d'ordonnancement de projet
avec des profils de ressource flexibles (FRCPSP)~\cite{NK}. Dans ces
problèmes, plusieurs types de ressources sont considérées: 
\begin{itemize}
\item principale (ou work-content dans~\cite{FT}): il s'agit de la
  ressource via laquelle la quantité d'énergie requise est donnée à
  l'activité. C'est elle qui sert à déterminer la durée de l'activité.
\item les ressources dépendantes: l'utilisation de ces ressources
  dépendent de l'utilisation de la ressource principale.
\item les ressources indépendantes: la consommation de ces ressources
  est indépendante des consommations des autres ressources mais ces
  utilisations doivent être synchrones. 
\end{itemize}

Bien que plusieurs différences existent entre ces problèmes et le
\CECSP -- l'utilisation de plusieurs ressources, ressource/temps discret
pour~\cite{FT}... -- les principales sont les suivantes: la longueur
minimale des blocs et les fonctions de rendement. La première
correspond au temps minimal qu'il faut attendre entre deux
ré-allocations de la ressource, que les auteurs de~\cite{FT} appellent
longueur minimale de bloc et qui est absente dans notre problème. La
seconde fait référence à l'absence de fonctions de rendement
dans~\cite{FT}. 

Enfin, la dernière extension du \RCPSP~présentée est celle où les
activités ont une intensité variable~\cite{Kis}. Ici, chaque activité
requiert une certaine quantité d'énergie durant toute son
exécution. Pour apporter cette énergie à l'activité, il faut décider,
dans chaque période de temps, l'{\it intensité} à laquelle est
exécutée l'activité. L'énergie apportée à l'activité est alors
proportionnelle à cette intensité. Dans ce cas, on peut introduire des
fonctions de rendement mais ces fonctions seraient alors contraintes à
être linéaires, i.e. $b \rightarrow a*b$. De plus, aucune borne
inférieure sur la consommation d'une activité n'est considérée.

Dans le cadre du \CUSP, d'autre variantes ainsi que des algorithmes de
filtrages dédiés ont été proposés. Parmi ceux-ci, on retrouve le cas
des activités complètement/partiellement élastiques de Baptiste et
al.~\cite{BLN}. Dans le premier cas, les activités ont une demande en
énergie constante mais la quantité de ressource consommée par
une activité à chaque instant (discret) peut varier entre $0$ et la
capacité de la ressource. Dans le second cas, les mêmes conditions
sont présentes mais les auteurs définissent des contraintes permettant
de limiter les variations dans l'utilisation de la ressource. Aucun de
ces deux problèmes ne considère de fonctions de rendement. 

Dans~\cite{BP}, les auteurs définissent une activité comme une
séquence de sous-activités trapézoïdales ayant des durées et hauteurs
(consommations) variables. Enfin, Vil{\'i}m~\cite{V09} considère des
activités pour lesquelles la durée et la hauteur sont définies par des
intervalles. Pour ces deux problèmes, aucune demande en énergie n'est
définie pour les activités. De plus, dans le second, l'énergie
manquante peut être achetée moyennant un certain coût.

Enfin, le \CECSP~ est aussi lié à d'autres problèmes à contraintes
d'énergie avec ressources continues~\cite{Blaz,Wali}.
Dans~\cite{Blaz}, plusieurs modèles représentant le temps d'exécution
d'une activité en fonction de la ressource qui lui est allouée sont
présentés. En particulier, les auteurs considèrent un problème où 
un ensemble de processeurs identiques et parallèles jouent le rôle de
la ressource. De plus, des fonctions représentant le temps d'exécution
d'une activité en fonction du nombre de processeurs qui lui est
allouée sont définies. Ce nombre de processeurs peut varier
continuellement au cours du temps et donc ces fonctions sont
équivalentes aux fonctions de rendement définies dans le cadre du
\CECSP. De plus, dans~\cite{Blaz,Wali}, l'énergie est calculée en
intégrant une fonction de rendement sur tout l'horizon de temps.
Cependant, aucune contrainte de consommation maximum et minimum n'est
considérée dans ces problèmes. Dans~\cite{Wali}, une partie des
ressources est continue et une partie est discrète.

Le tableau~\ref{tab:dif_CECSP} récapitule les principales différences
entre tous ces problèmes et le \CECSP. 

\begin{table}[!htb]
  \centering
\small
  \begin{tabular}{|>{\centering\arraybackslash} m{3cm}|>{\centering\arraybackslash} m{0.8cm}|>{\centering\arraybackslash} m{0.8cm}|>{\centering\arraybackslash} m{1.9cm}|>{\centering\arraybackslash} m{1.4cm}|>{\centering\arraybackslash} m{1.2cm}|>{\centering\arraybackslash} m{1cm}|>{\centering\arraybackslash} m{2.3cm}|}
    \hline    
    Problème & $\bmin$ & $\bmax$ & fonction de rend. ($f_i$) &
                                                                   activités
                                                                   non
                                                               rect.
    & 
                                                                 énergie
                                                                   ($W_i$)
    & res. cont. & autre différence \\
\hline
MRCPSP \cite{DDH} & $\surd$ & $\surd$ & $\surd$ &    & $\surd$ &  & \\
\hline
 FRCPSP \cite{NK} & $\surd$ & $\surd$ &    & $\surd$ & $\surd$ & $\surd$ & long. de bloc\\  
\hline
Work-content \cite{FT} & $\surd$ & $\surd$ & & $\surd$ & $\surd$ &  & long. de bloc\\  
\hline
Intensité variable \cite{Kis} &  & $\surd$&  & $\surd$ & $\surd$ & & \\   
\hline
Partiellement élastique \cite{BLPN} & & & & $\surd$ & $\surd$ & &\\
\hline
Complètement élastique \cite{BLPN} & & & $\surd$ & $\surd$ & & &\\
\hline
Activités trapézoïdales \cite{BP} & & & & $\surd$ & & &\#trapèzes fixe\\
\hline
Représentation par intervalles \cite{V09} & $\surd$ & $\surd$ & & & & & achat
                                                            d'énergie\\  
\hline
Modèle processeurs \cite{Blaz} & &  & $\surd$& $\surd$& $\surd$& $\surd$& \\ 
\hline
Continu/discret \cite{Wali}&  & &$\surd$ &$\surd$ & $\surd$& $\surd$& res. discrètes et continues\\ 
\hline
  \end{tabular}
  \caption{Principales différences entre les extensions des problèmes
    cumulatifs et le \CECSP.}
  \label{tab:dif_CECSP}
\end{table}

Le \CECSP~est donc un nouveau problème et les différences avec les
problèmes existants ne nous permettent pas d'appliquer directement des
techniques déjà définies pour d'autres problèmes. Cependant, certaines
techniques existantes peuvent être adaptées dans le cadre du
\CECSP. Ces techniques seront présentées plus tard dans le manuscrit.

La section suivante présente des propriétés du \CECSP~qui seront 
utilisées dans le cadre de sa résolution. 

\subsection{Propriétés du \CECSP}
\label{sec:ppte_CECSP}
Dans cette section, nous allons commencer par présenter la preuve de
NP-complétude du \CECSP. Ce problème pouvant être vu comme une
généralisation du \CUSP, nous utilisons ce problème pour montrer la
difficulté du \CECSP. 

\begin{theo}[\cite{Nattaf_Constraints}]
Le \CECSP~est NP-complet.
\end{theo}

\begin{proof}
Nous réduisons donc le \CUSP~vers le \CECSP. Soit $\Pi$ une instance
du \CUSP. Nous réduisons $\Pi$ en une instance du \CECSP, $\Pi'$, de
la manière suivante, $\forall i \in \A$:
\begin{itemize}
\item $ \bmin=\bmax=r_i$
\item $f_i(b)=b$
\item $W_i=p_ir_i$
\item $R,\ \ES$ and $\LE$ restent inchangés. 
\end{itemize}

On peut facilement vérifier  que $\Pi$ est une instance positive du \CUSP~si
et seulement si $\Pi'$ est une instance positive du \CECSP. Le
\CECSP~est donc NP-complet.
\end{proof}

Le problème de décision associé au \CECSP~est donc NP-complet. Dans ce
manuscrit, nous avons donc considérer ce problème sans fonction
objectif mais aussi avec la fonction objectif suivante: 
\[\text{minimiser } \sum_{i \in \A} \int_{st_i}^{et_i} b_i(t)dt\]
Cette fonction consiste en la minimisation de la consommation totale
de ressource. L'intérêt de cette fonction objectif dans le cas où les
fonctions de rendement sont égales à la fonction identité est discuté
dans le chapitre~\ref{sec:PLNE_CECSP}. Dans le cas où les fonctions de
rendement sont affines ou concaves et affines par morceaux, cet
objectif est pertinent puisque, même si la quantité d'énergie apportée
à une activité est fixée, la quantité de ressource que l'activité doit
consommer ne l'est pas et plusieurs profil de consommation peuvent
conduire à la même quantité d'énergie apportée. Trouver le profil qui
consomme le moins de ressource possible tout en apportant l'énergie
requise est donc un vrai problème.  Dans la suite, si rien n'est
précisé, cela veut dire que nous considérons le \CECSP~ sans fonction
objectif.

Nous présentons un exemple d'instance ne comprenant que des données
entières et ne possédant que des solutions non entières. Ceci permet
de justifier l'utilisation de modèles à temps continu.

\begin{ex}
  \label{exemple_NE}
  Dans cet exemple, nous considérons une instance à deux activités et
  une ressource de capacité $2$. Le tableau ci-dessous décrit les
  données de l'instance:
\begin{center}
    \begin{tabularx}{12cm}{|>{\centering\arraybackslash}p{0.6cm}|
        *5{>{\centering\arraybackslash}X}>{\centering\arraybackslash}p{2cm}|}
      \hline
      $i$ & $\ES$ & $\LE$ & $W_i$ & $\bmin$ & $\bmax$ & $f_i(b_i(t))$ \\
      \hline
      $1$ & $0$ & $2$ & $18$ & $2$ & $2$ & $3b_i(t)+6$\\
      $2$ & $1$ & $3$ & $3$ & $1$ & $2$ & $b_i(t)$\\
      \hline
    \end{tabularx}
  \end{center}
  
L'unique solution est décrite par la figure~\ref{figure_exemple_NE}.
  \begin{figure}[!htb]
    \centering
    \begin{tikzpicture}
      [xscale=2]
      \node (O) at (0,0) {} node[below=0.1cm] {$0$};
      \draw (1.5,0)  node[below=0.1cm] {$1.5$};
      \draw (3,0) node[below=0.1cm] {$3$};
      \node (T) at (3.5,0) {};
      \node at (0.75,1) {$1$};
      \node at (2.25,1) {$2$};
      \draw(0,2)  node[left,node
      distance=1.5pt] {$R=2$}; 
      \draw (O) rectangle (1.5,2);
      \draw (1.5,2) rectangle (3,0);
      \draw[->] (0,0) -- (T) node[below] {$t$};
      \draw[->] (0,0) -- (0,3) node[left] {$b(t)$};
    \end{tikzpicture}
    \caption{Exemple de solution non-entière pour une instance du
      \CECSP~à données entières.}
    \label{figure_exemple_NE}
  \end{figure}

Dans cette solution, la première activité doit finir au temps $t=1.5$
pour que la seconde activité puisse finir avant sa date échue
$\LE[2]=3$. En effet, l'activité $2$ doit commencer avant sa date de
début au plus tard, ici $\LS[2]= \LE[2] - W_i/f_i(\bmax) = 3 - 3/2=
1.5$, et l'activité $1$ ne peut finir avant sa date de fin au plus
tôt, $\EE[1]= \ES[1] + W_i / f_i(\bmax) = 0 + 18/12 = 1.5$. De plus,
l'activité $1$ consomme forcément $2$ unités de ressource durant son
exécution ($\bmin=\bmax=2$). Pour apporter exactement l'énergie
requise à l'activité $1$, il faut obligatoirement l'ordonnancer comme
sur la figure~\ref{figure_exemple_NE}.
\end{ex}

De ce fait, l'espace des solutions peut être réduit par l'utilisation
du modèle à temps discret mais ceci peut conduire à des infaisabilités ou
à des résultats sous-optimaux.

Une solution pour pallier ce problème est de mettre à l'échelle les
instances, i.e. multiplier les données par un certain coefficient $\alpha$
afin de s'assurer de l'existence d'une solution optimale entière,
avant de les résoudre. Cependant, l'utilisation d'un coefficient trop
grand peut conduire à une augmentation de la taille des modèles trop
importante pour permettre leur résolution. 

Le théorème suivant présente une des propriétés majeures du \CECSP. En
effet, il stipule que quelle que soit l'instance considérée, il existe
toujours une solution de cette instance où les fonctions $b_i(t)$ sont
constantes par morceaux, sous certaines conditions sur les fonctions de
rendement.  

\begin{theo}[\cite{Nattaf_CPDP}]
\label{theo_LPM_CECSP}
Soit $\Pi$ une instance réalisable du \CECSP~telle que: $\forall i \in
\A,\ f_i(0) =0$ et $f_i$ est croissante, continue, concave et affine
par morceaux. Une solution ayant la propriété que, $\forall i \in \A,\
b_i(t)$ soit constante par morceaux, existe.
\end{theo}

Afin de prouver le théorème~\ref{theo_LPM_CECSP}, nous commençons par
prouver l'affirmation suivante. Soit un intervalle $[t_1,t_2]$ et une
fonction d'allocation de ressource $b_i(t)$ non constante dans cet
intervalle, alors il existe une constante $b_{iq}$ pour laquelle
exécuter $i$ à $b_{iq}$ dans l'intervalle $[t_1,t_2]$,
i.e. $b'_i(t)=b_{iq},\ \forall t \in [t_1,t_2]$, apporte au moins
autant d'énergie tout en consommant la même quantité de ressource
qu'exécuter $i$ à $b_i(t)$ durant l'intervalle $[t_1,t_2]$. C'est ce
qu'affirme le lemme suivant:

\begin{lemma}
\label{lemmaEn}
Soit $b_{iq}= \frac{\int_{t_1}^{t_2}b_i(t)dt}{t_2-t_1}$. Alors, nous
avons:
\begin{align}
  &\int_{t_1}^{t_2}b_{iq}dt = \int_{t_1}^{t_2} b_i(t) dt \label{eq_LPM_res} \\
  & \int_{t_1}^{t_2}f_i(b_{iq})dt \ge \int_{t_1}^{t_2} f_i(b_i(t)) dt 
    \label{eq_LPM_nrj}
\end{align}
\end{lemma}

\begin{proof}
  L'équation~\eqref{eq_LPM_res} est trivialement vérifiée en remplaçant
  $b_{iq}$ par sa valeur. En effet, nous avons:
  \begin{align*}
    \int_{t_1}^{t_2}b_{iq}dt =&
    \int_{t_1}^{t_2}\left(\frac{\int_{t_1}^{t_2}b_i(t)dt}{t_2-t_1}\right)dt\\
    =& (t_2-t_1)\left(\frac{\int_{t_1}^{t_2}b_i(t)dt}{t_2-t_1}\right)\\
    =&\int_{t_1}^{t_2}b_i(t)dt
  \end{align*}

  Pour prouver que l'équation~\eqref{eq_LPM_nrj} est satisfaite,
  nous utilisons le théorème suivant:  
  \begin{theo}[\cite{Jensen}]
    Soit $\alpha(t)$ et $g(t)$ deux fonctions intégrables sur
    $[t_1,t_2] \subseteq \mathbb{R}$ telles que $\alpha(t) \ge 0,\
    \forall t \in [t_1,t_2]$. Alors, nous avons la propriété suivante: 
    \begin{equation}
      \phi\left( \frac{\int_{t_1}^{t_2} \alpha(t)g(t)dt }
        {\int_{t_1}^{t_2} \alpha(t)dt} \right) \ge
      \frac{\int_{t_1}^{t_2} \alpha(t)\phi(g(t))dt }
      {\int_{t_1}^{t_2} \alpha(t)dt}
    \end{equation}
    où $\phi$ est une fonction continue, concave sur $[\min_{t \in
      [t_1,t_2]} g(t),\max_{t \in [t_1,t_2]} g(t)]$. 
  \end{theo}
  Si nous remplaçons $\phi(t)$ par $f_i(t),\ g(t)$ par $b_i(t)$ et
  $\alpha(t)$ par la fonction constante égale à $1$, nous obtenons:
  \begin{align*}
    & f_i\left( \frac{\int_{t_1}^{t_2}b_i(t)dt }
      {t_2-t_1} \right) \ge
      \frac{\int_{t_1}^{t_2}f_i(b_i(t))dt }
      {t_2-t_1} \\
    & \Leftrightarrow (t_2-t_1)f_i\left( 
      b_{iq} \right) \ge
      \int_{t_1}^{t_2}f_i(b_i(t))dt\\
    & \Leftrightarrow \int_{t_1}^{t_2}f_i\left( 
      b_{iq} \right)dt \ge
      \int_{t_1}^{t_2}f_i(b_i(t))dt
  \end{align*}
Et donc, l'équation~\eqref{eq_LPM_nrj} est satisfaite.  
\end{proof}  

Nous pouvons maintenant prouver le théorème~\ref{theo_LPM_CECSP}. Pour
cela, nous allons montrer que, soit $S$ une solution d'une instance
$\Pi$, alors nous pouvons transformer $S$ en une solution $S'$ ayant
la propriété que chaque fonction $b'_i(t)$ est constante par morceaux.

\begin{proof}[Preuve du théorème~\ref{theo_LPM_CECSP}]  
  Soit $S$ une solution réalisable de $\Pi$ et soit
  $(t_q)_{q=1..Q}$ la suite des différentes dates de
  début et de fin d'activité triées par ordre croissant. Clairement,
  nous avons $Q\le 2n$. 

  Par souci de clarté, nous définissons la fonction intermédiaire
  $\tilde{b}_i(t),\ \forall i \in \A$, de la façon suivante:  

    \[\tilde{b}_i(t) =\left\{
        \begin{array}{lll}
          b_{i0} & & \text{si $t \in [t_0,t_1]$}\\
          \multicolumn{2}{c}{\vdots} &   \\
          b_{i(Q-1)} & & \text{si $t \in [t_{Q-1},t_Q]$}
        \end{array}
      \right.\]
    avec $b_{iq}=\frac{\int_{t_q}^{t_{q+1}} b_i(t) dt}{t_{q+1}-t_q}$.

    La solution $S'$ est alors construite de la manière suivante: 
    \begin{itemize}
    \item $st'_i=st_i$ 
    \item $et'_i=\min(\tau | \int_{st_i}^{\tau} f_i(\tilde{b}_i(t))dt=W_i)$
    \item $b'_i(t)= \left\{ 
        \begin{array}{lll}
          \tilde{b}_i(t) &\quad& \text{si $t \in [st_i,et'_i]$}\\
          0 &\quad& \text{sinon}\\
        \end{array}
      \right.$
    \end{itemize}

  Il est facile de voir que $S'$ satisfait les contraintes de fenêtres de
  temps~\eqref{tw_CECSP}, puisque, par le Lemme~\ref{lemmaEn},
  $et'_i\le et_i$. De plus, $S'$ vérifie la contrainte 
  d'énergie~\eqref{nrj_CECSP} puisqu'elle est définie de cette
  façon. Enfin, $S'$ vérifie aussi la contrainte de capacité de la
  ressource~\eqref{res_CECSP}. En effet, comme $S$ est une solution
  réalisable, nous avons $\forall q \in \{1,\dots,Q\}$ et $\forall t
  \in [t_q,t_{q+1}]$:  
  $\sum_{i\in \A}b_i(t) \le R \Rightarrow  
  \sum_{i\in \A} \int_{t_q}^{t_{q+1}} b_i(t)dt \le R(t_{q+1}-t_q)$.
 
  Donc, 
  \begin{align*}
    \sum_{i\in \A}b'_i(t) &\le 
                            \sum_{i\in \A} \tilde{b}_i(t)\\
                          &= 
                            \sum_{i\in \A} b_{iq}\\
                          &=
                            \sum_{i\in \A} \frac{\int_{t_q}^{t_{q+1}} b_i(t)dt}{t_{q+1}-t_q} \\
                          &\le R
  \end{align*}
  Nous pouvons montrer que $S'$ vérifie les contraintes de
  consommation minimale et maximale de la ressource d'une façon
  similaire. 
\end{proof}

Une remarque intéressante peut être faite à partir de la preuve du
théorème précédent. En effet, la nouvelle solution $S'$ possède la
propriété suivante: l'ensemble des points $t \in \H$ coïncidant avec
une variation de la consommation de ressource d'une activité $i$,
i.e. $\{t \in \H \ |\ \forall \epsilon>0,\ b_i(t) \neq b_i(t+\epsilon)\}$,
est contenue dans l'ensemble formé de toutes les dates de début et de
fin des activités. C'est ce qu'affirme le corollaire suivant: 

\begin{coro}[\cite{Nattaf_CPDP}]
$\{t \in \H \ |\ \forall \epsilon>0,\ b_i(t) \neq b_i(t+\epsilon)\}
\subseteq \{st_i,et_i\ |\ i \in \A \}$.
\end{coro}

De plus, nous pouvons en déduire que le \CECSP~à dates de début et de
fin fixées peut être résolu en temps polynomial. 

\begin{prop}[\cite{Nattaf_Constraints}]
Soit $\Pi$ une instance du \CECSP~avec des dates de début, $st_i$, et
des dates de fin, $et_i$, fixées. On peut vérifier que $\Pi$ est réalisable en temps 
polynomial en la taille de l'instance. 
\end{prop}

En effet, dans ce cas-là, il suffit de décider pour chaque intervalle
composé de deux dates de début/fin consécutives, i.e. de la forme
$[st_i,st_j],\ [st_i,et_j],\ [et_i,et_j]$ ou $[et_i,st_j]$, la
quantité de ressource consommée par chaque activité à l'intérieur de
cet intervalle. Ce problème peut facilement être modélisé par un
programme linéaire.

Soit $(t_q)_{q=1..Q}$ la suite définie dans la preuve du
théorème~\ref{theo_LPM_CECSP} et $b_{iq}$ (respectivement $w_{iq}$),
$\forall (i,q) \in \A\times\{1,\dots,Q-1\}$, la
quantité de ressource consommée par (resp. la quantité d'énergie
apportée à) l'activité $i$ dans l'intervalle
$[t_q,t_{q+1}]$. Rappelons que $Q \le 2n$. Le programme linéaire
s'écrit alors de la manière suivante:
{\small
\begin{align}
&\sum_{\i \in A} b_{iq} \le R(t_{q+1}-t_q)  & \forall q\in
\{1..Q-1\} \label{poly2}\\
& b_{iq} \le \bmax(t_{q+1}-t_q)  & \forall i \in \A,\ \forall q \in \{1..Q-1\} |\
t_q \in [st_i,et_i[\label{poly3}\\
& b_{iq} \ge \bmin (t_{q+1}-t_q) & \forall i \in \A,\ \forall q \in \{1..Q-1\} |\ t_q
\in [st_i,et_i[\label{poly4}\\
 & b_{iq}=0 & \forall i \in \A,\ \forall q \in \{1..Q-1\} |\ t_q
\not\in [st_i,et_i[\label{poly5}\\ 
& \sum_{q=1}^{Q-1} w_{iq} = W_i & \forall i \in
\A\label{poly6}\\
 & w_{iq} \le a_{ip}b_{iq} + c_{ip} & \forall i \in \A,\ \forall p \in
\P_i,\ \forall q \in \{1..Q-1\}\label{poly7}\\ 
& w_{iq} \le Mb_{iq} & \forall i \in \A,\ \forall q \in
\{1..Q-1\}\label{poly8}
\end{align} }
\noindent
pour $M$ une constante suffisamment grande et $\P_i=\{1,\dots,P_i\}$
le nombre d'intervalles de définition de la fonction $f_i$. La
contrainte~\eqref{poly2} modélise la contrainte de capacité de la
ressource. Les contraintes~\eqref{poly3} et~\eqref{poly4} assurent que
les contraintes de consommation minimale et maximale de la ressource
sont respectées tandis que la contrainte~\eqref{poly5} fixe la
consommation de la ressource à $0$ si l'activité n'est pas en
cours. La contrainte~\eqref{poly6} stipule que chaque activité doit
recevoir la quantité d'énergie requise. Enfin, les
contraintes~\eqref{poly7} et \eqref{poly8} assurent la conversion
ressource/énergie. De plus la contrainte~\eqref{poly8} fixe $w_{iq}$ à
$0$ si $b_{iq}=0$, i.e. modélise $f_i(0)=0$.

On peut remarquer que si $\forall i \in \A,\ \bmin=0$, alors le
\CECSP~devient polynomial. En effet, il suffit de prendre 
$(t_q)_{q=1..Q}$ la suite des différentes dates de début (resp. fin)
au plus tôt (resp. tard). Alors, le programme linéaire précédent nous
donne une solution réalisable. 

\begin{theo}[\cite{Nattaf_ORSpectrum}]
Le \CECSP~préemptif ($\forall i \in \A,\ \bmin=0$) peut être résolu en
temps polynomial.
\end{theo}

De ce fait, dans la suite, nous considérerons que $\exists i \in \A$
tel que $\bmin\neq 0$.


\section{Conclusion}

Dans ce chapitre, nous avons d'abord introduit les principales
caractéristiques des problèmes d'ordonnancement avant de nous intéresser
en particulier aux problèmes cumulatifs. Nous avons ensuite présenté
deux des principaux problèmes étudiés en ordonnancement sous
contraintes de ressource. Les limitations de ces problèmes en termes de
modélisation d'activités à profils variables ont ensuite
été démontrées et une nouvelle modélisation de la consommation de
ressource nous a permis de définir un nouveau problème: le
\CECSP. 

Dans un premier temps, nous avons comparé ce problème avec les
problèmes existant dans la littérature, puis nous avons présenté un
ensemble de propriétés qui va nous permettre, dans la suite de ce
manuscrit, de décrire des techniques pour sa résolution. La plupart de
ces techniques sont adaptées de techniques existantes et peuvent être
classées en deux catégories: 
\begin{itemize}
\item les techniques adaptées du \CUSP~et issues de la programmation
par contraintes. Ces techniques seront détaillées dans la
partie~\ref{part:PPC}.
\item les techniques adaptées du \RCPSP~et issues de la programmation
linéaire. Ces techniques seront détaillées dans la
partie~\ref{part:PLNE}.
\end{itemize}
\clearemptydoublepage%
\part{Programmation par contraintes}


\chapter{Programmation par
  contraintes et ordonnancement cumulatif}
\chaptermark{L'ordonnancement cumulatif en PPC}
\label{sec:PPC_CUSP}

 \section{La programmation par contraintes}

Cette section s'intéresse à la présentation des concept de base de la
programmation par contraintes (PPC). La programmation par contraintes
vise à résoudre des problèmes de satisfaction de contraintes (CSP)
mais aussi des problèmes d'optimisation (ce dernier cas ne sera pas
traité dans ce manuscrit). Pour cela, un problème est
modélisé à l'aide d'un réseau de contraintes et la recherche d'une
solution tend à trouver une affectation des variables satisfaisant
toutes les contraintes de ce réseau. Une présentation formelle des
problèmes de satisfaction de contraintes ainsi qu'un aperçu de
quelques méthodes permettant leur résolution est présentée dans les
paragraphes suivants. 

\subsection{Problème de satisfaction de contraintes}

Une instance d'une {\it problème de satisfaction de contraintes}, ou CSP est
la donnée d'un triplet $Q=(\X,\D,\C)$ où: 
\begin{itemize}
\item $\X=\{X_1,X_2,\dots,X_n\}$ est l'ensemble des variables du problème;
\item $\D=\{D_1,D_2,\dots,D_n\}$ est l'ensemble des domaines de ces
  variables, i.e. $X_i \in D_i,\ i=1,\dots,n$;
\item $\C=\{C_1,C_2,\dots,C_m\}$ définit l'ensemble des contraintes du
  problème où chaque $C_j$ définit un sous-ensemble du produit
  cartésien des domaines des variables sur lesquelles elle porte:
  $C_j(X_{j1},X_{j2},\dots,X_{jk}) \subseteq D_{j1} \times D_{j2}
  \times \dots \times D_{jk}$

La notion de domaine désigne l'ensemble des valeurs que peut prendre
une variable. La nature de ces domaines peut potentiellement être très
différente. Par exemple:
\begin{itemize}
\item un intervalle d'entiers;
\item un intervalle de réels;
\item un ensemble d'entiers non contigus: il est possible d'utiliser
  un ensemble d'entiers quelconque, e.g. $D=\{4,9,26\}$;
\item un ensemble de valeurs symboliques: on peut vouloir représenter
  des couleurs, ou encore des jours de la semaine... 
\end{itemize}
Formellement, une contrainte peut être définit comme une relation
définit sur un ensemble de variables. 

Une fois les variable définit, nous pouvons ajouter des contraintes
les liant entre elles. Ici aussi, plusieurs types de contraintes
existent. Nous détaillons trois d'entre elles. 

Le premières contraintes présentées sont les contraintes en {\it
  extension}. Pour ces contraintes, étant donnée un sous-ensemble de
variables, on définit explicitement la liste des tuples
autorisés. Suivant les cas, ces contraintes peuvent aussi être définit
comme une liste de tuples interdits. Le secondes contraintes
détaillées sont les contraintes en {\it intention}. Dans ce cas, les
contraintes sont décrites sous forme d'expression arithmétique
définissant  une relation entre les variables. Enfin, les dernières
contraintes décrites sont les contraintes globales. Ces contraintes
sont des relations prédéfinies, ayant une signification précise. Des
exemple de telles contraintes sont décrit dans
l'exemple~\ref{ex:contrainte}. Notons que si les variables sont à
valeurs dans $\mathbb{R}$ alors une définition en extension de la
contraintes peut comprendre un nombre infini de tuples.

\begin{ex}
  Soient trois variables $x_0,\ x_1$ et $x_2$ de domaine $D_0=[0,2],\
  D_1=[0,2]$ et $D_2=[1,2]$. 
  Nous considérons les contraintes $c_0,\ c_1$ et $c_2$ suivantes:
  \begin{itemize}
  \item $c_0$ est décrite en intention: $x_0 < x_1$;
  \item $c_1$ est décrite en extension: $\{(0,1) , (0,2), (1,2)\}$
  \item $c_2$ est une contrainte  globale portant sur les variable
    $x_0,x_1$ et $x_2$: allDifferent$(x_0,x_1,x_2)$. Cette contrainte
    stipule que les valeurs affectées à chaque variable soient
    différentes les une des autres. 
  \end{itemize}
  Les domaines des variables étant continus, un nombre infini de
tuples aurait été nécessaire pour écrire $c_0$ en extension.
\end{ex}

Pour trouver une solution à un CSP, il faut instancier toutes les
variables du problème de telles sorte que toutes les contraintes
soient satisfaites. Une variable est dite {\it instanciée} quand on
lui assigne une valeur de son domaine. Pour trouver une telle
instanciation, un certain nombres de techniques ont été mises en
place. Ces techniques reposent principalement sur deux éléments
centraux: le filtrage des domaines et l'exploration de l'espace de
recherche. Ces deux concepts sont donc décrits dans les paragraphes
suivants.

\subsection{Exploration de l'espace de recherche}
\label{sec:PPC_rech}

{\'E}tant donné un CSP, différentes techniques d'exploration de
l'espace de recherche peuvent être employées pour trouver des
solutions. Parmi elles, on trouve les techniques de recherche dites
systématiques. Celles-ci consistent à tester successivement toutes les
instanciations possibles des variables, jusqu'à trouver une solution
ou une incohérence remettant en cause la dernière instanciation
effectuée. Dans le second cas, on continue la recherche en supprimant
cette instanciations de l'espace de recherche. Ces combinaisons
peuvent être générées de différentes manière et donc testées dans des
ordres différents. 

Une représentation usuelle du processus de recherche d'une solution
est l'{\it arbre de recherche}. La racine de cette arbre représente le
problème que l'on cherche à résoudre et les sommets sont des problèmes
réduits, obtenus en décomposant le domaine d'une des variables du
problème père. Les feuilles de cet arbre correspondent donc à des
instanciations de toutes les variables du problèmes.

Une première approche alors à générer tous les tuples de valeurs
possibles et de tester s'ils sont solutions du problème, i.e. si cette
instanciation satisfait bien toutes les contraintes du problème.  Cela
revient à considérer toutes les affectations possibles des variables
et ce nombre, qui peut ne pas être fini dans le cas des variables
continues, est égal au produit cartésien du cardinal des domaines des
variables impliquées dans le CSP pour des variables dans
$\mathbb{Z}$.

La seconde méthode, le {\it backtracking}, cherche à étendre
progressivement une solution partielle en instanciant, à chaque
étape, une nouvelle variable à une valeur de son domaine. Si la
nouvelle instanciation partielle ainsi obtenue n'est pas une solution
partielle, on effectue un {\it backtrack} sur $X_i$, i.e. $X_i$ est
instanciée à une autre valeur de son domaine. Dans le cas où le
domaine de $X_i$ ne contient plus de valeurs non testées, alors il y a
deux possibilités: les domaine de toutes les autres variables sont
vides et, dans ce cas, le problème est insatisfiable; sinon, on
effectue un nouveau backtrack sur la dernière variable instanciée du
problème. Le processus se poursuit ainsi jusqu'à ce qu'une solution
complète soit trouvée. 

Dans le cas de CSP comprenant des variables continues, l'énumération
de tous les domaines des variables n'est pas possible. Une première
approche serait de considérer qu'une variable est instanciée quand son
domaine est réduit à un intervalle de taille suffisamment petite. Mais
même avec cette restriction supplémentaire, il est très coûteux
d'énumérer chacun de ces intervalles. L'idée est alors la suivante: au
lieu, à chaque étape de l'algorithme de backtracking, d'instancier une
variable à une valeur de son domaine, nous séparons le domaine de
cette variable en deux (ou plusieurs) sous-domaines.

consiste alors à parcourir l'arbre de recherche
jusqu'à arriver à une feuille et si cette feuille ne correspond pas à
une instanciation valide, alors l'exploration d'autre branche de
l'arbre de recherche est alors nécessaire. Si, au contraire,
l'instanciation des variables correspondant à cette feuille satisfait
toutes les contraintes du problème, alors une solution a été trouvée
et l'algorithme peut s'arrêter. Cependant, en pratique, une telle
méthode ne peut être appliquée. En effet, dans le pire des cas, il
faudrait parcourir tout l'arbre de recherche. Cela revient à
considérer toutes les affectations possibles des variables et ce
nombre, qui peut ne pas être fini dans le cas des variables continues,
est de l'ordre de $O(|\X|^{|\D|})$ pour des variables dans
$\mathbb{Z}$.

Plusieurs techniques de séparation du problème en sous-problème, aussi
appelées techniques de {\it branchement}, existent. Parmi les plus
classiques on trouve:
\begin{itemize}
\item l'affectation d'une valeur du domaine à une variable pour la
  première branche et la suppression de cette valeur du domaine pour
  la seconde branche. L'arbre de recherche est alors un arbre binaire.
\item la séparation du domaine en plusieurs sous-domaine, e.g. si,
dans un problème, une variable $x$ a le domaine $[0,6]$ alors on peut
crée $3$ sous-problèmes où $x$ a pour domaine respectif $[0,2[$,
$[2,4[$ et $[4,6[$. 
\end{itemize}

\begin{ex}
Soient $3$ variables, $x,\ y$ et $z$, de domaine
respectifs: $\{1,2\},\ \{1,2\},\ \{1,2,3\}$. Nous considérons la
contrainte AllDifferent($x,y,z$): chaque variable doit prendre une
valeur différente.   
  \begin{figure}[!htb]
    \centering
    \begin{tikzpicture}
      \node (O) at (0,0) {};    
    \end{tikzpicture}
    \caption{Exemple de parcours d'un arbre de recherche}
    \label{fig:ex_tree}
  \end{figure}
\end{ex}


\subsection{Filtrage et propagation}
\label{sec:PPC_propag}

Pour accélérer le processus de résolution, des techniques de filtrages 


 \section{L'ordonnancement cumulatif}
\label{sec:cumu}

\subsection{L'ordonnancement en programmation par contraintes}
\label{sec:cumu_ordo}

Les problèmes d'ordonnancement étudiés en programmation par
contraintes sont majoritairement regroupés en trois catégories: les
problème préemptifs, les problèmes non préemptifs et les problèmes
élastiques. Les problèmes préemptifs, respectivement non préemptifs,
sont des problèmes pour lesquels la préemption des activités est
autorisée, respectivement non autorisée. Une définition d'une activité
préemptive peut être trouvée dans le paragraphe~\ref{sec:ordo}. Les
{\it problèmes d'ordonnancement élastiques} correspondent aux
problèmes où la quantité de ressource attribuée à une activité peut
varier, à tout instant $t$ de l'horizon de temps, avec la contrainte
que la quantité totale de ressource consommée par l'activité durant
son exécution soit égale à une certaine valeur appelée {\it
  énergie}. Cette dernière notion peut aisément être étendue dans le
cas où l'énergie reçue par une activité n'est pas égale à la quantité
de ressource consommée mais où des fonctions de rendement modélisent 
cette conversion. Clairement, le \CECSP~est un exemple typique d'un
tel problème. Dans la suite de ce chapitre, nous considérons la notion
d'énergie au sens de la quantité totale de ressource
consommée. L'extension de cette notion pour prendre en considération
les fonctions de rendement sera détaillée dans le
chapitre~\ref{sec:PPC_CECSP}.

La majorité des problèmes d'ordonnancement non-préemptifs classiques
peuvent être modélisés à l'aide d'un problème de satisfaction de
contraintes. En général, trois variables représentant respectivement
la date de début d'une activité, notée $st_i$, sa date de fin, notée
$et_i$ et sa durée, notée $p_i$, sont définies. Les domaines de
chacune de ces variables sont définis par les données du problème. En
effet, pour chaque activité, nous pouvons calculer une date de début
au plus tôt, $\ES$, et au plus tard, $\LS$; ainsi, le domaine de la
variable $st_i$ est $[\ES,\LS]$. De même, le domaine de la variable
$et_i$ est $[\EE,\LE]$, avec $\EE$ la date de fin au plus tôt de $i$
et $\LE$ sa date de fin au plus tard. La durée d'une activité est
quant à elle définie comme la différence entre sa date de fin et sa
date de début, i.e. $p_i=et_i-st_i$.

Les problèmes préemptifs sont plus difficiles à modéliser. En effet,
dans ce cas-là, un ordonnancement valide ne peut seulement être
représenté par une date de début, de fin et une durée pour chaque
activité. Pour ces problèmes, au moins deux modélisations différentes
existent. La première consiste à associer à chaque activité une
variable d'ensemble, i.e. une variable dont la valeur sera un
ensemble. Cet variable représente l'ensemble des instants $t$ où
l'activité est en cours, définie comme un ensemble d'intervalles ou de
temps $t$ discret. Une second possibilité est de définir une variable
binaire, pour chaque activité et chaque instant $t$, qui prendra la
valeur $1$ si l'activité est en cours à l'instant $t$. Notons que,
dans le cas d'un problème d'ordonnancement continu, une telle solution
ne peut être envisageable car cela conduirait à un nombre infini de
telles variables.

Les problèmes élastiques sont, quant à eux, modéliser à travers les
contraintes de ressources. Il existe deux principaux types de
contraintes de ressource en PPC. Le premier permet de modéliser les
ressources disjonctives, i.e. qui ne peuvent exécuter qu'une activité
en parallèle, et le second sert à la modélisation des ressources
cumulatives. Dans ce manuscrit, nous nous intéressons seulement au
second cas. 

{\'E}tant donné une activité et une ressource de capacité $R$, une
variable $b_i$ sert généralement à modéliser la consommation de
l'activité sur cette ressource. Dans le cas des tâches élastiques,
cette variable est remplacée par une variable $W_i$ représentant
l'énergie requise par l'activité. Pour représenter un ordonnancement,
un ensemble de variables représentant la quantité de ressource
utilisée par une activité à l'instant $t$ est introduit. Ces variables
sont ensuite liés entre elles par un ensemble de contraintes
modélisant le fait que chaque activité doit recevoir une énergie
$W_i$. 

Ces différents concepts peuvent être étendus pour modéliser d'autre
types de contraintes telles que des contraintes de ressources
alternatives, des temps de préparation entre les activités, des
activités optionnelles ou des contraintes de réservoirs.

Dans la suite de ce manuscrit, nous nous intéressons aux problèmes
d'ordonnancement non-préemptifs avec ressource cumulative et aux
problèmes d'ordonnancement élastiques. Le premier cas correspond au
\CUSP, décrit dans le paragraphe~\ref{sec:ordo_res}. En PPC, le
problème est modélisé à l'aide d'une contrainte globale appelées
contrainte cumulative. Cette contrainte ainsi que différents
algorithmes de filtrage mis en place pour cette dernière sont
présentés dans les deux paragraphes suivant. 

Les problèmes d'ordonnancement élastiques seront représentés par le
\CECSP~(voir paragraphe~\ref{sec:ordo_nrj}), pour lequel une partie
des algorithmes de filtrage pour la contrainte cumulative sont adaptés
et un modèle de PPC dans le cas de la restriction discrète du
\CECSP~seront présentés dans le chapitre~\ref{sec:PPC_CECSP}.

\subsection{La contrainte cumulative}
\label{sec:cumu_cume}

En PPC, le \CUSP~est modélisé par la contrainte cumulative. Dans ce
contexte, une activité est souvent représentée par $4$ variables:
$st_i,\ et_i,\ p_i$ et $b_i$ correspondant respectivement à la date de
début de l'activité, sa date de fin, sa durée et sa consommation de
ressource. En règle générale, les domaines des variables $p_i$ et
$b_i$ sont restreint à un seul élément, i.e. la durée de l'activité
est fixe et sa consommation de ressource est constante durant toute
son exécution. 

La contrainte cumulative vise donc à déterminer seulement les dates de
début et de fin des activités -- liés par la condition d'intégrité
$et_i=st_i+p_i$. Cette contrainte prend donc en paramètre l'ensemble
d'activités $\A$ à ordonnancer et la capacité $R$ de la ressource utilisée
pour exécuter les activités. Avec ces notations la contrainte s'écrit
$cumulative(\A,R)$ et cette contrainte est satisfaites si et seulement 
si:
\begin{align}
  \forall i \in \A, & \quad  st_i+p_i=et_i \label{eq:CUSP_int}\\
  \forall t \in \H, &\quad \sum_{\substack{i \in A\\t \in
  [st_i,et_i[}} b_i \le R  \tag{\ref{eq:CUSP_res}} 
\end{align}
La seconde contrainte ainsi qu'un exemple de solution sont détaillés
dans le paragraphe~\ref{sec:ordo_res}.

L'existence d'une solution au problème cumulatif étant NP-complet, un
algorithme effectuant un filtrage complet des domaines des variables,
i.e. retirant toutes les valeurs incohérentes des domaine des
variables, ne peut s'exécuter en temps polynomial. De ce fait, les
algorithmes de filtrage se limite souvent à s'assurer de la cohérence
des bornes des domaines des variables. Ces algorithmes se présentent
le plus souvent sous la forme de règles permettant d'ajuster ces
bornes en fonction des données et contraintes du problème,
i.e. les {\it règles d'ajustement}. Malgré tout, ces techniques
restent encore trop coûteuse en temps pour être utilisées directement
sur la contrainte cumulative.  Différentes relaxations de la
contrainte cumulative ont donc été introduites et des règles
d'ajustement pouvant être appliquées en temps polynomial ont été mis
en place pour ces relaxations.

Dans ce manuscrit, nous présentons quelques une de ces
techniques. Pour une description plus détaillée des différentes
techniques de relaxation existantes pour la contrainte cumulative,
nous renvoyons le lecteur à~\cite{BLPN,DHP}. 
Baptiste {\it et al.}~\cite{BLPN} présentent un panorama de ces
différentes relaxations ainsi que des principales techniques issues de
la programmation par contraintes mis en place pour résoudre la
contrainte cumulative. Dans un contexte plus général, Dornoff {\it et
al.}~\cite{DHP} exhibent des règles de cohérence basées sur la
capacité des intervalles, i.e. la quantité de ressource disponible
dans un intervalle.

Le paragraphe suivant présente plusieurs des règles d'ajustement
définies pour la contrainte cumulative. 


\subsection{Les filtrages de la contrainte cumulative}
\label{sec:cumu_propag}

Le problème cumulatif étant un problème symétrique, les règles
d'ajustement sont souvent définie pour une seule des deux variables
$st_i$ et $et_i$. En effet, une fois ces règles définies pour une de
ces deux variables, les même règles peuvent être définies dans l'autre
cas de manière vraiment similaire. De ce fait, les raisonnements et
algorithmes décrit dans ce paragraphe ne présentent que le filtrage du
début au plus tôt.

Dans un premier temps, nous présentons des règles simples permettant
l'ajustement des bornes des variables. Ensuite, une partie sera
consacrée aux règles d'ajustement basée sur le concept
d'énergie. Enfin, ces règles seront ensuite combinées afin de créer
des algorithmes permettant un filtrage plus fort des domaines des
variables.

Plusieurs des règles présentées dans cette section seront ensuite
adaptées dans le cas du \CECSP~dans le chapitre~\ref{sec:PPC_CECSP}.

\subsubsection{Règles de filtrages simples}

Parmi les raisonnements les plus simples appliqués dans le cadre de la
contrainte cumulative, on retrouve le Time-Table~\cite{TTLah} et le
raisonnement disjonctif~\cite{BLPN}.  

\paragraph{Time-Table}
Le premier algorithme de filtrage de la contrainte cumulative
présenté est basé sur la notion de {\it partie obligatoire} d'une
activité. Cette partie obligatoire correspond à l'intervalle de temps
maximal pendant lequel on est sûr que l'activité devra être
exécutée. Cet intervalle est vide dans le cas où $\LS \ge \EE$, et est
égal à $[\LS,\EE{[}$ dans le cas contraire. Notons que dans le cas de la
contrainte cumulative, une borne supérieure sur la date de début d'une
activité peut aisément être calculée. En effet, il suffit de prendre
$\LS=\LE - p_i$. 


\begin{defi}
La partie obligatoire d'une activité $i$ est l'intervalle $[\LS,\EE{[}$.
\end{defi}

\begin{ex}
Considérons l'activité suivante: 

\vspace{-0.5cm}
\begin{center}
  \begin{tabular}{|P{1cm}P{1cm}P{1cm}P{1cm}|}
    \hline
    \ES & \LE & p_i & b_i  \\
    \hline
    1 & 13 & 8 & 2 \\
    \hline
  \end{tabular}
\end{center}

Nous pouvons calculer sa date de début au plus tard,
$\LS=\LE - p_i=13-8=5$, ainsi que sa date de fin au
plus tôt, $\EE=\ES + p_i=1+8=9$. Comme $\EE > \LS$,
l'activité possède une partie obligatoire qui est l'intervalle $[5,9[$
(voir figure~\ref{fig_mand_CUSP}).
  
\begin{figure}[htb!]
\subcaptionbox{Ordonnancement au plus tôt}[0.3\linewidth]{
    \begin{tikzpicture}
      [xscale=0.25, yscale= 0.4,node distance=0.5cm]
      \node (sil) at (1,0) {} ;
      \node (eil) at (9,0) {} ;
      \node [below of=eil,node distance=0.63cm]  {$\EE$};
      \draw (sil.center) node[below=0.2cm] {$\ES$};
   \node (sir) at (5,0) {} ;
      \node (eir) at (13,0) {} ;
      \node[below of= sir,node distance=0.63cm] {$\LS$};
      \draw (eir.center) node[below=0.2cm] {$\LE$};
      
      \draw (0,0) -- (14,0);
      \draw[line width=3pt] (1,0) -- (1,3);
      
      \draw[densely dotted] (9.05,0) -- (9.05,3);
     \draw[densely dotted] (4.95,0) -- (4.95,3);
      \draw[<->] (0,0.1) -- (0,2) node[midway,left] {$b_i$};
      \draw[fill=white] (1,0) rectangle (8.95,2) node[midway] {$i$};

      \foreach \i in {0,...,14} {
        \draw (\i,0)  -- (\i,-0.2);
      }
    \end{tikzpicture}
}
\hfill
\subcaptionbox{Ordonnancement au plus tard}[0.3\linewidth]{
    \begin{tikzpicture}
      [xscale=0.25, yscale= 0.4,node distance=0.5cm]
        \node (sil) at (1,0) {} ;
      \node (eil) at (9,0) {} ;
      \node [below of=eil,node distance=0.63cm]  {$\EE$};
      \draw (sil.center) node[below=0.2cm] {$\ES$};
   \node (sir) at (5,0) {} ;
      \node (eir) at (13,0) {} ;
      \node[below of= sir,node distance=0.63cm] {$\LS$};
      \draw (eir.center) node[below=0.2cm] {$\LE$};
      
      \draw (0,0) -- (14,0);
      \draw[line width=3pt] (13,0) -- (13,3);
      
      \draw[densely dotted] (4.95,0) -- (4.95,3);
      \draw[densely dotted] (9.05,0) -- (9.05,3);
      \draw[<->] (14,0.1) -- (14,2) node[midway,right] {$b_i$};
      \draw[fill=white] (5.05,0) rectangle (13,2) node[midway] {$i$};

      \foreach \i in {0,...,14} {
        \draw (\i,0)  -- (\i,-0.2);
      }
    \end{tikzpicture}
}
\hfill
\subcaptionbox{Partie obligatoire}[0.3\linewidth]{ 
  \begin{tikzpicture}
    [xscale=0.25, yscale= 0.4,node distance=0.5cm]
    \node (sir) at (5,0) {} ;
    \node (eil) at (9,0) {} ;
    \node [below of=eil,node distance=0.63cm]  {$\EE$};
    \node[below of= sir,node distance=0.63cm] {$\LS$};
    \draw[<->] (5,2.2) -- (9,2.2) node[midway,label={}]
    {};
    
    \draw (5.05,0) rectangle (8.95,2)
    node[midway,color=black,align=center,text width=1.4cm]
    {\scriptsize partie oblig.}; 
    
    
    \draw (0,0) -- (14,0);
    \draw[line width=3pt] (1,0) -- (1,3);
    \draw[line width=3pt] (13,0) -- (13,3);
    
    \draw[<->] (-0.1,0.1) -- (-0.1,2.4) node[midway,left] {$b_i$};

    \draw[densely dotted] (4.95,0) -- (4.95,3);
    \draw[densely dotted] (9.05 ,0) -- (9.05,3);

    \foreach \i in {0,...,14} {
      \draw (\i,0)  -- (\i,-0.2);
    }
  \end{tikzpicture}
}
\caption{Partie obligatoire d'une activité $i$}
\label{fig_mand_CUSP}
\end{figure}
\end{ex}

Le profil obligatoire des activités peut ensuite être agrégé de
manière à obtenir une fonction en escalier, appelée {\it profil
  obligatoire de la ressource}.  

\begin{defi}
\label{def:profil_oblig}
  Le {\bf profil obligatoire} d'une ressource $TT_{\cal A}$ est définie
  par la fonction suivante: $TT_{\cal A}(t)= \sum_{\substack{i \in {\cal
        A}\\ \LS \le t < \EE}} b_i ,\ \forall t \in \H$.  Le problème n'admet
  pas de solution si $\exists t \in \H \ : \ TT_{\cal A}(t) > R$.
\end{defi}


\begin{ex}
  \begin{figure}[!htb]
    \centering
    \begin{tikzpicture}
      [yscale=0.45,xscale=0.6]
      \node (O) at (0,0) {};
      \foreach \i in {0,...,13} {
        \draw (\i-1,0) -- (\i-1,-0.1) node[below] {$\i$};
      }
      
      \draw (0,0) -- (0,2) -- (2,2) -- (2,4) -- (4,4) -- (4,6) --
      (5,6) -- (5,1) -- (9,1) -- (9,3) -- (11,3) -- (11,0);
      \draw[->] (-1,0) -- (12.5,0);
      \draw[dashed] (-1,5) -- (12.5,5);
      \draw[->] (-1,0) -- (-1,5.5) node[left] {$R=5$};
    \end{tikzpicture}
    \caption{Exemple de profil obligatoire d'une ressource}
    \label{fig_profil_CUSP}
  \end{figure}
  Dans l'exemple décrit par la figure~\ref{fig_profil_CUSP}, nous
  remarquons que, au temps $t=1$, le profil obligatoire de la
  ressource vaut $2$. Comme la ressource est de capacité $5$, aucune
  incohérence n'est détectée. À l'inverse, au temps $t=5$, le profil
  obligatoire de la ressource a une valeur de $6$, ce qui est
  supérieur à la capacité de la ressource. Une incohérence est donc
  détectée et il n'existe pas de solution réalisable pour cette
  instance. 
\end{ex}

Pour réaliser un filtrage des bornes, nous devons fournir des règles
permettant d'identifier si débuter l'activité $i$ à $\ES$ (ou $\LS$)
ne viole pas les contraintes du problème. Cette règle peut être
formalisée de la façon suivante: 

\begin{reg}
\label{reg:TT}
Soit une activité $i \in \A$. S'il existe $t \in [\ES,\LS{[}$ tel que
$t < \EE$ et que $b_i + TT_{\A \setminus\{i\}}(t) > R$ alors la date
de début au plus tôt de $i$ peut être ajustée et on a: $ \ES > t$.
\end{reg}

\begin{ex}
 Soit l'activité définie ci-dessous.
\begin{center}
  \begin{tabular}{|P{1cm}P{1cm}P{1cm}P{1cm}|}
    \hline
    \ES & \LE & p_i & b_i  \\
    \hline
    1 & 14 & 8 & 2 \\
    \hline
  \end{tabular}
\end{center}
La figure~\ref{fig:TT_CUSP} présente le profil obligatoire de la
ressource ainsi que l'ajustement déduit de la règle~\ref{reg:TT}.
  \begin{figure}[!htb]
    \centering
    \begin{tikzpicture}
      [yscale=0.45,xscale=0.6]
      \node (O) at (0,0) {};
      \foreach \i in {0,2,...,13} {
        \draw (\i-1,0) -- (\i-1,-0.1) node[below] {$\i$};
      }
      
      \fill[draw=black,gray!50] (0,5) rectangle (8,3) node[midway,above,color=black]
      {$st_i=\ES$};
      \draw (0,0) -- (0,2) -- (2,2) -- (2,4) -- 
      (5,4) -- (5,1) -- (9,1) -- (9,3) -- (11,3) -- (11,0);
      \draw[->] (-1,0) -- (12.5,0);
      \draw[dashed] (-1,5) -- (12.5,5);
      \draw[->] (-1,0) -- (-1,5.5) node[left] {$R=5$};
      \draw[very thick=3pt] (0,-0.1) node[below=0.4cm] {$\ES$}-- (0,5.5);
      \draw[very thick=3pt] (5,-0.1) node[below=0.3cm] {$\ES^{'}$}-- (5,5.5);
      \draw[->] (0.8,-1.8) -- (4.2,-1.8);
    \end{tikzpicture}
    \caption{Règle d'ajustement du Time-Table}
    \label{fig:TT_CUSP}
  \end{figure}
Dans cet exemple, on voit bien que l'activité ne peut pas commencé au
temps $t=\ES=1$. En effet, le profil obligatoire de la ressource
montre que la quantité de ressource disponible dans l'intervalle
$[3,6[$ ne permet pas d'exécuter l'activité pendant cet intervalle. La
date de début au plus tôt peut donc être mise à jour ($\ES=6$).
\end{ex}

Beldiceanu {\it et al.}~\cite{BC} ont introduit un algorithme de
balayage permettant d'appliquer les ajustements décrits ci-dessus en
$O(n^2)$. D'autres algorithmes ont aussi été mis en place pour ce
raisonnement~\cite{OQ,LBC,GHS}. Ces algorithmes possèdent des
complexités théoriques moins élevée ou équivalente avec l'algorithme
de balayage de Beldiceanu {\it et al.} en proposant des filtrages
moins importants.

\paragraph{Raisonnement disjonctif}
Une deuxième règle permettant d'ajuster les bornes des variables
repose sur un raisonnement appelé le {\it raisonnement disjonctif}. Ce
raisonnement est brièvement décrit dans~\cite{BLPN} et présenté plus
en détail dans~\cite{Gay2015}. 

Ce raisonnement repose sur le concept d'ensembles disjonctifs. Ces
ensembles correspondent aux ensembles d'activités qui ne peuvent
s'exécuter en parallèle. Son application la plus basique s'intéresse
seulement aux ensembles de taille $2$, c'est-à-dire aux paires
d'activités disjonctives. 

Si nous considérons deux activités $(i,j) \in \A^2$ telles que
$b_i+b_j > R$ alors une des deux affirmations suivantes doit être
vérifiée:
\begin{itemize}
\item l'activité $i$ doit commencer après la fin de l'activité $j$;
\item l'activité $i$ doit finir avant le début de l'activité $j$.
\end{itemize}
En effet, comme les activités $i$ et $j$ ne peuvent être exécutée en
parallèle, l'une doit forcément finir avant que l'autre ne puisse
s'exécuter.

Cette propriété nous permet d'améliorer les bornes des variables de
début d'activité selon la règle suivante:
\begin{reg}
  Soient $i,\ j \in \A$, $i\neq j$ telles que $b_i+b_j > R$ et $\LS
  <\EE[j]$. Alors la date de début au plus tôt de l'activité peut
  être ajustée et on a: $ \ES[j] \ge \EE$. 
\end{reg}

\begin{ex}
\label{ex:disj_CUSP}
Soient $i$ et $j$ les deux activités suivantes: 
\begin{center}
  \begin{tabular}{|P{1cm}|P{1cm}P{1cm}P{1cm}P{1cm}|}
    \hline
    act & \ES & \LE & p_i & b_i  \\
    \hline
   i & 2 & 11 & 4 & 2 \\
   j & 1 & 20 & 7 & 2 \\    
    \hline
  \end{tabular}
\end{center}

  \begin{figure}[htb!]
    \centering
    \begin{tikzpicture}
     \begin{scope} [yscale=0.45,xscale=0.4]
      \node (O) at (0,0) {};
      \foreach \i in {0,5,...,10} {
        \draw (\i,0) -- (\i,-0.1) node[below] {\small $\i$};
      }
      \fill[gray!50] (2,0) rectangle (11,2.4);
      \fill[gray!50] (1,2.6) rectangle (14,5);
      
      \draw[fill=white] (7,0.2) rectangle (11,2.2) node[midway]
      {$i$};
      \draw[fill=white] (1,2.8) rectangle (8,4.8) node[midway]
      {$j$};
      \draw[white, pattern=north west lines] (7,0) rectangle (8,5.5);

      \draw[->] (0,0) -- (14,0);
      \draw[->] (0,0) -- (0,5.5) ;
      \draw (0,3) node[left] {$R=3$};
      \draw[densely dotted] (1,-0.1) -- (1,5.5) node[above] {$\ES[j]$};
      \draw[densely dotted] (8,-0.1) -- (8,5.5) node[above] {$\EE[j]$};
      \draw[densely dotted] (2,-0.1)  node[below] {$\ES$}-- (2,5.5);
      \draw[densely dotted] (6,-0.1 ) node[below=0.4cm] {$\EE$}-- (6,5.5) ;
      \draw[densely dotted] (7,-0.1) node[below] {$\LS$} -- (7,5.5) ;
      \draw[densely dotted] (11,-0.1) node[below right] {$\LE$} -- (11,5.5) ;
      % \draw[densely dotted] (6,-0.1) -- (6,5.5) node[above] {$\ES[j]^{'}$};
      % \draw[->] (1.8,5.8) -- (5.2,5.8);
    \end{scope}     
    \begin{scope} [yscale=0.45,xscale=0.4,xshift=20cm]
      \node (O) at (0,0) {};
      \foreach \i in {0,5,...,10} {
        \draw (\i,0) -- (\i,-0.1) node[below] {\small $\i$};
      }
      \fill[gray!50] (2,0) rectangle (11,2.4);
      \fill[gray!50] (6,2.6) rectangle (14,5);

      
      \draw[fill=white] (2,0.2) rectangle (6,2.2) node[midway]
      {$i$};
      \draw[fill=white] (6,2.8) rectangle (13,4.8) node[midway]
      {$j$};

      \draw[->] (0,0) -- (14,0);
      \draw[->] (0,0) -- (0,5.5) ;
      \draw (0,3) node[left] {$R=3$};
      \draw[densely dotted] (6,-0.1) -- (6,5.5) node[above] {$\ES[j]$};
      \draw[densely dotted] (13,-0.1) -- (13,5.5) node[above] {$\EE[j]$};
      \draw[densely dotted] (2,-0.1)  node[below] {$\ES$}-- (2,5.5);
      \draw[densely dotted] (6,-0.1 ) node[below=0.4cm] {$\EE$}-- (6,5.5) ;
      \draw[densely dotted] (7,-0.1) node[below] {$\LS$} -- (7,5.5) ;
      \draw[densely dotted] (11,-0.1) node[below right] {$\LE$} -- (11,5.5) ;
      % \draw[densely dotted] (6,-0.1) -- (6,5.5) node[above] {$\ES[j]^{'}$};
      % \draw[->] (1.8,5.8) -- (5.2,5.8);
    \end{scope}
  \end{tikzpicture}
  \caption{Raisonnement disjonctif}
  \label{fig:disj_CUSP}
\end{figure}
Dans l'exemple ci-dessus, nous avons $b_i+b_j=4 > 3$. Donc $i$ et $j$
ne peuvent s'exécuter en parallèle. Comme $\EE[j] = 8 > 7=\LS$, $i$
doit forcément être exécuter avant $j$. En effet, $i$ doit forcément
démarrer avant $\LS=7$ et $j$ ne peut finir avant $\EE=8$. Donc $j$ ne
peut commencer avant $\EE=6$.
\end{ex}

L'exemple~\ref{ex:disj_CUSP} montre que, dans certains cas, le
raisonnement disjonctif est plus fort que le Time-Table. En effet,
dans cette exemple, aucune des activités ne possèdent de partie
obligatoire. Aucun ajustement n'est donc détecté par la
règle~\ref{reg:TT}. 

{\`A} l'inverse, dans certains cas, le raisonnement Time-Table va
procéder à plus d'ajustements que le raisonnement disjonctif. En
effet, le raisonnement disjonctif présenté ici ne raisonne que sur des
paires d'activités tandis que le Time-Table est capable de détecter
des ajustements déduits de n-uplets d'activités. Il faut cependant que
les activités possèdent une partie obligatoire.

Dans le paragraphe~\ref{sec:mix_CUSP}, nous montrerons comme il est
possible de coupler ces deux raisonnements pour en obtenir un nouveau,
le Time-Table disjonctif. 

Le paragraphe suivant est dédié aux règles de filtrage utilisant le
concept d'{\it énergie}. 
 
\subsubsection{Règles de filtrage énergétiques}
 \label{sec:nrj_CUSP}
Parmi les règles de filtrage utilisant le concept d'énergie on
retrouve le Edge-Finding\cite{VilimEF,theseNuijten}, le raisonnement
énergétique~\cite{RELopez} ou encore les activités
élastiques~\cite{BLPN}.  Nous présenterons en détail le second cas,
tandis que les autres cas seront brièvement discutés. La raison
principale étant que nous adapterons le raisonnement énergétique dans
le cas du \CECSP.

L'idée principale des raisonnements énergétique est de comparer la
quantité de ressource disponible dans un intervalle avec l'énergie que
doit consommer un ensemble de tâches à l'intérieur de ce même
intervalle. Les raisonnements diffèrent principalement sur la manière
de calculer cette énergie et sur les intervalles considérés. 

 Le raisonnement Edge-Finding s'applique sur un sous-ensemble
d'activités $\Omega \subseteq \A$, le but étant de décider si une
activité $j$ doit commencer avant $\Omega$ dans toute solution
réalisable. Dans ce cas, l'énergie d'une activité $i$ représente sa
charge de travail $W_i$ et est simplement exprimée comme
$W_i=p_i\times b_i$. L'énergie nécessaire pour un ensemble d'activité
$\Omega$ est alors $\sum_{i\in \Omega} W_i$. Les intervalles
considérés dans ce raisonnement correspondent au fenêtre de temps des
activités $[\ES,\LE]$. Cette notion est étendue à $\Omega$ de la façon
suivante: $[\ES[\Omega],\LE[\Omega]=[\min_{j \in \Omega}\ES[j],\max_{j
\in \Omega} \LE[j] ]$. L'idée du raisonnement est alors  de vérifier
que la ressource disponible permet de faire débuter l'activité $i$
dans la fenêtre de temps correspondant à $\Omega$. Pour cela,
l'énergie requise par $\Omega \cup \{ i\}$ est comparée à la quantité
de ressource disponible dans l'intervalle $[\ES[\Omega],\LE[\Omega\cup
\{i\}]]$. Si la quantité de ressource n'est pas suffisante, alors
l'activité $i$ doit commencer avant $\ES[\Omega]$ et un ajustement
peut être effectué.

Dans le cas des activités partiellement et totalement élastiques, les
contraintes de consommation de ressource sont relâchées et une
activité n'est plus contrainte de consommer une quantité constante de
la ressource au cours du temps. Suivant que l'activité est totalement
ou partiellement élastique, ces contraintes sont plus ou moins
relâchées et la consommation de ressource des activités peut subir des
variations plus ou moins grande au cours du temps. L'avantage de ces
relaxations est que l'existence d'une solution peut être décider en
temps polynomial. La façon dont les ajustements sont calculés étant
très similaire à celle dont ils sont calculés dans le cadre du
raisonnement énergétique, ils ne sont pas détaillés dans ce
manuscrit. Une description de ces ajustements peut être trouvé
dans~\cite{BLN}. 

Le raisonnement énergétique, introduit par~\cite{RELopez}, est un des
raisonnements les plus forts existant pour la contrainte
cumulative. Le principe est de considérer un intervalle $[t_1,t_2[$ et
de calculer l'énergie requise par une activité à l'intérieur de cet
intervalle. Dans ce cas, l'énergie requise par une activité dans un
intervalle $[t_1,t_2[$  équivaut à
la quantité minimale de ressource qu'elle doit utiliser
à l'intérieur de cet intervalle. Cette quantité, notée $\bb$, est
ensuite utilisée pour calculer la fonction de marge $SL(t_1-t_2)$
correspondant à la différence entre la quantité de ressource requise
par toutes les activités à l'intérieur de $[t_1,t_2[$ et la quantité
de ressource disponible dans ce même intervalle, $R (t_2-t_1)$. Si
cette quantité est négative pour un intervalle $[t_1,t_2[$, alors une
incohérence est détectée. En effet, si c'est le cas, alors cela
signifie que la quantité de ressource disponible dans $[t_1,t_2[$
n'est pas suffisante pour ordonnancer les consommations minimales de
toutes les activités.

\begin{defi}
La fonction de marge $SL(t_1,t_2)$ est définie de la façon suivante: 
\[  SL(t_1,t_2) = R (t_2-t_1) - \sum_{i \in \A} \bb \]
\end{defi}

Avec cette définition, nous pouvons maintenant énoncer le théorème
central du raisonnement énergétique:
 
\begin{theo}
\label{th:centerRE}
Soit $\I$ une instance de la contrainte cumulative. S'il existe
$t_1 < t_2 \in \mathbb{N}^2$ tel que $SL(t_1,t_2) <0$ alors $\I$ ne
possède pas de solution, i.e.
\begin{equation}
\exists t_1 < t_2 \in \mathbb{N}^2 , SL(t_1,t_2) <0 \Rightarrow \I
\text{ ne possède pas de solution}
\label{eq:centerRE}
\end{equation}
\end{theo}

Pour calculer cette fonction de marge, il faut pouvoir calculer la
quantité de ressource requise par une activité dans l'intervalle
$[t_1,t_2[$, $\bb$. {\'E}tant donnée une activité $i$, cette quantité
correspond à l'aire minimale, parmi tous les ordonnancements possibles
de $i$, occupée par cette dernière à l'intérieur de l'intervalle
$[t_1,t_2[$. Pour atteindre ce minimum, seulement trois configurations
sont possibles et sont décrites ci-dessous:
\begin{itemize}
\item l'activité est {\it calée à gauche}: l'activité démarre à $\ES$; 
\item l'activité est {\it  calée à droite}: l'activité finit à $\LE$;
\item l'activité est {\it centrée}: elle occupe tout l'intervalle
  $[t_1,t_2[$.
\end{itemize}
Parmi ces trois cas, celui conduisant à la consommation minimale de
ressource est celui dont l'intersection avec l'intervalle $[t_1,t_2[$
est minimale.

Pour donner l'expression mathématique de ces quantités, nous
introduisons trois notations. $\bbLS$ (respectivement $\bbRS$ et
$\bbCS$) correspond à la quantité de ressource requise par l'activité $i$
dans l'intervalle $[t_1,t_2[$  quand l'activité est calée à gauche
(respectivement calée à droite et centrée). Formellement, ces trois
quantités peuvent être exprimer de la manière suivante: 
\begin{align}
\bbLS&=b_i * \max(0,p_i- \max(0,t_1 - \ES)) \label{eq:LS_CUSP}\\
\bbRS&=b_i * \max(0,p_i- \max(0, \LE-t_2))\label{eq:RS_CUSP}\\
\bbCS&=b_i * (t_2-t_1)\label{eq:CS_CUSP}
\end{align}
Alors, l'expression de la quantité minimale de ressource est le
minimum de ces trois quantités, i.e.
\begin{equation}
\bb=\min\left(\, \bbLS\, ,\,\bbRS\, ,\,\bbCS \, \right)
\end{equation} 


\begin{ex}
Considérons une ressource de capacité $R=4$ et les $4$ activités
suivantes: 
\begin{center}
  \begin{tabular}{|P{1cm}|P{1cm}P{1cm}P{1cm}P{1cm}|}
    \hline
    act & \ES & \LE & p_i & b_i  \\
    \hline
    1 & 0 & 6 & 3 & 1 \\
    2 & 1 & 6 &  5 & 2 \\    
    3 & 2 & 8 &  4 & 1 \\    
    4 & 2 & 8 &  5 & 1 \\    
    \hline
  \end{tabular}
\end{center}

Nous montrons comment appliquer le raisonnement énergétique sur
l'intervalle $[3,6]$. Nous commençons par présenter le calcul de la
consommation minimale de chaque activité. Dans la
figure~\ref{fig:ex_ER_CUSP}, nous représentons graphiquement les
activités calées à gauche ($LS$), à droite ($RS$) et centrées ($CS$).

\begin{figure}[!ht]
  \centering
  \subcaptionbox{activité $1$}[0.2\linewidth]{
  \begin{tikzpicture}[xscale=0.35,yscale=0.5]
   \node (O) at (0,0) {};
   \fill[gray!20!] (3.5,8) node[black,above] {$t_1$} node{} +(0,-8) 
   rectangle (6.5,8)  node[above,black] {$t_2$};
   \draw[dotted] (6.5,0) -- (6.5,8);
   \draw[dotted] (3.5,0) -- (3.5,8);
   \draw[->] (0,0) -- (9,0);
   \draw[thick] (6.5,0) node[below] {$\LE[1]$} -- (6.5, 8.5);
   \draw[thick] (0.5,0) node[below] {$\ES[1]$} -- (0.5, 8.5);
   \draw (0.5,0) rectangle (3.5,1) node[midway] {\small $LS$};
   \draw (3.5,3) rectangle (6.5,4) node[midway] {\small $RS$};
   \draw (3.5,6) rectangle (6.5,7) node[midway] {\small $CS$}; 

\foreach \i in {0.5,...,8.5} {
 \draw[thick] (\i,-0.1) -- (\i,0);
}
\end{tikzpicture}}
\hfill
  \subcaptionbox{activité $2$}[0.2\linewidth]{
  \begin{tikzpicture}[xscale=0.35,yscale=0.5]
    \node (O) at (0,0) {};
    \fill[gray!20!] (3.5,8) node[black,above] {$t_1$} node{} +(0,-8) 
    rectangle (6.5,8)  node[above,black] {$t_2$};
    \draw[dotted] (6.5,0) -- (6.5,8);
    \draw[dotted] (3.5,0) -- (3.5,8);
    \draw[->] (0,0) -- (9,0);
    \draw[thick] (6.5,0) node[below] {$\LE[2]$} -- (6.5, 8.5);
    \draw[thick] (1.5,0) node[below] {$\ES[2]$} -- (1.5, 8.5);
    \draw (1.5,0) rectangle (6.5,2) node[midway] {\small $LS$};
    \draw (1.5,3) rectangle (6.5,5) node[midway] {\small $RS$};
    \draw (1.5,6) rectangle (6.5,8) node[midway] {\small $CS$}; 

    \foreach \i in {0.5,...,8.5} {
      \draw[thick] (\i,-0.1) -- (\i,0);
    } 
  \end{tikzpicture}}
\hfill
\subcaptionbox{activité $3$}[0.2\linewidth]{
  \begin{tikzpicture}[xscale=0.35,yscale=0.5]
    \node (O) at (0,0) {};
    \fill[gray!20!] (3.5,8) node[black,above] {$t_1$} node{} +(0,-8) 
    rectangle (6.5,8)  node[above,black] {$t_2$};
    \draw[dotted] (6.5,0) -- (6.5,8);
    \draw[dotted] (3.5,0) -- (3.5,8);
    \draw[->](0,0) -- (9,0);
    \draw[thick] (8.5,0) node[below] {$\LE[3]$} -- (8.5, 8.5);
    \draw[thick] (2.5,0) node[below] {$\ES[3]$} -- (2.5, 8.5);
    \draw (2.5,0) rectangle (6.5,2) node[midway] {\small $LS$};
    \draw (4.5,3) rectangle (8.5,5) node[midway] {\small $RS$};
    \draw (3,6) rectangle (7,8) node[midway] {\small $CS$}; 

    \foreach \i in {0.5,...,8.5} {
      \draw[thick] (\i,-0.1) -- (\i,0);
    } 
  \end{tikzpicture}}
\hfill
\subcaptionbox{activité $4$}[0.2\linewidth]{
  \begin{tikzpicture}[xscale=0.35,yscale=0.5]
    \node (O) at (0,0) {};
    \fill[gray!20!] (3.5,8) node[black,above] {$t_1$} node{} +(0,-8) 
    rectangle (6.5,8)  node[above,black] {$t_2$};
    \draw[dotted] (6.5,0) -- (6.5,8);
    \draw[dotted] (3.5,0) -- (3.5,8);
    \draw[->](0,0) -- (9,0);
    \draw[thick] (8.5,0) node[below] {$\LE[4]$} -- (8.5, 8.5);
    \draw[thick] (2.5,0) node[below] {$\ES[4]$} -- (2.5, 8.5);
    \draw (2.5,0) rectangle (7.5,1) node[midway] {\small $LS$};
    \draw (3.5,3) rectangle (8.5,4) node[midway] {\small $RS$};
    \draw (2.5,6) rectangle (7.5,7) node[midway] {\small $CS$};
    \foreach \i in {0.5,...,8.5} {
      \draw[thick] (\i,-0.1) -- (\i,0);
    } 
  \end{tikzpicture}}
\caption{Calcul de l'énergie minimale}
\label{fig:ex_ER_CUSP}
\end{figure}

Le tableau suivant présente le calcul de l'énergie minimale de chaque
activité. Ces valeurs sont calculées à partir des
équations~\eqref{eq:LS_CUSP},~\eqref{eq:RS_CUSP}
et~\eqref{eq:CS_CUSP}.

\begin{center}
  \begin{tabular}{|P{1cm}|P{4cm}P{4cm}P{3.5cm}P{1.5cm}|}
    \hline
    act & \bbLS[i][3][6] & \bbRS[i][3][6] & \bbCS[i][3][6] & \bb[i][3][6]\\
    \hline
    1 & 1*(3-(3-0))=0 & 1*(3-(6-3))=3 & 1*(6-3)=3 & 0 \\
    2 & 2*(5-(3-1))=6 & 2*(5-(6-6))=10 & 2*(6-3)=6 & 6 \\    
    3 & 2*(4-(3-2))=6 & 2*(4-(8-6))=4 & 2*(6-3)=6 & 4 \\    
    4 & 1*(5-(3-2))=3 & 1*(4-(8-6))=3 & 1*(6-3)=3 & 3 \\    
    \hline
  \end{tabular}
\end{center}
La valeur de la fonction de marge $SL(3,6)$ est donc: 
\[\Big( 4*(6-3) \Big) - \Big( 0+6+4+3\Big) = 12 - 13 = -1 <0  \] 

L'instance n'est donc pas réalisable et une incohérence est détectée. 
\end{ex}

La règle de détection d'incohérence décrite par le
Théorème~\ref{th:centerRE} impose le calcul de la fonction de marge
pour tout intervalle $[t_1,t_2[ \in \mathbb{N}^2$, ce qui, en
pratique, n'est pas envisageable. Cependant, Baptiste {\it et al.} ont
proposé une caractérisation des intervalles sur lesquels il est
suffisant d'appliquer le raisonnement énergétique. Par suffisant, nous
entendons que, si une incohérence n'est pas détectée sur cet ensemble
d'intervalle, alors aucun autre intervalle ne pourra conduire à
l'insatisfiabilité de l'instance. La taille de cet ensemble
d'intervalle d'{\it intérêt} est de l'ordre de $O(n^2)$. Le calcul de
la fonction de marge pouvant s'effectuer incrémentalement, la
complexité du checker énergétique est donc de $O(n^2)$.

Récemment, Derrien {\it et al.}~\cite{DP} ont proposé une
caractérisation plus fine de l'ensemble des intervalles d'intérêt pour
le raisonnement énergétique. L'ordre de grandeur de la taille de cet
ensemble d'intervalle n'est pas modifié et est toujours de l'ordre de
$n^2$ mais le nombre d'intervalles considérés est divisé par $7$. En
effet, la caractérisation de Baptiste {\it et al.} considère, pour
chaque paire d'activités, $15$ intervalles tandis que la
caractérisation de Derrien {\it et al.} n'en considère que $2$.


Ces caractérisations se basent toutes les deux sur l'idée suivante:
pour décider de l'existence d'un point $(t_1,t_2)$ pour lequel la
fonction de marge évaluée en ce point $SL(t_1,t_2)$ est négative, il
suffit d'évaluer la fonction en ses minimums locaux. La principale
différence entre ces deux caractérisation est la caractérisation plus
ou moins fine de ces minimums.

Nous allons brièvement détailler la caractérisation de Derrien {\it et
al.} que nous adapterons dans le cadre du \CECSP~dans le
chapitre~\ref{sec:PPC_CECSP}.

Pour étudier les minimums locaux de la fonction de marge, les auteurs
de~\cite{DP} étudient les variations de cette dernière. L'expression
de la fonction de marge dépendant de $R(t_2-t_1)$ et de $\sum_{i \in
  \A} \bb$, ses variations dépendent des variations de la fonction
$(t_1,t_2) \rightarrow \sum_{i \in \A} \bb$. En particulier, ils
remarquent que tous les changements de variation de cette fonction ne
peuvent impliquer un minimum local. Pour qu'une variation implique un
tel minimum il faut que la dérivé à gauche de cette fonction soit
négative, que la dérivé à droite soit positive et qu'au moins l'une
d'entre elles soit non nulle. En effet, dans l'expression de la
fonction de marge, cette fonction est multiplié par $-1$.

Cette propriété, induite par le {\it test de la dérivé seconde}, leur
permet de décrire des conditions sur les consommations individuelles
des activités, $\bb$, sous lesquels $(t_1,t_2)$ peut être un minimum
local de la fonction de marge. Ces conditions sont exprimées dans le
lemme~\ref{lem:min_CUSP}. 

\begin{lemma}
\label{lem:min_CUSP}
La fonction de marge $SL(t_1,t_2)$ est localement minimale seulement
s'il existe deux activités $i$ et $j$ telles que les conditions
ci-dessous sont satisfaites. 
\begin{align} \frac{\delta^{-}\bb}{\delta t_1} &>
\frac{\delta^{+}\bb}{\delta t_1} \label{eq:deriv1_CUSP}\\ 
\frac{\delta^{-}\bb[j]}{\delta t_2}
& > \frac{\delta^{+}\bb[j]}{\delta t_2} \label{eq:deriv2_CUSP}
\end{align}
\end{lemma}

\begin{proof}
Soit $(t_1,t_2) \in \mathbb{N}^2$ tel qu'il n'existe aucune activité
$i$ vérifiant la condition~\eqref{eq:deriv1_CUSP}. Alors, $\sum_{i \in
  \A} \bb$ a sa dérivée à gauche supérieure à sa dérivé à droite. Par
le test de la seconde dérivé, un minimum local d'une fonction ne peut
exister que si la dérivé à gauche est plus grande que la dérivé à
droite. Donc $(t_1,t_2)$ ne peut pas être un minimum local. 

De la même façon, on peut prouver que si la
condition~\eqref{eq:deriv2_CUSP} n'est pas vérifié, alors $(t_1,t_2)$
ne peut pas être un minimum local. 
\end{proof}

Le lemme~\ref{lem:min_CUSP} peut ensuite être utilisé pour déterminer
les conditions nécessaires permettant de déterminer un l'ensemble des
intervalles d'intérêt. Pour cela, une étude des fonctions $t_1
\rightarrow \bb$ et $t_2 \rightarrow \bb$ est nécessaire. Nous montrons
comment obtenir une caractérisation des différentes dates de début
possibles pour un intervalle, la caractérisation des dates de fin
pouvant s'obtenir de manière similaire.

\begin{lemma}
  Pour chaque activité $i$ et pour tout début d'intervalle $t_1$ il
  existe au plus un intervalle $[t_1,t_2[$ tel que
  $\frac{\delta^{-}\bb[j]}{\delta t_2} >
  \frac{\delta^{+}\bb[j]}{\delta t_2} $:
  \begin{enumerate}
  \item si $t_1 < \ES$ alors seulement l'intervalle $[t_1,\LE {[}$ doit
    être considéré.
  \item si $t_1  > \LS \land t_1 \ge \EE$ alors aucun intervalle ne
    doit être considéré. 
  \item si $t_1  > \LS \land t_1 <\EE \land t_1 < \LS$ alors seulement
    l'intervalle $[t_1,\ES+\LE-t_1 [$ doit
    être considéré.
  \item si $t_1  > \LS \land t_1 <\EE \land t_1 \ge \LS$ alors
    seulement l'intervalle $[t_1,\EE {[}$ doit
    être considéré.
  \end{enumerate}
\end{lemma}

\begin{proof}
Pour prouver ce lemme, nous étudions les variations de la fonction
$t_2 \rightarrow \bb$, à $t_1$ fixé. Les variations de cette fonction
dépendent de la position relative de $t_1$ par rapport à $\ES,\ \LS$ et
$\EE$. Le cas où $t_1 \le \ES$ est présenté ci-dessous pour une
activité $i$ possédant les caractéristiques suivantes: $\ES=2,\
\LE=8,\ p_i=4$ et $b_i=1$.

\begin{minipage}{\linewidth}
  \begin{minipage}{0.4\linewidth}
    \begin{itemize}
      
      \vspace{0.4cm}
    \item si $t_2 \le \LS$ alors $\bb=0$
      
      \vspace{0.4cm}
    \item si $\LS < t_2 < \LE$ alors $\bb= b_i (t_2 - \LS)$
      
      \vspace{0.4cm}
    \item si $\LE \le t_2 $ alors $\bb = p_i$
    \end{itemize}
  \end{minipage}
  \hfill
  \begin{minipage}{0.55\linewidth}
    \begin{figure}[H]
      \centering
      \begin{tikzpicture}      
        [yscale=0.5,xscale=0.65,
        every node/.style={color=black},%
        dot/.style={circle,fill=black,minimum size=4pt,inner sep=0pt,%
        outer sep=-1pt},
      cross/.style={path picture={ 
          \draw
          (path picture bounding box.south east) -- (path picture bounding box.north west) (path picture bounding box.south west) -- (path picture bounding box.north east);
        }}]
      \fill[gray!20] (2,0) rectangle (8,1);

      \node (O) at (0,0) {};
      \draw[->] (0,0) -- (10.5,0); \foreach \i in {0,1,...,10}{ \draw
        (\i,-0.1) -- (\i,0); }

      \draw (1,0) -- (1,3) node[above] {\small $t_1=1$}; 
      \draw (1,2) -- node[midway,above] {\small $0$} (4,2) node {$\bullet$} node[above] {\small $
        \LS$} -- node[midway,above,rotate=35] {\small $b_i(t_2 - \LS)$} (8,5)
      node {$\bullet$}  node[above] {\small $ \LE$} --
      node[midway,above] {\small $p_i$} (9.5,5) ;
      \draw[dotted] (4,0) -- (4,2);
      \draw[dotted] (8,0) -- (8,5);
      \draw (4,0) rectangle (8,1) node[midway] {$i$};
    \end{tikzpicture}
    \caption{{\'E}volution de $t_2 \rightarrow \bb$}

  \end{figure}
\end{minipage}
\end{minipage}

Le seul point $t_2$ pour lequel $ \frac{\delta^{-}\bb[j]}{\delta t_2} >
\frac{\delta^{+}\bb[j]}{\delta t_2}$ est $[1,8[= [t_1, \LE{[}$. 

Les trois autres cas peuvent être prouver de façon très similaire. 
\end{proof}

Un raisonnement symétrique permet de caractériser l'ensemble des
points $t_1$ vérifiant la condition~\eqref{eq:deriv1_CUSP}. Une fois
que ces dates potentielles de début et de fin d'intervalle d'intérêt
sont caractérisées, elles sont rassemblées pour créer l'ensemble des
intervalles d'intervalles d'intérêt pour le raisonnement énergétique. 
La caractérisation complète de ces intervalles est décrit dans le
lemme suivant.

\begin{lemma}
  La fonction de marge $SL$ est localement minimale en $(t_1,t_2)$
  seulement s'il existe deux activités $i$ et $j$ telles que $(t_1,t_2)
  \in \O_C(i,j)$ avec 
  
  \[ \O_C(i,j)= \left\{ 
      \begin{aligned} 
        &{[}\ES,\LE[j]{[} & \quad \text{ si }  \quad & \ES \le \ES[j] \ \land \
                                       \LE[j] \ge \LE \\
        &[\ES,\ES[j]+\LE[j]-\ES{[} & \quad \text{ si }  \quad &  \ES > \ES[j] \ \land \
                                                  \ES > \EE[j] \ \land \
                                                  \\
                                                 & & &\ES[j] + \LE[j] - \ES \ge \LE\\
        &[\ES,\EE[j]{[} & \quad \text{ si }  \quad &  \ES > \ES[j] \ \land \ \ES <
                                       \EE[j] \ \land \  \ES \ge \LS[j]
                                       \ \land \  \\
                                       & & &\EE[j] \ge \LE \\
        &[\LS,\LE[j]{[} & \quad \text{ si }  \quad &  \LS \le \ES[j] \ \land \
                                       \LE[j] < \LE \ \land \ \LE[j] >
                                       \LS \ \land \  \\
                                       &  & &\LE[j] \le \EE[j]\\
        &[\LS,\ES[j]+\LE[j]-\LS{[} & \quad \text{ si }  \quad &  \LS > \ES[j] \ \land \
                                                  \LE < \EE[j]\ \land \ \LS <
                                                  \LS[j] \ \land \ \\
                                              & & &\LS < \ES[j] +
\LE[j] - \LS \ \land \ \ES[j] + \LE[j] -\LS[i] < \LE\\
        &[\LS,\EE[j]{[} & \quad \text{ si }  \quad &  \LS > \ES[j] \ \land \
                                       \LS < \EE[j] \ \land \ \LS \ge
                                       \LS[j] \ \land \  \\
& & &\EE[j] < \LE\
                                       \land \ \EE[j] > \LS\ \land \ \EE[j] \le \EE\\
        &[\ES+\LE - \LE[j],\LE[j]{[} & \quad \text{ si }  \quad &  \LE[j] < \LE \
                                                    \land \  \LE[j] >
                                                    \LS \ \land \
                                                    \LE[j] > \EE \
                                                    \land \ \\
& & &
                                                    \ES+\LE-\LE[j] \le
                                                    \LS[j] \\
        &[\ES+\LE - \EE[j],\EE[j]{[} & \quad \text{ si }  \quad &  \EE[j] <\LE \
                                                    \land \ \EE[j] >
                                                    \LS \ \land \
                                                    \EE[j] > \EE \
                                                    \land \  \\
& & &\LS[j]
                                                    \le \ES +\LE -
                                                    \EE[j] < \EE[j] \
                                                    \land \ \\
& & &\LS[j] <
                                                    \ES+\LE - \EE[j]
      \end{aligned} 
    \right.
  \]
\end{lemma}

Une démonstration de ce lemme ainsi que la preuve que ces intervalles
sont suffisant pour détecter toute incohérence due au raisonnement
énergétique peut être trouvé dans~\cite{Alb}.

Grâce à cette caractérisation, l'algorithme de détection d'incohérence
de Baptiste {\it et al.} peut être amélioré. Pour cela, nous
considérons l'ensemble $\O_1=\cup_{i \in \A} \{\ES,\LS\}$ des dates de
début possibles d'un intervalle et l'ensemble $\O_2=\cup_{i \in \A}
\{\EE,\LE \}$ correspondant aux dates de fin. Ceci nous permet de
décrire l'algorithme de détection d'incohérence en $O(n^2)$ de
Derrien~\cite{Alb} décrit par l'algorithme~\ref{algo:check_CUSP}.

\begin{algorithm}[!htb]
\setstretch{1.35}
  \caption{Algorithme de détection d'incohérence énergétique}
  \label{algo:check_CUSP}
  \PourTous {$(i,t_1) \in \O_1$}{

    $pente=\sum_j \bb[j][t_1][t_1+1]$

    $SL_{ER}=0$, $t_2^{old}=t_1$

  \PourTous {$(j,t_2) \in \O_2 \cup \O_{t_1}$ par ordre croissant}{

    $SL_{ER}=SL_{ER}+pente*(t_2-t_2^{old})$

    \Si {$SL_{ER } > R(t_2-t_1)$}{
      Le problème est incohérent.}

    $pente=pente + \bb[j][t_1][t_2-1] - 2\bb[j] +  \bb[j][t_1][t_2+1]$

    $t_2^{old}=t_2$
  }
}
\end{algorithm}

Cette caractérisation peut aussi être étendue pour déterminer les
intervalles $(t_1,t_2)$ sur lesquels appliquer les ajustements du
raisonnement énergétique. Ces ajustements sont décrits par les 
règles~\ref{reg:ajust1_ER_CUSP} et~\ref{reg:ajust2_ER_CUSP}.  Dans la
règle~\ref{reg:ajust1_ER_CUSP}, nous imposons à une activité $i$ de
commencer après $t_1$ et si la quantité de ressource
disponible n'est pas suffisante pour ordonnancer les consommations
minimales de toutes les activités -- excepté celle de l'activité $i$
qui est remplacée par $\bbRS$ -- alors, nous pouvons déduire que
l'activité doit commencer avant $t_1$.

\begin{reg}
  \label{reg:ajust1_ER_CUSP}
  S'il existe un intervalle $[t_1,t_2[$ avec $t_1 > \ES$ et une activité $i$ pour lesquels:
  \[ \sum_{\substack{j \in \A \\ j \neq i}} \bb[j] + \bbRS > R (t_2-t_1)\]
  alors 
  \[  \LS \le t_1 - \frac{1}{b_i} \left( \sum_{\substack{j \in \A \\ j
          \neq i}} \bb[j] + \bbRS - R (t_2-t_1) \right) \]
\end{reg}

\begin{ex}
Considérons une ressource de capacité $R=3$ et les $4$ activités
suivantes: 
\begin{center}
  \begin{tabular}{|P{1cm}|P{1cm}P{1cm}P{1cm}P{1cm}|}
    \hline
    act & \ES & \LE & p_i & b_i  \\
    \hline
    1 & 0 & 6 & 5 & 2 \\
    2 & 2 & 6 & 1 & 3 \\    
    3 & 2 & 5 & 3 & 1 \\    
    4 & 0 & 4 & 2 & 1 \\    
    \hline
  \end{tabular}
\end{center}

Pour pouvoir ajuster la date de début au plus tard de l'activité $4$,
nous allons appliquer la règle~\ref{reg:ajust1_ER_CUSP} sur
l'intervalle $[2,4[$. Pour cela, nous commençons par calculer
la quantité de ressource requise par les activités dans cet intervalle. 

Pour l'activité $1$, cette quantité est $\bb[1][2][4]=4$. Elle est atteinte en
calant l'activité à gauche ou à droite (l'énergie est la même dans les
deux cas). Pour l'activité $2$, en calant l'activité à droite, nous
obtenons $\bb[2][2][4]=0$. Enfin, pour l'activité $3$, la quantité de
ressource requise vaut $\bb[3][2][4]=2$.

La quantité de ressource obtenue en calant l'activité $4$ à droite est de
$\bbRS[4][2][4]=2$. Nous devons ensuite comparée la somme de toutes
ces consommations, i.e. $4+0+2+2=8$, à la quantité de ressource
disponible dans l'intervalle $[2,4[$,
i.e. $R(t_2-t_1)=3*(4-2)=6$. Comme nous avons $6<8$, nous pouvons
appliquer la règle~\ref{reg:ajust1_ER_CUSP} et nous obtenons $\LS[4]
\le 2- (6 + 2 -6)  / 1 = 0$. Le domaine de la date de début de l'activité
$4$ est donc réduit à $\{0\}$.


\begin{figure}[!ht]
  \centering
  \subcaptionbox{l'activité $4$ ne peut commencer après $t_1=2$}[0.45\linewidth]{
    \begin{tikzpicture}[xscale=0.6,yscale=0.5  ,     every node/.style={color=black},%
      dot/.style={circle,fill=black,minimum size=4pt,inner sep=0pt,%
        outer sep=-1pt}, cross/.style={path picture={ \draw (path picture
          bounding box.south east) -- (path picture bounding box.north west)
          (path picture bounding box.south west) -- (path picture bounding
          box.north east); }}]
      \node (O) at (0,0) {};
      \fill[gray!20!] (2.5,3.2) node[black,above] {$t_1$} node{} +(0,-3.2) 
      rectangle (4.5,3.2)  node[above,black] {$t_2$};
      \draw[dotted] (4.5,0) -- (4.5,3.5);
      \draw[dotted] (2.5,0) -- (2.5,3.5);
      \draw[->] (0,0) -- (7,0);
      \draw[decorate,decoration={brace,mirror}] (0.5,-0.3) --
      (2.5,-0.3) node[midway,below] {début de $4$};
      \draw (0.5,1) rectangle (5.5,3) node[midway] {\small $1$};
      \draw (5.5,0) rectangle (6.5,3) node[midway] {\small $2$};
      \draw (2.5,0) rectangle (5.5,1) node[midway] {\small $3$}; 
      
      \foreach \i in {0.5,...,6.5} {
        \draw[thick] (\i,-0.1) -- (\i,0);
      }
    \end{tikzpicture}}
  \hfill
  \subcaptionbox{le domaine de l'activité $4$ peut être ajuster}[0.45\linewidth]{
    \begin{tikzpicture}[xscale=0.6,yscale=0.5  ,     every node/.style={color=black},%
      dot/.style={circle,fill=black,minimum size=4pt,inner sep=0pt,%
        outer sep=-1pt}, cross/.style={path picture={ \draw (path picture
          bounding box.south east) -- (path picture bounding box.north west)
          (path picture bounding box.south west) -- (path picture bounding
          box.north east); }}]
      \node (O) at (0,0) {};
      \fill[gray!20!] (2.5,3.2) node[black,above] {$t_1$} node{} +(0,-3.2) 
      rectangle (4.5,3.2)  node[above,black] {$t_2$};
      \draw[dotted] (4.5,0) -- (4.5,3.5);
      \draw[dotted] (2.5,0) -- (2.5,3.5);
      \draw[->] (0,0) -- (7,0);

      \draw[decorate,decoration={brace,mirror}] (0.2,-0.3) --
      (0.8,-0.3) node[midway,below] {début de $4$};
      \draw (0.5,1) rectangle (5.5,3) node[midway] {\small $1$};
      \draw (5.5,0) rectangle (6.5,3) node[midway] {\small $2$};
      \draw (2.5,0) rectangle (5.5,1) node[midway] {\small $3$}; 
      
      \foreach \i in {0.5,...,6.5} {
        \draw[thick] (\i,-0.1) -- (\i,0);
      }
    \end{tikzpicture}}
  \caption{Ajustement de borne du raisonnement énergétique}
  \label{fig:ex_ER_CUSP}
\end{figure}

\end{ex}

Le même raisonnement que pour l'algorithme de détection d'incohérence
nous permet de caractériser l'ensemble des intervalles permettant de
réaliser des ajustements. Les conditions sous lesquelles un
intervalles peut conduire à un ajustement sont décrite dans le
lemme~\ref{lem:min_ajust_ER_CUSP}. Comment obtenir cette
caractérisation à partir de ce lemme sera aussi adaptée dans le cadre
du \CECSP. De ce fait, elle n'est pas décrite ici.

\begin{lemma}
\label{lem:min_ajust_ER_CUSP}
La fonction $R(t_2-t_1) -\sum_{\substack{j \in \A \\ j \neq i}} \bb[j]
- \bbRS $ est localement minimale seulement si une des conditions
ci-dessous est satisfaite.
\begin{align} 
\exists (j,k), \ &~\frac{\delta^{-}\bb[j]}{\delta t_1}
&>~&~\frac{\delta^{+}\bb[j]}{\delta t_1}
&\land~&~\frac{\delta^{-}\bb[k]}{\delta t_2} & >~&
~\frac{\delta^{+}\bb[k]}{\delta t_2} \label{eq:deriv1_adj_CUSP}\\
\exists j , \ &~\frac{\delta^{-}\bb[j]}{\delta t_1}
&>~&~\frac{\delta^{+}\bb[j]}{\delta t_1}
&\land~&\frac{\delta^{-}\bbRS[k]}{\delta t_2} & >~&
\frac{\delta^{+}\bbRS[k]}{\delta t_2} \label{eq:deriv2_adj_CUSP}\\
\exists k, \ &\frac{\delta^{-}\bbRS[j]}{\delta t_1}
&>~&\frac{\delta^{+}\bbRS[j]}{\delta t_1}
&\land~&~\frac{\delta^{-}\bb[k]}{\delta t_2} & >~&
~\frac{\delta^{+}\bb[k]}{\delta t_2} \label{eq:deriv3_adj_CUSP}\\
 &\frac{\delta^{-}\bbRS[j]}{\delta t_1}
&>~&\frac{\delta^{+}\bbRS[j]}{\delta t_1}
&\land~&\frac{\delta^{-}\bbRS[k]}{\delta t_2} & >~&
\frac{\delta^{+}\bbRS[k]}{\delta t_2} \label{eq:deriv4_adj_CUSP}
\end{align}
\end{lemma}

L'ordre de grandeur de l'ensemble des intervalles d'intérêt pour les
ajustements est aussi de $n^2$ et l'algorithme permettant de procéder
à tous les ajustments possibles a une complexité en $O(n^3)$. Ceci
fait du raisonnement énergétique un des moins utilisés en pratique,
même si ce raisonnement est un des plus forts pour la contrainte
cumulative. 


De manière similaire à la règle~\ref{reg:ajust1_ER_CUSP}, une autre
règle permettant d'ajuster la date de début au plus tôt d'une activité
peut être déduite.

\begin{reg}
  \label{reg:ajust2_ER_CUSP}
  S'il existe un intervalle $[t_1,t_2[$ avec $t_1 > \ES$ et une
activité $i$ pour lesquels:
  \[ \sum_{\substack{j \in \A \\ j \neq i}} \bb[j] +
    \min(\bbCS,\bbLS) > R (t_2-t_1)\] 
  alors 
  \[  \ES \ge t_2 - \frac{1}{b_i} \left(R (t_2-t_1)  -\sum_{\substack{j \in \A \\ j
          \neq i}} \bb[j]  \right) \]
\end{reg}

Le lemme~\ref{lem:min_ajust_ER_CUSP} peut aussi être appliqué dans ce
cas, en considérant la fonction suivante au lieu de $\bbRS$:
$b_i*\max\Big(\, 0\, , \, \min(\, \LE\, ,\, t_2\,) - \max(\, \LS\, ,
  \, t_1\,) \,\Big)$. 


Les différentes règles de filtrages présentées dans les paragraphes
précédents ont parfois été étendues ou couplées afin de définir de
nouvelles règles plus fortes que considérées indépendamment. Certaines
d'entre elles sont décrites dans le paragraphe suivant. 


\subsubsection{Règles de filtrage étendues}
\label{sec:mix_CUSP}

Parmi les règles de filtrage étendues ou couplés on trouve le
raisonnement Edge-Finding étendu~\cite{ExtEF}, le
Time-Table-Edge-Finding~\cite{V11} ou encore le Time-Table
disjonctif~\cite{Gay2015}. Dans ce paragraphe, nous montrerons comment
coupler deux raisonnements en présentant, en premier lieu, l'idée de
la méthode utilisée pour le Time-Table Edge-Finding et, dans un second
temps, en détaillant celle utilisée pour le TIme-Table Disjonctif. En
effet, une adaptation sera présentée dans le
chapitre~\ref{sec:PPC_CECSP} dans le cadre du \CECSP.

L'idée du Time-Table-Edge-Finding est de renforcer le raisonnement
existant du Edge-Finding en prenant en considération le profil
obligatoire des activités. L'algorithme proposé par Vil{\`i}m est en
$O(n^2)$ et est un des plus performant, en pratique, pour résoudre les
problème cumulatifs. 

Afin d'ajouter l'énergie déduite du raisonnement Time-Table à celle du
raisonnement Edge-Finding, les activités sont séparées en deux
parties, la partie obligatoire et la partie libre. Ceci permet
d'éviter de comptabiliser l'énergie deux fois lors du calcul de cette
dernière. L'énergie consommée par un ensemble d'activité dans un
intervalle est alors égale à la somme des énergie des parties libres
de ces dernières et de la fonction de profil obligatoire de la
ressource à l'intérieur de cet intervalle. 

Dans le même esprit, Gay propose de combiner le raisonnement
disjonctif et le raisonnement Time-Table. Pour ce faire, l'auteur
définit la notion d'{\it intervalle minimum de superposition}. Cette
notion va permettre de définir une nouvelle règle de filtrage dans le
cas où faire débuter une activité $j$ à sa date de début au plus tôt
$\ES[j]$ implique que cette activité chevauchera forcément une
activité $i$ vérifiant $b_i+b_j > R$ dans tout ordonnancement
réalisable. Ceci est dû au fait que la fenêtre de temps de $j$ ne peut
contenir un point que l'activité $i$ devra forcément chevaucher. Quand
l'activité $i$ n'a pas de partie obligatoire, l'intervalle que $j$ ne
peut intersecter peut être caractérisé. 

\begin{defi}
\label{des:moi_CUSP} 
L'intervalle minimal de superposition d’une activité $i$, dénoté
$moi_i$, est l’intervalle de temps le plus petit tel que $i$ s’exécute
au moins durant un point de temps de cet intervalle, et ce peu importe
le moment auquel l'activité $i$ est exécutée.  

Formellement,  l’intervalle minimal de superposition est défini par
$[\EE-1,\LS{]}$.
\end{defi}

Quand une activité possède une partie obligatoire, elle ne possède pas
d'intervalle minimum de superposition. 

\begin{ex}
La figure~\ref{fig:moi_CUSP} illustre l'intervalle minimum de
superposition d'une activité $i$ ne possédant pas de partie
obligatoire. \begin{figure}[!htb]
  \begin{center}
    \begin{tikzpicture}
      [xscale=0.5, yscale= 0.4,node distance=0.5cm][decoration={brace}]
      \node (sil) at (0,0) {} ;
      \node (sir) at (10,0) {}; 
      \draw (10,0) node[below] {$lst_i$};
      \node (eir) at (14,0) {} ;
      \node (eil) at (4,0) {} ;
      \draw (4,0) node [below]  {$eet_i$} ;
    %  \node [below of=eil]  {$eet_i$}; 
    %  \node[below of= sir] {$lst_i$};
      \draw (sil.center) node[below] {$est_i$}--
      (eir.center) node[below] {$let_i$};
      
      \draw[line width=3pt] (0,0.5) -- (0,2);
      \draw[line width=3pt] (14,0.5) -- (14,2);
      \draw[line width=3pt] (0,3) -- (0,4.5);
      \draw[line width=3pt] (14,3) -- (14,4.5);

      \fill[gray!80] (3,0.6) rectangle (10,1.9) node[midway,color=black] {$\mathbf{moi_i}$};
      \fill[gray!80] (3,3.1) rectangle (10,4.4) node[midway,color=black] {$\mathbf{moi_i}$};
      
      \draw (0,3.1) rectangle (4,4.4) node[midway] {$i$};
      \draw (10,0.6) rectangle (14,1.9) node[midway] {$i$};

      \draw[dashed] (3.9,0) -- (3.9,4.5);
      \draw[dashed] (10,0) -- (10,4.5);

      \foreach \i in {0,...,14} {
        \draw (\i,0)  -- (\i,-0.2);
      }
    \end{tikzpicture}
  \end{center}

  \caption{Intervalle minimum de superposition d'une activité}
  \label{fig:moi_CUSP}
\end{figure}
\end{ex}

Cette notion permet de définir une première règle d'ajustement. Cette
règle sera ensuite améliorée dans un second temps. 

\begin{reg}
\label{reg:RDR_CUSP}
  Soient deux activités $i$ et $j$ telles que $i$ ne possède pas de
  partie obligatoire et que $b_i+b_j > R$. Si ordonnancer l'activité
 $j$ à sa date de début au plus tôt la fait se superposer complètement
 à l’intervalle minimal de superposition de $i$ ($moi_i \subseteq
 [\ES[j],\EE[j]{]}$), alors $\ES[j] \ge \EE$.
\end{reg}

\begin{ex}
La règle~\ref{reg:RDR_CUSP} est illustrée par la
figure~\ref{fig:RDR_CUSP}. Dans la partie gauche de la fiigure, on
peut constater que les activités $i$ et $j$ ne peuvent être exécutée
en parallèle à cause de la capacité de la ressource. Si l'activité $j$
commence à $\ES[j]$, alors elle intersecterait complètement $moi_i$ et
il serait impossible d'ordonancer $i$. Sur la partie droite de la
figure, $\ES[j]$ a été ajusté et l'activité $j$ ne peut commencer
avant $t > \min{moi_i}$.

  \begin{figure}[!htb]
    \centering
    \begin{tikzpicture}
      \begin{scope} [yscale=0.45,xscale=0.4]
        \node (O) at (0,0) {};

        \foreach \i in {0,5,...,10} {
          \draw (\i,0) -- (\i,-0.1) node[below] {\small $\i$};
        }
        \fill[gray!30] (0,0) rectangle (14,2.4);
        \fill[gray!30] (2,2.6) rectangle (14,5);
        \draw[thick] (0,3.8)  node[left] {$R=3$} -- (14,3.8);
        
        
        \draw[densely dotted] (2,-0.1) -- (2,5.5) node[above] {$\ES[j]$};
        \draw[densely dotted] (11,-0.1) -- (11,5.5) node[above] {$\EE[j]$};
        \draw[densely dotted] (5,-0.1 ) node[below=0.3cm] {$\EE$}-- (5,5.5) ;
        \draw[densely dotted] (9,-0.1) node[below=0.3cm] {$\LS$} --
        (9,5.5) ;

        \draw[fill=black!70!] (4,2.2) rectangle
        (11,2.8) node[midway,white] {$\mathbf{moi_i}$};
        \draw[fill=white] (0,0.2) rectangle (5,2.2) node[midway]
        {$i$};

        \draw[fill=white] (2,2.8) rectangle (11,4.8) node[midway]
        {$j$};

        \draw[->] (0,0) -- (14,0);
        \draw[->] (0,0) -- (0,5.5) ;
        % \draw[densely dotted] (6,-0.1) -- (6,5.5) node[above] {$\ES[j]^{'}$};
        % \draw[->] (1.8,5.8) -- (5.2,5.8);
      \end{scope}     
      \begin{scope} [yscale=0.45,xscale=0.4,xshift=20cm]
        \node (O) at (0,0) {};
        \foreach \i in {0,5,...,10} {
          \draw (\i,0) -- (\i,-0.1) node[below] {\small $\i$};
        }

        \fill[gray!30] (5,2.6) rectangle (14,5);
        \fill[gray!30] (0,0) rectangle (14,2.4);
        \draw[thick] (0,3.8)  node[left] {$R=3$} -- (14,3.8);
        
        \draw[densely dotted] (13,-0.1) -- (13,5.5) node[above] {$\EE[j]$};
        \draw[densely dotted] (5,-0.1 ) node[below=0.3cm] {$\EE$}-- (5,5.5) node[above] {$\ES[j]$};
        \draw[densely dotted] (9,-0.1) node[below=0.3cm] {$\LS$} -- (9,5.5) ;
        \draw[fill=black!70!] (4,2.2) rectangle
        (11,2.8) node[midway,white] {$\mathbf{moi_i}$};
        
        \draw[fill=white] (0,0.2) rectangle (5,2.2) node[midway]
        {$i$};
        \draw[fill=white] (5,2.8) rectangle (13,4.8) node[midway]
        {$j$};


        \draw[->] (0,0) -- (14,0);
        \draw[->] (0,0) -- (0,5.5) ;
        % \draw[densely dotted] (6,-0.1) -- (6,5.5) node[above] {$\ES[j]^{'}$};
        % \draw[->] (1.8,5.8) -- (5.2,5.8);  

      \end{scope}
    \end{tikzpicture}
    \caption{Illustration de la règle~\ref{reg:RDR_CUSP}.}
    \label{fig:RDR_CUSP}
  \end{figure}
\end{ex}

Nous montrons maintenant comment les auteurs de~\cite{Gay2015} intègre
les informations déduites du Time-Table pour proposer une règle de
filtrage plus forte que celle décrite ci-dessus. Cette règle sera
décrite seulement dans le cas où les activités $i$ et $j$ ne possèdent
pas de partie obligatoire. Le cas inverse est décrit
dans~\cite{Gay2015} et repose sur la séparation des activités en une
partie obligatoire et une partie libre.


La règle~\ref{reg:RDR_CUSP} compare seulement les consommations de $i$
et de $j$ avec la capacité de la ressource $R$. Cependant, les
consommations obligatoires des autres activités peuvent ne pas laisser
$R$ unités de ressource durant l'intersection de $i$ et de $j$. La
règle suivante prend donc en considération le profil obligatoire de la
ressource.

\begin{reg}
\label{reg:TTDR_CUSP}
Soient donc $i$ et $j$ deux activités qui ne possèdent pas de partie
obligatoire et telles que $ b_i +b_j + \min_{t \in moi_i} TT_{\A}(t) >
R$.   Si ordonnancer l'activité
 $j$ à sa date de début au plus tôt la fait se superposer complètement
 à l’intervalle minimal de superposition de $i$ ($moi_i \subseteq
 [\ES[j],\EE[j]{]}$), alors $\ES[j] \ge \EE$.
\end{reg}

\begin{ex}
  Considérons les activités suivantes: 

  \vspace{-0.5cm}
  \begin{center}
    \begin{tabular}{|P{1cm}|P{1cm}P{1cm}P{1cm}P{1cm}|}
      \hline
      act &\ES & \LE & p_i & b_i  \\
      \hline
      1 & 2 & 11 & 3 & 2 \\
      2 & 1 & 20 & 9 & 1 \\
      3 & 2 & 11 & 9 & 1 \\
      \hline
    \end{tabular}
  \end{center}

  Les activités $1$ et $2$ ne possèdent pas de partie obligatoire tandis
  que l'activité $3$ est forcément en cours d'exécution durant
  l'intervalle $[2,11[$. 

  \begin{figure}[!htb]
    \centering
    \begin{tikzpicture}
      [yscale=0.45,xscale=0.6]
      \node (O) at (0,0) {};
      \foreach \i in {0,1,...,20} {
        \draw (\i,0) -- (\i,-0.15) node[below] {\small $\i$};
      }
      
      \draw (2,0) rectangle (11,1);
      \fill[gray!30] (5,3.8) rectangle (20,5.2);
      \fill[gray!30] (2,5.8) rectangle (11,8.2);
      
      \draw [->] (1.2,4.5) -- (4.8,4.5);
      \draw [->] (1.8,-1.7) -- (4.3,-1.7);
      \draw[densely dotted] (1,-0.1) node[below=0.3cm] {$\ES[j]$}-- (1,8.2) ;
      \draw[densely dotted] (5,-0.1 ) node[below=0.3cm] {$\EE$}--
      (5,8.2);
      \draw[densely dotted] (10,-0.1) node[below=0.3cm] {$\EE[j]$} -- (10,8.2) ;
      \draw[densely dotted] (8,-0.1) node[below=0.3cm] {$\LS$} -- (8,8.2) ;
      
      \draw[fill=black!70!] (4,5) rectangle
      (8,6) node[midway,white] {$\mathbf{moi_i}$};
      
      \draw[fill=white] (2,6) rectangle (5,8) node[midway]
      {$i$};
      \draw[fill=white] (5,4) rectangle (14,5) node[midway]
      {$j$};
      
      \draw[->] (0,0) -- (21,0);
      \draw[->] (0,0) -- (0,5.5) ;
       \draw[thick] (0,3)  node[left] {$R=3$} -- (20,3);
       % \draw[densely dotted] (6,-0.1) -- (6,5.5) node[above] {$\ES[j]^{'}$};
      % \draw[->] (1.8,5.8) -- (5.2,5.8);  
    \end{tikzpicture}
    \caption{Illustration du Time-Tabe disjonctif}
    \label{fig:TTDR_CUSP}
  \end{figure}
L'intervalle $moi_i=[4,8]$ est complètement inclus dans l'intervalle
formé par $\ES[j]$ et $\EE[j]$, i.e. $[1,10[$. De plus, le minimum du
profil de consommation de la ressource dans $moi_i$ est de $1$. Donc,
les activités $i$ et $j$, consommant respectivement $2$ et $1$ unités
de ressource, ne peuvent se chevaucher sur l'intervalle $[4,8]$. Donc
$\ES[j]$ peut être ajuster à $5$. 
\end{ex}

On peut remarquer qu'aucun des autres raisonnements présentés dans ce
chapitre n'aurait filtré de valeurs du domaine de $st_j$. En
particulier, le raisonnement énergétique, le raisonnement permettant
de filtrer le plus de valeur pour la contrainte cumulative, ne détecte
pas d'ajustement dans ce cas. Le raisonnement Time-Table disjonctif
n'est donc dominé par aucun autre raisonnement existant. 





\section*{Conclusion}

Dans ce chapitre, nous avons d'abord présenté les bases de la
programmation par contrainte. En particulier, nous avons vu que les
problèmes d'ordonnancement se traduisent naturellement en problème de
satisfaction de contraintes. 

Ce paradigme est ensuite appliqué au \CUSP, un des plus célèbres
problèmes d'ordonnancement traité en PPC. De ce fait, nous avons pu
présenté de nombreux algorithmes de filtrage, i.e. des algorithmes
filtrant les valeurs incohérentes des domaines des variables, pour ce
problème. Ces algorithmes sont classées en trois classes: les
algorithmes de filtrage simples, basés sur le concept d'énergie et
étendus. Pour chacune de ces trois classes, au moins un algorithme de
propagation est présenté en détails et de nombreux autres exemples
sont cités. 

Dans le paragraphe concernant les règles de filtrages simples, nous
avons présenté le raisonnement Time-Table et le raisonnement
disjonctif. Ces raisonnements sont parmi les moins fort pour le
\CUSP~mais leur facilité d'implémentation et leur rapidité en font des
algorithmes très utilisés pratiques. 

Le paragraphe suivant présente les règles de filtrages
énergétiques. Parmi elles sont mentionnées le raisonnement
Edge-Finding, les activités élastiques et le raisonnement énergétique
qui est présenté en détail. Ce dernier est, à ce jour, un des
raisonnement les plus fort pour le \CUSP. Cependant, la complexité
élevée de ce dernier fait de lui un des algorithmes les moins utilisé
en pratique.

Enfin, le dernier paragraphe explique comment certains auteurs ont
étendu ou couplé des règles de filtrages existantes pour définir de
nouvelles règles de filtrages plus fortes que prises
individuellement. Parmi celles citées, on trouve le raisonnement
Edge-Fiding étendu, le Time-Table-Edge-Finding et le Time-Table
disjonctif. {\'A} l'inverse du raisonnement énergétique, ces règles
sont très utilisées en pratique car leur complexité est relativement
faible. 

Dans le prochain chapitre, nous présenterons l'adaptation de plusieurs
de ces raisonnements dans le cadre du \CECSP. Un nouveau raisonnement,
basé sur le Time-Table, sera aussi présenté.
\chapter{Algorithmes issus de la programmation par contraintes pour le
\CECSP}
\label{sec:PPC_CECSP}
Dans ce chapitre, nous décrivons des algorithmes et modèles issus de
la programmation par contrainte. Les deux premières sections sont
dédiées à la présentation d'algorithmes de filtrage pour le problème
de décision du \CECSP. En effet, comme dans le cas du \CUSP, la
NP-complétude de ce problème implique qu'un algorithme assurant la
cohérence des bornes de chaque variable - assurant l'existence d'une
solution où les valeurs de chaque variables sont comprises dans ces
bornes - ne peut s'exécuter en temps polynomial. Par conséquent,
plusieurs relaxations sont proposées afin de supprimer les valeurs
possibles de début et fin de tâches incohérentes en temps polynomial.

Dans un premier temps, nous montrons que une partie des raisonnements
pour le \CUSP~s'adapte directement au cas du \CECSP. En effet, dans le
\CECSP, une activité peut être vue comme deux sous-activités: 
\begin{itemize}
\item l'une correspondant à une activité rectangulaire, appelée {\it
partie fixe}, dont la date de début et de fin doivent être
déterminées et consommant une quantité $\bmin$ de la ressource;
\item la seconde, appelée {\it partie malléable}, correspondant à une
activité préemptive de même date de début que la partie minimale et
dont chaque partie consomme une quantité de ressource comprise entre
$0$ et $\bmax-\bmin$.
\end{itemize}
La présence d'activités rectangulaires permet alors l'adaptation direct
de certains raisonnement existant pour le \CUSP. Ici, nous
présenterons les adaptations du raisonnement Time-Table, disjonctif et
Time-Table disjonctif. 

Dans un second temps, nous présenterons un nouveau raisonnement
étendu, couplant Time-Table et problème de flots. 

Enfin, une adaptation complète du raisonnement énergétique est décrit
dans le paragraphe~\ref{sec:ER_CECSP}.

Le dernier paragraphe sera quant à lui consacré à la présentation d'un
modèle de programmation par contraintes permettant de résoudre le
\CECSP~ discret, i.e. où les variables ne peuvent prendre que des
valeurs discrètes.

%%% Local Variables:
%%% mode: latex
%%% TeX-master: "../main_file"
%%% End:

\section{Algorithmes de filtrage basés sur le Time-Table}

La section suivante présente plusieurs algorithmes de filtrage pour le
\CECSP. Tous ces algorithmes utilisant une adaptation du Time-Table
pour le \CUSP, nous commençons donc par présenter brièvement comment
ce raisonnement est modifié pour être adapté dans le cadre du \CECSP.
Puis, nous présentons deux autres algorithmes permettant
de réduire l'ensemble des valeurs possibles pouvant être prises par
chaque variable. Le premier est adapté du Time-Table disjonctif pour
le \CUSP~\cite{Gay2015} et le dernier utilise une combinaison entre
problème de flots et profil obligatoire.

\subsection{Le Time-Table}
\index{Time-Table!CECSP}
Comme pour le \CUSP, le Time-Table pour le \CECSP~se base sur la
notion de partie obligatoire des activités, i.e. l'intervalle pendant
lequel une activité est en cours d'exécution dans tous les
ordonnancements réalisables. Cependant, comme dans le cas du \CECSP,
nous ne connaissons pas la durée exacte d'une activité, nous utilisons
une borne inférieure sur sa durée pour calculer la date de début au
plus tard, $\LS$, et la date de fin au plus tôt, $\EE$. Pour calculer
cette borne, remarquons que la configuration permettant de finir une
activité le plus rapidement possible, est celle où l'activité est
exécutée à son rendement maximal $\bmax$. De ce fait, une borne
inférieure sur la durée de l'activité vaut $W_i/f_i(\bmax)$. Nous
pouvons donc calculer la date de début au plus tard de l'activité,
$\LS=d_i-W_i/f_i(\bmax)$, et sa date de fin au plus tôt,
$\EE=r_i+W_i/f_i(\bmax)$. La partie obligatoire d'une activité $i$ est
alors définie de a même manière que pour le \CUSP, i.e la partie
obligatoire de $i$ est l'intervalle $[\LS,\EE]$
(cf. figure~\ref{fig_mand_CECSP}).

Cependant, dans le cas où $\LS \le \EE$, i.e. où l'activité possède
une partie obligatoire, nous pouvons seulement déduire que l'activité
$i$ va consommer au moins une quantité $\bmin$ de la ressource durant
toute sa partie obligatoire, et ce, quelque soit le moment où
l'activité est ordonnancée. La notion de profil obligatoire de la
ressource est donc légèrement différente de celle définie pour le
\CUSP~(cf. définition~\ref{def:profil_oblig},
page~\pageref{def:profil_oblig}).

\begin{defi}
Le profil obligatoire d'une ressource $TT_{\A}$ dans le cas du
\CECSP~est définie de la façon suivante: 
\[TT_{\A}(t)=\sum_{\substack{i \in \A\\\LS \le t \le \EE}} \bmin\quad
  \forall t \in \H\]
Le problème est donc insatisfiable dans le cas où $\exists t \in \H\
:\ TT_{\A}(t) > R$
\end{defi}

\begin{ex}
Considérons l'activité suivante: 

\vspace{-0.5cm}
\begin{center}
  \begin{tabular}{|P{1cm}P{1cm}P{1cm}P{1cm}P{1cm}P{1cm}|}
    \hline
    \ES & \LE & W_i & \bmin & \bmax & f_i(b)\\
    \hline
    1 & 14 & 72 & 2 & 5 & b+3\\
    \hline
  \end{tabular}
\end{center}

Nous pouvons calculer sa date de début au plus tard,
$\LS=d_i-W_i/f_i(\bmax)=14 - 72/8 =5$, ainsi que sa date de fin au
plus tôt, $\EE=r_i+ W_i/f_i(\bmax)=1 + 72/8 =10$. Comme $\EE > \LS$,
l'activité possède une partie obligatoire qui est l'intervalle
$[5,10]$ (voir
figure~\ref{fig_mand_CECSP_a},~\ref{fig_mand_CECSP_b}). Cependant,
nous pouvons seulement en déduire que l'activité sera en cours dans
cet intervalle et ,grâce à la borne inférieure sur la quantité de
ressource que peut consommer l'activité durant son exécution, nous
pouvons déduire que, sur l'intervalle $[\LS,\EE]$, l'activité est au
moins exécutée à $\bmin$ (voir figure~\ref{fig_mand_CECSP_c}).
  
\begin{figure}[htb!]
\vspace{-0.8cm}
\subcaptionbox{Ordonnancement au plus tôt\label{fig_mand_CECSP_a}}[0.3\linewidth]{
    \begin{tikzpicture}
      [xscale=0.25, yscale= 0.4,node distance=0.5cm]
      \node (sil) at (1,0) {} ;
      \node (eil) at (10,0) {} ;
      \node [below of=eil,node distance=0.63cm]  {$\EE$};
      \draw (sil.center) node[below=0.2cm] {$\ES$};
      
      \draw (0,0) -- (14,0);
      \draw[line width=3pt] (1,0) -- (1,5);
      
      \draw[<->] (0,0.1) -- (0,4.4) node[midway,left] {$\bmax$};
      \draw (1,0) rectangle (10,4.4) node[midway] {$i$};

      \draw[dashed] (10,0) -- (10,5);

      \foreach \i in {0,...,14} {
        \draw (\i,0)  -- (\i,-0.2);
      }
    \end{tikzpicture}
}
\hfill
\subcaptionbox{Ordonnancement au plus tard\label{fig_mand_CECSP_b}}[0.3\linewidth]{
    \begin{tikzpicture}
      [xscale=0.25, yscale= 0.4,node distance=0.5cm]
      \node (sir) at (5,0) {} ;
      \node (eir) at (14,0) {} ;
      \node[below of= sir,node distance=0.63cm] {$\LS$};
      \draw (eir.center) node[below=0.2cm] {$\LE$};
      
      \draw (0,0) -- (14,0);
      \draw[line width=3pt] (14,0) -- (14,5);
      
      \draw[<->] (0,0.1) -- (0,4.4) node[midway,left] {$\bmax$};
      \draw (5,0) rectangle (14,4.4) node[midway] {$i$};

      \draw[dashed] (5,0) -- (5,5);

      \foreach \i in {0,...,14} {
        \draw (\i,0)  -- (\i,-0.2);
      }
    \end{tikzpicture}
}
\hfill
\subcaptionbox{Ordonnancement réalisable à $\bmin$ dans
  $[\LS,\EE]$ \label{fig_mand_CECSP_c}}[0.3\linewidth]{ 
    \begin{tikzpicture}
      [xscale=0.25, yscale= 0.4,node distance=0.5cm]
      \node (sir) at (5,0) {} ;
      \node (eil) at (10,0) {} ;
      \node [below of=eil,node distance=0.63cm]  {$\EE$};
      \node[below of= sir,node distance=0.63cm] {$\LS$};
      \draw[<->] (5,2.7) -- (10,2.7) node[midway,above,text width=1.4cm]
      {\begin{center} \scriptsize partie oblig. \end{center}};
      
      \draw[white] (5,0) rectangle (10,2.4) node[midway,color=black] {$i$};
           
      \draw (1,4.4) -- (5,4.4) -- (5,2.4) -- (10,2.4) -- (10,3.3) --
      (13,3.3) -- (13,0);
      
      \draw (0,0) -- (14,0);
      \draw[line width=3pt] (1,0) -- (1,5);
      \draw[line width=3pt] (14,0) -- (14,5);
      
      \draw[<->] (-0.1,0.1) -- (-0.1,2.4) node[midway,left] {$\bmin$};
    


      \draw[dashed] (5,0) -- (5,5);
      \draw[dashed] (10,0) -- (10,5);

      \foreach \i in {0,...,14} {
        \draw (\i,0)  -- (\i,-0.2);
      }
    \end{tikzpicture}
}
\caption{Partie obligatoire d'une activité $i$}
\label{fig_mand_CECSP}
\end{figure}
\end{ex}

Le profil obligatoire peut aussi être calculé en $O(n)$ à l'aide d'un
algorithme de balayage en triant au préalable les activités par date
de début au plus tard et date de fin au plus tôt.

Nous détaillons maintenant l'adaptation du Time-Table disjonctif au
\CECSP. 

\subsection{Le Time-Table disjonctif}

Le second algorithme de filtrage proposé repose sur un raisonnement
appelé Time-Table disjonctif et utilisé, en premier lieu, pour le
\CUSP (cf. paragraphe~\ref{sec:mix_CUSP}). Ce dernier repose sur le
raisonnement Time-Table décrit précédemment et sur le raisonnement
disjonctif.

Le raisonnement disjonctif dans le cadre du \CECSP, est très similaire
à celui défini pour le \CUSP. La différence repose sur la construction
des ensembles disjonctifs. Dans le cas du \CECSP, un couple
d'activités ($i,j)$ sera dit disjonctif si $\bmin+\bmin[j] >R$. Dans
ce cas, nous savons que:
\begin{itemize}
\item l'activité $i$ doit commencer après l'activité $j$, ou, 
\item l'activité $i$ doit finir avant l'activité $j$.  
\end{itemize}

Cette propriété permet, entre autre, d'ajuster la date de début au
plus tôt de $j$. En effet, $\ES[j] \le \EE$ et $\LS \le \EE[j]$
implique que $j$ doit commencer après la fin de l'activité $i$ (voir
figure~\ref{fig:disj_CECSP}). De ce fait, la début de $j$ ne peut arriver
avant la date de fin au plus tôt de $i$, et donc: $\ES[j] \ge \EE$. La
règle de filtrage est ensuite similaire à celle mise en place pour le
\CUSP. 

\begin{reg}
Soient $i, j \in \A, i \neq j$ telles que $\bmin+\bmin[j] < R$ et $\LS
< \EE[j]$. Alors la date de début au plus tôt de l’activité peut être
ajustée et on a : $\ES[j] \ge \EE$.
\end{reg}

\begin{ex}
Considérons les deux activités suivantes: 
\begin{center}
\begin{tabular}{|P{1cm}|P{1cm}P{1cm}P{1cm}P{1cm}P{1cm}P{2cm}|}
    \hline
    act & \ES & \LE & W_i & \bmin & \bmax & f_i(b_i(t))  \\
    \hline
   i & 2 & 11 & 27 & 2 & 3 & 2*b_i(t) +1\\
   j & 1 & 20 & 49 & 2 & 4 & voir fig~\ref{fig:fonct_ CECSP}\\
    \hline
  \end{tabular}
\end{center}

La fonction $f_j(b)$ est définie par l'expression suivante: 
\[f_j(b)=\left\{
\begin{array}{lll}
2b & & b \in [2,3]\\
b+3 & & b \in [3,4]
\end{array}
\right.\] 
et décrite dans la figure~\ref{fig:fonct_CECSP}
\begin{figure}[!htb]
\centering
\begin{tikzpicture}
[xscale=0.8,yscale=0.56]
\node (O) at (1,2) {};
\draw[->] (1,2) -- (5.5,2);
\draw[->] (1,2) -- (1,8);

\path[draw] (2,4) -- (3,6) -- (4,7) ;

\draw[dotted] (2,2) node[below] {\footnotesize $2$} -- (2,8);
\draw[dotted,color=gray!70] (4,2) node[below,color=black] {\footnotesize $4$}
-- (4,8);
\draw[dotted] (3,2) node[below] {\footnotesize $3$} -- (3,8);

\draw (1,4) node[left] {\footnotesize $4$};
\draw (1,6) node[left] {\footnotesize $6$};
\draw (1,7) node[left] {\footnotesize $7$};
\end{tikzpicture}
\caption{Fonction $f_j(b_j(t))$}
\label{fig:fonct_CECSP}
\end{figure}

Dans cet exemple, comme $\bmin +\bmin[j] =4 > 3$, $i$ et $j$ ne
peuvent s'exécuter en parallèle. Si l'activité $i$ finit au temps
$\LE= 11$ alors, elle chevauche forcément l'activité $j $ (voir
figure~\ref{fig:disj_CECSPa} et~\ref{fig:disj_CECSPb}). Dans tous
ordonnancement réalisable, $i$ est donc exécuté avant $j$ et la date de
début au plus tôt de $j$ peut donc être ajustée, i.e. $j$ ne peut
commencer avant $\EE=6$ (voir
figure~\ref{fig:disj_CECSPc}).
  \begin{figure}[htb!] 
    \subcaptionbox{Si $i$ finit au temps $11$...\label{fig:disj_CECSPa}}[0.45\linewidth]{
    \centering
    \begin{tikzpicture} [yscale=0.4,xscale=0.4]   
        \node (O) at (0,0) {};
      \foreach \i in {0,5,...,10} {
        \draw (\i,0) -- (\i,-0.1) node[below] {\small $\i$};
      }
      \fill[gray!50] (2,0) rectangle (11,3.4);
      \fill[gray!50] (1,3.6) rectangle (14,8);
      
      \draw[fill=white] (6.4,0.2) -- (6.4,3.2)   -- (8.4,3.2) -- (8.4,2) --node[midway,below=0.2cm] {$i$}
      (11,2) -- (11,0.2) -- cycle;
      
      \draw[fill=white] (1,3.8) -- (1,7.8)  node[midway,right=0.6cm] {$j$} -- (6,7.8) -- (6,5.8) --
      (9.5,5.8) -- (9.5,3.8) -- cycle;
      \draw[white, pattern=north west lines] (6.4,0) rectangle (9.5,8);

      \draw[->] (0,0) -- (14,0);
      \draw[->] (0,0) -- (0,8) ;
      \draw (0,3) node[left] {$R=3$};
      \draw[densely dotted] (1,-0.1) -- (1,8) node[above] {$\ES[j]$};
      \draw[densely dotted] (8,-0.1) -- (8,8) node[above] {$\EE[j]$};
      \draw[densely dotted] (2,-0.1)  node[below] {$\ES$}-- (2,8);
      \draw[densely dotted] (6,-0.1 ) node[below=0.4cm] {$\EE$}-- (6,8) ;
      \draw[densely dotted] (7,-0.1) node[below] {$\LS$} -- (7,8) ;
      \draw[densely dotted] (11,-0.1) node[below right] {$\LE$} -- (11,8) ;
      \draw[<->] (14.5,0.2) -- (14.5,3.2) node[midway,right] {$\bmax$} ;
      \draw[<->] (14.5,3.8) -- (14.5,7.8) node[midway,right] {$\bmax[j]$} ;   
    \end{tikzpicture}
  }
    \subcaptionbox{...alors $i$ chevauche forcément l'activité $j$\label{fig:disj_CECSPb}}[0.45\linewidth]{
        \centering
        \begin{tikzpicture}[yscale=0.4,xscale=0.4]
      \node (O) at (0,0) {};
      \foreach \i in {0,5,...,10} {
        \draw (\i,0) -- (\i,-0.1) node[below] {\small $\i$};
      }
      \fill[gray!50] (2,0) rectangle (11,3.4);
      \fill[gray!50] (1,3.6) rectangle (14,8);
      
      \draw[fill=white] (7,0.2) rectangle (11,3.2) node[midway]
      {$i$};
      \draw[fill=white] (1,3.8) rectangle (8,7.8) node[midway]
      {$j$};
      \draw[white, pattern=north west lines] (7,0) rectangle (8,8);

      \draw[->] (0,0) -- (14,0);
      \draw[->] (0,0) -- (0,8) ;
      \draw (0,3) node[left] {$R=3$};
      \draw[densely dotted] (1,-0.1) -- (1,8) node[above] {$\ES[j]$};
      \draw[densely dotted] (8,-0.1) -- (8,8) node[above] {$\EE[j]$};
      \draw[densely dotted] (2,-0.1)  node[below] {$\ES$}-- (2,8);
      \draw[densely dotted] (6,-0.1 ) node[below=0.4cm] {$\EE$}-- (6,8) ;
      \draw[densely dotted] (7,-0.1) node[below] {$\LS$} -- (7,8) ;
      \draw[densely dotted] (11,-0.1) node[below right] {$\LE$} -- (11,8) ;
      % \draw[densely dotted] (6,-0.1) -- (6,8) node[above] {$\ES[j]^{'}$};
      % \draw[->] (1.8,5.8) -- (5.2,5.8);
      \draw[<->] (14.5,0.2) -- (14.5,3.2) node[midway,right] {$\bmax$} ;
      \draw[<->] (14.5,3.8) -- (14.5,7.8) node[midway,right]
      {$\bmax[j]$} ;
    \end{tikzpicture}
   }
    \subcaptionbox{$\ES[j]$ peut être ajusté\label{fig:disj_CECSPc}}[\linewidth]{    \centering
      \begin{tikzpicture} [yscale=0.4,xscale=0.4]
      \node (O) at (0,0) {};
      \foreach \i in {0,5,...,10} {
        \draw (\i,0) -- (\i,-0.1) node[below] {\small $\i$};
      }
      \fill[gray!50] (2,0) rectangle (11,3.4);
      \fill[gray!50] (6,3.6) rectangle (14,8);
      
      \draw[fill=white] (2,0.2) rectangle (6,3.2) node[midway]
      {$i$};
      \draw[fill=white] (6,3.8) rectangle (13,7.8) node[midway]
      {$j$};

      \draw[->] (0,0) -- (14,0);
      \draw[->] (0,0) -- (0,8) ;
      \draw (0,3) node[left] {$R=3$};
      \draw[densely dotted] (1,-0.1) -- (1,8) node[above] {$\ES[j]$};
      \draw[densely dotted] (8,-0.1) -- (8,8) node[above] {$\EE[j]$};
      \draw[densely dotted] (2,-0.1)  node[below] {$\ES$}-- (2,8);
      \draw[densely dotted] (6,-0.1 ) node[below=0.4cm] {$\EE$}-- (6,8) ;
      \draw[densely dotted] (7,-0.1) node[below] {$\LS$} -- (7,8) ;
      \draw[densely dotted] (11,-0.1) node[below right] {$\LE$} -- (11,8) ;
      \draw[<->] (14.5,0.2) -- (14.5,3.2) node[midway,right] {$\bmax$} ;
      \draw[<->] (14.5,3.8) -- (14.5,7.8) node[midway,right] {$\bmax[j]$} ;

  \end{tikzpicture}
}
  \caption{Raisonnement disjonctif}
  \label{fig:disj_CECSP}
\end{figure}
\end{ex}

Nous pouvons maintenant présenter l'adaptation du Time-Table
disjonctif au cas du \CECSP. Pour ce faire, nous commençons par
adapter la définition d'intervalle minimum de superposition, puis nous
présenterons les modifications apportées aux règles d'ajustement du
\CUSP~afin que ces dernières soient applicables dans le cas du
\CECSP. Enfin, l'extension de ces règles dans le cas où les activités
ne possèdent pas de partie obligatoire sera décrit. Cette extension
n'avait pas été présentée dans le chapitre sur le \CUSP~mais est
décrite dans l'article~\cite{Gay2015}. 

La notion d'intervalle minimum de superposition s'adapte naturellement
au cas du \CECSP. En effet, cette intervalle représente l'ensemble
minimal de points de temps tel que une activité $i$ est forcément en
cours durant un de ces points. 

\begin{defi}
\label{des:moi_CUSP} 
Soit $\EE^{-}$ le point le plus proche de $\EE$, i.e. $\forall \delta
>0 , |\EE^{-} -\EE| \le \delta$. L'intervalle minimum de
superposition d'une activité $i$,noté $moi_i$, est alors défini par
$moi_i=[\EE^{-},\LS{]}$ si $i$ ne possède pas de partie obligatoire et
$moi_i=\emptyset$ sinon.   

Il s'agit du plus petit intervalle de temps tel que $i$ s’exécute au
moins durant un point de temps de cet intervalle, et ce peu importe le
moment auquel l'activité $i$ est exécutée.
\end{defi}


\begin{ex}
La figure~\ref{fig:moi_CECSP} illustre l'intervalle minimum de
superposition d'une activité $i$ ne possédant pas de partie
obligatoire. En effet, quelque soit la position de l'activité $i$,
elle intersecte forcément $moi_i$.
\begin{figure}[!htb]
  \begin{center}
    \begin{tikzpicture}
      [xscale=0.5, yscale= 0.4,node distance=0.5cm][decoration={brace}]
      \node (sil) at (0,0) {} ;
      \node (sir) at (10,0) {}; 
      \draw (10,0) node[below] {$lst_i$};
      \node (eir) at (14,0) {} ;
      \node (eil) at (4,0) {} ;
      \draw (4,0) node [below]  {$eet_i$} ;
      \draw (sil.center) node[below] {$est_i$}--
      (eir.center) node[below] {$let_i$};
      
      \draw[<-] (3.95, 4.6) -- (4.3,5.1) node[above=0.2cm,right] {\scriptsize $\EE^{-}$};
      \draw[line width=3pt] (0,0.5) -- (0,2);
      \draw[line width=3pt] (14,0.5) -- (14,2);
      \draw[line width=3pt] (0,3) -- (0,4.5);
      \draw[line width=3pt] (14,3) -- (14,4.5);

      \fill[gray!80] (3.9,0.6) rectangle (10,1.9) node[midway,color=black] {$\mathbf{moi_i}$};
      \fill[gray!80] (3.9,3.1) rectangle (10,4.4) node[midway,color=black] {$\mathbf{moi_i}$};
      
      \draw (0,3.1) rectangle (4,4.4) node[midway] {$i$};
      \draw (10,0.6) rectangle (14,1.9) node[midway] {$i$};

      \draw[dashed] (3.9,0) -- (3.9,4.5);
      \draw[dashed] (10,0) -- (10,4.5);

      \foreach \i in {0,...,14} {
        \draw (\i,0)  -- (\i,-0.2);
      }
    \end{tikzpicture}
  \end{center}

  \caption{Intervalle minimum de superposition d'une activité}
  \label{fig:moi_CECSP}
\end{figure}
\end{ex}

Afin de pouvoir adapter le raisonnement au cas du \CECSP, 
\subsection{Le Time-Table basé sur les flots}




\section{Algorithme de filtrage du raisonnement énergétique}
\label{sec:ER_CECSP}

La section ci-dessous décrit l'adaptation du raisonnement
énergétique, introduit dans~\cite{RELopez} pour la contrainte cumulative,
et décrit dans la sous-section~\ref{sec:cumu_propag}. Nous commençons,
dans un premier temps, par décrire l'algorithme de vérification de ce
raisonnement, puis nous présenterons les règles d'ajustements qui
peuvent être mises en place pour filtrer les domaines des
variables. Enfin, la dernière partie de cette section sera consacrée à
la caractérisation des intervalles d'intérêt pour l'algorithme de
vérification et pour les règles d'ajustement.

\subsection{Algorithme de vérification}

\subsubsection{Condition nécessaire d'existence de solution}
Pour décrire l'algorithme de vérification, nous rappelons d'abord
l'idée principale sur laquelle repose le raisonnement énergétique. Le
principe est donc, étant donné un intervalle $[t_1,t_2[$, de calculer
les consommations minimales de ressource des activités dans cet 
intervalle et de les comparer à la quantité de ressource disponible
dans ce même intervalle. Si la ressource disponible n'est pas
suffisante pour ordonnancer les consommations minimales de toutes les
activités, une incohérence est détectée.

Dans le cas du \CUSP, la quantité de ressource requise par une
activité pouvait être calculée de manière directe. Ici, ce calcul sera
fait en deux fois: nous calculons d'abord la quantité d'énergie
requise par une activité à l'intérieur de l'intervalle $[t_1,t_2{[}$,
notée $\wb$, puis nous traduisons cette énergie en une quantité de
ressource, notée $\bb$. Ceci nous permettra ensuite de la comparer
avec la ressource disponible dans $[t_1,t_2{[}$.

Formellement, ces quantités sont représentées par les expressions
suivantes: 
\begin{align}
  \wb= \min \int_{t_1}^{t_2} f_i(b_i(t))dt & & \text{sous 
                                               \eqref{tw_CECSP}-\eqref{nrj_CECSP}}\\
  \bb= \min \int_{t_1}^{t_2} b_i(t)dt & & \text{sous 
                                          \eqref{tw_CECSP}-\eqref{nrj_CECSP}} \label{eq:minConso}
\end{align}

Comme pour le cas du \CUSP, la fonction de marge, notée $SL(t_1,t_2)$,
permet de mesurer l'écart entre la quantité de ressource disponible et
les consommations minimales de toutes les tâches dans l'intervalle
${[}t_1,t_2{]}$. Cette fonction est définie de la manière suivante:
\[ SL(t_1,t_2)=R(t_2-t_1)-\sum\limits_{i \in A} \bb \]

Ceci nous permet d'énoncer la condition nécessaire d'existence
d'une solution qui est à la base de l'algorithme de vérification du
raisonnement énergétique:

\begin{theo}
  \label{th:ER_CECSP}
  Soit $\I$ une instance du \CECSP. S'il existe $t_1 < t_2 \in
  \mathbb{R}^2$ tel que $SL(t_1,t_2) <0$ alors $\I$ ne peut pas avoir
  de solution.
\end{theo}

\begin{proof}
  Par l'absurde, supposons qu'il existe $t_1 < t_2 \in \mathbb{R}^2$ tel
  que $SL(t_1,t_2) <0$ et que l'instance $\I$ soit satisfiable. Par
  définition, $\bb$ est la quantité de ressource minimale que doit
  consommer l'activité $i$ dans l'intervalle $[t_1,t_2{[}$. 

  Donc, dans toute solution réalisable, nous avons: 
  \begin{align*}
    & \int_{t_1}^{t_2} b_i(t)dt \ge \bb\\
    \Rightarrow  & \sum_{i \in \A} \int_{t_1}^{t_2} b_i(t)dt \ge \sum_{i
                   \in \A}  \bb > R(t_2-t_1)
  \end{align*}
  Et ceci contredit le fait que $\sum_{i \in \A} b_i(t) \le
  R(t_2-t_1)$. 
\end{proof}

Dans un premier temps, nous allons nous intéresser au calcul de $\wb$,
le calcul de $\bb$ sera détaillé dans un second temps. 


\subsubsection{{\'E}nergie minimale dans un intervalle}

Pour calculer $\wb$, nous analysons les différentes configurations de
la consommation minimale d'une activité. Remarquons que les
configurations conduisant à une consommation minimale dans
l'intervalle $[t_1,t_2{[}$ sont celles où l'activité est ordonnancée à
$\bmax$ à l'extérieur de cet intervalle. Ces configurations, décrites
dans la figure~\ref{fig:conso_CECSP}, peuvent être regroupées en trois
catégories:
\begin{itemize}
\item l'activité est {\it calée à
    gauche} (figure~\ref{fig:conso_CECSPa}, \ref{fig:conso_CECSPb},
  \ref{fig:conso_CECSPd} et \ref{fig:conso_CECSPg}): l'activité démarre
  à $\ES$ et est ordonnancée à $\bmax$ pendant l'intervalle
  $[\ES,t_1{[}$;
\item l'activité est {\it calée à
    droite} (figure~\ref{fig:conso_CECSPb}, \ref{fig:conso_CECSPc},
  \ref{fig:conso_CECSPf} et \ref{fig:conso_CECSPi}): l'activité finit à
  $\LE$ et est ordonnancée à $\bmax$ pendant l'intervalle $[t_2,\LE{[}$;
\item l'activité est {\it centrée} (figure~\ref{fig:conso_CECSPe} et
  \ref{fig:conso_CECSPh}): l'activité occupe tout l'intervalle
  $[t_1,t_2[$, soit en étant ordonnancée à $\bmax$ pendant l'intervalle
  $[\ES,t_1{[} \cup [t_2,\LE{[}$, soit en étant ordonnancée à $\bmin$
  durant tout l'intervalle $[t_1,t_2{[}$.

  En effet, lorsque l'activité est ordonnancée à $\bmax$ pendant
  l'intervalle $[\ES,t_1{[} \cup [t_2,\LE{[}$, il peut arriver que la
  quantité d'énergie restant à apporter à l'activité durant l'intervalle
  $[t_1,t_2[$ ne soit pas suffisante pour assurer la satisfaction de la
  contrainte de consommation minimale~\eqref{req_CECSP}. Le cas où
  l'activité est ordonnancée à $\bmin$ durant tout l'intervalle
  $[t_1,t_2{[}$ doit donc être considéré.
\end{itemize}

\begin{figure}[!htb]  
  \centering
  \subcaptionbox{\label{fig:conso_CECSPa}}[0.3\linewidth]{
    \begin{tikzpicture}
      [xscale=0.37,yscale=0.3]
      \node[] (O) at (0,0) {};
      \node[label={[shift={(-0.4,-0.4)}]$\bmin$}] (bmin) at (0,1) {};
      \node[label={[shift={(-0.4,-0.4)}]$\bmax$}] (bmax) at (0,4) {};
      \node (t1) at (6,0) {}; 
      \node[label={[shift={(0,-0.8)}]$\ES$}] (ri) at (1,0) {};
      \node (t2) at (7,0) {};

      \draw[->] (O.center) -- (8,0)node[below] {$t$};
      \draw (O.south) -- (bmax.north);
      \draw (bmin.center) -- (8,1);
      \draw (bmax.center) -- (8,4);
      \draw(ri.south) -- (ri.center);
      \draw[fill=white] (ri.center) rectangle (5,4);
      \draw[pattern=north west lines] (ri.center) rectangle (5,4);
      \draw[thick] (t1.south) -- (6,4.1) node[above] {$t_1$};
      \draw[thick] (t2.south) -- (7,4.1) node[above] {$t_2$};
    \end{tikzpicture}}
  \hfill
  \subcaptionbox{\label{fig:conso_CECSPb}}[0.3\linewidth]{
    \begin{tikzpicture}
      [xscale=0.37,yscale=0.3]
      \node[] (O) at (0,0) {};
      \node[label={[shift={(-0.4,-0.4)}]$\bmin$}] (bmin) at (0,1) {};
      \node[label={[shift={(-0.4,-0.4)}]$\bmax$}] (bmax) at (0,4) {};
      \node (t1) at (1,0) {}; 
      \node[label={[shift={(0,-0.8)}]$\ES$}] (ri) at (2,0) {};
      \node (t2) at (7,0) {};
      \node[label={[shift={(0,-0.8)}]$\LE$}] (di) at (6,0) {};

      \draw[->] (O.center) -- (8,0)node[below] {$t$};
      \draw (O.south) -- (bmax.north);
      \draw (bmin.center) -- (8,1);
      \draw (bmax.center) -- (8,4);
      \draw(ri.south) -- (ri.center);
      \draw(di.south) -- (di.center);
      \draw[fill=white] (2.5,0) rectangle (5.5,4);
      \draw[pattern=north west lines] (2.5,0) rectangle (5.5,4);
      \draw[thick] (t1.south) -- (1,4.1) node[above] {$t_1$};
      \draw[thick] (t2.south) -- (7,4.1) node[above] {$t_2$};
    \end{tikzpicture}}
  \hfill
  \subcaptionbox{\label{fig:conso_CECSPc}}[0.3\linewidth]{
    \begin{tikzpicture}
      [xscale=0.37,yscale=0.3]
      \node[] (O) at (0,0) {};
      \node[label={[shift={(-0.4,-0.4)}]$\bmin$}] (bmin) at (0,1) {};
      \node[label={[shift={(-0.4,-0.4)}]$\bmax$}] (bmax) at (0,4) {};
      \node (t1) at (0.5,0) {};
      \node (t2) at (1.5,0) {};
      \node[label={[shift={(0,-0.8)}]$\LE$}] (di) at (7,0) {};
      
      \draw[->] (O.center) -- (8,0)node[below] {$t$};
      \draw (O.south) -- (bmax.north);
      \draw (bmin.center) -- (8,1);
      \draw (bmax.center) -- (8,4);
      \draw[fill=white] (3,0) rectangle (7,4);
      \draw[pattern=north west lines] (3,0) rectangle (7,4);
      \draw(di.south) -- (di.center);
      \draw[thick] (t1.south) -- (0.5,4.1) node[above] {$t_1$};
      \draw[thick] (t2.south) -- (1.5,4.1) node[above] {$t_2$};
    \end{tikzpicture}}


  \subcaptionbox{\label{fig:conso_CECSPd}}[0.3\linewidth]{
    \begin{tikzpicture}
      [xscale=0.37,yscale=0.3]
      \node[] (O) at (0,0) {};
      \node[label={[shift={(-0.4,-0.4)}]$\bmin$}] (bmin) at (0,1) {};
      \node[label={[shift={(-0.4,-0.4)}]$\bmax$}] (bmax) at (0,4) {};
      \node (t1) at (4,0) {};
      \node (t2) at (7,0) {};
      \node[label={[shift={(0,-0.8)}]$\LE$}] (di) at (6,0) {};
      \node[label={[shift={(0,-0.8)}]$\ES$}] (ri) at (1,0) {};
      
      \draw[->] (O.center) -- (8,0)node[below] {$t$};
      \draw (O.south) -- (bmax.north);
      \draw (bmin.center) -- (8,1);
      \draw (bmax.center) -- (8,4);
      \draw[fill=white] (ri.center) rectangle (4,4);
      \draw[pattern=north west lines] (ri.center) rectangle (4,4);
      \draw[fill=white] (t1.center) rectangle (5,1.5);
      \draw[pattern=north west lines] (t1.center) rectangle (5,1.5);
      \draw(di.south) -- (di.center);
      \draw(ri.south) -- (ri.center);
      \draw[thick] (t1.south) -- (4,4.1) node[above] {$t_1$};
      \draw[thick] (t2.south) -- (7,4.1) node[above] {$t_2$};
    \end{tikzpicture}}
  \hfill
  \subcaptionbox{\label{fig:conso_CECSPe}}[0.3\linewidth]{
    \begin{tikzpicture}
      [xscale=0.37,yscale=0.3]
      \node[] (O) at (0,0) {};
      \node[label={[shift={(-0.4,-0.4)}]$\bmin$}] (bmin) at (0,1) {};
      \node[label={[shift={(-0.4,-0.4)}]$\bmax$}] (bmax) at (0,4) {};
      \node (t1) at (2.5,0) {}; 
      \node[label={[shift={(0,-0.8)}]$\ES$}] (ri) at (1.5,0) {};
      \node (t2) at (6,0) {};
      \node[label={[shift={(0,-0.8)}]$\LE$}] (di) at (7,0) {};
      
      \draw[->] (O.center) -- (8,0)node[below] {$t$};
      \draw (O.south) -- (bmax.north);
      \draw (bmin.center) -- (8,1);
      \draw (bmax.center) -- (8,4);
      \draw(ri.south) -- (ri.center);
      \draw(di.south) -- (di.center);
      \draw[fill=white] (2.5,0) rectangle (1.5,4);
      \draw[pattern=north west lines] (2.5,0) rectangle (1.5,4);
      \draw[fill=white] (2.5,0) rectangle (6,2);
      \draw[pattern=north west lines] (2.5,0) rectangle (6,2);
      \draw[fill=white] (6,0) rectangle (7,4);
      \draw[pattern=north west lines] (6,0) rectangle (7,4);
      \draw[thick] (t1.south) -- (2.5,4.1) node[above] {$t_1$};
      \draw[thick] (t2.south) -- (6,4.1) node[above] {$t_2$};
    \end{tikzpicture}}
  \hfill
  \subcaptionbox{\label{fig:conso_CECSPf}}[0.3\linewidth]{
    \begin{tikzpicture}
      [xscale=0.37,yscale=0.3]
      \node[] (O) at (0,0) {};
      \node[label={[shift={(-0.4,-0.4)}]$\bmin$}] (bmin) at (0,1) {};
      \node[label={[shift={(-0.4,-0.4)}]$\bmax$}] (bmax) at (0,4) {};
      \node[label={[shift={(0,-0.8)}]$\LE$}] (di) at (7,0) {};
      \node (t1) at (1,0) {}; 
      \node[label={[shift={(0,-0.8)}]$\ES$}] (ri) at (2.5,0) {};
      \node (t2) at (4,0) {};
      
      \draw[->] (O.center) -- (8,0)node[below] {$t$};
      \draw (O.south) -- (bmax.north);
      \draw (bmin.center) -- (8,1);
      \draw (bmax.center) -- (8,4);
      \draw(di.south) -- (di.center);
      \draw(ri.south) -- (ri.center);
      \draw[fill=white] (t2.center) rectangle (7,4);
      \draw[pattern=north west lines] (t2.center) rectangle (7,4);
      \draw[fill=white] (t2.center) rectangle (3.2,2);
      \draw[pattern=north west lines] (t2.center) rectangle (3.2,2);
      \draw[thick] (t1.south) -- (1,4.1) node[above] {$t_1$};
      \draw[thick] (t2.south) -- (4,4.1) node[above] {$t_2$};
    \end{tikzpicture}}



  \subcaptionbox{\label{fig:conso_CECSPg}}[0.3\linewidth]{
    \begin{tikzpicture}
      [xscale=0.37,yscale=0.3]
      \node[] (O) at (0,0) {};
      \node[label={[shift={(-0.4,-0.4)}]$\bmin$}] (bmin) at (0,1) {};
      \node[label={[shift={(-0.4,-0.4)}]$\bmax$}] (bmax) at (0,4) {};
      \node (t1) at (3,0) {};
      \node (t2) at (5,0) {};
      \node[label={[shift={(0,-0.8)}]$\LE$}] (di) at (7,0) {};
      \node[label={[shift={(0,-0.8)}]$\ES$}] (ri) at (1,0) {};
      
      \draw[->] (O.center) -- (8,0)node[below] {$t$};
      \draw (O.south) -- (bmax.north);
      \draw (bmin.center) -- (8,1);
      \draw (bmax.center) -- (8,4);
      \draw[fill=white] (t1.center) rectangle (1,4);
      \draw[pattern=north west lines] (t1.center) rectangle (1,4);
      \draw[fill=white] (t1.center) rectangle (3.7,1);
      \draw[pattern=north west lines] (t1.center) rectangle (3.7,1);
      \draw(di.south) -- (di.center);
      \draw(ri.south) -- (ri.center);
      \draw[thick] (t1.south) -- (3,4.1) node[above] {$t_1$};
      \draw[thick] (t2.south) -- (5,4.1) node[above] {$t_2$};
    \end{tikzpicture}}
  \hfill
  \subcaptionbox{\label{fig:conso_CECSPh}}[0.3\linewidth]{
    \begin{tikzpicture}
      [xscale=0.37,yscale=0.3]
      \node[] (O) at (0,0) {};
      \node[label={[shift={(-0.4,-0.4)}]$\bmin$}] (bmin) at (0,1) {};
      \node[label={[shift={(-0.4,-0.4)}]$\bmax$}] (bmax) at (0,4) {};
      \node (t1) at (2.5,0) {}; 
      \node[label={[shift={(0,-0.8)}]$\ES$}] (ri) at (1.5,0) {};
      \node (t2) at (6,0) {};
      \node[label={[shift={(0,-0.8)}]$\LE$}] (di) at (7,0) {};

      \draw[->] (O.center) -- (8,0)node[below] {$t$};
      \draw (O.south) -- (bmax.north);
      \draw (bmin.center) -- (8,1);
      \draw (bmax.center) -- (8,4);
      \draw(ri.south) -- (ri.center);
      \draw(di.south) -- (di.center);
      \draw[fill=white] (2.5,0) rectangle (2,2.7);
      \draw[pattern=north west lines] (2.5,0) rectangle (2,2.7);
      \draw[fill=white] (2.5,0) rectangle (6,1);
      \draw[pattern=north west lines] (2.5,0) rectangle (6,1);
      \draw[fill=white] (6,0) rectangle (7,4);
      \draw[pattern=north west lines] (6,0) rectangle (7,3);
      \draw[thick] (t1.south) -- (2.5,4.1) node[above] {$t_1$};
      \draw[thick] (t2.south) -- (6,4.1) node[above] {$t_2$};
    \end{tikzpicture}}
  \hfill
  \subcaptionbox{\label{fig:conso_CECSPi}}[0.3\linewidth]{
    \begin{tikzpicture}
      [xscale=0.37,yscale=0.3]
      \node[] (O) at (0,0) {};
      \node[label={[shift={(-0.4,-0.4)}]$\bmin$}] (bmin) at (0,1) {};
      \node[label={[shift={(-0.4,-0.4)}]$\bmax$}] (bmax) at (0,4) {};
      \node[label={[shift={(0,-0.8)}]$\LE$}] (di) at (7,0) {};
      \node (t1) at (2.5,0) {}; 
      \node (t2) at (4,0) {};
      \node[label={[shift={(0,-0.8)}]$\ES$}] (ri) at (1,0) {};
      
      \draw[->] (O.center) -- (8,0)node[below] {$t$};
      \draw (O.south) -- (bmax.north);
      \draw (bmin.center) -- (8,1);
      \draw (bmax.center) -- (8,4);
      \draw(di.south) -- (di.center);
      \draw(ri.south) -- (ri.center);
      \draw[fill=white] (t2.center) rectangle (7,4);
      \draw[pattern=north west lines] (t2.center) rectangle (7,4);
      \draw[fill=white] (t2.center) rectangle (3.2,1);
      \draw[pattern=north west lines] (t2.center) rectangle (3.2,1);
      \draw[thick] (t1.south) -- (2.5,4.1) node[above] {$t_1$};
      \draw[thick] (t2.south) -- (4,4.1) node[above] {$t_2$}; 
    \end{tikzpicture}}
  \caption{Les différentes configurations menant à une consommation
    minimale à l'intérieur de $[t_1,t_2[$ pour le \CECSP.}
  \label{fig:conso_CECSP}
\end{figure}

Il est facile de calculer l'expression de la consommation minimale
d'énergie dans un intervalle pour une fonction $f_i$ croissante. En
effet, les différentes configurations possibles étant toujours celles 
où l'activité est exécutée à son rendement maximum en dehors de
l'intervalle ${[}t_1,t_2{[}$, il suffit de retrancher à $W_i$
l'énergie produite par l'exécution de l'activité à $\bmax$ en dehors
de ${[}t_1,t_2{[}$. Il existe une exception à cette règle, produite
par la contrainte de consommation minimale, mais ce cas est facilement
traité puisque l'énergie minimale correspond alors à la configuration
où la tâche est exécutée à son rendement minimal durant
${[}t_1,t_2{[}$.

Pour donner l'expression mathématique de $\wb$, nous introduisons
trois notations. $\wbLS$ (respectivement $\wbRS$ et $\wbCS$)
correspond à la quantité d'énergie apportée à l'activité $i$ dans
l'intervalle $[t_1,t_2[$ quand l'activité est calée à gauche
(respectivement calée à droite et centrée). Formellement, ces trois
quantités peuvent être exprimées de la manière suivante:
\begin{align}
  \wbLS&= \max\left(0\, ,\, W_i- f_i(\bmax)\max(0,t_1 -
         \ES)\right) \label{eq:LSnrj_CECSP}\\
  \wbRS&= \max\left(0\, ,\, W_i- f_i(\bmax)\max(0,
         \LE-t_2)\right)\label{eq:RSnrj_CECSP}\\
  \wbCS&=\max\left( f_i(\bmin) * (t_2-t_1)\, ,\, W_i -
         f_i(\bmax)|[\ES, \LE{]}\setminus[t_1,t_2[|\right)\label{eq:CSnrj_CECSP}
\end{align}
Alors, l'expression de l'énergie minimale est le minimum de ces trois
quantités, i.e.
\begin{equation}
  \wb=\min\left(\, \wbLS\, ,\,\wbRS\, ,\,\wbCS \, \right)
\end{equation} 

Nous allons maintenant utiliser l'expression de $\wb$ et la fonction
$f_i$ pour calculer $\bb$.

\subsubsection{Consommation minimale de la ressource}

Pour calculer l'expression de $\bb$, nous allons utiliser les
propriétés de la fonction $f_i$. En effet, une fois que nous avons
calculé $\wb$, nous voulons savoir quelle est la quantité minimale de
ressource que nous devons fournir à la tâche pour obtenir cette
quantité d'énergie dans l'intervalle ${[}t_1,t_2{[}$. Dans le cas où
la fonction $f_i$ est l'identité, nous avons que: $\bb=\wb$. Dans les
deux autres cas, i.e. $f_i$ est affine et $f_i$ est concave et affine
par morceaux, soit $I=[t_1,t_2[ \cap [\ES,\LE{[}$, alors trouver $\bb$
revient à résoudre le programme suivant:
\begin{align}
  \text{minimiser }   & \int_{I} b_i(t)dt \label{eq:conv_obj} \\
  \text{sous } & \int_{I} f_i(b_i(t))dt \ge \wb \label{eq:conv_nrj}\\
                      & \bmin \le b_i(t) \le \bmax \label{eq:conv_min}
\end{align}
En effet, l'objectif de ce programme est de minimiser la quantité de
ressource consommée dans l'intervalle $[t_1,t_2[$
(équation~\eqref{eq:conv_obj}), tout en s'assurant que l'énergie
requise, i.e. $\wb$, est bien apportée à l'activité
(équation~\eqref{eq:conv_nrj}). 

Nous utilisons ensuite le lemme~\ref{lemmaEn} pour simplifier ce
programme. En effet, le lemme affirme que, si $f_i$ est affine ou
concave et affine par morceaux, alors si une solution optimale
$b_i(t)$ pour le programme ci-dessus existe, cette solution peut être
transformée en une autre solution optimale vérifiant la propriété que
$b_i(t)$ soit constante et inférieure ou égale à
$\frac{\int_{I}b_i(t)dt}{|I|}$. Nous noterons $b_i$ cette
constante. Le programme simplifié s'écrit de la manière suivante:
\begin{align}
  \text{minimiser }   & b_i|I| \label{eq:convSimp_obj} \\
  \text{sous } & f_i(b_i)|I| \ge \wb \label{eq:convSimp_nrj}\\
                      &\bmin \le  b_i \le \bmax \label{eq:convSimp_nrj}
\end{align}


Ce programme peut être réécrit de la façon suivante: 
\[ b_i=\min\left( b \in [\bmin,\bmax]\ |\ f_i(b) \ge
    \frac{\wb}{|I|}\right)
\]

Nous pouvons remarquer que, si l'on suppose que $f_i(\bmax) (\LE - \ES
) \ge W_i$, nous sommes sûrs que la solution optimale vérifie $b_i \le
\bmax$. En effet, ceci est dû au fait qu'exécuter l'activité à une
faible consommation de ressource a un meilleur rendement que son
exécution à un rendement plus élevé, i.e. la fonction
$\frac{f_i(b_i(t))}{b_i(t)}$ est décroissante. 

De ce fait, seule la borne inférieure sur $b_i$, $\bmin$, doit être
considérée. Or, pour apporter l'énergie $\wb$ à l'activité dans
l'intervalle $I$, sa consommation $b_i$ doit vérifier $f_i(b_i) \ge
\frac{\wb}{|I|}$ et donc $b_i \ge
f_i^{-1}\left(\frac{\wb}{|I|}\right)$. En couplant cette contrainte
avec la contrainte de consommation minimale, nous obtenons, si $\bmin
\neq 0$:
\[
  b_i= \max\left(\, \bmin \, ,\, 
    f_i^{-1}\left(\frac{\wb}{|I|}\right)\, \right)
\]
et si $\bmin = 0$, nous avons:
\[
  b_i=  f_i^{-1}\left(\frac{\wb}{|I|}\right)
\]
Nous allons maintenant donner l'expression de $\bb$  en fonction des
coefficients $a_{ip}$  et $c_{ip}$ de la fonction $f_i$. Pour
simplifier la compréhension, nous commençons par détailler cette
expression dans le cas où $f_i$ est affine puis nous la généraliserons
dans le cas où $f_i$ est concave et affine par morceaux. 

\paragraph{Fonctions affines}

Dans ce paragraphe, nous allons décrire l'expression de $\bb$ dans le
cas où la fonction $f_i$ est affine, i.e. de la forme $a_ib_i(t)+c_i$. Dans
ce cas-là, nous avons:  
\[f_i^{-1}\left( \frac{\wb}{|I|}\right)= \frac{ \wb- c_i|I|}{a_i|I|}
\]
et donc, si $\bmin\neq 0$: 
\begin{equation}
  \bb= \max\left(b_i^{min} \frac{\wb}{f_i(b_i^{min})} \, ,\,  
    \frac{1}{a_i}\left(\wb-|I|c_{i})\right)\right)
\end{equation}
et si $\bmin=0$: 
\[
  \bb=  \frac{1}{a_i}\left(\wb-|I|c_{i})\right)
\]
Notons que le premier cas correspond au cas où l'intervalle $I$ est
suffisamment grand pour ordonnancer l'activité $i$ à $\bmin$ et lui
apporter l'énergie requise, i.e. $\wb$. En effet, comme ordonnancer
$i$ à $\bmin$ a le meilleur rendement, si l'on peut, i.e. l'intervalle
est assez grand, c'est cette valeur que l'on va choisir pour
$b_i$. Si, au contraire, l'intervalle n'est pas suffisamment grand,
c'est $f_i^{-1}(\wb/|I|) $ que l'on va choisir. Ceci est illustré dans
l'exemple~\ref{ex:convLin_CECSP}. 

\begin{ex}
  \label{ex:convLin_CECSP}
  Considérons les deux activités suivantes (voir figure~\ref{fig:ex_Lin}): 
  \begin{center}
  \begin{tabular}{|P{1cm}P{1cm}P{1cm}P{1cm}P{1cm}P{1cm}P{2cm}|}
    \hline
    act. & \ES & \LE & W_i & \bmin & \bmax & f_i(b_i(t))\\
    \hline
    i & 2 & 8 & 21 & 2 & 5 & b_i(t)+3\\
    j & 0 & 8 & 22 & 2 & 5 & \frac{1}{2}*b_i(t)+\frac{1}{2}\\
    \hline
  \end{tabular}
\end{center}

Si nous calculons $\wb$ et $\bb$ sur l'intervalle $[1,6[$, nous
obtenons: 
\begin{itemize}
\item $\wb[i][1][6] = \wbRS[i][1][6]  = 5$ et 
\item $\bb[i][1][6]  = \max (\bmin \frac{ \wb[i][1][6] }{f_i(\bmin)}, \frac{1}{a_i} ( \wb[i][1][6] 
  - c_i |I|) =\max (2 \frac{5}{5}, \frac{1}{1} ( 5
  - 3 \times 4) = \max ( 2 , -7) = 2 $
\item $\wb[j][1][6] = \wbCS[j][1][6] = 10$ et 
\item $\bb[j][1][6] = \max (\bmin \frac{ \wb[j][1][6] }{f_i(\bmin)},
\frac{1}{a_i} ( \wb[j][1][6] - c_i |I|) =\max (2
\frac{10}{\frac{3}{2}}, 2 \times ( 10 - \frac{1}{2} \times 5) = \max (
\frac{40}{3} , 15) = 15 $
\end{itemize}
\begin{figure}[!htb]
  \centering
  \subcaptionbox{Activité $i$}[0.45\linewidth]{
    \begin{tikzpicture}
      [xscale=0.5, yscale= 0.4,node distance=0.5cm]
      \fill[gray!20] (1,6) node[above,black] {$t_1$} +(0,-6)  rectangle (6,6) node[above,black] {$t_2$};
      \node (sil) at (1,0) {} ;
      \node (eil) at (8,0) {} ;
      \node [below of=eil,node distance=0.63cm]  {$\LE$};
      \draw (sil.center) node[below=0.2cm] {$\ES$};
      
      \draw (-1,0) -- (9,0);
      
      \draw[->] (-1,0) -- (-1,6);
      \draw[dashed] (10, 2) -- (1,2) -- (-1.1,2) node[left] {$\bmin$};
      \draw[dashed] (10, 5) -- (1,5) -- (-1.1,5) node[left] {$\bmax$};
      \path[draw] (8,0) -- (8,5) -- (6,5) -- (6,2) node[right=0.5cm, above]{$i$} -- (5,2) -- (5,0);


      \foreach \i in {0,...,9} {
        \draw (\i,0)  -- (\i,-0.2);
      }
    \end{tikzpicture}
}
\hfill
\subcaptionbox{Activité $j$}[0.45\linewidth]{
    \begin{tikzpicture}
      [xscale=0.5, yscale= 0.4,node distance=0.5cm]      
      \fill[gray!20] (1,6) node[above,black] {$t_1$} +(0,-6)  rectangle (6,6) node[above,black] {$t_2$};
      \node (sil) at (0,0) {} ;
      \node (eil) at (8,0) {} ;
      \node [below of=eil,node distance=0.63cm]  {$\LE$};
      \draw (sil.center) node[below=0.2cm] {$\ES$};
      
      \draw (-1,0) -- (9,0);
      
      \draw[->] (-1,0) -- (-1,6);
      \draw[dashed] (10, 2) -- (-1,2) -- (-1.1,2) node[left] {$\bmin$};
      \draw[dashed] (10, 5) -- (-1,5) -- (-1.1,5) node[left] {$\bmax$};
      \path[draw] (8,0) -- (8,5) -- (6,5) -- (6,3)  -- (1,3) node[midway,below]
     {$j$} -- (1,5)
      -- (0,5) -- (0,0);


      \foreach \i in {0,...,9} {
        \draw (\i,0)  -- (\i,-0.2);
      }
    \end{tikzpicture}
}
  \caption{Consommation de ressource dans $[t_1,t_2[$ pour le
    \CECSP~avec fonction de rendement affine.}
\label{fig:ex_Lin}
\end{figure}
\end{ex}

Ce raisonnement peut être étendu dans le cas où $f_i$ est concave et
affine par morceaux. 

\paragraph{Fonctions concave et affine par morceaux}

Dans le cas des fonctions concaves et affines par morceaux, le
raisonnement décrit au paragraphe précédent peut être généralisé. En
effet, comme dans le cas des fonctions affines, la fonction de
rendement, $f_i(b_i(t))/b_i(t)$, est décroissante. Donc, il est
toujours préférable d'exécuter l'activité avec une consommation de
ressource aussi basse que possible. La condition selon laquelle on
peut ou non exécuter l'activité $i$ avec une consommation $b_i$ dépend
donc de la taille de l'intervalle $I$. 

Si $\bmin\neq 0$ et si l'intervalle est suffisamment grand, on va donc
exécuter l'activité à $\bmin$. Sinon, et si l'intervalle est
suffisamment grand, nous essayons d'exécuter l'activité avec une
consommation $b_i \in ]\bmin,x_2^i]$, etc. Nous rappelons que les
points $x_\ell^i$ correspondent aux points de rupture de la fonction
$f_i$. Dans le cas où $\bmin \neq 0$, l'expression de $\bb$ est
formalisée ci-dessous:
\[f_i^{-1}\left(\frac{\wb}{|I|}\right)=\left\{
    \begin{aligned}
      \bmin & \qquad & \text{si } |I| \ge \frac{\wb}{f_i(\bmin)}\\
      \frac{\wb -c_{i1}|I|}{a_{i1}|I|} & \qquad & \text{si } \frac{\wb}{f_i(\bmin)}
      > |I| > \frac{\wb}{f_i(x^i_2)} \\
      \frac{\wb -c_{i2}|I|}{a_{i2}|I|} & \qquad & \text{si }  \frac{\wb}{f_i(x^i_2)}
      \ge |I| > \frac{\wb}{f_i(x^i_3)}\\
      & \qquad \vdots& \\
      \frac{\wb -c_{iP_i}|I|}{a_{iP_i}|I|} & \qquad & \text{si }  \frac{\wb}{f_i(x^i_{P_i})}
      \ge |I| > \frac{\wb}{f_i(\bmax)}\end{aligned}
  \right.
\]
Et ceci nous donne l'expression de $\bb$ suivante:
\begin{equation}
  \bb= 
  \max \left\{b_i^{min} \frac{\wb}{f_i(b_i^{min})},
    \max_{ p \in \P_i} \left(\frac{1}{a_{ip}}(\wb-|I|c_{ip})\right)\right\}
\label{eq:conv_CECSP}
\end{equation}
Dans le cas où $\bmin=0$, nous avons: 
\[
  \bb= 
  \max_{ p \in \P_i} \left(\frac{1}{a_{ip}}(\wb-|I|c_{ip})\right)
\]
Un exemple du calcul de $\bb$ dans le cas d'une fonction $f_i$ concave
et linéaire par morceaux est décrit dans
l'exemple~\ref{ex:convPWL_CECSP}.

\begin{ex}
  \label{ex:convPWL_CECSP}
  Considérons l'activité suivante (voir figure~\ref{fig:ex_PWL}): 
  \begin{center}
  \begin{tabular}{|P{1cm}P{1cm}P{1cm}P{1cm}P{1cm}P{1cm}P{3cm}|}
    \hline
    act. & \ES & \LE & W_i & \bmin & \bmax & f_i(b)\\
    \hline
    i & 2 & 8 & 70 & 2 & 5 & 3b+1 \text{ si } b \in [2,3]\\
    &  & &  &  &  & 2b+4 \text{ si } b \in ]3,4]\\
    &  & &  &  &  & b+8 \text{ si } b \in ]4,5]\\
    \hline
  \end{tabular}
\end{center}

Si nous calculons $\wb$ et $\bb$ sur l'intervalle $[2,6[$, nous
obtenons: 
\begin{itemize}
\item $\wb[i][2][6] = \wbRS[i][2][6]  = 44$ et 
\item $\begin{aligned}[t]
        \bb[i][2][6]  &= 
  \max (\bmin \frac{ \wb[i][2][6] }{f_i(\bmin)},
  \max_{p  \in \{1,2,3\}} \frac{1}{a_{ip}} ( \wb[i][2][6] 
  - c_{ip} |I|)) \\
  &=\max  \left(2 \frac{44}{7}, \max\left\{\frac{1}{3} ( 44
  - 4) , \frac{1}{2} (44  - 4 \times 4 ), \frac{1}{1} (44  - 8 \times 4 )
\right\}\right) \\ &=  
\max ( \frac{88}{7},
\max\{\frac{40}{3}, 14, 12 \}) \\ &= 14 \end{aligned}$

~

\end{itemize}
Dans ce cas, c'est la seconde partie de la fonction $f_i$ qui est
utilisée, i.e. $f_i(b)=2b+4 \text{ si } b \in ]3,4]$.

Si maintenant nous calculons $\wb$ et $\bb$ sur l'intervalle $[4,6[$,
nous obtenons: 
\begin{itemize}
\item $\wb[i][4][6] = \wbCS[j][4][6] =18$ et 

\item $\begin{aligned}[t]
  \bb[i][4][6] &=\max (\bmin \frac{ \wb[i][4][6] }{f_i(\bmin)},
  \max_{p  \in \{1,2,3\}} \frac{1}{a_{ip}} ( \wb[i][4][6] 
  - c_{ip} |I|)) \\
  &=\max  \left(2 \frac{18}{7}, \max\left\{\frac{1}{3} ( 18
      - 2) , \frac{1}{2} (18  - 4 \times 2), \frac{1}{1} (18  - 8 \times 2 )
    \right\}\right) \\ &=  
  \max ( \frac{36}{7},
  \max\{\frac{16}{3}, 5 , 2 \}) \\ &= \frac{16}{3} 
\end{aligned}$

~

\end{itemize}
Et dans ce cas, c'est la première partie de la fonction $f_i$ qui est
utilisée, i.e. $f_i(b)=3b+1 \text{ si } b \in ]2,3]$.

\begin{figure}[!htb]
  \centering
  \subcaptionbox{Consommation minimale sur $[2,6[$}[0.45\linewidth]{
    \begin{tikzpicture}
      [xscale=0.5, yscale= 0.4,node distance=0.5cm]
      \fill[gray!20] (2,6) node[above,black] {$t_1$} +(0,-6)  rectangle (6,6) node[above,black] {$t_2$};
      \node (sil) at (2,0) {} ;
      \node (eil) at (8,0) {} ;
      \node [below of=eil,node distance=0.63cm]  {$\LE$};
      \draw (sil.center) node[below=0.2cm] {$\ES$};
      
      \draw (-1,0) -- (9,0);
      
      \draw[->] (-1,0) -- (-1,6);
      \draw[dashed] (10, 2) -- (-1,2) -- (-1.1,2) node[left] {$\bmin$};
      \draw[dashed] (10, 5) -- (-1,5) -- (-1.1,5) node[left] {$\bmax$};
      \path[draw] (8,0) -- (8,5) -- (6,5) -- (6,3.5)  -- (2,3.5)
      node[midway, below=0.5cm]{$i$} -- (2,0) ;


      \foreach \i in {0,...,9} {
        \draw (\i,0)  -- (\i,-0.2);
      }
    \end{tikzpicture}
}
\hfill
\subcaptionbox{Consommation minimale sur $[4,6[$}[0.45\linewidth]{
    \begin{tikzpicture}
      [xscale=0.5, yscale= 0.4,node distance=0.5cm]      
      \fill[gray!20] (4,6) node[above,black] {$t_1$} +(0,-6)  rectangle (6,6) node[above,black] {$t_2$};
      \node (sil) at (2,0) {} ;
      \node (eil) at (8,0) {} ;
      \node [below of=eil,node distance=0.63cm]  {$\LE$};
      \draw (sil.center) node[below=0.2cm] {$\ES$};
      
      \draw (-1,0) -- (9,0);
      
      \draw[->] (-1,0) -- (-1,6);
      \draw[dashed] (10, 2) -- (-1,2) -- (-1.1,2) node[left] {$\bmin$};
      \draw[dashed] (10, 5) -- (-1,5) -- (-1.1,5) node[left] {$\bmax$};
      \path[draw] (8,0) -- (8,5) -- (6,5) -- (6,2.67)  -- (4,2.67) node[midway,below=0.2cm]
     {$i$} -- (4,5) -- (2,5) -- (2,0);
      -- (0,5) -- (0,0);


      \foreach \i in {0,...,9} {
        \draw (\i,0)  -- (\i,-0.2);
      }
    \end{tikzpicture}
}
  \caption{Consommation de ressource dans $[t_1,t_2[$ pour le
    \CECSP~avec fonction de rendement concave et affine par morceaux.}
\label{fig:ex_PWL}
\end{figure}
\end{ex}

\subsection{Les ajustements de bornes}
\label{sec:adjustment_tw}

Dans cette sous-section, nous décrivons les ajustements qui peuvent être
faits sur les fenêtres de temps des activités. Tout d'abord, nous
introduisons les notations suivantes: $\bbLS$ (respectivement $\bbRS$
et $\bbCS$) correspond à la quantité de ressource consommée par
l'activité $i$ dans l'intervalle $[t_1,t_2[$ quand l'activité est
calée à gauche (respectivement calée à droite et
centrée). Formellement, ces trois quantités peuvent être exprimées de
la manière suivante:
\begin{align}
  \bbLS&=   \max \left\{b_i^{min} \frac{\wbLS}{f_i(b_i^{min})},
         \max_{ p \in \P_i} \left(\frac{1}{a_{ip}}(\wbLS-|I|c_{ip})\right)\right\}\\
  \bbRS&=   \max \left\{b_i^{min} \frac{\wbRS}{f_i(b_i^{min})},
         \max_{ p \in \P_i} \left(\frac{1}{a_{ip}}(\wbRS-|I|c_{ip})\right)\right\}\\
  \bbCS&=  \max \left\{b_i^{min} \frac{\wbCS}{f_i(b_i^{min})},
         \max_{ p \in \P_i} \left(\frac{1}{a_{ip}}(\wbCS-|I|c_{ip})\right)\right\}
\end{align}

Nous allons décrire deux règles d'ajustements pour les fenêtres de
temps du \CECSP. La première permet d'ajuster $\LS$ et la seconde
$\ES$;nous avons des ajustements symétriques pour $\EE$ et $\LE$. 

Pour les premiers ajustements décrits, nous allons essayer de faire
démarrer l'activité après $t_1$ et si la quantité de ressource
disponible n’est pas suffisante pour ordonnancer les consommations
minimales de toutes les activités -– excepté celle de l’activité $i$ qui
est remplacée par $\bbRS$~–- alors, nous pouvons déduire que l’activité
doit commencer avant $t_1$.

\begin{reg}
  \label{reg:ajust_CECSP}
  S’il existe un intervalle $[t_1 , t_2 [$ avec $t_1 > \ES$ et une
  activité $i$ pour lesquels:
  \[ \sum_{\substack{j \in \A\\ j\neq i}} \bb[j] + \bbRS > R(t_2-t_1)
  \]
  alors, on a:
  \[ \LS \le t_1 - \frac{1}{\bmax}\left(\sum_{\substack{j \in \A\\ j\neq i}} \bb[j] + \bbRS - R(t_2-t_1)\right)
  \]
\end{reg}

\begin{proof}
  Soient $i$ et $[t_1,t_2[$ vérifiant la condition de la
  règle~\ref{reg:ajust_CECSP}. Supposons que l'activité $i$ démarre
  après   $t_1$. 

  La consommation minimale de $i$ dans $[t_1,t_2[$ est donc $\bbRS$. On
  a donc que $\sum_{\substack{j \in \A\\ j\neq i}} \bb[j] + \bbRS $ est
  la consommation minimale de toutes les activités dans $[t_1,t_2[$
  quand l'activité $i$ commence après $t_1$. Donc, si cette quantité est
  plus grande que la quantité de ressource disponible dans l'intervalle
  $[t_1,t_2[$, i.e. $R(t_2-t_1)$, on obtient une contradiction avec le
  fait que l'instance soit réalisable et $i$ doit commencer avant
  $t_1$. 

  Pour calculer la nouvelle date de début au plus tard de $i$,
  remarquons que la quantité de ressource devant être consommée avant
  $t_1$ est d'au moins $\sum_{\substack{j \in \A\\ j\neq i}} \bb[j] +
  \bbRS - R(t_2-t_1)$, i.e. ce qui ne ``rentrait'' pas dans
  $[t_1,t_2[$. L'objectif étant d'obtenir une borne supérieure sur la date de
  début de $i$, nous cherchons donc à exécuter cette partie de
  l'activité le plus rapidement possible. 

  Le temps minimal requis pour réaliser cette
  consommation étant obtenu en exécutant l'activité à $\bmax$, nous
  obtenons comme borne supérieure sur la date de début de $i$:
  $t_1 - 1/\bmax \times \left(\sum_{\substack{j \in \A\\ j\neq
        i}} \bb[j] + \bbRS - R(t_2-t_1)\right)$
\end{proof}

De manière similaire, nous avons les ajustements suivants sur la date
de début au plus tôt d'une activité. 
\begin{reg}
  \label{reg:ajustRi_CECSP}
  S'il existe un intervalle $[t_1,t_2[$ avec $ t_2 > \LE$ et une
  activité $i$ telle que $\bmin \neq 0$ pour lesquels:
  \[ \sum_{\substack{j \in \A \\ j \neq i}} \bb[j] +
    \min(\bbCS,\bbLS) > R (t_2-t_1)\] 
  alors
  \[ \ES \ge t_2 - \frac{1}{\bmin} \left(R (t_2-t_1) -\sum_{\substack{j
          \in \A \\ j \neq i}} \bb[j] \right) \]
\end{reg}

\begin{proof}
  Soient $i$ et $[t_1,t_2[$ vérifiant la condition de la
  règle~\ref{reg:ajust_CECSP}. Nous allons décider si $i$ peut commencer
  avant $t_1$. Les configurations où l'activité $i$ démarre avant $t_1$
  et consomme le moins de ressource possible dans $[t_1,t_2[$ sont les
  configurations où $i$ est soit calée à gauche, soit centrée.

  La consommation minimale de $i$ dans l'intervalle $[t_1,t_2[$ quand
  $i$ commence avant $t_1$ est donc $\min( \bbLS, \bbCS)$. On a donc que
  $\sum_{\substack{j \in \A\\ j\neq i}} \bb[j] + \min( \bbLS, \bbCS) $
  est la consommation minimale de toutes les activités dans $[t_1,t_2[$
  quand l'activité $i$ commence avant $t_1$. Donc, si cette quantité est
  plus grande que la quantité de ressource disponible dans l'intervalle
  $[t_1,t_2[$, i.e. $R(t_2-t_1)$, on obtient une contradiction avec le
  fait que l'instance soit réalisable et $i$ doit commencer après $t_1$.

  Pour calculer la nouvelle date de début au plus tard de $i$,
  remarquons que la quantité de ressource disponible dans $[t_1,t_2[$
  pour exécuter $i$ est de $R(t_2-t_1) -\sum_{\substack{j \in \A\\
      j\neq i}} \bb[j]$. L'objectif étant d'obtenir une borne inférieure sur la date de
  début de $i$, nous cherchons donc à ordonnancer cette partie de
  l'activité le moins rapidement possible. 

  Le temps maximal requis pour réaliser cette consommation étant
  obtenu en exécutant l'activité à $\bmin$, et, comme $\bmin \neq 0$,
  l'activité ne peut être interrompue, nous savons que l'activité doit
  être en cours à $t_2$. Nous obtenons alors comme borne inférieure sur
  la date de début de $i$:
  $t_2 - 1/\bmin \left(R (t_2-t_1) -\sum_{\substack{j
        \in \A \\ j \neq i}} \bb[j] \right) $
\end{proof}

\begin{ex}
  Considérons l'instance à $3$ activités et avec $R=5$ suivante:

  \begin{center}
    \begin{tabular}{|P{1cm}|P{1cm}P{1cm}P{1cm}P{1cm}P{1cm}P{2cm}|} 
      \hline 
      act. & \ES & \LE & W_i & \bmin& \bmax & f_i(b_i(t))\\ 
      \hline 
      1 & 0 & 6 & 28 & 1 & 5 & 2b_i(t)+1\\ 
      2 & 2 & 6 & 32 & 2 & 5 & b_i(t)+5\\ 
      3 & 2 & 5 & 6 & 2 & 2 & b_i(t)\\ 
      \hline
    \end{tabular}
  \end{center}
  Une solution réalisable est décrite par la figure~\ref{exSol}.
  \begin{figure}[!htb]
    \begin{center} 
      \begin{tikzpicture}
        [scale=0.7]
        \node (O) at (0,0) {};
        \node (2) at (4,2) {\LARGE $2$};
        \node (1) at (1,2) {\LARGE $1$};
        \node (3) at (3.5,4) {\LARGE $3$};
        \node[label={[shift={(-0.4,0)}]$R=5$}] (B) at (0,5) {};


        \node (r1) at (0,-0.5) {$\ES[1]$}; 
        \node (r2) at (2,-0.5) {$\ES[2]$};
        \node (r3) at (2,-0.9) {$\ES[3]$};
        \node (d1) at (6,-0.5) {$\LE[1]$};
        \node (d2) at (6,-0.9) {$\LE[2]$};
        \node (d3) at (5,-0.5) {$\LE[3]$};


        \draw[->,>=latex] (6,0) -- (6.5,0) node[below] {$t$};

        % \draw (0,0) rectangle (6,5);
        \draw (4,0) -- (6,0) -- (6,5) -- (5,5);

        \draw (2,5) -- (2,1) -- (4,1) -- (4,0) -- (0,0) -- (0,5) -- cycle;

        \draw (2,3) -- (5,3) -- (5,5) -- (2,5) ;


        \foreach \i in {0,...,5}
        {
          \draw (\i,-0.3) -- (\i,0);
          \draw (-0.3,\i) -- (0,\i);
        }

        \draw (6,-0.3) -- (6,0);

      \end{tikzpicture}
      \caption{Une solution réalisable du \CECSP.}
      \label{exSol}
    \end{center}
  \end{figure}

  Nous allons ajuster la fenêtre de temps de l'activité $1$. Pour cela,
  considérons l'intervalle $[t_1,t_2[=[2,5[$. On a:
  \begin{itemize}
  \item $\bb[2][2][5]=7$
  \item $\bb[3][2][5]=6$
  \item $\bbRS[1][2][5]=7$ et $\bbCS[1][2][5]=3$
  \end{itemize}
  Nous avons donc: $\sum_{j\in A;\ j\neq i} \bb[j][2][5] +
  \bbRS[1][2][5]= 7+7+6 =20 > 5 (5-2) =15$. Donc $\LS[1]$ peut être
  ajustée 
  à $ 2 -\frac{1}{5} (20-15) = 1$ .
  En effet, la quantité de ressource disponible dans l'intervalle
  $[2,5[$ pour ordonnancer l'activité $1$ est de $15-7-6=2$. Donc, si
  l'activité $1$ commence après $t_1$, elle ne peut consommer que $2$
  unités de ressource dans $[2,5[$.  Or, il faudrait $\bbRS[1][2][5]= 7$
  unités de ressource disponibles dans $[2,5[$ pour que $1$ puisse
  démarrer avant $t_1$. Donc l'activité $1$ ne peut commencer après $t_1$
  et, au moins $5= 7-2$ unités de ressource doivent être exécutées avant
  $t_1$. Donc $\LS[1]$ peut être ajustée à $2 - \frac{1}{5}(20-15) =1 $.

  De plus, $\LE[1]$ peut être ajusté à $t_1 + 1/ \bmin \times (R(t_2-t_1) -
  \sum_{j\in A;\ j\neq i} \bb[j] )= 2+(15-13)=4$. En effet, si l'activité
  $1$ finit après $t_2$, alors elle doit consommer au moins
  $\min(\bbRS[1][2][5],\bbCS[1][2][5])=\min(11,6)=6$ unités de ressource
  dans l'intervalle $[2,5[$. Or, seulement $2$ unités sont
  disponibles. L'activité $1$ ne peut donc pas finir après $t_2$ et nous
  pouvons ajuster $\LE[1]$ à $4$.
\end{ex}

L'algorithme~\ref{algo:ajust_CECSP} présente les ajustements pour le
\CECSP. 
\begin{algorithm}[!htb]
  \caption{Ajustement des fenêtres de temps pour le \CECSP.}
  \label{algo:ajust_CECSP}
    \PourTous {intervalles d'intérêt $[t_1,t_2[$}
    { $W\gets 0$
    
      $B\gets 0$
      \PourTous {$i \in \A$}{
        $W \gets W+\min(\wbLS,\wbCS,\wbRS)$
    
        $B \gets B+\max(\bmin\frac{W}{f_i(\bmin)},\frac{1}{a_i}(W-c_i|I|))$
      }
      \Si {$B>R(t_2-t_1)$}{
        L'instance est infaisable.}
      \Sinon{
        \PourTous {$i \in A$}{
          $W \gets \min(\wbLS,\wbCS,\wbRS)$
          
          $slack \gets R(t_2-t_1)-B+\max(\bmin\frac{W}{f_i(\bmin)},\frac{1}{a_i}(W-c_i|I|))$
    \Si {$ slack < \bbLS$}{
      $\emin \gets \max(\emin,t_2+\frac{1}{\bmax}(\bbLS-slack))$
    }
    \Si {$ slack < \min(\bbLS,\bbCS)$}{
      $\ES \gets \max(\ES,t_2-\frac{slack}{\bmin})$
    }
    \Si {$slack < \bbRS$}{
      $\smax \gets \min(\smax,t_1-\frac{1}{\bmax}(\bbRS-slack))$
    }
    \Si {$slack < \min(\bbRS,\bbCS)$}{
      $\LE \gets \min(\LE,t_1+\frac{slack}{\bmin})$
    }
  }
}
}    
\end{algorithm}


Nous avons montré qu'il était possible de calculer, étant donné un
intervalle et une activité, sa consommation minimale et les
ajustements pouvant être faits sur ses fenêtres de temps en
$O(1)$. {\'E}tant donné un intervalle, la fonction de marge ainsi que
tous les ajustements peuvent donc être calculés en $O(n)$. Dans la
sous-section suivante, nous montrons qu'il suffit d'exécuter l'algorithme
de vérification, ainsi que les ajustements sur un nombre polynomial
d'intervalles $[t_1,t_2[$.

\subsection{Caractérisation des intervalles d'intérêt}
\label{sec:intervalle_CECSP}
Dans un premier temps, nous prouvons que nous pouvons seulement
considérer un nombre polynomial d'intervalles pour détecter une
incohérence. En effet, à cause de la nature continue du problème, on
aurait pu être amené à considérer un nombre potentiellement infini
d'intervalles. 

\begin{theo}
  L'algorithme de vérification du raisonnement énergétique a seulement
  besoin d'être appliqué sur un nombre polynomial d'intervalles $[t_1,t_2[$.
\end{theo}

\begin{proof}
  La fonction de marge étant la différence entre une fonction affine,
  $R(t_2-t_1)$, et la somme de fonction affine par morceaux,
  $\bb$, c'est aussi une fonction affine par morceaux à deux
  dimensions. Le minimum de cette fonction est donc atteint en un point
  extrême d'un des polygones convexes dans lequel elle est
  affine. 

  Les segments de définitions, i.e. les segments où l'expression de la
  fonction est la même, de la fonction de marge sont les mêmes que ceux
  de la somme des consommations individuelles de chaque activité. Donc,
  d'un point de vue géométrique, un point extrême de la fonction de marge
  est l'intersection de deux segments chacun correspondant à un segment
  de définition de la fonction de consommation d'une activité. 

  Pour calculer la fonction de marge, il suffit donc d'énumérer tous
  ces points d'intersections et, pour chacun d'entre eux, d'appliquer le
  test de satisfiabilité décrit par le
  théorème~\ref{th:ER_CECSP}. Comme, pour chaque activité, il existe un
  nombre constant de segments de définition, le nombre de points
  d'intersection est de l'ordre de $O(n^2)$.
\end{proof}

De la même manière, nous pouvons prouver que les ajustements
peuvent être appliqués sur un nombre polynomial d'intervalles. Pour
cela, il suffit de considérer, à la place de la fonction de marge, la
fonction $ R(t_2-t_1) - \sum_{\substack{j \in \A\\ j \neq i}} \bb[j] -
\bbRS$ pour la règle~\ref{reg:ajust_CECSP} et $ R(t_2-t_1) - \sum_{\substack{j \in
    \A\\ j \neq i}} \bb[j] - \min(\bbCS,\bbLS)$ pour la
règle~\ref{reg:ajustRi_CECSP}. Dans la suite, nous appellerons ces
fonctions {\it fonction d'ajustement de $\LS$ et $\ES$} respectivement. 

\begin{theo}
  L'algorithme de filtrage du raisonnement énergétique a seulement
  besoin d'être appliqué sur un nombre polynomial d'intervalles $[t_1,t_2[$.
\end{theo}

Dans les paragraphes suivants, nous allons présenter trois méthodes
permettant de calculer les intervalles d'intérêt du raisonnement
énergétique. Les deux premières méthodes s'appuient sur une analyse
des segments de définition des fonctions de consommation individuelle
des activités tandis que la seconde est une adaptation du travail
de~\cite{DP} pour la contrainte cumulative et se base sur une analyse
des variations des fonctions de marge et d'ajustements.

\subsubsection{Analyse des segments de définition des fonctions de 
  consommation individuelle}

Dans ce paragraphe, nous allons donc étudier les segments de
définitions des fonctions de consommation individuelle en fonctions
des caractéristiques des activités. Plus précisément, nous devons
considérer trois cas:
\begin{itemize}
\item $\bmin=0$
\item $W_i \le f_i(\bmin)(\LE-\ES)$
\item $W_i \ge f_i(\bmin)(\LE-\ES)$
\end{itemize}
Ces trois cas devront être considérés séparément dans chacune des
méthodes proposées pour calculer les intervalles d'intérêt du
raisonnement énergétique (pour l'algorithme de vérification et les
ajustements). Cependant, seul le troisième cas sera détaillé dans
le manuscrit, les deux autres cas étant traités de manière similaire. 
% Les deux autres cas seront quant à eux détaillés dans
% l'annexe~\ref{annexe:breakline}. 

Ce raisonnement peut aussi s'appliquer pour calculer les intervalles
d'intérêt pour les ajustements mais ceci ne sera pas détaillé dans ce
manuscrit, cette méthode n'étant pas la plus efficace. De même, nous
allons seulement présenter cette technique dans le cas des fonctions
affines mais elle peut être étendues facilement au cas des fonctions
concaves et affines par morceaux. La caractérisation des intervalles
d'intérêt dans ces deux cas sera présentée dans le paragraphe suivant
et calculée à l'aide de l'analyse de la fonction de marge et des
fonctions d'ajustements. 

Pour analyser les segments de définition des fonctions de consommation
individuelle, nous allons, dans un premier temps, montrer que si nous
analysons seulement les segments de définition de $\bb$ pour $t_1 \ge
\ES$, $t_2 \le \LE$ et $t_1+t_2 \le \ES + \LE$, alors le reste des
segments peut être déduit par symétrie.

\begin{lemma}
  \label{lem:sym1}
  \[\bb=\bb[i][\ES][t_2], \ \forall t_1,t_2 \in \mathbb{R},\ t_1 \le \ES\]
  \[\bb=\bb[i][t_1][\LE], \ \forall t_1,t_2 \in \mathbb{R},\ t_2 \ge
    \LE\]
\end{lemma}

\begin{proof}
  Pour tout intervalle $[t_1,t_2[$ tel que $t_1 \le \ES$, nous avons:
  \[ \int_{t_1}^{t_2} b_i(t)dt = \int_{t_1}^{\ES} b_i(t)dt+\int_{\ES}^{t_2} b_i(t)dt\]
  Or, $b_i(t)=0, \forall t \le \ES$. Nous avons donc que:
  \[\int_{t_1}^{\ES} b_i(t)dt+\int_{\ES}^{t_2} b_i(t)dt=
    \int_{\ES}^{t_2} b_i(t)dt\]
  De plus, d'après l'expression de la consommation
  minimale~\eqref{eq:minConso}, nous avons aussi que:
  \begin{align*}
    \bb &= \min \int_{t_1}^{t_2} b_i(t)dt \\
        &=\min \int_{t_1}^{\ES} b_i(t)dt+ \min \int_{\ES}^{t_2} b_i(t)dt\\
        &= \min \int_{\ES}^{t_2} b_i(t)dt\\
        & = \bb[i][\ES][t_2]
  \end{align*}
  Nous pouvons montrer de la même manière la seconde égalité et ceci
  achève la démonstration.
\end{proof}


\begin{lemma}
  \label{lem:sym2}
  Pour tout $t_1 \ge 0,\ t_2 \ge t_1,\ \bb= \bb[i][\LE+\ES-t_2][\LE +
  \ES -t_1]$.
\end{lemma}

\begin{proof}
  Pour montrer le lemme, on peut montrer que $ \wb=
  \wb[i][\LE+\ES-t_2][\LE +\ES -t_1]$. En effet, pour une même quantité
  d'énergie, la conversion de $\wb$ en $\bb$ ne dépend que de la
  taille de l'intervalle et $\LE +\ES -t_1 - \LE - \ES+t_2 =t_2 -
  t_1$. Nous montrons donc le lemme pour $\wb$.

  \begin{itemize}
  \item $\wbLS[i][t_1'][t_2']=\max(0,W_i-f_i(\bmax)(\max(0,\LE+\ES-t_2-\ES)))=\wbRS$
    \vspace{0.2cm}
  \item $\wbRS[i][t_1'][t_2']=\max(0,W_i-f_i(f_i(\bmax))(\max(0,\LE-\LE-\ES+t_1)))=\wbLS$
  \item$
    % \setlength{\extrarowheight}{0.5cm}
    \begin{array}{rl}
      \wbCS[i][t_1'][t_2']=&\max\left(
                             \begin{array}{l}
                               f_i(\bmin) (\LE+\ES-t_1 - \ES- \LE + t_2),\\
                               W_i - f_i(\bmax)(\max(0,\LE - \LE -\ES +t_1+\\
                               \hspace{2cm}\LE +\ES - t_2 - \ES)
                             \end{array}
      \right)\\
                           & \\
      = &\max\left(
          \begin{array}{l}
            f_i(\bmin) (t_2-t_1),\\
            W_i - f_i(\bmax)(\max(0,\LE-t_2 +t_1 - \ES))
          \end{array}
      \right)\\
      = & \wbCS
    \end{array}$
  \end{itemize}
  Et donc: 
  \begin{align*}
    \wb[i][t_1'][t_2'] &=
                         \min(\wbLS[i][t_1'][t_2'],\wbCS[i][t_1'][t_2'],\wbRS[i][t_1'][t_2'])
    \\
                       & =\min(\wbRS,\wbCS,\wbLS)\\
                       & =\wb 
  \end{align*}
  avec $t_1'=\LE+\ES-t_2$ et
  $t_2'=\LE+\ES-t_1,\ \forall t_1 \le 0$ et $\forall t_2 \ge t_1$.
\end{proof}

Donc, grâce aux lemmes~\ref{lem:sym1} et~\ref{lem:sym2}, nous avons
seulement besoin d'établir l'expression de $\bb$ à l'intérieur du
polygone (triangle) délimité par les inégalités $t_1 \ge \ES, \ t_2
\le \LE,\ t_1+t_2 \le \ES+\LE$ et $ t_2 \ge t_1$. Ce triangle est
défini par les points:

$A=(\ES,\LE)$ \hfill $\ B=(\frac{\ES+\LE}{2},\frac{\ES+\LE}{2})$
\hfill $C=( \ES,\ES)$

Les différentes expressions de $\wb$ en fonction de $t_1,t_2$ et des
paramètres de l'activité $i$ sont décrits
dans~\cite{ArtiguesLopez}. Nous commençons donc par brièvement
présenter ces résultats, puis nous expliquerons comment les étendre
pour obtenir les différentes expressions de $\bb$.

Pour ce faire et pour simplifier les calculs, nous introduisons les
deux notations suivantes: $\smax$ est la borne supérieure $ \LE -
W_i/f_i(\bmax)$ sur la date de début de $i$ et $\emin$ la borne
inférieure $\ES + W_i/f_i(\bmax)$ sur la date de fin de $i$. 

Afin de décrire les zones où l'expression de $\wb$ est identique, il
faut analyser les conditions sur $t_1$ et $t_2$ selon lesquelles une
expression est choisie plutôt qu'une autre. Pour cela, les auteurs
de~\cite{ArtiguesLopez} se proposent d'étudier les segments
correspondant au cas où deux expressions ont la même valeur,
e.g. $f_i(\bmin)(t_2-t_1) = W_i - f_i(\bmax) (t_1-\ES)$. Ces segments
permettent de délimiter des zones dans lesquelles on doit décider
l'expression de $\wb$, i.e. quelle expression est minimale à
l'intérieur de cette zone. 

Par exemple, dans le cas où $W_i \ge f_i(\bmin)(\LE -\ES)$ l'unique
point pour lequel l'énergie requise est maximale est le point
$A$. Donc, $\wb[i][\ES][\LE]= W_i$ et, dans la zone délimitée par les
lignes $t_1 \le \ES$ et $t_2 \ge \LE$ (en rouge sur la
figure~\ref{fig:breakline_nrj}), l'expression de $\wb$ est $W_i$. 

Notons aussi que le cas $W_i \ge f_i(\bmin)(\LE -\ES)$ correspond au
cas où l'activité ne peut être ordonnancée à $\bmin$ durant toute sa
durée d'exécution. C'est-à-dire que l'énergie requise par l'activité
$i$ dans l'intervalle $[t_1,t_2[$ peut dépasser $f_i(\bmin)(t_2-t_1)$
et le cas où $\wb= W_i - f_i(\bmax)(\LE-t_2+t_1-\ES)$ doit être
considéré. De plus, le cas $W_i \ge f_i(\bmin)(\LE -\ES)$ est
séparé en deux sous cas: $\emin \le \smax$ et $\emin \ge \smax$. Le
premier cas correspond au cas où le point d'intersection des lignes
$t_1\le \emin$ et $t_2 \ge \smax$ vérifie $t_2 \ge t_1$ et donc le
point est à l'intérieur du triangle $(A,B,C)$. Le second cas
correspond au cas où ce point se trouve à l'extérieur du
triangle. Dans ces deux cas, les segments de définition à considérer
ne sont pas les mêmes.

Plaçons nous dans le premier cas, i.e. le cas où $\emin \le
\smax$. Dans ce cas, le segment $\overline{DD'}$ sépare les cas où
$\wb= f_i(\bmin)(t_2-t_1)$ et $\wb= W_i -
f_i(\bmax)(\LE-t_2+t_1-\ES)$, le segment $\overline{DG}$ délimite
quant à lui les zones où $\wb=f_i(\bmin)(t_2-t_1)$ et $\wb= W_i -
f_i(\bmax)(\LE-t_2)$. Les autres zones peuvent être définies de
manière similaire. Pour une analyse complète des différents
expressions de $\wb$, nous renvoyons le lecteur
à~\cite{ArtiguesLopez}. 

La figure~\ref{fig:breakline_nrj} présente ces différentes expressions
avec les expressions de $\wb$ suivantes: 
\begin{itemize}
\item la zone rouge correspond à $\underline{w}(i,t_1,t_2)=W_i$;
\item la zone verte à $\underline{w}(i,t_1,t_2)=W_i-(\LE-t_2)f_i(\bmax)$;
\item la zone bleue à $\underline{w}(i,t_1,t_2)=W_i-(t_1-\ES)f_i(\bmax)$;
\item la zone jaune à
  $\underline{w}(i,t_1,t_2)=W_i-(\LE-t_2+t_1-\ES)f_i(\bmax))$;
\item et la blanche à $\underline{w}(i,t_1,t_2)=(t_2-t_1)f_i(\bmin)$.
\end{itemize}
Les autres zones correspondent à $\underline{w}(i,t_1,t_2)=0$.

De plus, les coordonnées de chaque point sont:
\begin{itemize}
\item $A=(\ES,\LE),\ B=(\emin,\smax),\
  C=(\smax,\smax)$ and $C'=(\emin,\emin)$
\item
  $D=(\ES,\frac{\LE f_i(\bmax)-\ES f_i(\bmin)-W_i}{f_i(\bmax)-f_i(\bmin)}),\
  D'=(\frac{\ES f_i(\bmax)-\LE f_i(\bmin)+W_i}{f_i(\bmax)-f_i(\bmin)},\LE)$
\item
  $G=(\frac{\ES(f_i(\bmax)-f_i(\bmin))-\LE f_i(\bmin)+W_i}{f_i(\bmax)-2f_i(\bmin)},
  \frac{\LE(f_i(\bmax)-f_i(\bmin))-\ES f_i(\bmin)-W_i}{f_i(\bmax)-2f_i(\bmin)})$
\end{itemize}



\begin{figure}[!htb]
  \centering
  \subcaptionbox{Cas $\bmin = 0$}[0.45\linewidth]
  {\begin{tikzpicture}
      [scale=0.4]
      \fill[gray!20] (0,0) -- (10,10) -- (0,10) -- cycle;
      \draw[fill=LightCoral!80!] (0,10) rectangle (2,9);
      \draw[fill=LightGoldenrod] (0,6.5) -- (2,6.5) -- (4.5,9) -- (4.5,10)
      -- (2,10) -- (2,9) -- (0,9) -- cycle; 
      \node (O) at (0,0) {};
      \node (T1) at (11,0) {};
      \node (T2) at (0,11) {};

      \node at (0,0) {};
      \node[label={[shift={(-0.4,-0.6)}]\scriptsize$\LE$}] (di) at (0,9) {$\bullet$};
      \node[label={[shift={(-0.4,-0.6)}]\scriptsize$E_{max}$}] at (0,10) {};

      \node[label={[shift={(0,-0.8)}]\scriptsize$S_{min}$}] at (0,0) {};
      \node[label={[shift={(0,-0.8)}]\scriptsize$\ES$}] (ri) at (2,0) {};
      \node[label={[shift={(0,-0.8)}]\scriptsize$E_{max}$}] at (10,0) {};
      \node[label={[shift={(-0.4,-0.6)}]\scriptsize$\smax$}] (smax2) at (0,6.5) {$\bullet$};

      \node[label={[shift={(0,-0.8)}]\scriptsize$\emin$}] (emin1) at (4.5,0) {};

      \node at (2,10) {$\bullet$};
      \node at (4.5,10) {$\bullet$};

      \node at (2,3) {};
      \node at (5,5) {};
      \node[label={[shift={(-0.2,-0.3)}]\scriptsize$A$}] (A) at (2,9) {$\bullet$};
      \node[label={[shift={(0.2,-0.6)}]\scriptsize$F$}] at (2,6.5) {$\bullet$};
      \node[label={[shift={(0.2,-0.7)}]\scriptsize$F'$}] at (4.5,9) {$\bullet$};
      \draw[dashed] (ri.center) -- (2,10);
      \draw[dashed] (di.center) -- (10,9);
      \path[draw] (O.center) -- (10,10) -- (0,10) -- cycle;
      \draw[dashed] (emin1.center) -- (4.5,10);
      \draw[dashed] (smax2.center) -- (10,6.5);

      \draw[->] (O.center) -- (T1.center) node[above left] {\scriptsize$t_1$};
      \draw[->] (O.center) -- (T2.center) node[below left] {\scriptsize$t_2$};
    \end{tikzpicture}}

  \vspace{0.7cm}
  Cas $ W_i \le f_i(\bmin) (\LE-\ES)$


  \subcaptionbox{$ \emin \le \smax $}[0.45\linewidth]{
    \begin{tikzpicture}
      [scale=0.4]
      % non common
      \fill[gray!20] (0,0) -- (10,10) -- (0,10) -- cycle;
      \draw[fill=LightGreen] (0,9) -- (2,9) -- (4.5,6.5) --(0,6.5) -- cycle;
      \draw[fill=LightBlue] (2,9) -- (2,10) -- (4.5,10) --(4.5,6.5) --cycle;
      \node[label={[shift={(0.3,-0.7)}]\scriptsize$B$}] (B) at (4.5,6.5) {$\bullet$};
      \node[label={[shift={(-0.4,-0.6)}]\scriptsize$\smax$}] (smax2) at (0,6.5) {$\bullet$};

      \node[label={[shift={(0,-0.8)}]\scriptsize$\emin$}] (emin1) at (4.5,0) {};
      \node at (4.5,10) {$\bullet$};
      % common

      \draw[fill=LightCoral!80!] (0,10) rectangle (2,9);
      \node (O) at (0,0) {};
      \node (T1) at (11,0) {};
      \node (T2) at (0,11) {};


      \node[label={[shift={(-0.4,-0.6)}]\scriptsize$\LE$}] (di) at (0,9) {$\bullet$};
      \node[label={[shift={(-0.4,-0.6)}]\scriptsize$E_{max}$}] at (0,10) {};

      \node[label={[shift={(0,-0.8)}]\scriptsize$S_{min}$}] at (0,0) {};
      \node[label={[shift={(0,-0.8)}]\scriptsize$\ES$}] (ri) at (2,0) {};
      \node[label={[shift={(0,-0.8)}]\scriptsize$E_{max}$}] at (10,0) {};

      \node at (2,10) {$\bullet$};

      \node[label={[shift={(-0.2,-0.3)}]\scriptsize$A$}] (A) at (2,9) {$\bullet$};

      \draw[dashed] (ri.center) -- (2,10);
      \draw[dashed] (emin1.center) -- (4.5,10);
      \draw[dashed] (di.center) -- (10,9);
      \draw[dashed] (smax2.center) -- (10,6.5);
      \path[draw] (O.center) -- (10,10) -- (0,10) -- cycle;
      \draw[->] (O.center) -- (T1.center) node[above left] {\scriptsize$t_1$};
      \draw[->] (O.center) -- (T2.center) node[below left] {\scriptsize$t_2$};
    \end{tikzpicture}}
  \hfill
  \subcaptionbox{$\emin \ge \smax$}[0.45\linewidth]{
    \begin{tikzpicture}
      [scale=0.4]
      \fill[gray!20] (0,0) -- (10,10) -- (0,10) -- cycle;
      \draw[fill=Cornsilk!99!black!40!] (6,6) -- (5,5) -- (4.5,6.5) -- cycle;
      \draw[fill=LightGreen] (0,9) -- (2,9)-- (4.5,6.5) -- (5,5) -- (0,5) -- cycle;
      \draw[fill=LightBlue]  (2,9) -- (2,10)-- (6,10) -- (6,6)-- (4.5,6.5) -- cycle;
      \node[label={[shift={(0,-0.8)}]\scriptsize$C$}] (C) at (5,5)  {$\bullet$};
      \node[label={[shift={(0.3,-0.7)}]\scriptsize$C'$}] (C2) at (6,6)  {$\bullet$};

      \node[label={[shift={(0.2,-0.3)}]\scriptsize$G$}] at (4.5,6.5)  {$\bullet$};
      \node[label={[shift={(-0.4,-0.6)}]\scriptsize$\smax$}] (smax2) at (0,5) {$\bullet$};

      \node[label={[shift={(0,-0.8)}]\scriptsize$\emin$}] (emin1) at (6,0) {};

      \node at (6,10) {$\bullet$};
      % common

      \draw[fill=LightCoral!80!] (0,10) rectangle (2,9);
      \node (O) at (0,0) {};
      \node (T1) at (11,0) {};
      \node (T2) at (0,11) {};

      \node[label={[shift={(-0.4,-0.6)}]\scriptsize$\LE$}] (di) at (0,9) {$\bullet$};
      \node[label={[shift={(-0.4,-0.6)}]\scriptsize$E_{max}$}] at (0,10) {};

      \node[label={[shift={(0,-0.8)}]\scriptsize$S_{min}$}] at (0,0) {};
      \node[label={[shift={(0,-0.8)}]\scriptsize$\ES$}] (ri) at (2,0) {};
      \node[label={[shift={(0,-0.8)}]\scriptsize$E_{max}$}] at (10,0) {};

      \node at (2,10) {$\bullet$};

      \node at (2,3) {};
      \node at (5,5) {};
      \node[label={[shift={(-0.2,-0.3)}]\scriptsize$A$}] (A) at (2,9) {$\bullet$};

      \draw[dashed] (ri.center) -- (2,10);
      \draw[dashed] (emin1.center) -- (6,10);
      \draw[dashed] (di.center) -- (10,9);
      \draw[dashed] (smax2.center) -- (10,5);
      \path[draw] (O.center) -- (10,10) -- (0,10) -- cycle;
      \draw[->] (O.center) -- (T1.center) node[above left] {\scriptsize$t_1$};
      \draw[->] (O.center) -- (T2.center) node[below left] {\scriptsize$t_2$};

    \end{tikzpicture}
  }

  \vspace{0.7cm}
  Cas $  W_i \ge f_i(\bmin) (\LE - \ES)$

  \subcaptionbox{$\emin \le \smax$}[0.45\linewidth]{
    \begin{tikzpicture}
      [scale=0.4]
      % non common
      \fill[gray!20] (0,0) -- (10,10) -- (0,10) -- cycle;
      \path[fill=Cornsilk!99!black!40!] (2,7.5) -- (3.5,9)-- (3.75,7.3) -- cycle;
      \draw[fill=LightGoldenrod] (2,7.5) -- (3.5,9) -- (2,9) -- cycle;
      \draw[fill=LightGreen] (0,9) -- (2,9) -- (2,7.5)  --  (3.75,7.3) -- (4.5,6.5) -- (0,6.5)  -- (0,7.5) --cycle;
      % \draw[fill=PaleGreen!70!] (0,7.5) -- (2,7.5) -- (3.75,7.3) --
      % (4.5,6.5) -- (0,6.5) -- cycle;
      \draw[fill=LightBlue] (2,9) -- (2,10) -- (4.5,10) --  (4.5,6.5) --
      (3.75,7.3) -- (3.5,9) -- cycle;
      % \draw[fill=PowderBlue!70!] (4.5,6.5)-- (3.75,7.3) -- (3.5,9) -- (3.5,10) -- (4.5,10) --cycle;
      \node[label={[shift={(-0.2,-0.7)}]\scriptsize$D$}] (D) at (2,7.5) {$\bullet$};
      \node[label={[shift={(0.2,-0.3)}]\scriptsize$D'$}] (D2) at (3.5,9) {$\bullet$};
      \node[label={[shift={(0.1,-0.3)}]\scriptsize$G$}] (E) at (3.75,7.3)  {$\bullet$};
      \node[label={[shift={(0.3,-0.7)}]\scriptsize$B$}] (B) at (4.5,6.5) {$\bullet$};
      \node[label={[shift={(-0.4,-0.6)}]\scriptsize$\smax$}] (smax2) at (0,6.5) {$\bullet$};

      \node[label={[shift={(0,-0.8)}]\scriptsize$\emin$}] (emin1) at (4.5,0) {};
      \node at (4.5,10) {$\bullet$};
      % \node at (3.5,10) {$\bullet$};
      % \node at (0,7.5) {$\bullet$};
      % common


      \draw[fill=LightCoral!80!] (0,10) rectangle (2,9);
      \node (O) at (0,0) {};
      \node (T1) at (11,0) {};
      \node (T2) at (0,11) {};


      \node[label={[shift={(-0.4,-0.6)}]\scriptsize$\LE$}] (di) at (0,9) {$\bullet$};
      \node[label={[shift={(-0.4,-0.6)}]\scriptsize$E_{max}$}] at (0,10) {};

      \node[label={[shift={(0,-0.8)}]\scriptsize$S_{min}$}] at (0,0) {};
      \node[label={[shift={(0,-0.8)}]\scriptsize$\ES$}] (ri) at (2,0) {};
      \node[label={[shift={(0,-0.8)}]\scriptsize$E_{max}$}] at (10,0) {};

      \node at (2,10) {$\bullet$};

      \node at (2,3) {};
      \node at (5,5) {};
      \node[label={[shift={(-0.2,-0.3)}]\scriptsize$A$}] (A) at (2,9) {$\bullet$};

      \draw[dashed] (ri.center) -- (D.center);
      \draw[dashed] (emin1.center) -- (4.5,10);
      \draw[dashed] (di.center) -- (10,9);
      \draw[dashed] (smax2.center) -- (10,6.5);
      \path[draw] (O.center) -- (10,10) -- (0,10) -- cycle;
      \draw[->] (O.center) -- (T1.center) node[above left] {\scriptsize$t_1$};
      \draw[->] (O.center) -- (T2.center) node[below left] {\scriptsize$t_2$};

    \end{tikzpicture}
  }
  \hfill
  \subcaptionbox{$\emin \ge \smax$}[0.45\linewidth]{
    \begin{tikzpicture}
      [scale=0.4]
      % non common
      
      \fill[gray!20] (0,0) -- (10,10) -- (0,10) -- cycle;
      \path[fill=Cornsilk!99!black!40!] (2,6.25) -- (4.75,9)-- (6,6) -- (5,5) -- cycle;
      \draw[fill=LightGoldenrod] (2,6.25) -- (2,9) -- (4.75,9) -- cycle;
      \draw[fill=LightGreen] (0,9) -- (2,9) -- (2,6.25) --  (5,5)
      -- (0,5) --  cycle;
      % \draw[fill=PaleGreen!70!] (0,6.25) -- (2,6.25) -- (5,5) -- (0,5) -- cycle;
      \draw[fill=LightBlue] (2,9) -- (2,10) -- (4.75,10) -- (6,10) -- (6,6) -- (4.75,9) -- cycle;
      % \draw[fill=PowderBlue!70!] (4.75,10) -- (6,10) -- (6,6)-- (4.75,9) -- cycle;
      \node[label={[shift={(-0.2,-0.7)}]\scriptsize$D$}] (D) at (2,6.25) {$\bullet$};
      \node[label={[shift={(0.2,-0.3)}]\scriptsize$D'$}] (D2) at (4.75,9) {$\bullet$};
      \node[label={[shift={(0,-0.8)}]\scriptsize$C$}] (C) at (5,5)  {$\bullet$};
      \node[label={[shift={(0.3,-0.5)}]\scriptsize$C'$}] (C2) at (6,6)  {$\bullet$};
      \node[label={[shift={(-0.4,-0.6)}]\scriptsize$\smax$}] (smax2) at (0,5) {$\bullet$};

      \node[label={[shift={(0,-0.8)}]\scriptsize$\emin$}] (emin1) at (6,0) {};

      \node at (6,10) {$\bullet$};
      % \node at (4.75,10) {$\bullet$};
      % \node at (0,6.25) {$\bullet$};
      % common

      \draw[fill=LightCoral!80!] (0,10) rectangle (2,9);
      \node (O) at (0,0) {};
      \node (T1) at (11,0) {};
      \node (T2) at (0,11) {};

      \node[label={[shift={(-0.4,-0.4)}]\scriptsize$S_{min}$}] at (0,0) {};
      \node[label={[shift={(-0.4,-0.6)}]\scriptsize$\LE$}] (di) at (0,9) {$\bullet$};
      \node[label={[shift={(-0.4,-0.6)}]\scriptsize$E_{max}$}] at (0,10) {};

      \node[label={[shift={(0,-0.8)}]\scriptsize$S_{min}$}] at (0,0) {};
      \node[label={[shift={(0,-0.8)}]\scriptsize$\ES$}] (ri) at (2,0) {};
      \node[label={[shift={(0,-0.8)}]\scriptsize$E_{max}$}] at (10,0) {};

      \node at (2,10) {$\bullet$};

      \node at (2,3) {};
      \node at (5,5) {};
      \node[label={[shift={(-0.2,-0.3)}]\scriptsize$A$}] (A) at (2,9) {$\bullet$};

      \draw[dashed] (ri.center) -- (D.center);
      \draw[dashed] (emin1.center) -- (6,10);
      \draw[dashed] (di.center) -- (10,9);
      \draw[dashed] (smax2.center) -- (10,5);
      \path[draw] (O.center) -- (10,10) -- (0,10) -- cycle;
      \draw[->] (O.center) -- (T1.center) node[above left] {\scriptsize$t_1$};
      \draw[->] (O.center) -- (T2.center) node[below left] {\scriptsize$t_2$};

    \end{tikzpicture}

  }
  \caption{Les différentes expressions de $\wb$ en fonction des
    paramètres du problème \CECSP.}
  \label{fig:breakline_nrj}
\end{figure}

Nous avons donc défini les différentes zones correspondant chacune à
une expression de $\wb$. Dans le cas où $f_i$ est la fonction
identité, cela suffit pour définir les segments à considérer, i.e. les
segments pour lesquels on va tester l'intersection. La méthode
générale pour calculer les intervalles d'intérêt en fonction de ces
segments sera détaillée un peu plus loin dans le paragraphe. 

Les segments à considérer sont les segments délimitant deux zones avec
une expression de $\wb$ différentes, i.e. reliant deux points
représentés par une boule noire sur la
figure~\ref{fig:breakline_nrj}. Par exemple, dans le cas où $W_i \ge
f_i(\bmin)(\LE-\ES)$ les segments à considérer sont donc:
\begin{itemize}
\item dans le cas où $\emin \le \smax$: $(\ES ,E_{max})A,\ (S_{min},
  \LE)A,\ (S_{min},\smax)B,\ B(\emin,E_{max})$,  $AD,\ AD',\
  DD' ,\ DG,\ D'G$ et $GB$.
\item dans le cas où $\emin\ge \smax$: $(\ES ,E_{max})A,\ (S_{min},
  \LE)A,\ (S_{min},\smax)C,\ C'(\emin,E_{max}),\ AD$, $ AD',\
  DD' ,\ DC$ et $D'C$.
\end{itemize}
avec $D_{t_1}$ (resp. $D_{t_2}$) la projection sur l'axe $x$
(resp. sur l'axe $y$) du point $D$.


Nous allons maintenant calculer, à partir de la
figure~\ref{fig:breakline_nrj}, les zones correspondant à des
expressions différentes de $\bb$. 

Pour ce faire, nous considérons, à l'intérieur de chaque zone définie
pour $\wb$, l'inégalité suivante: 
\[ \frac{\wb}{f_i(\bmin)} \le |I|\]
avec $I=[t_1,t_2[ \cap [\ES,\LE]$. Si cette condition est vérifiée,
alors l'activité peut être ordonnancée à $\bmin$ et l'expression de
$\bb$ sera alors $\frac{\wb}{f_i(\bmin)}\times \bmin$. Dans le cas
inverse, nous avons $\bb=\frac{1}{a_i}(\wb-c_i|I|)$. 

Dans le cas où $\bmin=0$, l'inégalité n'a pas besoin d'être considérée et
nous avons directement $\bb=\frac{1}{a_i}(\wb-c_i|I|)$ dans chaque
zone. 

Dans le cas où $W_i\le f_i(\bmin)(\LE-\ES)$, $|I|$ peut valoir
$t_2-t_1$, $t_2-\ES$, $\LE-t_1$ ou $\LE-\ES$. 

Dans le premier cas,
l'inégalité donne $\wb \le f_i(\bmin)(t_2-t_1)$. Or, ce cas correspond
au cas où $[t_1,t_2[ \subset [\ES,\LE]$ et donc, $\wb=\min(W_i -
f_i(\bmax)(\LE-t_2), W_i - f_i(\bmax)(t_1-\ES),
f_i(\bmin)(t_2-t_1))$. Donc, dans ce cas, l'inégalité est toujours
vérifiée et nous avons $\bb = \bmin \times \wb/f_i(\bmin)$. 

Dans le second cas,
i.e. $|I|=t_2-\ES$, l'inégalité donne $\wb \le
f_i(\bmin)(t_2-\ES)$. Or, ce cas correspond au cas où $\wb = W_i -
f_i(\bmax)(\LE -t_2)$ , i.e. la zone verte. En remplaçant $\wb$ par
son expression, nous obtenons l'inégalité suivante: 
\[t_2 \le \frac{f_i(\bmax)\LE - f_i(\bmin)\ES - W_i }{f_i(\bmax) -
    f_i(\bmin)} \]
Or, $\LE  \le \frac{f_i(\bmax)\LE - f_i(\bmin)\ES - W_i }{f_i(\bmax) -
  f_i(\bmin)}$. Donc, l'inégalité est vérifiée si $t_2 \le d_i$
et ceci est vrai dans ce cas. 

Les cas où $|I| = \LE -\ES$ et $|I| =
\LE -t_1$ sont traités de la même façon. Donc, dans le cas où $W_i \le
f_i(\bmin) (\LE - \ES)$, $\bb=\bmin \times \wb/f_i(\bmin)$ et les
zones où $\bb$ a la même expression sont les mêmes que pour $\wb$ dans
ce cas. 

Dans le cas où $W_i \ge f_i(\bmax)(\LE - \ES)$, si $|I|$ vaut
$t_2-t_1$, l'inégalité donne $\wb \le f_i(\bmin)(t_2-t_1)$. Donc, si
$\wbCS=f_i(\bmin) (t_2-t_1)$ alors $\wb=\min(W_i -
f_i(\bmax)(\LE-t_2), W_i - f_i(\bmax)(t_1-\ES),
f_i(\bmin)(t_2-t_1))$ et l'inégalité est vérifiée. A l'inverse, si
$\wbCS=W_i - f_i(\bmax)(\LE - t_2 +t_1 - \ES)$ et comme $[t_1,t_2[
\subset [\ES,\LE]$, on a forcément $\wb=W_i - f_i(\bmax)(\LE - t_2
+t_1 - \ES)$. Donc l'inégalité n'est jamais vérifiée et
$\bb=\frac{1}{a_i}(\wb-c_i|I|)$. 

Le cas où $|I|$ vaut $t_2 - \ES$ correspond au cas où
$\wb=W_i-f_i(\bmax)(\LE-t_2)$. Dans ce cas, l'inégalité donne: 
\[ \frac{\LE f_i(\bmax) - \ES f_i(\bmin) - W_i}{f_i(\bmax)-f_i(\bmin)}
  \le t_2 \]
La partie gauche de cette inégalité correspond exactement à l'ordonnée
du point $D$ et donc la zone verte est divisée en deux parties:
\begin{itemize}
\item la partie claire où $\bb= \bmin \times \wb / f_i(\bmin)$
\item  et la partie plus foncée où $\bb=\frac{1}{a_i}(\wb - c_i|I|)$.
\end{itemize}
Les cas où $|I| = \LE -\ES$ et $|I| = \LE -t_1$ sont traités de la
même façon.

La figure~\ref{fig:breakline_res} présente ces résultats avec les
 expressions de $\bb$ suivantes:
 \begin{itemize}
 \item la zone rouge claire correspond à $\bb=\bmin \times W_i/f_i(\bmin)$;
 \item la zone rouge foncée correspond à $\bb=\frac{1}{a_i}(W_i - c_i (\LE - \ES))$;
 \item la zone verte claire à $\bb=\bmin \times
   \left\{W_i-(\LE-t_2)f_i(\bmax)\right\}/ f_i(\bmin)$;
 \item la zone verte foncée à
   $\bb=\frac{1}{a_i}(W_i-(\LE-t_2)f_i(\bmax) - c_i |I|)$ où $I$ vaut
   $[t_1,t_2[$ ou $[\ES,t_2[$;
 \item la zone bleue claire à $\bb=\bmin \times \left\{W_i-(t_1-\ES)f_i(\bmax)\right\}/f_i(\bmin)$;
 \item la zone bleue foncée à $\bb=\frac{1}{a_i}(W_i-(t_1-\ES)f_i(\bmax) - c_i
   |I|)$ où $I$ vaut  $[t_1,t_2[$ ou $[t_1,\LE{[}$;
 \item la zone jaune à
   $\bb=\frac{1}{a_i}(W_i-(\LE-t_2+t_1-\ES)f_i(\bmax) - c_i (t_2 - t_1))$;
 \item et la blanche à $\bb=(t_2-t_1) \bmin$.
 \end{itemize}
 Les autres zones correspondent à $\bb=0$.

 Les segments à considérer sont aussi décrits par la
 figure~\ref{fig:breakline_res} et sont, par exemple, dans le cas où
 $W_i \ge f_i(\bmin) (\LE-\ES)$:
 \begin{itemize}
 \item si $\emin \le \smax$: $(\ES ,E_{max})A,\ (S_{min},
   \LE)A,\ (S_{min},\smax)B, \\B(\emin,E_{max}),\ AD,\ AD',\
   D(0,D_{t_2}),\ D'(D'_{t_1},E_{max}),\ DD' ,\ DG,\ D'G$ et $GB$.
 \item si $\emin \ge \smax$: $(\ES ,E_{max})A,\ (S_{min},
   \LE)A,\ (S_{min},\smax)C,\\ C'(\emin,E_{max}),\ AD,\ AD',\
   D(0,D_{t_2}),\ D'(D'_{t_1},E_{max}),\ DD' ,\ DC$ et $D'C$.
 \end{itemize}
avec $E_{max}=\max_{i \in \A}\{\LE\}$ et $S_{min}=\min_{i \in
  \A}\{\ES\}$. 

\begin{figure}[!htb]
  \centering
  \subcaptionbox{Cas $\bmin = 0$}[0.45\linewidth]{
    \begin{tikzpicture}
      [scale=0.4]
      \fill[gray!20] (0,0) -- (10,10) -- (0,10) -- cycle;
      \draw[fill=LightCoral!80!] (0,10) rectangle (2,9);
      \draw[fill=LightGoldenrod] (0,6.5) -- (2,6.5) -- (4.5,9) -- (4.5,10) -- (2,10) -- (2,9) -- (0,9) -- cycle;
      \node (O) at (0,0) {};
      \node (T1) at (11,0) {};
      \node (T2) at (0,11) {};
      
      \node at (0,0) {};
      \node[label={[shift={(-0.4,-0.6)}]\scriptsize$\LE$}] (di) at (0,9) {$\bullet$};
      \node[label={[shift={(-0.4,-0.6)}]\scriptsize$E_{max}$}] at (0,10) {};

      \node[label={[shift={(0,-0.8)}]\scriptsize$S_{min}$}] at (0,0) {};
      \node[label={[shift={(0,-0.8)}]\scriptsize$\ES$}] (ri) at (2,0) {};
      \node[label={[shift={(0,-0.8)}]\scriptsize$E_{max}$}] at (10,0) {};
      \node[label={[shift={(-0.4,-0.6)}]\scriptsize$\smax$}] (smax2) at (0,6.5) {$\bullet$};
      
      \node[label={[shift={(0,-0.8)}]\scriptsize$\emin$}] (emin1) at (4.5,0) {};

      \node at (2,10) {$\bullet$};
      \node at (4.5,10) {$\bullet$};

      \node at (2,3) {};
      \node at (5,5) {};
      \node[label={[shift={(-0.2,-0.3)}]\scriptsize$A$}] (A) at (2,9) {$\bullet$};
      \node[label={[shift={(0.2,-0.6)}]\scriptsize$F$}] at (2,6.5) {$\bullet$};
      \node[label={[shift={(0.2,-0.7)}]\scriptsize$F'$}] at (4.5,9) {$\bullet$};
      \draw[dashed] (ri.center) -- (2,10);
      \draw[dashed] (di.center) -- (10,9);
      \path[draw] (O.center) -- (10,10) -- (0,10) -- cycle;
      \draw[dashed] (emin1.center) -- (4.5,10);
      \draw[dashed] (smax2.center) -- (10,6.5);

      \draw[->] (O.center) -- (T1.center) node[above left] {\scriptsize$t_1$};
      \draw[->] (O.center) -- (T2.center) node[below left] {\scriptsize$t_2$};
    \end{tikzpicture}}

  \vspace{0.7cm}
  Cas $ W_i \le f_i(\bmin) (\LE-\ES)$


  \subcaptionbox{$ \emin \le \smax $}[0.45\linewidth]{
    \begin{tikzpicture}
      [scale=0.4]
      \fill[gray!20] (0,0) -- (10,10) -- (0,10) -- cycle;
      % non common
      \draw[fill=LightGreen] (0,9) -- (2,9) -- (4.5,6.5) --(0,6.5) -- cycle;
      \draw[fill=LightBlue] (2,9) -- (2,10) -- (4.5,10) --(4.5,6.5) --cycle;
      \node[label={[shift={(0.3,-0.7)}]\scriptsize$B$}] (B) at (4.5,6.5) {$\bullet$};
      \node[label={[shift={(-0.4,-0.6)}]\scriptsize$\smax$}] (smax2) at (0,6.5) {$\bullet$};

      \node[label={[shift={(0,-0.8)}]\scriptsize$\emin$}] (emin1) at (4.5,0) {};
      \node at (4.5,10) {$\bullet$};
      % common

      \draw[fill=LightCoral!40!] (0,10) rectangle (2,9);
      \node (O) at (0,0) {};
      \node (T1) at (11,0) {};
      \node (T2) at (0,11) {};


      \node[label={[shift={(-0.4,-0.6)}]\scriptsize$\LE$}] (di) at (0,9) {$\bullet$};
      \node[label={[shift={(-0.4,-0.6)}]\scriptsize$E_{max}$}] at (0,10) {};

      \node[label={[shift={(0,-0.8)}]\scriptsize$S_{min}$}] at (0,0) {};
      \node[label={[shift={(0,-0.8)}]\scriptsize$\ES$}] (ri) at (2,0) {};
      \node[label={[shift={(0,-0.8)}]\scriptsize$E_{max}$}] at (10,0) {};

      \node at (2,10) {$\bullet$};

      \node[label={[shift={(-0.2,-0.3)}]\scriptsize$A$}] (A) at (2,9) {$\bullet$};

      \draw[dashed] (ri.center) -- (2,10);
      \draw[dashed] (emin1.center) -- (4.5,10);
      \draw[dashed] (di.center) -- (10,9);
      \draw[dashed] (smax2.center) -- (10,6.5);
      \path[draw] (O.center) -- (10,10) -- (0,10) -- cycle;
      \draw[->] (O.center) -- (T1.center) node[above left] {\scriptsize$t_1$};
      \draw[->] (O.center) -- (T2.center) node[below left] {\scriptsize$t_2$};
    \end{tikzpicture}}
  \hfill
  \subcaptionbox{$\emin \ge \smax$}[0.45\linewidth]{
    \begin{tikzpicture}
      [scale=0.4]
      \fill[gray!20] (0,0) -- (10,10) -- (0,10) -- cycle;
      \draw[fill=Cornsilk!99!black!40!] (6,6) -- (5,5) -- (4.5,6.5) -- cycle;
      \draw[fill=LightGreen] (0,9) -- (2,9)-- (4.5,6.5) -- (5,5) -- (0,5) -- cycle;
      \draw[fill=LightBlue]  (2,9) -- (2,10)-- (6,10) -- (6,6)-- (4.5,6.5) -- cycle;
      \node[label={[shift={(0,-0.8)}]\scriptsize$C$}] (C) at (5,5)  {$\bullet$};
      \node[label={[shift={(0.3,-0.7)}]\scriptsize$C'$}] (C2) at (6,6)  {$\bullet$};

      \node[label={[shift={(0.2,-0.3)}]\scriptsize$G$}] at (4.5,6.5)  {$\bullet$};
      \node[label={[shift={(-0.4,-0.6)}]\scriptsize$\smax$}] (smax2) at (0,5) {$\bullet$};

      \node[label={[shift={(0,-0.8)}]\scriptsize$\emin$}] (emin1) at (6,0) {};

      \node at (6,10) {$\bullet$};
      % common

      \draw[fill=LightCoral!40!] (0,10) rectangle (2,9);
      \node (O) at (0,0) {};
      \node (T1) at (11,0) {};
      \node (T2) at (0,11) {};

      \node[label={[shift={(-0.4,-0.6)}]\scriptsize$\LE$}] (di) at (0,9) {$\bullet$};
      \node[label={[shift={(-0.4,-0.6)}]\scriptsize$E_{max}$}] at (0,10) {};

      \node[label={[shift={(0,-0.8)}]\scriptsize$S_{min}$}] at (0,0) {};
      \node[label={[shift={(0,-0.8)}]\scriptsize$\ES$}] (ri) at (2,0) {};
      \node[label={[shift={(0,-0.8)}]\scriptsize$E_{max}$}] at (10,0) {};

      \node at (2,10) {$\bullet$};

      \node at (2,3) {};
      \node at (5,5) {};
      \node[label={[shift={(-0.2,-0.3)}]\scriptsize$A$}] (A) at (2,9) {$\bullet$};

      \draw[dashed] (ri.center) -- (2,10);
      \draw[dashed] (emin1.center) -- (6,10);
      \draw[dashed] (di.center) -- (10,9);
      \draw[dashed] (smax2.center) -- (10,5);
      \path[draw] (O.center) -- (10,10) -- (0,10) -- cycle;
      \draw[->] (O.center) -- (T1.center) node[above left] {\scriptsize$t_1$};
      \draw[->] (O.center) -- (T2.center) node[below left] {\scriptsize$t_2$};

    \end{tikzpicture}
  }

  \vspace{0.7cm}
  Cas $  W_i \ge f_i(\bmin) (\LE - \ES)$

  \subcaptionbox{$\emin \le \smax$}[0.45\linewidth]{
    \begin{tikzpicture}
      [scale=0.4]
      % non common
      \fill[gray!20] (0,0) -- (10,10) -- (0,10) -- cycle;
      \path[fill=Cornsilk!99!black!40!] (2,7.5) -- (3.5,9)-- (3.75,7.3) -- cycle;
      \draw[fill=LightGoldenrod] (2,7.5) -- (3.5,9) -- (2,9) -- cycle;
      \draw[fill=LightGreen] (0,9) -- (2,9) -- (2,7.5) -- (0,7.5) --cycle;
      \draw[fill=PaleGreen!70!] (0,7.5) -- (2,7.5) -- (3.75,7.3) -- (4.5,6.5) -- (0,6.5) -- cycle;
      \draw[fill=LightBlue] (2,9) -- (2,10) -- (3.5,10) --(3.5,9)--cycle;
      \draw[fill=PowderBlue!70!] (4.5,6.5)-- (3.75,7.3) -- (3.5,9) -- (3.5,10) -- (4.5,10) --cycle;
      \node[label={[shift={(-0.2,-0.7)}]\scriptsize$D$}] (D) at (2,7.5) {$\bullet$};
      \node[label={[shift={(0.2,-0.3)}]\scriptsize$D'$}] (D2) at (3.5,9) {$\bullet$};
      \node[label={[shift={(0.1,-0.3)}]\scriptsize$G$}] (E) at (3.75,7.3)  {$\bullet$};
      \node[label={[shift={(0.3,-0.7)}]\scriptsize$B$}] (B) at (4.5,6.5) {$\bullet$};
      \node[label={[shift={(-0.4,-0.6)}]\scriptsize$\smax$}] (smax2) at (0,6.5) {$\bullet$};

      \node[label={[shift={(0,-0.8)}]\scriptsize$\emin$}] (emin1) at (4.5,0) {};
      \node at (4.5,10) {$\bullet$};
      \node at (3.5,10) {$\bullet$};
      \node at (0,7.5) {$\bullet$};
      % common

      \draw[fill=LightCoral!80!] (0,10) rectangle (2,9);
      \node (O) at (0,0) {};
      \node (T1) at (11,0) {};
      \node (T2) at (0,11) {};


      \node[label={[shift={(-0.4,-0.6)}]\scriptsize$\LE$}] (di) at (0,9) {$\bullet$};
      \node[label={[shift={(-0.4,-0.6)}]\scriptsize$E_{max}$}] at (0,10) {};

      \node[label={[shift={(0,-0.8)}]\scriptsize$S_{min}$}] at (0,0) {};
      \node[label={[shift={(0,-0.8)}]\scriptsize$\ES$}] (ri) at (2,0) {};
      \node[label={[shift={(0,-0.8)}]\scriptsize$E_{max}$}] at (10,0) {};

      \node at (2,10) {$\bullet$};

      \node at (2,3) {};
      \node at (5,5) {};
      \node[label={[shift={(-0.2,-0.3)}]\scriptsize$A$}] (A) at (2,9) {$\bullet$};

      \draw[dashed] (ri.center) -- (D.center);
      \draw[dashed] (emin1.center) -- (4.5,10);
      \draw[dashed] (di.center) -- (10,9);
      \draw[dashed] (smax2.center) -- (10,6.5);
      \path[draw] (O.center) -- (10,10) -- (0,10) -- cycle;
      \draw[->] (O.center) -- (T1.center) node[above left] {\scriptsize$t_1$};
      \draw[->] (O.center) -- (T2.center) node[below left] {\scriptsize$t_2$};

    \end{tikzpicture}
  }
  \hfill
  \subcaptionbox{$\emin \ge \smax$}[0.45\linewidth]{
    \begin{tikzpicture}
      [scale=0.4]
      % non common
      \fill[gray!20] (0,0) -- (10,10) -- (0,10) -- cycle;
      \path[fill=Cornsilk!99!black!40!] (2,6.25) -- (4.75,9)-- (6,6) -- (5,5) -- cycle;
      \draw[fill=LightGoldenrod] (2,6.25) -- (2,9) -- (4.75,9) -- cycle;
      \draw[fill=LightGreen] (0,9) -- (2,9) -- (2,6.25) -- (0,6.25) -- cycle;
      \draw[fill=PaleGreen!70!] (0,6.25) -- (2,6.25) -- (5,5) -- (0,5) -- cycle;
      \draw[fill=LightBlue] (2,9) -- (2,10) -- (4.75,10) -- (4.75,9) -- cycle;
      \draw[fill=PowderBlue!70!] (4.75,10) -- (6,10) -- (6,6)-- (4.75,9) -- cycle;
      \node[label={[shift={(-0.2,-0.7)}]\scriptsize$D$}] (D) at (2,6.25) {$\bullet$};
      \node[label={[shift={(0.2,-0.3)}]\scriptsize$D'$}] (D2) at (4.75,9) {$\bullet$};
      \node[label={[shift={(0,-0.8)}]\scriptsize$C$}] (C) at (5,5)  {$\bullet$};
      \node[label={[shift={(0.3,-0.5)}]\scriptsize$C'$}] (C2) at (6,6)  {$\bullet$};
      \node[label={[shift={(-0.4,-0.6)}]\scriptsize$\smax$}] (smax2) at (0,5) {$\bullet$};

      \node[label={[shift={(0,-0.8)}]\scriptsize$\emin$}] (emin1) at (6,0) {};

      \node at (6,10) {$\bullet$};
      \node at (4.75,10) {$\bullet$};
      \node at (0,6.25) {$\bullet$};
      % common

      \draw[fill=LightCoral!80!] (0,10) rectangle (2,9);
      \node (O) at (0,0) {};
      \node (T1) at (11,0) {};
      \node (T2) at (0,11) {};

      \node[label={[shift={(-0.4,-0.4)}]\scriptsize$S_{min}$}] at (0,0) {};
      \node[label={[shift={(-0.4,-0.6)}]\scriptsize$\LE$}] (di) at (0,9) {$\bullet$};
      \node[label={[shift={(-0.4,-0.6)}]\scriptsize$E_{max}$}] at (0,10) {};

      \node[label={[shift={(0,-0.8)}]\scriptsize$S_{min}$}] at (0,0) {};
      \node[label={[shift={(0,-0.8)}]\scriptsize$\ES$}] (ri) at (2,0) {};
      \node[label={[shift={(0,-0.8)}]\scriptsize$E_{max}$}] at (10,0) {};

      \node at (2,10) {$\bullet$};

      \node at (2,3) {};
      \node at (5,5) {};
      \node[label={[shift={(-0.2,-0.3)}]\scriptsize$A$}] (A) at (2,9) {$\bullet$};

      \draw[dashed] (ri.center) -- (D.center);
      \draw[dashed] (emin1.center) -- (6,10);
      \draw[dashed] (di.center) -- (10,9);
      \draw[dashed] (smax2.center) -- (10,5);
      \path[draw] (O.center) -- (10,10) -- (0,10) -- cycle;
      \draw[->] (O.center) -- (T1.center) node[above left] {\scriptsize$t_1$};
      \draw[->] (O.center) -- (T2.center) node[below left] {\scriptsize$t_2$};

    \end{tikzpicture}
  }
  \caption{Les différentes expressions de $\bb$ en fonction des
    paramètres du problème \CECSP.}
  \label{fig:breakline_res}
\end{figure}

Pour calculer les intervalles d'intérêt pour l'algorithme de
vérification du raisonnement énergétique, il faut, pour chaque paire
de segments de définition, calculer leur point d'intersection, s'il
existe. Les coordonnées de ce point d'intersection correspondent à un
intervalle d'intérêt. 

Pour calculer ces points d'intersection, deux méthodes existent. La
première est la plus naïve et consiste à considérer toutes les paires
de segments et calculer leur intersection. La seconde méthode utilise
l'algorithme de balayage de Bentley-Ottmann~\cite{sweep}.  L'idée
principale sur laquelle repose l'algorithme de balayage est que deux
segments ne peuvent s'intersecter si leurs projections sur l'axe
des ordonnées et sur l'axe des abscisses ne se chevauchent pas. Une
ligne horizontale fictive est utilisée pour balayer l'axe des
abscisses et à certains ``événements'', on teste l'intersection entre
deux segments s'ils intersectent tous les deux cette ligne fictive et
s'ils sont consécutifs dans l'ordre vertical. Avec cet algorithme, le
nombre de tests d'intersection peut grandement décroître par rapport à
l'algorithme naïf.

Dans le cas de l'algorithme naïf, la complexité totale de l'algorithme
de vérification est de $O(n^3)$ tandis que, pour l'algorithme de
balayage la complexité est de $O((n^2+nk) \log n)$, avec $k$ le nombre
de points d'intersection. En théorie, la complexité du second
algorithme peut être plus élevée ($k$ peut être de l'ordre de
$O(n^2)$), mais en pratique, cette complexité peut devenir très faible
si le nombre de points d'intersection est petit. Une comparaison de
ces deux algorithmes sera effectué dans le chapitre~\ref{sec:expe}.

\subsubsection{Analyses des variations de la fonction de marge et des
  fonctions d'ajustements}

Nous allons décrire ici une troisième méthode permettant le calcul des
intervalles d'intérêt du raisonnement énergétique. Cette méthode est
l'adaptation des travaux de~\cite{DP} dans le cadre du \CUSP. Dans un
premier temps, nous décrivons la méthode permettant de calculer ces
intervalles pour l'algorithme de vérification, puis nous présenterons
ce même résultat dans le cadre des ajustements de fenêtres de temps. 

 La méthode décrite dans ce paragraphe repose sur le fait que, pour
 chercher un intervalle $[t_1,t_2[$ vérifiant $SL(t_1,t_2) < 0$, il
 suffit de s'intéresser au point $(t_1,t_2)$ pour lesquels la
 fonction de marge est minimale. Comme, en pratique, il est difficile
 de trouver le minimum global d'une telle fonction, nous nous
 intéressons à ces minima locaux. Grâce au test de la dérivée
 seconde (voir sous-section~\ref{sec:nrj_CUSP}), nous pouvons dériver
 des conditions selon lesquels $SL$ est localement minimale.

\begin{lemma}[\cite{DP}]
\label{lem:min_CECSP}
La fonction de marge $SL(t_1,t_2)$ est localement minimale seulement
s'il existe deux activités $i$ et $j$ telles que les conditions
ci-dessous sont satisfaites: 
\begin{align} \frac{\delta^{-}\bb}{\delta t_1} &>
\frac{\delta^{+}\bb}{\delta t_1} \label{eq:deriv1_CECSP}\\ 
\frac{\delta^{-}\bb[j]}{\delta t_2}
& > \frac{\delta^{+}\bb[j]}{\delta t_2} \label{eq:deriv2_CECSP}
\end{align}
\end{lemma}

Le preuve est identique à celle du lemme~\ref{lem:min_CUSP}. Le
lemme~\ref{lem:min_CECSP} peut être utilisé pour déterminer les
conditions nécessaires permettant de déterminer l'ensemble des
intervalles d'intérêt. Pour cela, une étude des fonctions $t_1
\rightarrow \bb$ et $t_2 \rightarrow \bb$ est nécessaire. 

\begin{theo}
  Pour chaque activité $i$ et pour tout début d'intervalle $t_1$ il
  existe au plus deux intervalles $[t_1,t_2[$ tels que
  $\frac{\delta^{-}\bb}{\delta t_2} >
  \frac{\delta^{+}\bb}{\delta t_2} $. 
  De même,  pour chaque activité $i$ et pour toute fin d'intervalle $t_2$, il
  existe au plus deux intervalles $[t_1,t_2[$ tels que
  $\frac{\delta^{-}\bb}{\delta t_1} >
  \frac{\delta^{+}\bb}{\delta t_1} $. 
  Ces intervalles sont décrits
  par le tableau~\ref{fig:interCheck_CECSP} avec:
  \begin{center}
    \begin{tabular}{lcr}
      $\Delta'(t_2)=\frac{t_2(f_i(\bmin)-f_i(\bmax))+\LE f_i(\bmax)-W_i}{f_i(\bmin)}$
      & ;
      &$\Gamma'(t_2)=\frac{W_i-t_2f_i(\bmin)+\ES f_i(\bmax)}{f_i(\bmax)-f_i(\bmin)}$
    \end{tabular}
  \end{center}

les points de cassures de la fonction $t_1 \rightarrow \bb$ et 

\begin{center}
\begin{tabular}{lcr}
  $\Gamma(t_1)=\frac{W_i-t_1(f_i(\bmin)-f_i(\bmax))+\ES f_i(\bmax)}{f_i(\bmin)}$&
  ;
  &$\Delta(t_1)=\frac{W_i-f_i(\bmin)\LE +t_1f_i(\bmax)}{f_i(\bmax)-f_i(\bmin)}$
\end{tabular}
\end{center}

de la fonction $t_2 \rightarrow \bb$.

\begin{table} 
  \input{part2/CECSP/tabular.tex}
  \caption{Intervalles d'intérêt pour l'algorithme de vérification du
    raisonnement énergétique pour le \CECSP.}
  \label{fig:interCheck_CECSP}
\end{table}
\end{theo}

Notons que le cas $\bmin=\bmax$, i.e. le cas cumulatif, est inclus
dans le cas $W_i \le f_i(\bmin)(\LE - \ES )$ et que dans ce cas: 
\begin{center}
\begin{tabular}{lcr}
$\Delta'(t_2)=\smax$ & et &
$\Gamma(t_1)=\emin$
\end{tabular}
\end{center}
 \begin{proof}
Nous présentons seulement comment nous obtenons les $t_2$ pertinents
pour la huitième colonne du tableau~\ref{tab:interCheck_t2}, tous les
autres cas étant obtenus de manière similaire. La signification de
cette colonne est la suivante: si $W_i \ge f_i(\bmax)(\LE-\ES)$, $\ES
\le t_1 \le \EE$ et $t_1 \le D_{t_1}$, alors les intervalles à
considérer sont $[t_1,\LE {[}$ et $[t_1, \Gamma(t_1){[}$.

Pour prouver le lemme, nous analysons les variations de la fonction
$t_2 \rightarrow \underline{b}(i,t_1,t_2)$. Pour analyser ces
variations, nous utilisons les expressions de $\bb$ en fonction de
$t_1$ et $t_2$ décrites dans le paragraphe précédent. Pour adapter ces
zones dans le cas des fonctions concaves et affines par morceaux, il
faut déterminer, dans les zones où l'expression de $\bb$ est
$\frac{1}{a_i}(\wb - c_i|I|)$, la valeur de $\max_{p \in
\P_i}(\frac{1}{a_{ip}}(\wb - c_{ip}|I|))$. Pour ce faire, notons que
le segment $\overline{DD'}$ dans la figure~\ref{fig:breakline_res},
sépare la zone où $|I|= t_2 - t_1 \ge \wb/ f_i(\bmin)$ de la zone où
cette inégalité n'est pas vérifiée. De la même manière, nous pouvons
définir une série de segments parallèles ,
$\overline{D_{p}D_{p}^{'}}$, séparant les zones où $|I|= t_2 - t_1 \ge
\wb/f_i(x_p^i)$ et celle où cette inégalité n'est pas vérifiée,
i.e. séparant les zones où $\bb= \frac{1}{a_{ip}}(\wb -
c_{ip}(t_2-t_1))$. La figure~\ref{fig:variation} représente ces
variations.
   \begin{figure}[!htb] \input{part2/CECSP/derivative4.tex}
     \caption{Intervalles d'intérêt du raisonnement énergétique du
\CECSP~dans le cas où $\ES <t_1 \le D'_{t_1}$.}
     \label{fig:variation}
   \end{figure}


Les intervalles $\inter$ pour lesquels $t_2 \rightarrow \bb$ vérifie
que sa dérivée à 
gauche est inférieure à sa dérivée à droite sont 
${[}t_1,\LE{[}$ et $[t_1,\bd{t_1}[$. En effet, comme $f_i$ est une
fonction concave et affine par morceaux, nous avons: $a_{ip} >
a_{ip+1}$ et $c_{ip} <c_{ip+1}$, et donc
$\frac{\delta^{-}expr_{ip}}{\delta t_2}<
\frac{\delta^{+}expr_{ip+1}}{\delta t_2}$.

On peut remarquer que dans la figure~\ref{fig:variation}, l'indice
de $expr_{ip}$ va seulement jusqu'à $4$. Ceci est dû à une
simplification. Dans le cas général, l'indice de $expr_{ip}$ s'arrête
au rang $\ell$ si $W_i \le f_i(x^i_{\ell +1})(\LE - \ES)$.
\end{proof}

En appliquant un raisonnement symétrique pour toute fin d'intervalle
$t_2$ fixée, nous obtenons une liste d'intervalles d'intérêt, décrite
par le lemme~\ref{listinterval}.

\begin{lemma}
  \label{listinterval} 
Soient $i$ et $j$ deux activités telles que: $W_l\ge
f_l(\bmin[l])(\LE[l]-\ES[l]),\ l=i,j$. Alors, la fonction de marge est
localement minimum seulement si $(t_1,t_2)$ correspond à l'un des
intervalles suivants:
\[ \left\{
    \begin{aligned} 
      &{[}\ES[j],\LE{]} & & \text{si } \Big(\ES[j] \le \ES \lor (\ES[j]
      \le \emin \land \ES[j] \le \itu)\Big) \land \\*
      & & &\Big(\LE \ge \LE[j] \lor (\LE \ge \smax[j] \land \LE \ge \itd)\Big)\\
      \rule[-0.8em]{0pt}{2em}      
      &{[}\Delta'(\LE),\LE{]} & & \text{si } \Big(\bdp{\LE} \le \ES \lor
      (\bdp{\LE}\le \emin \land \bdp{\LE}\le \itu)\Big) \land \\*
      & & &\LE[j] \ge \LE \ge \smax[j] \land \LE \le \itd \land \LE \ge
      \htd\\
      \rule[-0.8em]{0pt}{2em}
      &{[}\Gamma'(\LE),\LE{]} & & \text{si } \Big(\bup{\LE} \le \ES \lor
      (\bup{\LE} \le \emin \land \bup{\LE} \le \itu)\Big) \\*
      & & &\land \LE[j] \ge \LE \ge \smax[j] \land \Big(\LE \ge \emin[j]
      \lor \LE \ge E_{t_2}\Big)\\
      \rule[-0.8em]{0pt}{2em}
      &{[}\LE[j]+\ES[j]-\LE,\LE{]} & & \text{si } \Big(\LE[j]+\ES[j]-\LE \le \ES \lor
      (\LE[j]+\ES[j]-\LE \le \emin \land \\*
      & & & \LE[j]+\ES[j]-\LE \le D'_{t_1})\Big) \land \LE[j] \ge \LE
      \ge \smax[j] \land \LE\le \htd\\ 
      \rule[-0.8em]{0pt}{2em}
      &{[}\ES[j],\Gamma(\ES[j]){]} & &\text{si } \ES \le \ES[j] \le \emin \land
      \ES[j] \ge \itu \land \ES[j] \le \htu \land \\* 
      & & &\Big(\bu{\ES[j]} \ge \LE[j] \lor (\bu{\ES[j]} \ge \smax[j] \land \bu{\ES[j]} \ge
      \itd)\Big)\\
      \rule[-0.8em]{0pt}{2em}
      &{[}\ES[j],\Delta(\ES[j]){]} & &\text{si } \Big(\Delta(\ES[j]) \ge \LE[j] \lor
      (\Delta(\ES[j]) \ge \smax[j] \land \Delta(\ES[j]) \ge D_{t_2})\Big) \land
      \\* 
      & & &\ES \le \ES[j] \le \emin \land \Big(\smax \ge \ES[j] \lor \ES[j] \le
      E_{t_1}\Big)\\
      \rule[-0.8em]{0pt}{2em}
      &{[}\ES[j],\LE+\ES-\ES[j]{]} & & \text{si } \Big(\LE+\ES-\ES[j] \ge \LE[j] \lor
      (\LE+\ES-\ES[j] \ge \smax[j] \\*
& & &\land \LE+\ES-\ES[j] \ge \itd)\Big) \land 
      \ES \le \ES[j] \le \emin \land \ES[j] \ge \htu
    \end{aligned} \right.
\]
\end{lemma}

Le lemme~\ref{listinterval} présente les résultats pour le cas où $i$
et $j$ vérifient $W_l\ge f_l(\bmin[l])(\LE[l]-\ES[l]),\ l=i,j$. Les autres
cas à considérer sont:
\begin{itemize}
\item $\bmin=0$ et $\bmin[j]=0$
\item $\bmin=0$ et $W_j\le f_j(\bmin[j])(\LE[j]-\ES[j])$
\item $\bmin=0$ et $W_j\ge f_j(\bmin[j])(\LE[j]-\ES[j])$
\item $W_l\le f_l(\bmin[l])(\LE[l]-\ES[l]),\ l=i,j$
\item $W_i\ge f_i(\bmin)(\LE-\ES)$ et $W_j\ge
f_j(\bmin[j])(\LE[j]-\ES[j])$
\end{itemize} 
Ces cas peuvent être déduits de manière similaire. De plus,
l'algorithme décrit ci-dessous pour appliquer le test de
satisfiabilité du raisonnement énergétique n'utilise pas directement
cette liste.

Pour décrire cet algorithme, nous introduisons les notations
suivantes: 
\begin{itemize}
\item $O_1=\{\ES ,\ \forall i \in \A\} \cup \{\LS ,\ \forall i \in \A \, |\,
  \bmin = \bmax \} $
\item $O_2=\{\LE ,\ \forall i \in \A\} \cup \{\EE ,\ \forall i \in \A \, |\,
  \bmin = \bmax \}$
\end{itemize}

L'algorithme de détection d'incohérence est alors le suivant:
\begin{algorithm}[!htb]
\setstretch{1.35}
  \caption{Algorithme de détection d'incohérence énergétique du \CECSP.}
  \label{algo:check_CECSP}
  \PourTous {$t_1 \in O_1$}{
    \PourTous {$i \in \A$}{
      Calculer $O_{t_2}$ l'ensemble des $t_2$ vérifiant la
      condition~\eqref{eq:deriv2_CECSP} à l'aide du 
      tableau~\ref{tab:interCheck_t2}
      \PourTous {$t_2 \in O_{t_2}$}{
        \Si {$R(t_2-t_1) < \sum_{j \in \A} \bb[j]$}{
          Le problème est infaisable}
      }
    }
}  
\PourTous {$t_2 \in O_2$}{
  \PourTous {$i \in \A$}{
      Calculer $O_{t_1}$ l'ensemble des $t_1$ vérifiant la
      condition~\eqref{eq:deriv1_CECSP} à l'aide du
      tableau~\ref{tab:interCheck_t1}
      \PourTous {$t_1 \in O_{t_1}$}{
        \Si {$R(t_2-t_1) < \sum_{j \in \A} \bb[j]$}{
         Le problème est infaisable}
      }
    }
}
\end{algorithm}

L'algorithme~\ref{algo:check_CECSP} est de complexité 
$O(n^3)$. En effet, pour chaque couple d'activités $(i,j)$ au plus
deux intervalles sont considérés. La complexité est donc la même que
dans la première méthode présentée. Cependant, la constante cachée est
beaucoup plus petite dans ce cas-là. Une comparaison des performances
de chaque méthode sera présentée dans le chapitre~\ref{sec:expe}.

Nous allons maintenant montrer comment ce raisonnement peut aussi
s'appliquer dans le cas du calcul des intervalles d'intérêt pour les
ajustements. Pour ce faire, nous utilisons le
lemme~\ref{lem:min_ajust_ER_CUSP}. Comme nous avons déjà calculé les
points $t_1$ vérifiant $\frac{\delta^{-}\bb}{\delta
t_1}>\frac{\delta^{+}\bb}{\delta t_1}$ et les points $t_2$ vérifiant
$\frac{\delta^{-}\bb}{\delta t_2} >\frac{\delta^{+}\bb}{\delta
t_2}$, il suffit, pour caractériser les intervalles d'intérêt pour les
ajustements, de calculer les points $t_1$ vérifiant:
\begin{equation}
\frac{\delta^{-}\bbRS}{\delta t_1} >\frac{\delta^{+}\bbRS}{\delta
t_1}\label{eq:derivt1_CECSP} \end{equation}
et les points $t_2$ vérifiant \begin{equation}
\frac{\delta^{-}\bbRS}{\delta t_2} > \frac{\delta^{+}\bbRS}{\delta
t_2}\label{eq:derivt2_CECSP}\end{equation}
Pour cela, nous analysons les variations des fonctions $t_1
\rightarrow \bbRS$ et $t_2 \rightarrow \bbRS$. Nous allons montrer
comment cela a été fait pour $t_1 \rightarrow \bbRS$ et dans le cas où
$W_i \ge f_i(\bmin) (\LE - \ES)$, les autres cas étant déduits de
manière similaire. Le tableau~\ref{tab:interAjust_CECSP} présente 
les résultats dans ces autres cas. 


\begin{theo}
  Pour chaque activité $i$ et pour tout début d'intervalle $t_1$, il
  existe au plus deux intervalles $[t_1,t_2[$ tels que
  l'équation~\eqref{eq:derivt2_CECSP} soit vérifiée.
  De même,  pour chaque activité $i$ et pour toute fin d'intervalle $t_2$, il
  existe au plus deux intervalles $[t_1,t_2[$ tels que
  l'équation~\eqref{eq:derivt1_CECSP} soit vérifiée. 
  Ces intervalles sont décrits
  par le tableau~\ref{tab:interAjust_CECSP}.
  \begin{table}[!htb]
  \input{part2/CECSP/tabularRS.tex}
  \caption{Intervalles d'intérêt pour les ajustements du
    raisonnement énergétique pour le \CECSP (placement à droite).}
  \label{tab:interAjust_CECSP}
\end{table}
\end{theo}

\begin{proof}
  Nous présentons seulement comment nous obtenons les $t_2$ pertinents
dans le cas où $W_i \ge f_i(\bmax)(\LE - \ES)$, i.e. les deux dernières
colonnes du tableau~\ref{tab:interAjust_t2}. Les autres cas sont
obtenus de manière similaire. Pour prouver le lemme, nous analysons
les variations de la fonction $t_2 \rightarrow \bbRS$.


\begin{figure}[!htb]
 \subcaptionbox{Cas $t_1 \le \ES$  \label{fig:variation1}}[\linewidth] {
   \input{part2/CECSP/derivative1.tex} }
 \subcaptionbox{Cas $\ES < t_1 < \smax$
   \label{fig:variation2}}[\linewidth]{
  \input{part2/CECSP/derivative2.tex}}
 \subcaptionbox{Cas $\smax \le  t_1 < \LE $
  \label{fig:variation3}}[\linewidth]
{\input{part2/CECSP/derivative3.tex}}
\caption{Variation de $\bbRS$ en fonction de $t_2$.}
\end{figure}


Les intervalles $\inter$ pour lesquels $t_2 \rightarrow \bbRS$ vérifie
que sa dérivée à gauche est supérieure à sa dérivée à droite sont:
\begin{itemize}
\item $[t_1, \LE {[}$ dans le cas où $t_1 \le \ES$;
\item $[t_1, \LE {[}$ et $[t_1, \Delta(t_1) [$ dans le cas où $\ES <
  t_1 < \smax$;
\item $[t_1, \LE {[}$ dans le cas où $\smax \le t_1 < \LE$.
\end{itemize}
En simplifiant et rassemblant les cas ci-dessus, nous obtenons: \begin{itemize}
\item $[t_1, \LE {[}$ dans le cas où $t_1 < \LE$;
\item $[t_1, \Delta(t_1) [$ dans le cas où $\ES <
  t_1 < \smax$.
\end{itemize}
\end{proof}

Un raisonnement identique amène à la caractérisation des intervalles
d'intérêt pour $\bbLS$. Ces résultats sont décrits dans le
tableau~\ref{tab:interAjustLS_CECSP}.
  \begin{table}[!htb]
  \input{part2/CECSP/tabularLS.tex}
  \caption{Intervalles d'intérêt pour les ajustements du
    raisonnement énergétique pour le \CECSP (placement à gauche).}
  \label{tab:interAjustLS_CECSP}
\end{table}

Comme dans le cas de l'algorithme de vérification, cette
caractérisation nous permet de définir une liste d'intervalles sur
lesquels appliquer les ajustements. Pour cela, il suffit, pour chaque
couple $(i,t_1)$ et pour chaque activité $j$, de calculer les
$t_2$ d'intérêt à l'aide du tableau~\ref{tab:interAjust_CECSP} pour
les ajustements des dates de début et à l'aide du
tableau~\ref{tab:interAjustLS_CECSP} pour 
les ajustements des dates de fin. Pour une activité $i$, le nombre
d'intervalles d'intérêt est donc au plus de $4(n-1)^2 + 2* (2n+3)$.












\section{Modèle de programmation par contraintes pour le cas discret}
%
\clearemptydoublepage%
%TODO justification modèle RCPSP discret + phrases de transition
%entre modele discret et modele à evenement + justification
%présentation des modèle

%!TeX root =../main_file.tex 

\cleardoublepage
\begin{minipage}{0.95\linewidth}
\part{Programmation linéaire en nombres entiers}
\label{part:PLNE}
\vspace{15mm} % l'espacement souhaité
\parttoc 
\end{minipage}
\newpage
\thispagestyle{empty}
\vspace*{\stretch{1}}
\begin{center}
  \begin{minipage}{\textwidth}
    \hrule
    \vspace{0.5cm}
    {\it Dans cette partie consacrée à l'adaptation des techniques de
      résolution issues de la programmation linéaire mixtes et en
      nombres entiers, les concepts généraux qu'elles utilisent sont
      décrits dans le paragraphe~\ref{sec:PLNE}. Ces techniques sont
      ensuite présentées dans le cadre d'un célèbre problème
      d'ordonnacement: le RCPSP. 

      Pour ce problème, trois modèles sont présentés. Le premier est
      un modèle indexé par le temps et fait partie des classes de modèles
      les plus performants pour la résolution des instances de la
      PSPLIB~\cite{PSPLIB} pour le \RCPSP. Cependant, ces modèles ont leurs
      limitations puisqu'entre autre, leur taille dépent grandement de
      l'horizon de temps du projet. Pour des plannification à long terme,
      d'autres modèles plus performants sont introduits: les modèles à
      événements. Ils ont l'avantage d'avoir un nombre polynomiale de
      variables et de contraintes et permettent la modélisation des
      problèmes continus. Cependant, leur relaxation linéaire sont
      relativement moins fortes que celles des modèles indexés par le temps,
      ce qui les rend non compétitivés dans le cas des instances ayant un
      petit horizon de temps. Ces modèles dans le cadre du \RCPSP~ont été
      introduit par Kone et al.~\cite{modele_RCPSP}.

      Dans le chapitre~\ref{sec:PLNE_CECSP}, nous présentons
      l'adaptation de ces modèles dans le cadre du \CECSP. Ces modèles
      sont présentés
      dans~\cite{Nattaf_Constraints,Nattaf_ORSpectrum,Nattaf_CPDP}. De plus,
      pour chacun des trois modèles, des techniques visant à améliorer leurs
      performances sont présentées. Pour le modèle indexé par le
      temps, nous utilisons le raisonnement énergétique afin de
      décrire des inégalités valides pouvant être ajoutées au modèle
      (avant ou pendant la résolution). Cette technique a été
      présentée dans~\cite{Nattaf_JFPC}.

      Pour les modèles à événements, plusieurs ensembles d'inégalités
      sont présentées: des inégalités permettant de borner la distance
      entre deux événements, d'autres permettant de borner la date des
      événements ou des inégalités dérivées du problème du
      sac-à-dos. De plus, des coupes appelées inégalités de
      non-préemption sont définies et une étude polyhédrale de
      l'ensemble des vecteurs binaires solution d'un des modèles
      présenté est détaillé. Ces résultats ont été présenté
      dans~\cite{Nattaf_ROADEF16,Nattaf_ECCO16}. }
    \vspace{0.5cm}
    \hrule
  \end{minipage}
\end{center}
\vspace*{\stretch{1}}

\chapter{Programmation linéaire et ordonnancement
  de projet}

%!TeX root =../main_file.tex

\section{La programmation linéaire en nombres entiers}
\label{sec:PLNE}
Dans ce paragraphe, nous présentons les concepts de base de la
programmation linéaire ainsi que les notions et outils qui seront
utilisés dans la suite de cette partie. Cette présentation n'étant que
partielle, nous renvoyons le lecteur à~\cite{LP} pour une description
plus détaillée des différentes techniques existantes.

\subsection{La programmation linéaire}

La programmation linéaire vise à résoudre des problèmes
d'optimisation ayant la particularité de pouvoir s'exprimer à
l'aide de contraintes, se présentant sous la forme d'inégalités
et/ou d'égalités, linéaires en fonction des variables du problème.
De plus, la fonction objectif du problème, i.e. ce que l'on
cherche à maximiser/minimiser, s'écrit aussi sous la forme d'une
fonction linéaire.

Sous forme canonique, un programme linéaire (PL) s'écrit de la manière
suivante: 
 \[ \begin{array}{lcl}
\text{maximiser } & & \displaystyle cx\\ 
\text{tel que }& & \displaystyle Ax \le b\\
 & & \displaystyle x \in \mathbb{R}^n_+
 \end{array}
\]
avec $c \in \mathbb{R}^n$, $b \in \mathbb{R}^m$ et $A \in
\mathbb{R}^{m,n}$. 

Nous pouvons supposer, sans perte de généralité, que les variables du
problème sont positives et que la fonction objectif $x\rightarrow cx$
doit être maximisée. Les vecteurs $x$ de $\mathbb{R}^n$ qui vérifient
$Ax \le b$ sont appelés les solutions réalisables du problème et les
vecteurs $x^*$ qui, en plus, maximisent le critère $cx^*$ sont appelés
solutions optimales.    

La force de ces formulations repose sur le fait que l'ensemble des
contraintes du problème définit alors un polyèdre convexe nommé
polyèdre des solutions réalisables. De plus, si ce polyèdre est non
vide, alors toute solution optimale se trouve forcément sur un sommet,
appelé point extrême, de celui-ci. Les solutions optimales peuvent aussi
se trouver sur une face du polyèdre, i.e. l'intersection de celui-ci
avec un plan défini par une des contraintes du programme linéaire $\{x
\in \mathbb{R}^n\ | \ A_jx=b_j\}$. 

Ceci a permis la mise en place de plusieurs algorithmes pour résoudre
ces programmes linéaires. Un des plus efficaces en pratique et donc
des plus utilisés est l'algorithme du Simplexe. Le principe de cet
algorithme est le parcours "intelligent" des points extrêmes du
polyèdre des solutions admissibles: à chaque itération, l'algorithme
se "déplace" vers un sommet adjacent de meilleur (ou même)
coût. L'algorithme se termine donc lorsque tous les sommets adjacents
possèdent un coût moins bon que le sommet courant.

Un des inconvénients de cet algorithme est qu'il a, dans le pire des
cas, une complexité exponentielle. D'autres algorithmes permettent de
résoudre un programme linéaire en temps polynomial. Cependant, en
pratique, l'algorithme du simplexe reste plus efficace et c'est
souvent ce dernier qui est implémenté dans les solveurs d'optimisation
linéaire. 

\subsection{Contrainte d'intégrité}

De nombreux problèmes d'optimisation combinatoire ne peuvent s'écrire
sous la forme d'un programme linéaire. En effet, pour des problèmes
tels que les problèmes d'ordonnancement, tout ou partie des
variables doivent être restreintes à prendre des valeurs entières.  

Quand toutes les variables du problème sont contraintes à prendre des
valeurs entières, on parle de programme linéaire en nombres
entiers (PLNE). Lorsque seulement une partie des variables sont
autorisées à prendre des valeurs réelles, on parle de programme
linéaire mixte (PLM). 

De manière générale, un programme linéaire mixte 
peut s'écrire de la façon suivante: 

\[ \begin{array}{lcl}
\text{maximiser } & & \displaystyle \sum_{i=1}^p c_ix_i +
\sum_{j=1}^{n-p} f_jy_j\\ \text{tel que }& & \displaystyle
\sum_{i=1}^p a_{ki}x_i + \sum_{j=1}^{n-p} d_{kj}y_j \le b_k, \quad
k=1,\dots,m\\ & & \displaystyle x_i \ge0,\quad i=1,\dots,p\\ & &
\displaystyle y_j \in \mathbb{N},\quad j=1,\dots,n-p 
\end{array}
\] $y_j,\ j=1,\dots,n-p$ sont les variables de décisions ne
pouvant prendre que des valeurs entières tandis que les variables
$x_i,\ i=1,\dots,p$ peuvent prendre n'importe quelles valeurs
réelles positives.

La difficulté de la résolution de ces programmes repose sur le fait
que l'ensemble des solutions réalisables du problème ne forme plus un
polyèdre convexe et les algorithmes traditionnels ne peuvent être
appliqués directement dans ce cas. Le problème de décision associé à
un PLNE, ou à un PLM, est NP-complet~\cite{NP_bible}. De ce fait,
plusieurs techniques permettant de trouver la solution optimale en
temps raisonnable ont été développées.

\subsection{Techniques de résolution}

Parmi les techniques les plus utilisées, on retrouve les techniques
utilisant des algorithmes de recherche arborescente par séparation et
évaluation ainsi que des méthodes de coupes.

Le principe de la {\it recherche arborescente par séparation et
évaluation} repose sur deux idées principales: 
\begin{itemize}
\item {\bf la séparation} qui consiste à décomposer selon une
partition l'ensemble des solutions en plusieurs sous-ensembles;
\item {\bf l'évaluation} qui consiste à examiner la qualité des
solutions d'un sous-ensemble de façon optimiste, i.e. trouver une
borne supérieure (resp. inférieure) de la meilleure solution de ce
sous-ensemble lorsqu'il s'agit d'un problème de maximisation
(resp. minimisation).
\end{itemize} 
L’algorithme propose de parcourir l’arborescence des solutions
possibles en évaluant chaque sous-ensemble de solutions de façon
optimiste. Lors de ce parcours, il maintient la valeur de la meilleure
solution trouvée jusqu’à présent. Quand l’évaluation d’un
sous-ensemble donne une valeur plus faible (plus forte pour un
problème de minimisation) ou égale, il est inutile d’explorer plus loin ce
sous-ensemble. L’ensemble des solutions admissibles est ainsi
représenté par une arborescence dans laquelle un grand nombre de nœuds
peuvent être éliminés. L’évaluation des sous-problèmes se fait
typiquement par {\it relaxation linéaire} (on ignore la contrainte
d’intégrité) en utilisant le Simplexe.

Les {\it méthodes de coupes} cherchent à caractériser le plus petit
polyèdre contenant l'ensemble de toutes les solutions
admissibles. Pour cela, un ensemble de coupes, i.e. des inégalités
permettant de supprimer une partie du polyèdre ne contenant que des
solutions non entières, est généré. Si ces coupes sont assez fortes,
le polyèdre des solutions admissibles du modèle devient alors
exactement l'enveloppe convexe des solutions admissibles
entières. Dans ce cas, la résolution de la relaxation du programme
linéaire  réduit à cet ensemble donne une solution optimale
entière.



La section suivante est dédiée aux modèles de programmation linéaire
mis en place pour le problème d'ordonnancement de projet sous
contraintes de ressource. 

%!TeX root =../main_file.tex 
\section[Application au RCPSP]{Application à l'ordonnancement de projet sous contraintes de
  ressources}   
\label{sec:PLNE_ordo_res}

La facilité de modélisation des problèmes d'ordonnancement sous
contraintes de ressources à l'aide de la programmation linéaire en a
fait une des principales méthodes de résolution pour ces derniers. De
ce fait, de nombreuses techniques de modélisation ont été développées
et la littérature sur ce sujet est très vaste. Dans ce paragraphe,
nous nous intéressons principalement aux modèles développés pour le
problème d'ordonnancement de projet sous contraintes de ressources, le
\RCPSP. Ces modèles sont regroupés en plusieurs familles:
\begin{itemize}
\item les formulations indexées par le temps (ou à temps discret)
sont, en général, des formulations purement discrètes. Ces
formulations ont la particularité d'avoir les meilleures relaxations
linéaires au détriment du nombre de
variables~\cite{CAVT,ex_RCPSP_discret};
\item les formulations avec variables de séquencement sont des
formulations qui ont l'avantage d'être plus compactes que celles
indexées par le temps mais possèdent de moins bonnes relaxations
linéaires que ces dernières~\cite{AVT,AMR};
\item les formulations à événements, plus récentes, possèdent aussi de
faibles relaxations linéaires mais ont un nombre polynomial de
variables, moins élevé que  les formulations à temps
discret~\cite{modele_RCPSP}.
\end{itemize}

Dans ce paragraphe, nous présentons quelques modèles de
programmation linéaire en nombres entiers ou mixte développés pour les
problèmes d'ordonnancement sous contraintes de ressources. Nous nous
intéressons particulièrement aux modèles décrits
dans~\cite{modele_RCPSP} pour le \RCPSP. En effet, ce sont ces modèles
qui seront adaptés dans le cas du \CECSP. Nous commençons donc, dans
un premier temps, par présenter les modèles indexés par le temps puis,
dans un second temps, nous présentons les modèles à événements. Les
modèles avec variables de séquencement ne sont pas traités ici. Des
exemples de telles formulations dans le cadre du \RCPSP~sont décrites
dans~\cite{ADN}.

\subsection{Modèles indexés par le temps}
\label{sec:time_RCPSP}

Une des méthodes permettant la modélisation des problèmes
d'ordonnancement de projet consiste à discrétiser l'horizon de
temps. Ces modèles de programmation linéaire en nombres entiers
indexés par le temps, aussi appelés modèles à temps discret, ont été
largement étudiés dans le cadre des problèmes d'ordonnancement en
général. Ceci étant principalement dû à leur relativement forte
relaxation linéaire et à la facilité d’extension à de nombreux
objectifs et contraintes.

Dans ces formulations, l'horizon de temps est discrétisé en périodes de
temps, i.e en intervalles, généralement unitaires, et un
ensemble de variables est défini pour modéliser le statut d'une
activité $i$ en période $t$, en cours, commencé ou terminé. Pour
chaque activité $i$ et pour chaque période $t$ (discret), une variable
$x_{it} \in \{0,1\}$ est donc définie.

Dans la formulation présentée pour le \RCPSP, cette variable prendra
la valeur $1$ si et seulement si l'activité $i$ commence au début de
la période $t$. Le nombre de ces variables dépend donc de la taille de
l'horizon de temps du problème, i.e. du nombre de périodes, qui peut
être, pour certains problèmes, aussi grand que la somme de toutes les
durées des activités. Le calcul d'une borne supérieure raisonnable
pour la durée du projet $T$ est donc indispensable.  Dans la suite,
nous noterons ${\cal H}=\{0,\dots,T-1\}$ l'ensemble des périodes du
problème.

En pratique, pour chaque activité, nous calculons l'ensemble des
périodes de temps pendant lesquels l'activité peut commencer. Pour
cela, nous ajoutons deux activités fictives au problème: $0$ et
$n+1$. Ces activités ont les caractéristiques suivantes: leur
durée est égale à 0, elles ne consomment aucune ressource durant
leur exécution et l'activité $0$ (resp. $n+1$) doit être
ordonnancée avant (resp. après) toutes les autres activités. Ces
deux activités fictives vont nous permettre d'associer à chaque
activité $i$, une date de début au plus tôt $\ES$ et une date de
début au plus tard $\LS$. Donc, l'intervalle $[\ES,\LS]$ est la
fenêtre de temps pendant laquelle l'activité $i$ peut commencer.

Pour calculer ces fenêtres de temps, remarquons que, si chaque
arc $(i,j)$ du graphe des précédences $G$ est pondéré par
$p_i$, la date de début au plus tôt de $i$, $\ES$ peut prendre la
valeur du plus long chemin entre l'activité $0$ et l'activité $i$
et la date de début au plus tard $\LS$, $T$ moins la valeur du
plus long chemin entre $i$ et $n+1$.

Dans la suite, nous notons ${\cal A}$ l'ensemble des
activités réelles du projet et par $V={\cal A} \cup \{0,n+1\}$
l'ensemble des activités réelles et fictives du projet.

Nous pouvons maintenant exhiber la formulation à temps discret
suivante~\cite{ex_RCPSP_discret}:
{\small \begin{align} &\text{min }
\sum_{t \in {\cal H}} tx_{n+1,t} \label{obj_RCPSP_discret}\\
&\sum_{t \in \H} tx_{jt} - \sum_{t \in {\cal H}} tx_{it} \ge p_i &
&\forall (i,j) \in E \label{prec_RCPSP_discret}\\ &\sum_{i \in
V}\,\sum_{\tau=\max(0,t-p_i+1)}^t r_{ik}x_{i\tau} \le R_k& &\forall t\in
{\cal H},\ \forall k \in {\cal R} \label{res_RCPSP_discret}\\
&\sum_{t\in {\cal H}} x_{it}=1& & \forall i \in V
\label{preem_RCPSP_discret}\\ & x_{it} \in \{0,1\} & &\forall i
\in V,\ \forall t \in {\cal H} \end{align}
 } 
Dans cette formulation, l'objectif est donné par
l'expression~\eqref{obj_RCPSP_discret}. Cette expression signifie que
l'on cherche à minimiser la date de début de l'activité $n+1$. Or
comme cette activité est forcément placée à la fin de
l'ordonnancement du fait des relations de précédences, ceci
revient bien à minimiser la durée totale du projet.

La contrainte~\eqref{prec_RCPSP_discret} modélise les relations de
précédences entre les activités. En effet, toute paire d'activités
$(i,j)$ ayant la propriété que $i$ doit précéder $j$ vérifie la
relation suivante: la date de début de $j$ est supérieure ou égale
à la date de début de $i$ plus sa durée, i.e. la date de fin de
$i$.

La contrainte~\eqref{res_RCPSP_discret} formalise les contraintes de
ressource. En effet, la contrainte s'assure que, pour chaque ressource
$k \in {\cal R}$, la somme des consommations instantanées sur $k$ des
activités en cours durant la période $t$ est bien inférieure ou égale à la
capacité $R_k$ de cette ressource.

La contrainte~\eqref{preem_RCPSP_discret} impose que chaque activité
n'ait qu'une et une seule date de début. Ceci peut aussi être vu comme
une contrainte de non-préemption puisque ceci revient à empêcher que
l'activité soit interrompue et redémarrée plus tard dans
l'ordonnancement final.

Le modèle possède donc $(n+2)T$ variables binaires et
$|E|+T*m+n$ contraintes.

Il existe d'autres formulations à temps discret comme la formulation
désagrégée de Christofides et al.~\cite{CAVT} ou la formulation basée
sur les ensembles réalisables de Mingozzi et al.~\cite{MMRB}. Ces
formulations ne seront pas détaillées dans ce manuscrit. Une
description des modèles de PLNE existants pour le \RCPSP~peut être
trouvée dans~\cite{ADN}.

\subsection{Modèles à événements}
\label{sec:event_RCPSP}

Nous allons présenter deux formulations à événements: une formulation
dite start/end et une formulation dite on/off.  Ce paragraphe montre,
dans un premier temps, l'intérêt de l'utilisation des modèles à
événements à la place des modèles à temps discret, en comparant
notamment leur nombre respectif de variables et de contraintes.
Dans un second temps, les modèles seront détaillés.

\subsubsection{Motivations des modèles à événements}
\label{sec:motiv_event_RCPSP}

Parmi les formulations existantes pour le \RCPSP, celles basées sur des
modèles indexés par le temps sont les plus utilisées dû au fait de la
qualité (relative) de leur relaxation.  Cependant, comme la taille de
ces modèles, et donc leur complexité, dépend grandement de la taille
de l'horizon de temps du problème, ces modèles peuvent s’avérer moins
efficaces sur certains types d'instances~\cite{modele_RCPSP}. Pour
pallier ce problème, des formulations basées sur des {\it événements}
ont été mises en place.

La notion d'événement permet de caractériser seulement les dates
"importantes" de l'ordonnancement, i.e. les dates de début et de
fin de chaque activité. Ainsi, la considération de chaque date de
l'horizon de temps n'est plus nécessaire. Cela permet de réduire
le nombre de variables qui ne dépend alors que du nombre
d'activités à ordonnancer. De plus, pour le \RCPSP, seuls les
événements correspondant au début d'une activité ont besoin d'être
considérés. En effet, si l'on considère que les activités sont
ordonnancées au plus tôt, i.e. dès que les ressources requises sont
disponibles, alors la date de début de chaque activité correspond soit
à l'instant $0$, soit à la date de fin d'une tâche.

Les formulations à événements possèdent aussi la caractéristique
suivante: elles permettent de résoudre des instances pouvant
contenir des valeurs (de paramètre et/ou de solution) non
entières.

Dans~\cite{modele_RCPSP}, les auteurs montrent les limitations des
modèles indexés par le temps pour le \RCPSP. Les instances considérées
sont des instances pour lesquelles l'horizon de temps, i.e. $T$, est
très grand. Dans ce cas-là, le nombre de variables et de contraintes
des modèles à temps discret est nettement supérieur à ceux des modèles
à événements. 

Par exemple, la famille d'instances KSD15\_d~\cite{theseOumar} possède
480 instances de $15$ activités et avec $4$ ressources dont l'horizon
de temps peut varier entre $187$ et $999$ et possédant entre $77$ et
$144$ relations de précédence. Pour le \RCPSP, le modèle à temps
discret peut alors posséder jusqu'à $17000$ variables binaires et
 $4200$ contraintes. Les modèles à événements possèdent quant à
eux jusqu'à $500$ variables binaires, $16$ variables continues et
$4200$ contraintes pour le modèle start/end et jusqu'à $250$ variables
binaires, $16$ variables continues et $4000$ contraintes.

Considérant des instances de taille similaire pour le \CECSP~ avec
fonction de rendement affine, le modèle à temps discret peut alors
posséder jusqu'à $30000$ variables binaires, le même nombre de
variables continues et $90000$ contraintes.  Les modèles à événements
possèdent quant à eux jusqu'à $900$ variables binaires, le même nombre
de variables continues et $7200$ contraintes pour le modèle start/end
et jusqu'à $450$ variables binaires, $900$ variables continues et
$7100$ contraintes.

Dans ce cas-là, la faiblesse de la relaxation linéaire des modèles à
événements est largement compensée par le grand nombre de variable et
de contraintes des modèles indexés par le temps.

Dans ce contexte, l'amélioration des modèles à événements s'avère être
une direction de recherche nécessaire dans l'élaboration de méthodes de
résolution efficaces pour le \RCPSP. 



\subsubsection{Modèle start/end}

Dans cette formulation, la date de chaque événement est représentée
par une variable continue $t_e$, pour tout $e \in {\cal E}$, avec
${\cal E}$ l'ensemble des indices des événements. Afin d'associer
ces dates aux débuts et fins des activités, nous définissons, pour
toute activité $i \in \A$ et pour tout événement $e \in \E$, deux
variables binaires, $x_{ie}$ et $y_{ie}$, ayant les propriétés
suivantes: 
\begin{itemize}
 \item $x_{ie}=1$ si et seulement si
l'activité $i$ commence à l'événement $e$, c'est-à-dire commence à
la date $t_e$. 
\item $y_{ie}=1$ si et seulement si l'activité $i$
finit à l'événement $e$, c'est-à-dire finit à la date $t_e$.
\end{itemize} 
Notons que les événements correspondant à la fin
d'une activité correspondent aussi au début d'une autre activité
(à l'exception de l'événement correspondant à la date de fin de la
dernière activité), le nombre d'événements à considérer est $n+1$. De
ce fait, nous avons comme ensemble d'événements ${\cal
  E}=\{1,\dots,n+1\}$. 

Nous définissons aussi une variable continue $b_{ek}$, pour chaque
événement $e \in \E$ et pour chaque ressource $k \in \R$, modélisant
la consommation totale de la ressource $k$ à l'événement $e$ (et
donc durant tout l'intervalle compris entre $t_e$ et $t_{e+1}$).
Ceci nous permet de présenter la formulation
suivante issue de~\cite{modele_RCPSP} et corrigées dans~\cite{ABKKLM}: 
{\small
\begin{align} 
& \text{min } t_{n+1}
\label{obj_RCPSP_SE}\\ 
& t_1 =0 & & \label{t0_RCPSP_SE}\\ 
&t_e \le
t_{e+1} & & \forall e \in {\cal E} \label{ordre_RCPSP_SE}\\
&\sum_{e\in {\cal E}} x_{ie} =1 & & \forall i \in \A
\label{start_RCPSP_SE}\\ 
&\sum_{e\in {\cal E}} y_{ie} =1 & &
\forall i \in \A\label{end_RCPSP_SE}\\ 
&\ES x_{ie} \le t_e \le \LS
x_{ie}+\LS[n+1]\left(1-x_{ie}\right) & & \forall i \in \A,\ \forall
e \in {\cal E} \label{twx_RCPSP_SE}\\ 
&\left(\LS+p_i\right)y_{ie}
+\left(1-y_{ie}\right)\LS[n+1] \ge t_e & & \forall i \in \A ,\
\forall e \in {\cal E} \label{twy1_RCPSP_SE}\\ 
& t_e \ge
y_{ie}\left(\ES+p_i\right) & & \forall i \in \A ,\ \forall e \in
{\cal E} \label{twy2_RCPSP_SE}\\ 
&b_{1k} - \sum_{i \in \A}
r_{ik}x_{i1}=0 & & \forall k \in {\cal R} \label{res0_RCPSP_SE}\\
& b_{ek} - b_{e-1,k} + \sum_{i\in \A}r_{ik}
\left(y_{ie}-x_{ie}\right)=0 & & \forall e \in {\cal E}
\setminus\{1\},\ \forall k \in {\cal R} \label{resCons_RCPSP_SE}\\ 
&
b_{ek} \le R_k & & \forall e \in {\cal E},\ \forall k\in {\cal R}
\label{res_RCPSP_SE}\\
 &t_f \ge t_e + (x_{ie} + y_{if} -1) p_i & &
\forall i \in \A,\ \forall (e,f) \in {\cal E}^2,\ f>e
\label{dur_RCPSP_SE}\\ 
&\sum_{f=1}^e y_{if} +\sum_{f=e}^n x_{if}
\le 1 & & \forall i \in \A,\ \forall e \in {\cal E}
\label{xby_RCPSP_SE}\\
 &\sum_{f=e}^n y_{if} - \sum_{f=1}^{e-1}
x_{jf} \le 1 & & \forall (i,j) \in E,\ \forall e \in {\cal E}
\label{prec_RCPSP_SE}\\ 
& \ES[n+1] \le t_n \le \LS[n+1]&
&\label{Btn_RCPSP_SE}\\ 
&t_e \ge 0 & & \forall e \in {\cal E}
\label{Bte_RCPSP_SE}\\ 
& b_{ek} \ge 0 & & \forall k \in {\cal R},\
\forall e \in {\cal E} \label{Bbek_RCPSP_SE}\\ 
&x_{ie} \in
\{0,1\},\ y_{ie} \in \{0,1\} & & \forall i \in \A,\ \forall e \in
{\cal E} \label{Bxy_RCPSP_SE} \end{align}
 }
Dans cette formulation, l'objectif est donné par
l'équation~\eqref{obj_RCPSP_SE}. En effet, comme pour la
formulation à temps discret, on cherche à minimiser la date de fin
du projet. Or, l'événement $n+1$ représente, par définition, le
dernier événement, c'est à dire le début de l'activité fictive $n+1$
et donc la fin de l'ordonnancment.

La contrainte~\eqref{t0_RCPSP_SE} fixe le premier événement à la date
0, tandis que la contrainte~\eqref{ordre_RCPSP_SE} ordonne les dates
des événements par ordre croissant.

La contrainte~\eqref{start_RCPSP_SE} (resp.~\eqref{end_RCPSP_SE})
stipule que chaque activité ne peut commencer (resp. finir) qu'une
et une seule fois. En effet, chaque début (resp. fin) d'activité
ne peut être associé qu'à un et un seul événement.
 
La contrainte~\eqref{twx_RCPSP_SE} garantit que la date d'un
événement correspondant au début d'une activité soit bien comprise
dans sa fenêtre de temps, i.e. entre sa date de début au plus tôt
$\ES$ et sa date de début au plus tard $\LS$. En effet, si
l'événement $e$ correspond au début de l'activité $i$, i.e.
$x_{ie}=1$, alors l'inégalité devient: $\ES \le t_e \le \LS$. Au
contraire, si $x_{ie}=0$, i.e. $e$ ne correspond pas au début de
$i$, alors l'inégalité devient: $0 \le t_e \le \LS[n+1]$ et, grâce
aux contraintes~\eqref{ordre_RCPSP_SE} et~\eqref{Btn_RCPSP_SE},
ceci est vrai pour tout $e \in {\cal E}$. De même, les
contraintes~\eqref{twy1_RCPSP_SE} et~\eqref{twy2_RCPSP_SE}
s'assurent que, si un événement $f$ correspond à la fin d'une
activité $i$, alors $t_f$ est bien compris entre $\ES+p_i$ et
$\LS+p_i$. 

Les contraintes~\eqref{res0_RCPSP_SE}
et~\eqref{resCons_RCPSP_SE} représentent les contraintes de
conservation des ressources. La contrainte~\eqref{res0_RCPSP_SE}
modélise la consommation totale de chaque ressource $k$ à
l'événement $1$. Pour une ressource $k$, cette quantité est égale
à la somme des consommations sur $k$ de chaque activité commençant
à l'événement $1$. La contrainte \eqref{resCons_RCPSP_SE} donne
la consommation totale de chaque ressource $k$ à l'événement $e$,
$b_{ek}$. En effet, pour chaque événement $e$, cette quantité est
égale à la consommation totale à l'événement $e-1$, i.e.
$b_{e-1,k}$, à laquelle on retranche les consommations des
activités finissant à l'événement $e$ et on ajoute les
consommations de chaque activité commençant à l'événement $e$.
Enfin, la contrainte~\eqref{res_RCPSP_SE} limite la demande totale
sur chaque ressource à sa capacité.

La contrainte~\eqref{dur_RCPSP_SE} assure que si l'activité $i$
commence à l'événement $e$ et se termine à l'événement $f$, alors
les dates correspondant à ces deux événements sont séparées par
au moins la durée de cette tâche, i.e. $p_i$. Dans ce cas,
l'inégalité s'écrit $t_f \ge t_e+p_i$, et pour toute autre
combinaison de $x_{ie}$ et $y_{if}$, on obtient soit $t_f \ge
t_e$ ou $t_f\ge t_e- p_i$ ce qui est vérifié par la
contrainte~\eqref{ordre_RCPSP_SE}.

La contrainte~\eqref{xby_RCPSP_SE} garantit qu'une activité ne peut se
terminer avant d'avoir commencé. Si l'activité $i$ finit entre
l'événement $1$ et $e$, i.e. $\sum_{f=1}^{e} y_{if}=1$, alors elle ne
peut pas commencer entre l'événement $e$ et $n$, i.e. $\sum_{f=e}^{n}
x_{if}=0$, et inversement.

Enfin, la contrainte~\eqref{prec_RCPSP_SE} modélise les relations
de précédences entre les activités: si une activité $i$, devant
précéder une activité $j$, finit à l'événement $e$ ou après, alors
l'activité $j$ ne peut commencer avant l'événement $e$, i.e.
$\sum_{f=e}^n x_{if}=1 \Rightarrow \sum_{f=1}^{e-1} y_{jf}=0$.

Le modèle possède $2n^2+n$ variables binaires, $n+1$ variables
continues et $(n+1)(m+n^2/2+|E|)+3n$ contraintes. Le nombre de
variables et de contraintes du modèle est donc bien polynomial en
fonction du nombre d'activités et de ressources.
 
\subsubsection{Modèle on/off} 

Dans le modèle on/off, comme pour la
formulation start/end, la variable $t_e$ représente la date
de chaque événement, pour tout $e \in {\cal E}$. Pour associer ces
dates aux débuts et fins de chaque activité, nous définissons, pour
chaque activité $i \in \A$ et pour chaque événement $e \in \E$, la
variable $z_{ie} \in \{0,1\}$. Cette variable prendra la valeur
$1$ si et seulement si l'activité $i$ est en cours durant
l'intervalle $[t_e,t_{e+1}]$, et 0 sinon. Ainsi, une activité
commencera (resp. finira) à l'événement $e$ si
$z_{ie}-z_{i,e-1}=1$ (resp.$z_{i,e-1}-z_{ie}=1$).

Notons que, comme dans le cas du modèle start/end, nous pouvons
limiter le nombre d'événements au début de chaque activité. Ainsi,
${\cal E}=\{1,\dots,n\}$; l'utilisation d'activités fictives
n'est, ici, plus nécessaire pour modéliser l'événement de fin de
la dernière activité. De plus, la variable $z_{ie}$ modélisant le fait
qu'un variable soit en cours entre $t_e$ et $t_{e+1}$, nous pouvons
limiter le nombre de ces variables en ne les définissant que pour les
événements $e \in \Em[n]$. Ceci conduit à la formulation
suivante~\cite{modele_RCPSP}: {\small \begin{align} & \text{min } C_{max}
\label{obj_RCPSP_OO}\\ &C_{max} \ge t_e+ (z_{ie}-z_{i,e-1})p_i & &
\forall e \in {\cal E}\setminus\{1\},\ \forall i \in {\cal A}
\label{Cmax_RCPSP_OO}\\ &t_1=0 & & \label{t0_RCPSP_OO}\\ &t_e \le
t_{e+1} & &\forall e \in {\cal E}\setminus\{n\}
\label{ordre_RCPSP_OO}\\ &\sum_{e \in \Em[n]} z_{ie} \ge 1 & &
\forall i \in {\cal A} \label{start_RCPSP_OO}\\ & \ES z_{ie}\le
t_e & & \forall e \in \Em[n],\ \forall i \in {\cal A}
\label{ES_RCPSP_OO} \\ & t_e \le
\LS(z_{ie}-z_{i,e-1})+(1-(z_{ie}-z_{i,e-1}))\LS[n+1] & & \forall e
\in \Em[n]\setminus\{1\},\ \forall i \in {\cal A}
\label{LS_RCPSP_OO}\\ &t_f\ge t_e
+\left((z_{ie}-z_{i,e-1})-(z_{if}-z_{i,f-1})-1\right)p_i & & \forall
e>f \in (\Em[1,n])^2,\ \forall i \in {\cal
A}\label{dur_RCPSP_OO}\\ &\sum_{f=1}^{e} z_{if} \le
e(1-(z_{ie}-z_{i,e-1})) & & \forall e \in \Em[1,n],\
\forall i \in {\cal A}\label{preem1_RCPSP_OO}\\ &\sum_{f=e}^{n-1}
z_{if} \le (n-e)(1+(z_{ie}-z_{i,e-1})) & & \forall e \in \Em[1,n],\ 
\forall i \in {\cal A}\label{preem2_RCPSP_OO}\\
&\sum_{i \in {\cal A}} r_{ik}z_{ie} \le R_k & &\forall e \in \Em[n]
,\ \forall k \in {\cal R} \label{res_RCPSP_OO}\\
&z_{ie}+\sum_{f=1}^e z_{jf} \le 1 + (1-z_{ie})e & & \forall e \in
\Em[1,n],\ \forall (i,j) \in E \label{prec_RCPSP_OO}
\end{align}
}
L'objectif, donné par l'équation~\eqref{obj_RCPSP_OO}, 
consiste à minimiser la date de fin du projet, ici
représentée par la variable $C_{max}$. La
contrainte~\eqref{Cmax_RCPSP_OO} s'assure que $C_{max}$ prend bien
la valeur de la date de fin du projet.

Les contraintes~\eqref{t0_RCPSP_OO} et~\eqref{ordre_RCPSP_OO} jouent
le même rôle que dans la formulation start/end, à savoir, ordonner
les événements.

La contrainte~\eqref{start_RCPSP_OO} stipule que chaque activité
doit être ordonnancée sur au moins un événement dans toute la durée du projet.
 
Les contraintes~\eqref{ES_RCPSP_OO} et~\eqref{LS_RCPSP_OO}
garantissent que la date d'un événement correspondant au début
d'une activité soit bien comprise dans sa fenêtre de temps, i.e.
entre sa date de début au plus tôt $\ES$ et sa date de début au
plus tard $\LS$. En effet, si l'événement $e$ correspond au début
de l'activité $i$, i.e. $z_{ie}=1$ et $z_{i,e-1}=0$ , l'inégalité
devient alors: $\ES \le t_e \le \LS$. Nous distinguons 3 autres
cas: \begin{itemize} \item {\it $z_{ie}=z_{i,e-1}=1$ :} l'activité
est en cours entre les événements $t_{e-1}$ ($z_{i,e-1}=1$) et
$t_{e+1}$ ($z_{ie}=1$). L'inégalité devient: $\ES\le t_e \le
\LS[n+1]$. Comme l'activité $i$ a déjà commencé à un événement $f
<e$, on a: $\ES \le t_f \le t_e$. L'autre inégalité est triviale.
\item {\it$z_{ie}=0$ et $z_{i,e-1}=1$:} l'activité se termine à
l'événement $e$. L'inégalité donne alors $0 \le t_e \le
2*\LS[n+1]-\LS$. Or, $2*\LS[n+1] -\LS \ge \LS[n+1]$. Donc
l'inégalité est vérifiée. \item {\it $z_{ie}=z_{i,e-1}=0$:}
l'activité n'est pas en cours entre les événements $t_{e-1}$ et
$t_{e+1}$. L'inégalité devient $0 \le t_e \le \LS[n+1]$ et est
trivialement vérifiée. \end{itemize}

La contrainte~\eqref{dur_RCPSP_OO} assure une séparation
suffisante, i.e. la durée de l'activité, entre un événement
correspondant au début d'une activité et un événement
correspondant à la fin de cette même activité. La validité de
cette contrainte suit la même logique que pour la
contrainte~\eqref{dur_RCPSP_SE} du modèle précédent.

Les contraintes~\eqref{preem1_RCPSP_OO}
et~\eqref{preem2_RCPSP_OO}, appelées {\it contraintes de
contiguïté}, assure la non-préemption des activités. La validité
de cette contrainte n'est pas détaillée ici mais une preuve
formelle peut être trouvée dans~\cite{modele_RCPSP}.

La contrainte~\eqref{res_RCPSP_OO} représente les limites de
capacité de chaque ressource. En effet, une activité $i$ consomme
une quantité $r_{ik}$ de la ressource $k$ entre $t_e$ et $t_{e+1}$
si et seulement si elle est en cours entre ces deux dates, i.e.
$z_{ie}=1$. On pourra remarquer que dans ce modèle le nombre de
contraintes nécessaires pour modéliser les contraintes de capacité
des ressources est nettement inférieur à celui du modèle précédent.

Enfin, la contrainte~\eqref{prec_RCPSP_OO} modélise les relations
de précédences entre les activités: si une activité $i$, devant
précéder une activité $j$, est en cours entre les événements $e$
et $e+1$, i.e. $z_{ie}=1$, alors l'activité $j$ ne peut être en
cours avant l'événement $e$, i.e. $z_{ie}=1 \Rightarrow
\sum_{f=0}^e z_{jf}=0 $.

Le modèle possède $n^2$ variables binaires, deux fois moins que
pour le modèle start/end, $n+1$ variables continues et
$(n-1)(3+|E|+m+n^2/2)+n^2+n$ contraintes. Le nombre de variables
et de contraintes du modèle est donc bien polynomial en fonction
du nombre d'activités et de ressources.


La partie suivante est consacrée à la présentation des modèles de
programmation linéaire mixtes pour le \CECSP. 
\chapter[Modèles pour le \CECSP~et renforcement des modèles]{Programmation linéaire pour le \CECSP~et renforcement des modèles}
\label{sec:PLNE_CECSP}




%!TeX root =../main_file.tex
\section{Modèles de programmation linéaire mixte pour le
\CECSP}
\label{sec:modele_CECSP}
Le \CECSP ~et le \RCPSP~étant deux problèmes relativement proches --
utilisation d'une ou plusieurs ressources cumulatives, de fenêtres de
temps -- la modélisation du \CECSP~à l'aide de la programmation
linéaire mixte est une approche naturelle pour
sa résolution. 

De plus, le théorème~\ref{theo_LPM_CECSP} nous assurant que toute solution
$S$ du \CECSP~peut être transformée en une solution $S'$ ayant la
propriété que $\forall i \in \A,\ b_i(t)$ soit constante par morceaux,
le problème peut être modélisé à l'aide de la programmation linéaire
mixte. Nous pouvons aussi remarquer que, comme la transformation
proposée par le théorème n'augmente ni la date de début des activités,
ni leur date de fin, ni leur consommation totale de ressource, tout modèle
ayant un objectif impliquant seulement la minimisation de ces trois
quantités sera valide pour le \CECSP.

Dans cette section, nous commençons par décrire un modèle indexé par
le temps, puis, nous présentons deux modèles à événements. Ces trois
modèles sont adaptés des modèles pour le \RCPSP~présentés dans la
section~\ref{sec:PLNE_ordo_res} et sont présentés
dans~\cite{Nattaf_ORSpectrum}.   


\subsection{Modèle indexé par le temps}

La première formulation proposée est une formulation indexée par le
temps. Ce modèle a été conçu dans le cadre d'une collaboration avec
David Rivreau~\cite{Nattaf_ORSpectrum}. Comme pour le modèle à temps discret du \RCPSP~l'horizon de
temps est ici divisé en périodes de taille unitaire. Le calcul d'une
borne supérieure $T$ sur la durée totale du projet est trivial: il
suffit de prendre la plus grande date échue, i.e. $T=\max_{i \in \A}
\LE$. L'ensemble des périodes, noté $\H$, peut donc être défini par:
$\H=\{0,\dots,T-1\}$.  Notons que, par translation, nous pouvons
toujours supposer que $\min_{i\in \A} \ES=0$.

Une des principales différences entre le modèle à temps discret du
\RCPSP~et celui du \CECSP~ repose sur le fait que, dans le
second, la durée des activités n'est pas connue à l'avance et doit
être déterminée par le modèle. De ce fait, nous devons définir,
pour chaque activité $i \in \A$ et pour chaque instant $t
\in \H$, deux variables binaires $x_{it}$ et $y_{it}$ pour
modéliser le début et la fin des activités. La variable $x_{it}$
(resp. $y_{it}$) prendra la valeur $1$ si et seulement si
l'activité $i$ commence (finit) à l'instant $t$.

Une seconde différence entre les deux modèles est le calcul des 
fenêtres de temps, effectué de manière triviale pour le \CECSP. 
En effet, pour une activité $i$,
la fenêtre correspondante à sa date de début est $[\ES,\LS]$, avec
$\LS=\LE-W_i/f_i(\bmax)$, et celle correspondante à sa date de fin
$[\EE,\LE]$, avec $\EE=\ES+W_i/f_i(\bmax)$. Enfin, l'ajout des
activités fictives marquant le début et la fin du projet n'est 
pas nécessaire ici.

\paragraph{Fonction de rendement identité}

Dans le cas où la fonction de rendement de chaque activité est la
fonction identité, i.e. $f_i(b_i(t))=b_i(t),\ \forall i \in \A$, nous
définissons une variable $b_{it}$, pour chaque activité $i \in \A$ et
pour chaque période de temps $t \in \H$, qui représentera la quantité
de ressource consommée par l'activité $i$ dans la période de temps
$t$, i.e. dans l'intervalle $[t,t+1]$.

Ceci conduit à la formulation suivante:
{\small
 \begin{align} &\text{min }
\sum_{i\in \A}\sum_{t \in \H} b_{it}
\label{obj_CECSP_TI}\\ &\sum_{t=\ES}^{\LS} x_{it} = 1 & &\forall i
\in \A \label{start_CECSP_TI}\\ &\sum_{t=\EE}^{\LE} y_{it} =
1 & & \forall i \in \A
\label{end_CECSP_TI}\\&\left(\sum_{\tau=\ES}^{t} x_{i\tau}
-\sum_{\tau=\ES+1}^{t} y_{i\tau}\right)\bmin \le b_{it} & &
\forall t \in \{\ES,\dots,\LE-1\},\ \forall i \in \A
\label{bmin_CECSP_TI}\\ &\left(\sum_{\tau=\ES}^{t} x_{i\tau} -
\sum_{\tau=\ES+1}^{t} y_{i\tau}\right)\bmax\ge b_{it}& & \forall t
\in \H ,\ \forall i \in \A \label{bmax_CECSP_TI}\\
&\sum_{t=\ES}^{\LE} b_{it} \ge W_i & & \forall i \in \A
\label{nrj_CECSP_TI}\\ &\sum_{i \in \A} b_{it} \le R & &
\forall t \in \H \label{res_CECSP_TI}\\ &b_{it} = 0 & &
\forall t \not\in \{\ES,\dots,\LE-1\},\ \forall i \in \A
\label{consNul_CECSP_TI}\\ &x_{it} = 0 & & \forall t \not\in
\{\ES,\dots,\LS\},\ \forall i \in \A \label{twx_CECSP_TI}\\
&y_{it} = 0 & & \forall t \not\in \{\EE,\dots,\LE\},\ \forall i
\in \A \label{twy_CECSP_TI}\\ & b_{it} \ge 0 & & \forall t
\in \H; \forall i \in \A \label{Bb_CECSP_TI}\\
&x_{it}\in \{0,1\},\ y_{it} \in \{0,1\} & & \forall t \in {\cal
H},\ \forall i \in \A \label{Bxy_CECSP_TI} \end{align}
} 
Ce modèle s'inspire grandement du modèle de~\cite{ALR}, la principale
différence résidant dans la considération de fenêtres de temps plus
fines pour les variables, i.e. $[\ES,\LS]$ pour $x_{it}$ au lieu de
$[\ES,\LE-1]$ et $[\EE,\LE]$ pour $y_{it}$ au lieu de
$[\ES-1,\LE]$. Ces fenêtres de temps pouvant être raffinées à l'aide
d'heuristiques, de la programmation par contraintes (cf.
sous-section~\ref{sec:adjustment_tw}) ou par le biais d'autres méthodes,
ces modifications peuvent grandement améliorer les performances du
modèle.

Dans cette formulation, l'objectif est décrit par
l'équation~\eqref{obj_CECSP_TI}. Ici, l'objectif est de minimiser la
consommation totale de la ressource durant tout le projet. Cet
objectif a moins d'impact dans le cas où la fonction de rendement de
chaque activité est la fonction identité mais se révèle très pertinent
pour d'autres types de fonctions de rendement.  Cependant, comme la
formulation à temps discret ne nous permet pas de s'assurer que la
quantité de ressource consommée par une activité soit exactement égale
à la quantité nécessaire pour apporter l'énergie requise,
i.e. $\sum_{t=\ES}^{\LE} b_{it} \ge W_i$, cet objectif reste valide
dans le cas des fonctions rendement identités.

Si l'on souhaite modifier l'objectif pour minimiser la date de fin
du projet, il suffit d'introduire une variable $C_{max}$ ainsi que
la contrainte suivante: 
\begin{equation} 
C_{max} \ge \sum_{i \in \A} ty_{it} \quad \forall t \in \H \notag
\end{equation} 
et alors l'objectif s'écrit facilement comme: 
\begin{equation}
\text{min } C_{max} \notag
\end{equation} 
Les contraintes~\eqref{start_CECSP_TI} et~\eqref{end_CECSP_TI}
stipulent que chaque activité est exécutée une et une seule fois
durant la durée du projet. En effet, grâce à ces contraintes, chaque
activité n'a qu'une et une seule date de début et une et une seule
date de fin.

Les contraintes~\eqref{bmin_CECSP_TI} et~\eqref{bmax_CECSP_TI}
permettent de s'assurer que la consommation instantanée de chaque
activité est bien comprise entre $\bmin$ et $\bmax$ durant toute sa
durée d'exécution. Pour s'en assurer, il suffit de remarquer que
$\sum_{\tau=\ES}^{t} x_{i\tau} -\sum_{\tau=\ES+1}^{t}y_{i\tau}=1$ si
et seulement si l'activité $i$ est en cours à l'instant $t$. Les
autres configurations possibles sont $\sum_{\tau=\ES}^{t} x_{i\tau}
-\sum_{\tau=\ES+1}^{t} y_{i\tau}=$0 et $\sum_{\tau=\ES}^{t} x_{i\tau}
-\sum_{\tau=\ES+1}^{t} y_{i\tau}=-1$. Dans le premier cas, ceci
signifie que l'activité n'a pas encore débuté et les inégalités
imposent que $b_{it} \le 0$ et donc $b_{it}=0$. Dans le second cas,
l'équation devient $-\bmax \ge b_{it}$, ce qui est impossible. Notons
que ce dernier cas nous assure ainsi qu'une activité ne peut finir avant
d'avoir commencé. Cependant, afin de renforcer la relaxation linéaire
du modèle, des contraintes spécifiques, empêchant que le début d'une
activité ne se produise avant sa fin, peuvent être ajoutées. Ces
inégalités sont de la forme: 
\begin{equation}
 \sum_{\tau=\ES}^{t} x_{i\tau}
-\sum_{\tau=\ES+1}^{t} y_{i\tau} \le 0 \quad \forall t \in \{\ES,\dots,\LE-1\}
\end{equation}

La contrainte~\eqref{nrj_CECSP_TI} modélise le fait qu'une activité
doit recevoir au moins la quantité d'énergie requise par la donnée du
problème.

La contrainte~\eqref{res_CECSP_TI} limite la quantité de ressource
utilisée simultanément à la capacité de cette dernière.

Enfin, les contraintes~\eqref{consNul_CECSP_TI},~\eqref{twx_CECSP_TI}
et~\eqref{twy_CECSP_TI} fixent la valeur des variables à zéro en
dehors de leurs fenêtres de temps, i.e. $[\ES,\LE]$ pour $b_{it}$,
$[\ES,\LS]$ pour $x_{it}$ et $[\EE,\LE]$ pour $y_{it}$.

Cette formulation possède $2nT$ variables binaires, $nT$ variables
continues et au plus $3n+T*(5n+1)$ contraintes.

\paragraph{Fonction de rendement affine}

Dans le cas où les fonctions de rendement sont toutes la fonction
identité, l'énergie apportée à une activité durant une certaine
période est égale à la quantité de ressource consommée par celle-ci
durant la même période de temps. Or, lorsque les fonctions de
rendement deviennent affines, ce n'est plus le cas.  Pour s'assurer
qu'une activité $i$ reçoive bien l'énergie requise, nous devons
déclarer une nouvelle variable qui permettra de mesurer l'énergie
apportée à cette dernière durant la période $t$, $w_{it},\ \forall
(i,t) \in \A \times \H$. Dans le cas
précédent, nous avions $w_{it}=b_{it}$.

Nous devons donc vérifier, pour chacune des contraintes du modèle
précédent impliquant la variable $b_{it}$, si cette dernière
modélisait une quantité de ressource ou d'énergie, i.e. si nous devons
la remplacer par $w_{it}$. De plus, une contrainte liant les variables
$b_{it}$ et $w_{it}$ devra être rajoutée pour assurer la bonne
conversion entre les deux quantités.

La contrainte~\eqref{bmin_CECSP_TI} (resp.~\eqref{bmax_CECSP_TI})
représentant la quantité minimale (resp. maximale) de ressource qu'une
activité doit consommer à chaque instant de son exécution, reste donc
inchangée. De même, la contrainte~\eqref{res_CECSP_TI}, modélisant la
contrainte sur la capacité de la ressource, n'a pas besoin d’être
modifiée.

La contrainte~\eqref{nrj_CECSP_TI}, qui s'assure qu'une activité
reçoive bien la quantité d'énergie requise doit être réécrite en
utilisant la variable $w_{it}$. Ce qui nous donne l'inégalité
suivante:
\begin{equation} 
\sum_{t=\ES}^{\LE} w_{it} \ge W_i\quad \forall i \in \A \tag{\ref{nrj_CECSP_TI}'} \label{nrj2_CECSP_TI}
\end{equation} 
De plus, nous devons ajouter une contrainte permettant de lier la
quantité de ressource consommée par une activité et la quantité
d'énergie apportée à celle-ci. Nous ajoutons donc la contrainte
suivante au modèle:
\begin{equation}
w_{it}=a_ib_{it}+c_i\left(\sum_{\tau=\ES}^t
x_{i\tau}-\sum_{\tau=\ES+1}^t y_{i\tau}\right) \quad \forall t\in
\H,\ \forall i \in \A \label{conv_CECSP_TI}
\end{equation} 

Cette contrainte nous permet de modéliser la fonction de rendement
$f_i,\ \forall i \in \A$. En effet, $\left(\sum_{\tau=\ES}^t
x_{i\tau}\right.$ $\left.-\sum_{\tau=\ES+1}^t y_{i\tau}\right) $ est égale à $1$ si et
seulement si l'activité $i$ est en cours à l'instant $t$.  Dans ce
cas-là, la valeur de l'énergie apportée à $i$ est bien
$w_{it}=a_ib_{it}+c_i$. Le second cas se produit quand l'activité $i$
n'est pas en cours à $t$. Dans ce cas, la
contrainte~\eqref{bmax_CECSP_TI} nous dit que $b_{it}=0$ et donc
$w_{it}=0$.

Le modèle possède donc $2nT$ variables binaires, $2nT$ variables
continues et au plus $3n+T*(6n+1)$ contraintes.

\paragraph{Fonction de rendement concave affine par morceaux}

Lorsque les fonctions de rendements sont des fonctions concaves
affines par morceaux, nous utilisons aussi la variable $w_{it}$ pour
représenter l'énergie reçue par l'activité $i$ dans la période $t$. La
contrainte~\eqref{nrj2_CECSP_TI} est utilisée pour modéliser le fait
que cette activité doive recevoir une quantité d'énergie $W_i$ durant
son exécution. 

Pour s'assurer de la bonne conversion entre la quantité de ressource
consommée par une activité $i$ dans une période $t$ et la quantité
d'énergie reçue par cette dernière dans la même période,
l'égalité~\eqref{conv_CECSP_TI} est remplacée par l'inégalité
suivante: 
\begin{equation}
w_{it} \le a_{ip}b_{it} + c_{ip}\left(\sum_{\tau=\ES}^t
x_{i\tau}-\sum_{\tau=\ES+1}^t y_{i\tau}\right) \quad  \forall i \in
\A,\ \forall p \in \P_i,\ \forall t \in \H 
\label{conv_CECSP_TI2}
\end{equation}
avec $\P_i$ l'ensemble des intervalles de définition de la fonction
$f_i$.

Notons que l'utilisation d'une inégalité peut impliquer que la
variable $w_{it}$ ne représente pas exactement la quantité d'énergie
apportée à $i$ dans la période $t$. Cependant, la contrainte nous
assure que $w_{it} \le f_i(b_{it})$. De ce fait, à une solution
optimale donnée par le modèle correspond toujours une solution
optimale du \CECSP~à dates de début et de fin entières. En effet,
supposons que dans une solution optimale renvoyée par le PLNE, il
existe $(i,t)$ tel que $w_{it} < f_i(b_{it})$, alors, deux cas sont
possibles:
\begin{itemize}
\item $\bmin < f_i^{-1}(w_{it})$ et dans ce cas-là, $b_{it}$ peut
prendre la valeur $f_i^{-1}(w_{it})$ et cette solution aura un coût,
i.e. une consommation totale de ressource, plus faible que la solution
précédente et ceci contredit l'optimalité de la solution.
\item $\bmin \ge  f_i^{-1}(w_{it})$ et dans ce cas-là, $b_i(t)$ prend la
valeur $\bmin$ et $et_i=\min \{t \in\mathbb{N}\ |\ \\
\int_{\tau=st_i}^{t} f_i(b_i(t))dt \le W_i\}$. Ici, deux cas sont
possibles: le premier correspond au cas où $et_i=\{t\ | \ y_{it}=1\}$
et la solution renvoyée par le PLNE est optimale; dans le second cas
$et_i<\{t\ |\ y_{it}=1\}$ mais alors, $y_{it-1}$ peut prendre la
valeur $1$ et $y_{it}$ la valeur $0$ et cette solution est de plus
faible coût que la solution précédente et ceci contredit l'optimalité
de la solution renvoyée par le PLNE.
\end{itemize}


Le modèle ainsi défini possède alors $2nT$ variables binaires, $2nT$
variables continues et au plus $3n + (5n +nP+1)T$ contraintes, avec
$P=\max_{i\in A}|\P_i|$.  

\subsection{Modèles à événements}

Dans cette sous-section, nous proposons deux formulations à événements pour
le \CECSP. Ces deux formulations sont grandement inspirées des
formulations start/end et on/off du \RCPSP, présentées dans la
section~\ref{sec:PLNE_ordo_res} et issues 
de~\cite{modele_RCPSP}. Comme dans le cas du \RCPSP, ces modèles se
justifient par leur nombre polynomial de variables et de contraintes,
ce qui peut s'avérer très pertinent dans la cas de grands horizons de
temps. Un argument supplémentaire vient s'ajouter pour
le \CECSP. En effet, il peut arriver qu'une instance de ce problème ne
possède que des solutions à valeurs non entières et ce, même si toutes
les données sont entières (voir exemple~\ref{exemple_NE},
page~\pageref{exemple_NE}).


Comme dans le cas de la formulation à temps discret, ici, les dates de
fin des activités ne sont plus totalement définies par leur date de
début. De ce fait, dans les formulations à événements du \CECSP, nous
avons besoin de modéliser deux types d'événements, les débuts et les
fins des activités, soit, au plus, $2n$ événements. Ces événements,
indexés par l'ensemble $\E=\{1,\dots,2n\}$, sont représentés par un
ensemble de variables continues, notées $t_e$.

Nous rappelons aussi les notations suivantes: $T=\max_{i \in \A}
\LE$ est une borne supérieure sur la date de fin du projet;
$[\ES,\LS]$, avec $\LS= \LE-W_i/f_i(\bmax)$, est la fenêtre de temps
dans laquelle l'activité $i$ peut débuter; de même, $[\EE,\LE]$,
avec $\EE=\ES + W_i/f_i(\bmax)$, la fenêtre de temps durant laquelle
l'activité $i$ peut finir.

Nous allons présenter, en premier lieu, la formulation start/end du
\CECSP~avec fonctions de rendement identité dont nous dériverons 
les cas de fonctions de rendement affines et affines par morceaux.
Ensuite, nous présenterons la formulation on/off du même problème et
de ses dérivés.

\subsubsection{Modèle start/end}

Dans le modèle start/end, deux variables binaires $x_{ie}$ et
$y_{ie}$, $\forall (i,e) \in \A\times \E$, 
servent à affecter les dates des différents événements, modélisés par
les variables $t_e$, aux débuts et fins des activités. En effet, la
variable $x_{ie}$ prend la valeur $1$ si et seulement si l'événement
$e$ correspond au début de l'activité $i$, i.e. l'activité $i$
commence à la date $t_e$, et est égale à $0$ sinon. De même, la
variable $y_{ie}$ est égale à $1$ si et seulement si l'événement $e$
correspond à la fin de l'activité $i$, et vaut $0$ sinon.

\paragraph{Fonction de rendement identité}

Dans le cas où toutes les fonctions de rendement sont la fonction
identité, un seul ensemble de variables est nécessaire pour modéliser
la consommation de la ressource. En effet, comme la quantité de
ressource consommée par une activité $i$ est égale à la quantité
d'énergie apportée à cette même activité, il n'est pas utile de
définir une variable représentant cette quantité d'énergie.

Un ensemble de variables supplémentaires, $b_{ie}$, $\forall (i,e) \in
\A\times \Em$, est donc défini. La variable $b_{ie}$ représente la
quantité de ressource consommée par une activité $i$ entre les dates
$t_e$ et $t_{e+1}$.  Ceci nous permet de modéliser le problème de la
façon suivante:
{\small
\begin{align}
& \text{min } \sum_{i\in A}\ \sum_{e\in \E\setminus\{2n\}} b_{ie} 
\label{obj_CECSP_SE}\\
&t_e \le t_{e+1} & & \forall e \in
\E\setminus\{2n\} \label{ordre_CECSP_SE}\\
 &\sum_{e\in \E} x_{ie} =1 & & \forall i \in
A \label{start_CECSP_SE}\\
 &\sum_{e\in \E} y_{ie} =1 & & \forall i \in A \label{end_CECSP_SE}\\
 &x_{ie}\ES \le t_e & & \forall i \in A,\ \forall e \in
\E \label{twx1_CECSP_SE}\\
 &t_e \le x_{ie}\LS+ \left(1-x_{ie}\right)T & & \forall i \in A,\
\forall e \in \E \label{twx2_CECSP_SE}\\
 &t_e \ge y_{ie}\EE & & \forall i \in A ,\ \forall e \in \E 
 \label{twy1_CECSP_SE}\\
 &\LE y_{ie} + \left(1-y_{ie}\right)T \ge t_e & & \forall i \in A,\ \forall e \in \E 
 \label{twy2_CECSP_SE}\\
 &\sum_{i \in A} b_{ie} \le R \left(t_{e+1}- t_e\right) & & 
 \forall e \in \E\setminus\{2n\} \label{res_CECSP_SE}\\
 &t_f \ge t_e +  \left(x_{ie} + y_{if} -1\right) W_i/f_i(\bmax) & & \forall i \in A,\ 
 \forall e,f \in \E\ ; f\ge e \label{dur_CECSP_SE}\\
 &\sum_{e\in \E\setminus\{2n\}} b_{ie} = W_i  & & \forall i \in A 
 \label{nrj_CECSP_SE}\\
&b_{ie} \ge \bmin \left(t_{e+1}-t_e\right) 
- M \left(1 - \sum_{f=0}^e x_{if} +\sum_{f=0}^e y_{if}\right) 
& &  \forall i \in A,\ \forall e \in \E\setminus\{2n\} \label{bmin_CECSP_SE}\\
&b_{ie} \le \bmax  \left(t_{e+1} - t_e\right) & &
\forall i \in A,\ \forall e \in \E\setminus\{2n\} \label{bmax_CECSP_SE}\\
& \left(\sum_{f=0}^{e} x_{if} - \sum_{f=0}^e y_{if}\right)M\ge b_{ie} & &
 \forall i \in A,\ \forall e \in \E\setminus\{2n\} \label{res0_CECSP_SE}\\
&t_e \ge 0 & & \forall e \in \E \label{eq36}\\
& b_{ie} \ge 0 & & \forall i \in A,\ \forall e \in \E\setminus\{2n\} 
\label{Bb_CECSP_SE}\\
&x_{ie} \in \{0,1\},\ y_{ie} \in \{0,1\} & & 
\forall i \in A,\ \forall e \in \E \label{eq39}
\end{align}
}

L'objectif, décrit par l'équation~\eqref{obj_CECSP_SE}, est de
minimiser la consommation totale de la ressource, i.e. la somme des
consommations de toutes les tâches. On peut facilement modifier cet
objectif afin de minimiser la date de fin du projet en remplaçant
l'objectif par:
\begin{equation}
\text{min } t_{|\E|} \notag
\end{equation}
La contrainte~\eqref{ordre_CECSP_SE} ordonne les événements. La
contrainte~\eqref{start_CECSP_SE} (resp.~\eqref{end_CECSP_SE})
s'assure que chaque activité ne commence (resp. finisse) qu'une et une
seule fois. En effet, chaque début (resp. fin) d'activité ne peut être
associé qu'à un et un seul événement.
 
Les contraintes~\eqref{twx1_CECSP_SE} et~\eqref{twx2_CECSP_SE}
garantissent que la date d'un événement correspondant au début d'une
activité soit bien comprise dans sa fenêtre de temps, i.e. dans
l'intervalle $[\ES,\LS]$. En effet, si l'événement $e$ correspond au
début de l'activité $i$, i.e. $x_{ie}=1$, alors la première inégalité
devient $\ES \le t_e$ et la seconde $t_e \le \LS$. Les autres
configurations donnent: $0 \le t_e \le T$ et ceci est trivialement
vérifié. De même, les contraintes~\eqref{twy1_CECSP_SE}
et~\eqref{twy2_CECSP_SE} s'assurent que, si un événement $e$
correspond à la fin d'une activité $i$, alors $t_e$ est bien compris
entre $\EE$ et $\LE$.

La contrainte~\eqref{res_CECSP_SE} s'assure que, dans la période de
temps comprise entre $t_e$ et $t_{e+1}$, la quantité de ressource
consommée n’excède pas la capacité de cette dernière.

La contrainte~\eqref{dur_CECSP_SE} modélise le fait que si l'activité
$i$ commence à l'événement $e$ et se termine à l'événement $f$, alors
les dates correspondant à ces deux événements sont au moins séparées
par la durée de cette tâche. Or, comme nous ne connaissons pas cette
durée, nous utilisons une borne inférieure: $W_i/f_i(\bmax)$. 

La contrainte~\eqref{nrj_CECSP_SE} modélise le fait qu'une activité
doit recevoir au moins la quantité d'énergie requise par la donnée du
problème.

Les contraintes~\eqref{bmin_CECSP_SE} et~\eqref{bmax_CECSP_SE}
permettent de s'assurer que la consommation instantanée de chaque
activité est bien comprise entre $\bmin$ et $\bmax$ durant toute sa
durée d'exécution. Pour s'en assurer, il suffit de remarquer que, dans
la première inégalité, $\sum_{f=0}^{e} x_{if}-\sum_{f=0}^{e}y_{if}=1$
si et seulement si l'activité $i$ est en cours entre les événements
$e$ et $e+1$.  L'inégalité devient donc $b_{ie} \ge
\bmin(t_{e+1}-t_e)$. Les autres configurations possibles donnent
$b_{ie} \ge \bmin(t_{e+1}-t_e)-M$, avec $M$ une constante
suffisamment grande pour dépasser $\bmin(t_{e+1}-t_e)$.

La contrainte~\eqref{res0_CECSP_SE} garantit que si l'activité $i$
n'est pas en cours entre les événements $e$ et $e+1$, alors
$b_{ie}=0$. En effet, si l'activité $i$ n'est pas en cours,
i.e. $\sum_{f=0}^{e} x_{if}-\sum_{f=0}^{e}y_{if}=0$, la contrainte
s'écrit $0\ge b_{ie}$ et donc on a bien $b_{ie} =0$. Dans le cas où
l'activité $i$ est en cours, i.e. $\sum_{f=0}^{e} x_{if}
-\sum_{f=0}^{e}y_{if}=1$, la contrainte s'écrit $M\ge b_{ie}$ et, si
$M$ est une constante suffisamment grande pour servir de borne
supérieure à $b_{ie}$, alors la contrainte est valide. Notons aussi
que ces contraintes empêchent qu'une activité ne puisse commencer
avant d'avoir fini. De plus, pour renforcer la formulation, nous
pouvons ajouter les contraintes~\eqref{xby_RCPSP_SE} modélisant
explicitement le fait qu'un événement ``début d'activité'' se produit
forcément avant l'événement ``fin d'activité'' lui
correspondant. 

Cette formulation possède $4n^2$ variables binaires, $2n^2+n$
variables continues et $2n^3+13n^2+4n-2$ contraintes.

\paragraph{Fonction de rendement affine}

Pour adapter cette formulation au cas des fonctions de rendement
affines, nous avons, dans un premier temps, besoin de définir une
variable $w_{ie},\ \forall i \in \A$ et $\forall e \in \E
\setminus\{2n\}$. Cette variable sert à modéliser la quantité
d'énergie apportée à une activité $i$ durant l'intervalle de temps
$[t_e,t_{e+1}]$. De plus, nous devons identifier les contraintes
impliquant la variable $b_{ie}$ pour lesquelles cette variable doit
être remplacée par la variable $w_{ie}$. En d'autres termes, nous
devons identifier les contraintes traitant de la ressource et les
contraintes liées à l'énergie.

La contrainte~\eqref{res_CECSP_SE}, modélisant la contrainte de
capacité de la ressource, n'est pas modifiée. De même, les
contraintes~\eqref{bmin_CECSP_SE} et~\eqref{bmax_CECSP_SE},
garantissant la contrainte de consommation maximale et minimale de la
ressource, restent inchangées. Enfin, la
contrainte~\eqref{res0_CECSP_SE}, s'assurant de la non consommation de
ressource des activités en dehors de leur exécution, ne nécessite
aucun changement.

La contrainte~\eqref{nrj_CECSP_SE} cependant, étant en charge du
respect de l'apport requis en énergie de chaque activité, doit être
réécrite en utilisant la variable adéquate, $w_{ie}$. Ceci donne
l'inégalité suivante:
\begin{equation} 
\sum_{e\in \E\setminus\{2n\}} w_{ie} = W_i\quad \forall i \in \A
\tag{\ref{nrj_CECSP_SE}'}
\label{nrj2_CECSP_SE}
\end{equation} 

Il nous reste à écrire les contraintes permettant de lier les deux
variables, i.e.  de s'assurer de la bonne conversion
ressource/énergie. Nous ajoutons donc les contraintes suivantes au
modèle:
\begin{align}
  &w_{ie} \le a_ib_{ie}+c_i(t_{e+1}-t_e) & & \forall i \in A,\ \forall
  e \in \E \setminus\{2n\} \label{conv1_CECSP_SE}\\
     &w_{ie} \le W_i (\sum_{f=0}^{e} x_{if} -\sum_{f=0}^{e} y_{if}) &
    & \forall i \in A,\ \forall e \in {\cal
      E}\setminus\{2n\} \label{conv2_CECSP_SE}\\
& w_{ie} \ge 0 & & \forall i \in A,\ \forall e \in {\cal
      E}\setminus\{2n\}
\label{Bw_CECSP_SE}
\end{align}

La contrainte~\eqref{conv1_CECSP_SE} permet de modéliser l'énergie
apportée à l'activité $i$ en fonction de la quantité de ressource
consommée par cette même activité dans l'intervalle
$[t_e,t_{e+1}]$. Notons d'abord l'utilisation d'une inégalité et non
d'une égalité. Ceci est dû au fait que, lorsque l'activité n'est pas
en cours entre $t_e$ et $t_{e+1}$, i.e. $b_{ie}=0$, la contrainte
devient $w_{ie} \le c_i(t_{e+1}-t_e)$. Dans le cas d'une égalité, ceci
aurait été faux puisqu'on devrait avoir $w_{ie}=0$. Le rôle de la
seconde contrainte est donc de s'assurer que lorsque l'activité n'est
pas en cours, on ait bien $w_{ie}=0$. En effet, lorsque l'activité
n'est pas en cours, la contrainte devient $w_{ie}\le 0$ et dans le cas
contraire, l'inégalité s'écrit $w_{ie} \le W_i$.

La seconde remarque qui peut être faite est que, puisque la
contrainte~\eqref{conv1_CECSP_SE} utilise une inégalité, $w_{ie}$ peut
ne pas être réellement égale à la quantité d'énergie apportée à $i$
entre $t_e$ et $t_{e+1}$. En fait, grâce à
l'objectif~\eqref{obj_CECSP_SE}, ceci ne peut arriver.

\begin{theo}
  \label{th:conv}
  Soit $S$ une solution optimale du modèle décrit
  par~\eqref{obj_CECSP_SE}--\eqref{dur_CECSP_SE},
  \eqref{nrj2_CECSP_SE} et
  \eqref{bmin_CECSP_SE}--\eqref{Bw_CECSP_SE}. Alors $S$ vérifie la
  condition suivante: \[\forall i \in \A,\ \forall e \in \E\setminus\{2n\},\
  w_{ie}=f_i\left(\frac{b_{ie}}{t_{e+1}-t_e}\right)(t_{e+1}-t_e)\]
\end{theo}

\begin{proof}
Pour prouver le théorème, commençons par remarquer que la
contrainte~\eqref{conv1_CECSP_SE} implique
$w_{ie} \le f_i\left(\frac{b_{ie}}{t_{e+1}-t_e}\right)(t_{e+1}-t_e)$. En effet, 
\begin{align*}
  f_i\left(\frac{b_{ie}}{t_{e+1}-t_e}\right)(t_{e+1}-t_e)&
  =\left(a_i\frac{b_{ie}}{t_{e+1}-t_e}+c_i\right)(t_{e+1}-t_e)\\  
   &=a_ib_{ie}+c_i(t_{e+1}-t_e)
\end{align*}
Il nous reste donc à montrer que $w_{ie}\ge
f_i\left(\frac{b_{ie}}{t_{e+1}-t_e}\right)(t_{e+1}-t_e)$. Par l'absurde,
supposons que $\exists i \in \A,\ \exists e \in \E\setminus\{2n\}$
tel que $w_{ie} < 
f_i\left(\frac{b_{ie}}{t_{e+1}-t_e}\right)(t_{e+1}-t_e)$. Soit $\S_C$
l'ensemble des solutions du \CECSP~et soit  $c :\S_C
\rightarrow \mathbb{R}$ la fonction qui associe à une solution $S$ sa
consommation de ressource $c(S)$. Nous allons créer une nouvelle
solution $S'$ qui vérifie que $c(S)>c(S')$, ce qui invalidera
l'optimalité de $S$.

Pour cela, nous considérons la solution du \CECSP~associée à
$S$. Notons $\widehat{S}$ cette solution. $\widehat{S}$ est obtenue à
partir de $S$ de la manière suivante, $\forall (i,e)
\in \A \times \E$: 
\begin{itemize}
\item si $x_{ie}=1$ alors $st_i$ est fixé à $t_e$,
\item si $y_{ie}=1$ alors $et_i$ est fixé à $t_e$,
\item $b_i(t)=\frac{b_{ie}}{t_{e+1}-t_e}, \forall t \in [t_e,t_{e+1}]$. 
\end{itemize}

$S'$ est définie de la façon suivante: 
\begin{itemize}
\item $st_i'=st_i$,
\item $et_i'=\min(t \in \H | \int_{st_i}^t f_i(b_i(t)) = W_i)$,
\item $b_i'(t)=\left\{ \begin{array}{ll}
                         b_i(t)& \text{si } st_i' \le t \le et_i'\\ 
                         0 & \text{sinon}
                       \end{array}
                     \right.$
\end{itemize}
Clairement, $S'$ est une solution pour le \CECSP. Nous montrons
maintenant que $c(S) = c(\widehat{S})> c(S')$. Pour cela, nous allons montrer qu'il
existe une tâche $i$ telle que $et_i > et_i'$ et donc
$\int_{st_i}^{et_i} b_i(t)dt > \int_{st_i'}^{et_i'} b_i(t)dt$. De ce
fait, nous aurons bien que $\sum_{i \in \A} \int_{st_i}^{et_i}
b_i(t)dt > \sum_{i \in \A}\int_{st_i'}^{et_i'} b_i(t)dt$. 

On sait que $\exists (i,e) \in \A \times \E$ tel que:
\begin{align*}
  &w_{ie} < f_i\left(\frac{b_{ie}}{t_{e+1}-t_e}\right)(t_{e+1}-t_e)\\  
  \Rightarrow &\sum_{e \in  \E} w_{ie} < \sum_{e \in \E} f_i\left(
                \frac{b_{ie}}{t_{e+1}-t_e}\right)(t_{e+1}-t_e)\\ 
  \Rightarrow & W_i=\sum_{e \in  \E} w_{ie} < \sum_{e \in \E}
                f_i(b_i(t))(t_{e+1}-t_e)=\int_{st_i}^{et_i}
                f_i(b_i(t))dt \\
  \Rightarrow & \min\left(t \in \H |
                \int_{st_i}^{t}f_i(b_i(t))dt\right) = et_i' < et_i                 
\end{align*}

Donc nous avons bien $et_i > et_i'$ et ceci achève la démonstration.
\end{proof}

Notons que pour d'autres objectifs cela peut ne pas être valide. Mais, la
solution obtenue par le modèle peut toujours être transformée en une
solution du \CECSP~en temps polynomial. Pour cela, il suffit de suivre
la transformation proposée dans la preuve du théorème~\ref{th:conv}.

Le modèle ainsi défini possède $4n^2$ variables binaires, $4n^2$
variables continues et $2n^3+17n^2+2n-2$ contraintes.

\paragraph{Fonction de rendement affine par morceaux}

Lorsque les fonctions de rendement sont des fonctions concaves
affines par morceaux, nous utilisons aussi la variable $w_{ie}$ pour
représenter l'énergie reçue par l'activité $i$ dans la période
$[t_e,t_{e+1}]$. La contrainte~\eqref{nrj2_CECSP_SE} est utilisée pour
modéliser le fait que cette activité doive recevoir une quantité
d'énergie $W_i$ durant son exécution.

Pour s'assurer de la bonne conversion entre la quantité de ressource
consommée par une activité $i$ dans une période $[t_e,t_{e+1}]$ et la
quantité d'énergie reçue par cette dernière dans la même période,
l'inégalité~\eqref{conv1_CECSP_SE} est remplacée par l'inégalité
suivante:
\begin{equation}
w_{ie} \le a_{ip}b_{ie} + c_{ip}\left( t_{e+1} - t_e\right) \quad  
\forall i \in \A,\ \forall p \in \P_i,\ \forall e \in \E\setminus\{2n\}  
\label{conv2_CECSP_SE}
\end{equation}

Notons que la même argumentation que celle de la preuve du
théorème~\ref{th:conv} permet de nous assurer que la conversion faite
par le modèle entre la quantité d'énergie reçue par l'activité et la
quantité de ressource consommée par cette dernière est valide.

Le modèle ainsi défini possède $4n^2$ variables binaires, $4n^2$
variables continues et $2n^3+ (15 + 2P)n^2+\left(1-P\right)n-2$
contraintes, avec $P=\max_{i\in A}|\P_i|$.

\subsubsection{Modèle on/off}
\label{sssection:OO_CECSP}
Dans le modèle on/off, une variable binaire $z_{ie}$ se charge
d'assigner à chaque événement les dates de début et de fin des
activités. La variable $z_{ie}$ prendra la valeur $1$ si et seulement
si l'activité $i$ est en cours d'exécution dans la période
$[t_e,t_{e+1}]$ et sera égale à $0$ sinon. Ceci permet de diviser par
deux le nombre de variables binaires utilisées par rapport à la
formulation start/end.

\paragraph{Fonction de rendement identité}

Dans le cas où les fonctions de rendement sont la fonction identité,
nous définissons la variable $b_{ie}$ qui modélise la quantité de
ressource consommée par l'activité $i$ entre les dates $t_e$ et
$t_{e+1}$. Comme pour le modèle start/end, nous avons seulement besoin
de définir la variable $b_{ie},\ \forall i \in \A$ et $\forall e \in
\Em$. Ceci conduit à la formulation suivante: 
{\small
\begin{align}
& \text{min } \sum_{i\in A}\ \sum_{e\in \E\setminus
    \{2n\}}b_{ie}
 \label{obj_CECSP_OO}\\ 
&t_e \le t_{e+1} & &\forall e \in \E\setminus
 \{2n\} \label{ordre_CECSP_OO}\\
&\sum_{e \in \E} z_{ie} \ge 1 & & \forall i \in {\cal
   A}\label{start_CECSP_OO}\\
& \ES z_{ie}\le t_e \le \LS(z_{ie}-z_{i,e-1})+(1-(z_{ie}-z_{i,e-1}))T
 & & \forall e \in \E\setminus \{1\},\ \forall i \in {\cal
   A}\label{twx_CECSP_OO}\\
&\EE(z_{i,e-1}-z_{ie})\le t_e & & \forall e \in \E\setminus
 \{1\},\ \forall i \in \A\label{twy1_CECSP_OO}\\
&t_e \le \LE(z_{i,e-1}-z_{ie})+(1-(z_{i,e-1}-z_{ie}))T & & \forall e
 \in \E\setminus \{1\},\ \forall i \in {\cal
   A}\label{twy2_CECSP_OO}\\
&t_f\ge t_e +((z_{ie}-z_{i,e-1})-(z_{if}-z_{i,f-1})-1)W_i/f_i(\bmax) &
 & \forall e,f \in \E,\ f>e\neq 1,\ \forall i \in {\cal
   A} \label{dur_CECSP_OO}\\
&\sum_{e'=1}^{e} z_{ie'} \le e(1-(z_{ie}-z_{i,e-1})) & & \forall e \in
       \E\setminus \{1\},\ \forall i \in \A
\label{preem1_CECSP_OO}\\
&\sum_{e'=e}^{2n} z_{ie'} \le (2n-e)(1+(z_{ie}-z_{i,e-1})) & & \forall
e \in \E\setminus \{1\},\ \forall i \in \A
\label{preem2_CECSP_OO}\\
&\sum_{i\in \A} b_{ie} \le R(t_{e+1}-t_e) & & \forall e \in
      \E\setminus \{2n\}\label{res_CECSP_OO}\\
&\sum_{e \in \E\setminus \{2n\}} b_{ie} = W_i & & \forall i \in
        \A\label{nrj_CECSP_OO}\\
&b_{ie} \ge \bmin(t_{e+1}-t_e) - M(1-z_{ie}) & & \forall e \in {\cal
  E}\setminus \{2n\}, \ \forall i \in \A\label{bmin_CECSP_OO}\\
&b_{ie} \le \bmax(t_{e+1}-t_e) & &\forall e \in \E\setminus
\{2n\},\ \forall i \in \A\label{bmax_CECSP_OO}\\
&z_{ie}M\ge b_{ie} & & \forall e\in \E\setminus \{2n\},
\ \forall i \in \A\label{res0_CECSP_OO}\\
&t_e \ge 0 & & \forall e \in \E \label{eq54}\\
& b_{ie} \ge 0 & & \forall i \in A,\ \forall e \in \E\setminus
\{2n\}
 \label{Bb_CECSP_OO}\\
&z_{ie} \in \{0,1\} & & \forall i \in A,\ \forall e \in {\cal
   E} \label{eq57}
\end{align}
}
Dans cette formulation, l'objectif est décrit
par~\eqref{obj_CECSP_OO}. Ici, l'objectif est de minimiser la
consommation totale de la ressource durant tout le projet. Si 
l'on souhaite minimiser la date de fin du projet, l'objectif s'écrit 
facilement comme: 
\begin{equation}
\text{min } t_{|\E|} \notag
\end{equation} 

La contrainte~\eqref{ordre_CECSP_OO} joue le même rôle que dans la
formulation start/end, à savoir, ordonner les événements.

La contrainte~\eqref{start_CECSP_OO} stipule que chaque activité doit
être ordonnancée une fois dans toute la durée du projet.
 
La contrainte~\eqref{twx_CECSP_OO} s'assure que la date d'un événement
correspondant au début d'une activité soit bien comprise dans sa
fenêtre de temps.  En effet, si l'événement $e$ correspond au début de
l'activité $i$, i.e. $z_{ie}=1$ et $z_{i,e-1}=0$ , l'inégalité devient
alors: $\ES \le t_e \le \LS$. Notons que pour le cas $e=1$,
$z_{i,e-1}-z_{ie}$ est remplacé par $z_{ie}$. De même, les
contraintes~\eqref{twy1_CECSP_OO} et~\eqref{twy2_CECSP_OO}
garantissent qu'un événement correspondant à la fin d'une activité
soit bien compris entre $\EE$ et $\LE$. Ici, il n'est pas
nécessaire de considérer le cas $e=1$ car cet événement ne peut pas
correspondre à la fin d'une activité. 

La contrainte~\eqref{dur_CECSP_OO} assure une séparation suffisante
entre un événement correspondant au début d'une activité et un
événement correspondant à la fin de cette même activité. Ici, comme
nous ne connaissons pas la durée d'une activité, nous utilisons une
borne inférieure sur cette dernière. Pour le cas où $e=1$, il suffit
de remplacer $z_{i,e-1}-z_{ie}$ par $z_{ie}$.

Les contraintes~\eqref{preem1_CECSP_OO} et~\eqref{preem2_CECSP_OO},
appelées {\it contraintes de contiguïté}, assure la non-préemption des
activités. Une preuve formelle de la validité de ces contraintes est
décrite dans~\cite{modele_RCPSP}. Ici aussi, le cas $e=1$, est 
traité en remplaçant $z_{i,e-1}-z_{ie}$ par $z_{ie}$.

La contrainte~\eqref{res_CECSP_OO} représente la capacité de
la ressource tandis que la contrainte~\eqref{nrj_CECSP_OO} s'assure
que chaque activité reçoit bien la quantité d'énergie requise par la
donnée du problème.

Les contraintes~\eqref{bmin_CECSP_OO},~\eqref{bmax_CECSP_OO} et
\eqref{res0_CECSP_OO} -- permettant respectivement de modéliser les
contraintes de consommation minimale, de consommation maximale et de
consommation nulle en dehors de l'intervalle d'exécution de $i$ -- se
déduisent directement des
contraintes~\eqref{bmin_CECSP_SE},~\eqref{bmax_CECSP_SE} et
\eqref{res0_CECSP_SE}. Dans un premier temps, remarquons que nous
avons la relation suivante:
\begin{equation}
\label{SEversOO}
z_{ie} = \sum_{f=1}^e x_{if} -  \sum_{f=1}^e y_{if} \quad \forall i
\in \A,\ \forall e\in \Em
\end{equation}
En effet, $z_{ie}$ vaut $1$ si et seulement si l'activité $i$ est en
cours entre $t_e$ et $t_{e+1}$, c'est-à-dire si l'activité a commencé à
l'événement $e$ ou avant, i.e. $\sum_{f=1}^ex_{if}=1$, et si elle
ne finit pas à l'événement $e$ ou avant, i.e. $\sum_{f=1}^ey_{if}=
0$. Dans ce cas, nous avons bien $\sum_{f=1}^ex_{if} -
\sum_{f=1}^ey_{if}= 1 = z_{ie}$. Les
contraintes~\eqref{bmin_CECSP_OO},~\eqref{bmax_CECSP_OO} et
\eqref{res0_CECSP_OO} s'obtiennent donc facilement à partir
de~\eqref{bmin_CECSP_SE},~\eqref{bmax_CECSP_SE} et
\eqref{res0_CECSP_SE} en remplaçant $\sum_{f=1}^{e}
x_{if}-\sum_{f=1}^{e}y_{if}$ par $z_{ie}$.


Le modèle ainsi défini possède $2n^2-n$ variables binaires, $2n^2+n$
variables continues et $2n^3+13n^2-n-2$ contraintes.    

\paragraph{Fonction de rendement affine}

Pour modéliser la conversion de la ressource en énergie, nous
définissons une variable $w_{ie}$, $\forall (i,e) \in \A \times \Em$,
représentant la quantité d'énergie 
apportée à l'activité $i$ dans l'intervalle $[t_e,t_{e+1}]$. Puis,
nous identifions les contraintes du précédent modèle qui ont besoin
d'être réécrites à l'aide de cette variable. Ici, seule la
contrainte~\eqref{nrj_CECSP_OO} est concernée. La réécriture donne la
contrainte suivante:
\begin{equation}
\sum_{e\in \E} w_{ie}=W_i \quad \forall i \in \A
\tag{\ref{nrj_CECSP_OO}'}
\end{equation}

Dans un second temps, nous écrivons les contraintes liant les variables
$b_{ie}$ et $w_{ie}$:
\begin{align}
  &w_{ie} \le a_ib_{ie}+c_i(t_{e+1}-t_e) & & \forall i \in A,\ \forall
  e \in \E \setminus\{2n\} \label{conv1_CECSP_OO}\\
  &w_{ie} \le W_i z_{ie} & & \forall i \in A,\ \forall e \in
                             \E\setminus\{2n\} \label{conv2_CECSP_OO}\\ 
  & w_{ie} \ge 0 & & \forall i \in A,\ \forall e \in \E\setminus\{2n\}
  \label{Bw_CECSP_OO}
\end{align}

Notons que la même argumentation que celle de la preuve du
théorème~\ref{th:conv} permet de nous assurer de la bonne conversion
entre énergie et ressource du modèle. 

Le modèle ainsi défini possède $2n^2-n$ variables binaires, $4n^2$
variables continues et $2n^3+17n^2-3n-2$ contraintes.

\paragraph{Fonction de rendement affine par morceaux}

Lorsque les fonctions de rendement sont des fonctions concaves
affines par morceaux, nous utilisons aussi la variable $w_{ie}$ pour
représenter l'énergie reçue par l'activité $i$ dans la période
$[t_e,t_{e+1}]$. La contrainte~\eqref{nrj2_CECSP_SE} est utilisée pour
modéliser le fait que cette activité doive recevoir une quantité
d'énergie $W_i$ durant son exécution.

Pour s'assurer de la bonne conversion entre la quantité de ressource
consommée par une activité $i$ dans une période $[t_e,t_{e+1}]$ et la
quantité d'énergie reçue par cette dernière dans la même période,
l'inégalité~\eqref{conv1_CECSP_SE} est remplacée par l'inégalité
suivante:
\begin{equation}
w_{ie} \le a_{ip}b_{ie} + c_{ip}\left( t_{e+1} - t_e\right) \quad  
\forall i \in \A,\ \forall p \in \P_i,\ \forall e \in \Em
\label{conv2_CECSP_SE}
\end{equation}

Le modèle ainsi défini possède $2n^2-n$ variables binaires, $4n^2$
variables continues et $2n^3+(15 + 2P)n^2+\left(2-P\right)n-2$
contraintes, avec $P=\max_{i\in A}|\P_i|$.    


Dans la partie suivante, nous allons montrer comment améliorer ces
modèles par le biais d'une étude polyédrale et la proposition
d'inégalités valides.

\section{Renforcement des modèles}
\label{sec:amelioration_modele}
Dans ce paragraphe, nous présentons des améliorations mises en place
pour les modèles indexés par le temps et les modèles à événements du
\RCPSP~et du \CECSP. 

Dans un premier temps, nous nous sommes intéressé aux modèles à temps
discret pour lesquels nous proposons des inégalités valides déduites
directement du raisonnement énergétique présenté aux
paragraphes~\ref{sec:cumu_propag} et~\ref{sec:ER_CECSP}. En effet, ces modèles restent les modèles 
les plus efficaces pour résoudre le \CECSP~ou le \RCPSP, ceci étant
principalement dû à la qualité relative de leur relaxation. 

Cependant, les limitations de ces modèles nous ont poussé à nous
intéresser plus particulièrement à l'amélioration des modèles à
événements. En effet, le nombre de variables et de contraintes de ces
modèles dépend de la taille de l'horizon de temps du problème. Ce
nombre peut donc s'avérer très important si l'horizon de temps est grand.
De plus, dans les modèles indexés par le temps du \CECSP~, la
discrétisation de l'horizon de temps conduit à une réduction de
l'espace des solutions et peut donc mener à des solutions
sous-optimales (voir exemple~\ref{exemple_NE},
page~\pageref{exemple_NE}). 

Pour améliorer les relaxations linéaires des modèles à événements, nous
avons étudié le polyèdre défini par l'ensemble de toutes les
affectations possibles des variables binaires pour une seule activité,
pour lequel nous exhibons un ensemble minimal d'inégalités permettant
de le décrire. De plus, plusieurs ensembles d'inégalités valides sont
proposées pour contribuer à l'amélioration des performances des
modèles à événements.
 
\subsection{Amélioration des modèles indexés par le temps}
\label{sec:ER_TI}
Dans ce paragraphe, nous montrons comment est utilisé le
raisonnement énergétique décrit au paragraphe~\ref{sec:ER_CECSP} pour
exhiber des inégalités valides pour les modèles de programmation
linéaire mixte indexés par le temps du \RCPSP~et du \CECSP. Nous
commençons par détailler le cas du \CECSP~avant de montrer comment
appliquer le même raisonnement pour le \RCPSP.

Pour intégrer le raisonnement énergétique dans les modèles indexés par
le temps, nous utilisons l'équation qui se trouve au centre de ce
raisonnement (cf. équation~\eqref{eq:centerRE}). Soit $\I$ l'ensemble des
intervalles d'intérêt du raisonnement énergétique, alors l'équation
s'écrit:  
\begin{align} 
  SL(t_1,t_2) &\ge 0 \quad \forall [t_1,t_2] \in \I \nonumber\\
  \Rightarrow \bb + \sum_{\substack{j\in \A\\j\neq i}}\bb[j] &\le
  R(t_2-t_1)\quad \forall [t_1,t_2] \in \I \nonumber
\end{align}  

Soit $CR: \mathbb{R} \times \H \times \H \rightarrow \mathbb{R}$ qui
associe à une quantité d'énergie $w$ et à un intervalle $[t_1,t_2]$,
la quantité de ressource minimale à consommer dans $[t_1,t_2]$ pour
apporter à une activité une énergie $w$ (voir
équation~\eqref{eq:conv_CECSP}). En considérant toutes les
expressions possibles pour $\bb$ et en utilisant les variables
$x_{it}$ et $y_{it}$ pour activer ou non l'équation correspondant à la
consommation minimale de ressource dans $[t_1,t_2]$, nous obtenons les
inégalités suivantes: 
\begin{align}
  &\left(1-\sum_{t\le t_1}x_{it} - \sum_{t\ge t_2}y_{it}\right) \, \conv[W_i] +
    \sum_{j\neq i} \bb[j] \le R(t_2-t_1) \nonumber\\
  & \hspace{9.55cm} \forall i \in \A,\ \forall [t_1,t_2] \in \I
    \label{total}
\end{align}

L'inégalité~\eqref{total} représente le cas où l'intervalle
$[\ES,\LE]$ est complètement inclus dans $[t_1,t_2]$. En effet,
les différents cas possibles sont: 
\begin{itemize}
\item l'activité commence après $t_1$ et finit avant $t_2$. Dans ce
  cas-là, l'inégalité devient: 

$\conv[W_i] +\sum_{j\neq i} \bb[j] \le
  R(t_2-t_1)$. Ceci correspond au cas où la contrainte est activée. 
\item l'activité commence avant $t_1$ et finit avant $t_2$. Ici,
  l'inégalité devient: $\sum_{j\neq i} \bb[j] \le R(t_2-t_1)$ et ceci
  est vérifié quel que soit $i$ et $[t_1,t_2]$. Ce cas correspond au
  cas où la contrainte n'est pas activée. 
\item l'activité commence après $t_1$ et finit après $t_2$. Ce cas est
  similaire au précédent. 
\item l'activité commence avant $t_1$ et finit après
  $t_2$. L'inégalité s'écrit alors: $\sum_{j\neq i} \bb[j]-\conv[W_i]
  \le R(t_2-t_1)$. Ce cas est aussi similaire au second. 
\end{itemize}

\begin{align}
&\left(x_{i,\ES} + y_{i,\LE} -1 \right) \, \conv[W_i-f_i(\bmax)\left(t_1-\ES + \LE
-t_2\right)] \nonumber \\
&+ \sum_{j\neq i} \bb[j] \le R\left(t_2-t_1\right) \hspace{4.35cm} \forall i \in
{\cal A},\ \forall [t_1,t_2] \in \I  \label{both}
\end{align}

L'inégalité~\eqref{both} correspond au cas où l'activité est centrée,
i.e. $\ES\le t_1<t_2 \le \LE$ et $W_i-\left(t_1-\ES + \LE -t_2\right)
\ge f_i(\bmin)(t_2-t_1)$. Dans ce cas, l'inégalité sera activée si et
seulement si l'activité commence à $\ES$ ($x_{i,\ES}=1$) et finit à
$\LE$ ($y_{i,\LE}=1$). 

\begin{align}
&\left(x_{i,\ES} + \sum_{t=t_1}^{t_2}y_{it} -1\right) \,
\conv[W_i-f_i\left(\bmax\right)\left(t_1-\ES\right)]+ \nonumber\\
&\sum_{j\neq i} \bb[j] \le R\left(t_2-t_1\right) \hspace{4.75cm} \forall i \in
  {\cal A},\ \forall [t_1,t_2] \in \I
\label{left}
\end{align}

L'inégalité~\eqref{left} correspond au cas où l'activité est calée à
gauche, i.e. exécutée à $\bmax$ pendant l'intervalle
$[\ES,t_1]$. Cette inégalité est activée quand l'activité commence à
$\ES$ et finit dans l'intervalle $[t_1,t_2]$. 

\begin{align}
&\left(\sum_{t=t_1}^{t_2}x_{it} + y_{i,\LE}-1\right) \,
\conv[W_i-f_i\left(\bmax\right)\left(t_2-\LE\right)]\nonumber\\
& + \sum_{j\neq i} \bb[j] \le R\left(t_2-t_1\right) \hspace{4.29cm} \forall i \in
{\cal A},\ \forall [t_1,t_2] \in \I
\label{right}
\end{align} 

L'inégalité~\eqref{right} correspond au cas où l'activité est calée à
droite, i.e. exécutée à $\bmax$ pendant l'intervalle
$[t_2,\LE]$. Cette inégalité est activée quand l'activité commence
dans l'intervalle $[t_1,t_2]$ et finit a $\LE$.

\begin{align}
&\left(\sum_{t\le t_1}x_{it} + \sum_{t\ge t_2}y_{it} -1 \right) \,
  \bmin\left(t_2-t_1\right) + \sum_{j\neq i} \bb[j] \nonumber\\
& \le R\left(t_2-t_1\right)\hspace{7cm} \forall i \in {\cal A},\ \forall [t_1,t_2]
\in \I
\label{min}
\end{align}

Enfin, l'inégalité~\eqref{min} correspond au cas où l'activité est
exécutée à $\bmin$ durant l'intervalle $[t_1,t_2]$. Cette inégalité
assure que si l'activité commence à $t_1$ ou avant et finit à $t_2$ ou
après alors la quantité de ressource disponible dans $[t_1,t_2]$ est
suffisante pour exécuter l'activité à $\bmin$ durant tout cet
intervalle.


Notons que ces inégalités peuvent aussi être définies dans le cas du
\RCPSP. Dans ce cas là, nous devons appliquer le même raisonnement
pour chaque ressource. Cependant, il y aura seulement trois cas à
considérer: l'activité est calée à gauche, l'activité est calée à
droite et l'activité est en cours d'exécution durant l'intervalle
$[t_1,t_2]$. Ceci donne les inégalités suivantes:
\begin{multline} \left(x_{i,\ES} + \sum_{t=t_1}^{t_2}y_{it} -1\right)
\, r_{ik}p_i^+(t_1)+ \sum_{j\neq i} \bb[j] \le R\left(t_2-t_1\right)
\qquad \\ \forall i \in {\cal A},\ \forall k \in \R ,\ \forall
[t_1,t_2] \in \I
    \label{RCPSP_left}
\end{multline} \vspace{-1.3cm}
\begin{multline} \left(\sum_{t=t_1}^{t_2}x_{it} + y_{i,\LE}-1\right)
\, b_ip_i^-(t_2) + \sum_{j\neq i} \bb[j] \le R\left(t_2-t_1\right)
\qquad \\ \forall i \in {\cal A},\ \forall k \in \R ,\ \forall
[t_1,t_2] \in \I
    \label{RCPSP_right}
\end{multline} \vspace{-1.3cm}
\begin{multline} \left(\sum_{t\le t_1}x_{it} + \sum_{t\ge t_2}y_{it}
-1 \right) \, b_i\left(t_2-t_1\right) + \sum_{j\neq i} \bb[j] \le
R\left(t_2-t_1\right) \qquad \\ \forall i \in {\cal A},\ \forall k \in
\R ,\ \forall [t_1,t_2] \in \I
    \label{RCPSP_min}
\end{multline}
avec $p_i^+(t_1)$ et $p_i^-(t_2)$ comme définis dans le
paragraphe~\ref{sec:cumu_propag} pour le \CUSP. 

De telles inégalités ont été proposées dans le cadre du modèle
préemptif sur les ensembles admissibles~\cite{BD}. Ces inégalités sont
ajoutées au modèle indexé par le temps du \CECSP~et du \RCPSP~et les
performances de ces modèles avec ou sans ces inégalités sont évaluées
dans le chapitre~\ref{sec:expe} afin de montrer leur intérêt.
 
Les modèles à temps discret sont particulièrement efficaces pour le
\RCPSP~et le \CECSP. Cependant, ces modèles possèdent des limitations
qui sont décrites dans la section suivante. 



\subsection{Amélioration des modèles à événements}
\label{sec:amelioration_OO}
Dans ce paragraphe, nous présentons les améliorations possibles des
modèles à événements. Dans un premier temps, nous montrons que le
modèle start/end possède de meilleures relaxations que le modèle
on/off, puis nous présentons un ensemble d'inégalités que nous pouvons
rajouter au modèle on/off pour renforcer sa relaxation. Cet ensemble
d'inégalités est ensuite utilisé pour donner une description minimale
du polyèdre défini par l'ensemble de toutes les affectations possibles
des variables binaires $z_{ie}$ pour une seule activité. 

D'autres ensembles d'inégalités, permettant l'ajout de nouvelles
contraintes au modèle, sont ensuite présentées , la suppression des
coefficients big-$M$ dans les contraintes existantes ou l'amélioration
des contraintes existantes. Dans ce paragraphe, nous présentons les
résultats pour le \CECSP~car, dans la plupart des cas, le remplacement
de $\Em=\{1,\dots,2n-1\}$ par $\{1,\dots,n\}$ suffit à l'adaptation de
la preuve au \RCPSP. Cependant, lorsque ce n'est pas le cas ou que le
résultat n'a pas été prouvé pour le \RCPSP, nous le précisons.

\subsubsection{Comparaison des relaxations linéaires}

Pour montrer que le modèle start/end possède de meilleures relaxations
linéaires que le modèle on/off, nous commençons par montrer que le
modèle start/end possède des relaxations au moins aussi bonnes que
celles du modèle on/off. Pour cela, il suffit de  montrer que toute solution du 
modèle start/end, entière ou non, est solution du modèle on/off. Ceci
peut être fait en écrivant les variables $z_{ie}$ en fonction des
variables $x_{ie}$ et $y_{ie}$ et nous avons la relation suivante,
décrite dans le paragraphe~\ref{sssection:OO_CECSP}: 
\begin{equation}
\label{SEversOO}
z_{ie} = \sum_{f=1}^e x_{if} -  \sum_{f=1}^e y_{if} \quad \forall i
\in \A,\ \forall e\in \Em
\end{equation}

Il nous reste maintenant à montrer que les relaxations du modèle
on/off ne sont pas aussi bonnes que celles du modèle start/end. Pour
cela, nous relâchons les contraintes d'intégrité des deux modèles et
nous trouvons une solution pour la relaxation du modèle on/off qui
n'en est pas une pour la relaxation du modèle start/end. 

Considérons le sous-modèle formé des
équations~\eqref{start_CECSP_OO}, \eqref{preem1_CECSP_OO} et
\eqref{preem2_CECSP_OO} et le sous-modèle formé des
équations~\eqref{start_CECSP_SE}, \eqref{end_CECSP_SE},
\eqref{SEversOO} et de l'ensemble d'équations suivant: $\sum_{f=1}^e
y_{if} +\sum_{f=e}^n x_{if}\le 1,\ \forall i \in \A$. Clairement, si
nous trouvons une solution pour le premier sous-modèle qui n'en est
pas une pour le second, ceci montrera bien que les relaxations du modèle
on/off ne sont pas aussi bonnes que celles des modèles start/end. En
effet, toute solution d'un des sous-modèles est solution du modèle
entier (start/end ou on/off) en considérant l'instance suivante:
$\forall i \in \A,\ \ES=0,\ \LE=T,\ \bmin=0,\ \bmax=1,\ W_i=1,\
f_i(b_i(t))=b_i(t)$ et $B=n$. 

Nous nous plaçons dans le cas où le nombre d'activités est de 2. Alors,
nous avons $|\E|=4$.  Dans la suite, les tâches sont notées par les
indices $a$ et $b$ et les événements par $0,\ 1,\ 2$ et $3$.

Nous constatons tout d'abord que $z_a=(0,6\,;\,0\,;\,0,7\,;\,0)$ est bien une
solution du sous-modèle on/off. 

Nous devons maintenant montrer que le sous modèle start/end ne possède
pas de solution avec $z_a=(0,6\,;\,0\,;\,0,7\,;\,0)$. Pour cela, étudions les contraintes du
modèle qui n'impliquent que l'activité $a$:

\begin{align} &x_{a0}+x_{a1}+x_{a2}+x_{a3}= 1 \label{c9}\\
              &y_{a0}+y_{a1}+y_{a2}+y_{a3}= 1 \label{c10}\\
              &y_{a0}+x_{a0}+x_{a1}+x_{a2}+x_{a3}\le1 \label{c11}\\
              &y_{a0}+y_{a1}+x_{a1}+x_{a2}+x_{a3}\le1 \label{c12}\\
              &y_{a0}+y_{a1}+y_{a2}+x_{a2}+x_{a3}\le1 \label{c13}\\
              &y_{a0}+y_{a1}+y_{a2}+y_{a3}+x_{a3}\le1 \label{c14}\\
              &0.6=x_{a0}-y_{a0}\label{c17}\\
              &0=x_{a0}-y_{a0}+x_{a1}-y_{a1}\label{c18}\\
              &0.7=x_{a0}-y_{a0}+x_{a1}-y_{a1}+x_{a2}-y_{a2}\label{c19}\\
              &0=x_{a0}-y_{a0}+x_{a1}-y_{a1}+x_{a2}-y_{a2}+x_{a3}-y_{a3}\label{c20}
\end{align}

Clairement, dans le modèle start/end, nous avons les contraintes
suivantes: $x_{a3}=0$ et $y_{a0}=0$. Donc, les contraintes~(\ref{c17})
et~(\ref{c20}) impliquent que $x_{a0}=0,6$ et $y_{a3}=0,7$. 

Or, la contrainte~(\ref{c18}) implique $y_{a1}=0,6+x_{a1}$
et la contrainte~(\ref{c9}) implique $0,6+x_{a1}+x_{a2}=1 \Rightarrow
x_{a2}=0,4-x_{a1}$. 

La contrainte~(\ref{c12}) peut donc s'écrire
$0,6+2x_{a1}+x_{a2}\le 1 \Rightarrow 1+x_{a1}\le 1 \Rightarrow
x_{a1}=0$. 

Ceci implique $x_{a2}=0,4$ et $y_{a1}=0,6$, et l'on obtient
une contradiction avec la contrainte~(\ref{c10}) puisque $0+0,6+y_{a2}+0,7>1$.

Donc, nous avons bien montré que le modèle start/end possède en
théorie de meilleures relaxations que le modèle on/off. Cependant, si
nous ajoutons un ensemble particulier d'inégalités à ce modèle, ceci
n'est plus le cas. Ces inégalités sont décrites dans le paragraphe
suivant.


\subsubsection{Inégalités de non préemption}
\label{sec:nonPreem}
Dans ce paragraphe, nous présentons un ensemble d'inégalités que nous
utilisons pour donner une description minimale du polyèdre défini par
l'ensemble de toutes les affectations possibles des variables binaires
$z_{ie}$ pour une seule activité. 

L'ensemble d'inégalités, appelées inégalités de non préemption, est
défini comme suit. Puisque, dans tout ordonnancement réalisable,
chaque activité doit être exécutée sans préemption, $z_{ie}$ doit
satisfaire:
\begin{equation}
  \sum_{e_u \in {\cal F}} (-1)^{u} z_{i,e_u} \le 1
\label{non_preem_ineg}
\end{equation}
où ${\cal F}=\{e_0,e_1,\dots,e_{2v}\}$ est un sous-ensemble ordonné de
cardinalité impaire de $\E^*=\Em$.  

Soit le polyèdre $ZP_i=\{z_i \in [0,1]^{\E^*}\ | \ z_i\text{ satisfait
\eqref{start_CECSP_OO} et \eqref{non_preem_ineg}}\}$ et soit
$ZQ_i=\mathrm{conv}\{z_i \in \{0,1\}^{\E^*}\ | \ z_i\text{ satisfait
\eqref{start_CECSP_OO},\eqref{preem1_CECSP_OO} et
\eqref{preem2_CECSP_OO}}\}$. Nous allons montrer le théorème suivant: 

\begin{theo}
$ZP_i=ZQ_i$.
\end{theo}

\begin{proof}

Dans un premier temps, nous rappelons le lemme de Farkas que nous
utiliserons plus tard dans la preuve. 

\begin{lemma}[Lemme de Farkas]
\label{Farkas_Lemma}
Soit $A$ une matrice de taille $m \times n$ et $b$ un vecteur de
$\mathbb{R}^m$. Il existe un vecteur $x \in \mathbb{R}^n$ vérifiant
$Ax = b$ si et seulement si pour tout vecteur $y \in \mathbb{R}^m$
tel que $ y^TA\le 0$ on a $y^Tb \le 0$. 
\end{lemma}

Nous allons exhiber un ensemble d'inégalités linéaires décrivant
$ZQ_i$. Pour cela, commençons par remarquer que les sommets de $ZQ_i$
sont exactement les vecteurs de dimension $|\E^*|$ suivants:
\[
z_{i,e}^{uv}=\left\{
\begin{array}{ll}
1 & \text{ si }u \le e \le v\\
0 & \text{ sinon}
\end{array}
\right. \qquad \forall u,\ v \in \E^*,\ u \le v
\]
Donc, un point $\overline{z}_i \in ZQ_i$ si et seulement si le système
linéaire suivant admet une solution réalisable:
\begin{align}
& \sum_{u\le v} z_{i,e}^{uv} \lambda_{uv} = \overline{z}_{i,e} & &
\forall e \in \E^* \label{kis1}\\
&\sum_{u \le v} \lambda_{uv} =1 & & \label{kis2}\\
&\lambda \ge 0 & & \label{kis3}
\end{align}

Or, d'après le lemme de Farkas, le
système~\eqref{kis1}--\eqref{kis3} admet une solution réalisable si et
seulement si $\forall \mu \in \mathbb{R}^{|\E^*|}$ tel que: 
\begin{equation}
\sum_{e=u}^{v} \mu_e + \mu_0 \le 0, \quad u \le v 
\label{kis4} 
\end{equation}
$\mu$ satisfait aussi la condition suivante: 
\begin{equation}
\sum_{e \in \E^*} \mu_e \overline{z}_{i,e}+ \mu_0 \le 0
\label{kis5}
\end{equation}
Donc, puisque les rayons extrêmes du cône caractérisé par~\eqref{kis4}
définissent toutes les inégalités linéaires requises pour décrire
$ZQ_i$, il suffit, pour prouver le théorème, de trouver tous ces
rayons extrêmes. 

Nous allons ensuite montrer qu'il existe une bijection entre les
rayons extrêmes du cône~\eqref{kis4} vérifiant~\eqref{kis5} et les
inégalités de $ZP_i$.

Pour trouver les rayons extrêmes du cône~\eqref{kis4}, il suffit de
considérer les trois cas suivants: 
\paragraph{\boldmath $\mu_0 > 0$.} Quitte à normaliser, nous pouvons
suppposer que $\mu_0=1$. Dans ce cas-là, pour tout $e \in \E^*$, nous
avons $\mu_e \le -1$. En effet, ceci est déduit de~\eqref{kis4} en
considérant l'équation pour $u=v=e$. Alors, en prenant
$\mu_e=-1,\forall e \in \E^*$, ~\eqref{kis5} implique:
\[ \sum_{e \in \E^*} - \overline{z}_{i,e} \le -1 \] D'après le lemme
de Farkas, ceci est une inégalité valide pour $ZQ_i$. On peut
remarquer que ces inégalités sont équivalentes
à~\eqref{start_CECSP_OO}. 
\paragraph{\boldmath $\mu_0 = 0$.} Alors, nous avons toujours un cône
dont les rayons extrêmes sont les vecteurs unité négatifs de
$\mathbb{R}^{\E^*}$. Ces rayons extrêmes nous donnent les inégalités
$z_{i,e} \ge 0$ qui sont les contraintes de non négativité valides
pour $ZQ_i$.
\paragraph{\boldmath $\mu_0 < 0$.} Quitte à normaliser, nous pouvons
supposer que $\mu_0=-1$. Nous allons prouver que dans ce cas, il
existe une bijection entre les points extrêmes du polyèdre $H$ défini
par:
  \begin{equation} \sum_{e=u}^{v} \mu_e \le 1 \quad u \le
    v \label{kis6}
  \end{equation} 
et les inégalités~\eqref{non_preem_ineg}. Comme~\eqref{kis6}
implique~\eqref{kis5}, ceci montrera bien la bijection entre les
rayons extrêmes de~\eqref{kis4} vérifiant~\eqref{kis5} et les
inégalités~\eqref{non_preem_ineg}.

  Dans un premier temps, nous montrons que le vecteur formé des
  coefficients de la partie gauche de chaque inégalité
  de~\eqref{non_preem_ineg} est une solution de~\eqref{kis6}
  correspondant à un point extrême de $H$. 

  Soit ${\cal F}=\{e_0,e_1,\dots,e_{2v}\}$ un ensemble d'événements
  vérifiant $e_i<e_{i+1}$ pour $i=0,\dots,2v-1$. Le vecteur
  $\overline{\mu}$ formé des coefficients de la partie gauche de chaque
  inégalité de~\eqref{non_preem_ineg} est défini par:
  \[ \overline{\mu}_e=\left\{ 
      \begin{array}{ll}
        (-1)^u & \text{ si } e=e_u \in \cal F \\
        0 & \text{ si } e \in \E^* \setminus \cal F
      \end{array}
    \right.
  \]
  Pour prouver que $\overline{\mu}$ est une solution de~\eqref{kis6}
  correspondant à un point extrême de $H$, nous exhibons un
  sous-système $L$ de~\eqref{kis6} 
  contenant $|\E^*|$ inégalités linéairement indépendantes telles que
  chaque inégalité de $L$ soit vérifiée à l'égalité par
  $\overline{\mu}$. 

  Le sous-système $L$ est formé des inégalités suivantes: 
  \begin{align*}
    & \sum_{e=u}^{e_0} \mu_e \le 1 & & u=1,\dots, e_0-1\\
    & \sum_{e=e_{2v}}^u \mu_e \le 1& & u=e_{2v},\dots,|\E^*|
  \end{align*}
  De plus, $L$ contient l'ensemble d'inégalités suivant, défini pour
  tout ensemble formé de 3 événements consécutifs
  $e_{2u},e_{2u+1},e_{2u+2} \in \cal F$:   
  \begin{align*}
    & \sum_{e=e_{2u}}^{e_{2u+2}} \mu_e \le 1 & & \\
    & \sum_{e=e_{2u}}^t \mu_e \le 1& & t=e_{2u},\dots,e_{2u+2}-1\\
    & \sum_{e=t}^{e_{2u+2}} \mu_e \le 1& & t=e_{2u+1}+1,\dots,e_{2u+2}-1
  \end{align*} 
  On peut facilement vérifier que le système ci-dessus est formé de
  $|\E^*|$ inégalités linéairement indépendantes et que
  $\overline{\mu}$ vérifie chacune d'entre elle à l'égalité et ceci
  prouve notre affirmation. 

  Nous montrons maintenant que toute solution $\overline{\mu}$
  correspondant à un point extrême de $H$ équivaut à une inégalité
  de~\eqref{non_preem_ineg}. Dans un premier temps, remarquons que la
  matrice formée par les coefficients de la partie gauche
  de~\eqref{kis6} est totalement unimodulaire. En effet, les colonnes
  de cette matrice peuvent être réordonnées de telle sorte que chaque
  ligne ne contienne que des $1$ consécutifs. Donc, tout sommet de
  ce polyèdre est un vecteur à valeurs entières. Nous remarquons aussi
  que $\overline{\mu}_e\le 1,\ \forall e \in \E$, puisque $\mu_e \le 1$
  est une inégalité de~\eqref{kis6} ($u=v$) pour tout $e \in \E^*$.
  
  Soit $u_1$ le premier indice tel que $\overline{\mu}_{u_1} \neq 0$. Nous
  affirmons que $\overline{\mu}_{u_1}=1$. Supposons que ce ne soit pas
  le cas, i.e. $\overline{\mu}_{u_1} \le -1$ (les coordonnées de
  $\overline{\mu}$ sont entières). Puisque $\overline{\mu}$ est un point
  extrême de $H$, il existe un sous-ensemble $L$ formé de $|\E^*|$
  inégalités linéairement indépendantes de~\eqref{kis6} qui sont
  satisfaites à l'égalité par $\overline{\mu}$. 

  Remarquons que $L$ doit contenir une inégalité impliquant la
  variable $\mu_{u_1}$. En effet, si ce n'est pas le cas, cette variable
  peut être arbitrairement fixée à une valeur négative et toujours
  satisfaire toutes les inégalités de $L$. Et donc $\overline{\mu}$ ne
  serait pas un point extrême de $H$ ce qui est une contradiction. Comme
  $\overline{\mu}_e=0$ pour $e< u_1$, une telle inégalité doit être de
  la forme $ \sum_{e=u_1}^{v_1} \mu _e \le 1$. Comme elle doit être
  satisfaite par $\overline{\mu}$ à l'égalité et que $\overline{\mu}_e
  \le 1$, nous avons que $\overline{\mu}$ doit contenir au moins deux
  coordonnées $q_1$ et $q_2$ tels que $ u_1 \le q_1 \le q_2 \le v_1$
  avec $\overline{\mu}_{q_1}=\overline{\mu}_{q_2}=1$ et
  $\overline{\mu}_{e}=0$ pour $q_1 < e < q_2$. Mais, dans ce cas-là,
  $\overline{\mu}$ violerait l'inégalité $\sum_{e=q_1}^{q_2} \mu_e \le
  1$ et ceci est une contradiction. Donc, la première coordonnée non
  nulle de $\overline{\mu}$ doit 
  prendre la valeur $1$. 
  
  La seconde coordonnée non nulle, disons
  $u_2$, ne peut prendre la valeur $1$ par le même raisonnement que
  précédemment. Donc, cette valeur doit être négative et
  entière. Mais, dans ce cas-là, nous pouvons suivre la même
  argumentation que précédemment pour montrer que $u_2$ doit
  apparaître dans une des inégalités de $\ell \in L$ et aboutir à la même
  contradiction que précédemment. Donc, la seconde coordonnée non nulle
  de $\overline{\mu}$ doit être égale à $-1$. De plus, $\ell$ doit
  contenir une inégalité impliquant une variable $\mu_{u_3}$ de valeur
  $1$ dans $\overline{\mu}$. En effet, dans le cas contraire,
  $\overline{\mu}$ ne peut satisfaire $\ell$ à l'égalité. Si nous
  poursuivons cette argumentation tant que $\mu$ possède des
  coefficients non nuls après $u_3$, nous reconnaissons la suite de
coefficients $1/-1$ présente dans~\eqref{non_preem_ineg}.
\end{proof}

Nous avons donc défini un ensemble d'inégalités permettant une
description complète du polyèdre formé par les vecteurs $z_i$ solution
du modèle on/off. Cependant, nous ne pouvons ajouter directement ces
inégalités au modèle car leur nombre est exponentiel. Dans le
chapitre~\ref{sec:expe}, nous présenterons un algorithme de 
séparation polynomial qui nous permettra de définir un algorithme de
branch-and-cut pour le \CECSP~et le \RCPSP~(rappelons que les
résultats ci-dessus sont valides dans le cas du \RCPSP~en posant
$\E^*=\{1,\dots,n\}$). 

Le paragraphe suivant présente d'autres inégalités valides pour le
\CECSP~et le \RCPSP. 

\subsubsection{Autres inégalités valides}

\label{sec:maxDist}
Dans ce paragraphe, nous décrivons plusieurs ensembles d'inégalités
valides pour le \RCPSP~ et le \CECSP. Dans la suite, nous considérons
qu'un événement correspond à une et une seule date de début/fin,
i.e. $\forall e \in \Em[1]$, il existe une et une seule activité $i$
vérifiant $(z_{i,e-1}-z_{ie}=1) \vee (z_{ie}-z_{i,e-1}=1)$. Cela est
toujours possible puisque $|\E|=2n$. De plus, l'ajout de cette
supposition ne change pas la véracité de ce qui précède. 

\paragraph{Séparation maximale entre deux événements} 
Les inégalités définies dans ce paragraphe sont des inégalités bornant
supérieurement la valeur de $t_{e+1}- t_e, \forall e\in \Em$. Pour
définir ces inégalités, nous étudions les fenêtres de temps de chaque
date de début et de fin d'une activité, i.e. $[\ES,\LS]$ et
$[\EE,\LE],\ \forall i \in \A$. L'idée principale repose sur le fait
que, dans chacun de ces intervalles, un événement doit forcément avoir
lieu. De ce fait, nous savons qu'il y a au moins deux événements
consécutifs dans l'union de deux fenêtres de temps consécutives. 

Soit ${\cal D}$ l'ensemble de toutes les fenêtres de temps, i.e. ${\cal
D}=\{[\ES,\LS] , [\EE,\LE],\ \forall i \in \A\}$. Nous commençons par
trier les intervalles de $\cal D$ suivant la règle suivante: 
$[a,b] \le [c,d]
\Leftrightarrow a<c \lor \left( a=c \land b\le d\right)$

Alors, soit $\underline{{\cal D}_e}$ (resp. $\overline{{\cal D}_e}$)
la borne inférieure (resp. supérieure) de l'intervalle ${\cal D}_e$,
nous avons la propriété suivante:
\begin{equation} \label{sep_CECSP_OO} 
t_{e+1}-t_e \le \max(\overline{{\cal D}_e},\overline{{\cal D}_{e+1}}) -
\min(\underline{{\cal D}_e},\underline{{\cal D}_{e+1}}) \qquad \forall e \in {\cal E}\setminus\{2n\}
\end{equation}

\begin{ex}
\label{ex:evt_sep} Considérons l'ensemble d'intervalles suivant:
\begin{center}
\begin{tikzpicture} [yscale=0.8,xscale=0.6] \node (O) at (0,0) {};

\draw[->] (0,0) -- (23,0);
 \draw (1,0) node[red] {$[$} node[above=0.2cm,red]
{$\ES[1]$}-- +(0:3cm)node[red] {$]$} node[right=0.1cm,above=0.2cm,red]{$\LS[1]$};

\draw(7.05,0) node[red] {$[$} node[above=0.2cm,red] {$\EE[1]$}-- +(0:3cm) node[red]
{$]$} node[above=0.2cm,red] {$\LE[1]$};
 \draw (3,0) node[Green] {$[$} node[below=0.3cm,Green]
{$\ES[2]$}-- +(0:4cm) node[Green] {$]$} node[below=0.2cm,Green] {$\LS[2]$};
 \draw
(16,0) node[Green] {$[$} node[above=0.2cm,Green] {$\EE[2]$}-- +(0:3cm) node[Green] {$]$}
node[right=0.1cm,above=0.2cm,Green] {$\LE[2]$};
 \draw(10.1,0) node[blue] {$[$}
node[below=0.3cm,blue] {$\ES[3]$}-- +(0:3cm) node[blue] {$]$} node[below=0.2cm,blue]
{$\LS[3]$};
 \draw (15,0) node[blue] {$[$} node[below=0.2cm,blue] {$\EE[3]$}--
+(0:7cm) node[blue] {$]$} node[below=0.25cm,blue] {$\LE[3]$};

\end{tikzpicture}
\end{center}

Après avoir trié les intervalles, nous obtenons:
$[\ES[1],\LS[1]] \le [\ES[2],\LS[2]] \le
[\EE[1],\LE[1]] \le [\ES[3],\LS[3]] \le 
[\EE[3],\LE[3]] \le [\EE[2],\LE[2]] $.

Alors, nous avons l'ensemble de contraintes suivantes:

\begin{itemize}
\item $t_2-t_1 \le \LS[2]-\ES[1]$
\item $t_3-t_2 \le \LE[1]-\ES[2]$
\item $t_4-t_3 \le \LS[3]-\EE[1]$
\item $t_5-t_4 \le \LE[3]-\ES[3]$
\item $t_6-t_5 \le \LE[3]-\EE[3]$
\end{itemize}

\end{ex}

Notons aussi que l'ensemble d'inégalités
$\underline{{\cal D}_e} \le t_e \le \overline{{\cal D}_e}$ peut ne pas
être valide. En effet, ici $t_6$ peut correspondre à la fin de
l'activité $3$ et $t_5$ à la fin de l'activité $2$, alors que
${\cal D}_5=[\EE[3],\LE[3]] \le [\EE[2],\LE[2]]={\cal D}_6$. On aurait
alors $\underline{{\cal D}_6} \le t_5 \le \overline{{\cal D}_6}$.

Ces contraintes peuvent être ajoutées au modèle on/off ou utilisées
comme borne supérieure sur la valeur de $\bmin(t_{e+}-t_e)$
dans~\eqref{bmin_CECSP_OO}. L'inégalité se réécrit donc comme:
\[ b_{ie} \ge \bmin(t_{e+}-t_e) - \bmin\left(\max(\overline{{\cal
D}_e},\overline{{\cal D}_{e+1}}) - \min(\underline{{\cal
D}_e},\underline{{\cal D}_{e+1}})\right)(1-z_{ie})\qquad \forall (i,e)
\in {\cal A}\times{\cal E}
\]

Ces inégalités peuvent être généralisées à tout sous-ensemble de $k$
intervalles ordonnés $\{{\cal D}_{e_1},\dots,{\cal D}_{e_k}\}$ avec
$t_{e_k}-t_{e_1} \le \max(\overline{{\cal D}_{e_1}},\overline{{\cal
D}_{e_k}}) - \min(\underline{{\cal D}_{e_k}},\underline{{\cal
D}_{e_1}}) $.

\paragraph{Date maximale d'un événement}


Une idée similaire à celle décrite dans le paragraphe précédent peut
être utilisée pour ordonner les événements et calculer des bornes
supérieures sur leur date. 

Pour faire cela, nous commençons par trier les bornes supérieures des
fenêtres de temps de chaque activité, i.e. $\LS$ et $\ES,\ \forall i
\in \A$, par ordre croissant. Alors, puisqu'un événement doit avoir
lieu dans chaque fenêtre de temps, i.e. avant chaque borne supérieure de
chaque fenêtre, nous pouvons déduire une borne supérieure sur la date
de chaque événement.

En effet, soit ${\cal UP}$ l'ensemble formé de toutes les bornes
supérieures de toutes les fenêtres de temps. Alors, nous avons la
propriété suivante: 
\begin{equation} \label {Bte_CECSP_OO} t_e \le {\cal UP}_e \qquad
\forall e \in {\cal E}
\end{equation}

\begin{ex} 
Considérons les intervalles définis dans
l'exemple~\ref{ex:evt_sep}. Alors, nous pouvons déduire l'ensemble de
contraintes suivantes:

\begin{itemize}
\item $t_1 \le \LS[1]$
\item $t_2 \le \LS[2]$
\item $t_3 \le \LE[1]$
\item $t_4 \le \LS[3]$
\item $t_5 \le \LE[2]$
\item $t_6 \le \LE[3]$
\end{itemize}
\end{ex}

Comme précédemment, nous pouvons utiliser ces inégalités comme
contraintes additionnelles des modèles à événements ou les utiliser
à la place de  $T$ dans les contraintes \eqref{twx_CECSP_OO}
et\eqref{twy2_CECSP_OO}. Les contraintes s'écrivent alors: 
\begin{align*}
& \ES z_{ie}\le t_e \le \LS(z_{ie}-z_{i,e-1})+(1-(z_{ie}-z_{i,e-1})){\cal UP}_e 
 & & \forall e \in \E\setminus \{1\},\ \forall i \in {\cal
   A}\\
&t_e \le \LE(z_{i,e-1}-z_{ie})+(1-(z_{i,e-1}-z_{ie})){\cal UP}_e  & & \forall e
 \in \E\setminus \{1\},\ \forall i \in {\cal
   A}
\end{align*}

Pour le \RCPSP, $t_n$ correspond à borne supérieure sur la durée
totale du projet, i.e. sur $T$.

\paragraph{Inégalités valides dérivées du problème de sac-à-dos}

Le rendement minimal de chaque activité pouvant être positif, nous
pouvons considérer les contraintes de type sac-à-dos suivantes pour
tout $e \in \Em$ et les transformer facilement en inégalités valides: 
\begin{equation}
\sum_{i\in \A^+} \bmin z_{ie} \leq B  \qquad  \forall e \in \E
\end{equation}
où $\A^+$ est le sous-ensemble d'activités avec $\bmin > 0$. 


\paragraph{Inégalités de cliques}

Les inégalités de clique permettent de modéliser le fait que plusieurs
variables binaires $z_{ie}$ ne peuvent prendre la valeur $1$
simultanément. Ces inégalités, déjà établies dans le cas du
\RCPSP~\cite{CAVT_clique}, correspondent aux sous-ensembles
disjonctifs d'activités. Elles sont facilement adaptables au cas du
\CECSP~et sont définies de la manière suivante. Soit $C$ un ensemble
minimal d'activités ne pouvant s'exécuter en parallèle, i.e. telles que
$\sum_{i \in C} \bmin > B$, alors l'ensemble d'inégalités suivantes
est valide pour le \CECSP: 
\begin{equation} 
\sum_{i\in C} z_{ie}  \le |C| -1 \qquad \forall C,\ \forall e \in \E
\end{equation}


Différentes techniques permettant l'intégration des inégalités
ci-dessus seront présentées et comparées dans le
chapitre~\ref{sec:expe}.

\chapter*{Conclusion}

Dans ce chapitre, nous avons présenté des modèles de programmation
linéaire en nombres entiers pour le problème du \RCPSP~ et pour le
\CECSP. Pour chacun de ces problèmes, trois modèles sont présentés,
un modèle utilisant une discrétisation de l'horizon de temps et deux
modèles basés sur une représentation des événements pertinents du
problème. 

Enfin, des améliorations de ces modèles sont proposées dans la
dernière partie du chapitre. Ces améliorations sont basées sur le
raisonnement énergétique, la mise en place d'inégalités valides et des
études polyédrales. De plus, les avantages et inconvénients de chacun
des modèles sont décrits ce qui permet de justifier l'intérêt de ces
améliorations. 

Des résultats numériques évaluant les performances de ces formulations
ainsi que l'intérêt de chaque amélioration sur diverses instances du
\CECSP~et du \RCPSP~feront l'objet d'un paragraphe dans le chapitre
portant sur les expérimentations (cf. Chapitre~\ref{sec:expe}). 
\chapter*{Conclusion}

Cette partie consacrée à la programmation linéaire mixte et en nombres
entiers a commencé par décrire les concepts fondamentaux de cette
théorie. Puis, dans un second temps, nous avons présenté plusieurs
modèles pour un problème d'ordonnancement à contraintes de ressource:
le \RCPSP. Pour ce problème, plusieurs types de modèles ont été
présentés et, pour chacun d'eux, nous avons discuté leurs avantages et
inconvénients. Les modèles indexés par le temps sont plus performants
sur des instances ayant de petits horizon de temps tandis que les
modèles à événements s'avèrent plus efficaces sur des instances
disposant de plus grands horizon de temps. 

Ces deux types de modèles ont ensuite été adaptés pour être appliquer
au \CECSP. Pour ce problème, en plus des avantages décrits ci-dessus,
les modèles à événements disposent d'un avantage supplémentaire. En
effet, pour le \CECSP, il est possible qu'une instance ne possède que
des solutions rationnelles et ce même si cette dernière est seulement
pourvu de données entières. Les modèles indexés par le temps ne
permettent d'obtenir que des solutions à date de début et fin
d'activités entières. A l'inverse, les modèles à événements permettant
d'obtenir des solutions non entières, sont donc plus adaptés au cas du
\CECSP. De ce fait, une grande partie de nos travaux a porté sur le
renforcement de ces modèles par le biais de l'ajout de coupes et
d'inégalités valides. De plus, nous avons montré qu'un de ces
ensembles d'inégalités permettait de décrire exactement l'enveloppe
convexes des vecteurs binaires solutions du modèle. 

Des inégalités valides pour le modèle indexé par le temps ont aussi
été décrites. Ces inégalités sont directement déduites d'un algorithme
présenté dans la partie dédiée à la programmation par contrainte de
ce manuscrit: le raisonnement énergétique. Dans la continuité de ces
travaux, d'autres raisonnements issues de la programmation par
contraintes pourraient être envisagés afin de déduire des ensembles
d'inégalités pouvant être ajoutés au modèles (indexé par le temps ou à
événements). De plus, les modèles présentées souffrent d'un grand
nombre de symétrie. Ces dernières pourraient être éviter par l'ajout
de nouvelles contraintes. La définition de telle contraintes ainsi
qu'une étude du polyèdre associé fait aussi partie des poursuites de
recherche sur ce sujet. La dernière perspective dans ce domaine
seraient l'établissement de nouveaux modèles indexés par le temps, il
en existe beaucoup d'autres pour le \RCPSP, ou le mise en place de
nouvelles formulations étendues.

%
\clearemptydoublepage%
\cleardoublepage
\begin{minipage}{0.95\linewidth}

\part{Implémentations et Expérimentations}
\label{sec:expe}
\vspace{15mm} % l'espacement souhaité
\parttoc 
\end{minipage}
\newpage
\thispagestyle{empty}
\vspace*{\stretch{1}}
\begin{center}
  \begin{minipage}{\textwidth}
    \hrule
    \vspace{0.5cm}
    {\it  La dernière partie de cette thèse est consacrée à la
      présentation des résultats des expérimentations que nous avons
      conduites pour valider les méthodes présentées dans les
      chapitres précédents. Nous commençons, dans un premier temps,
      par présenter les instances sur lesquelles ces expérimentations
      ont été menées ainsi que les algorithmes de prétraitement
      utilisés. 

      La suite de cette partie est ensuite consacrée à la présentation
      des résultats des méthodes décrites dans le
      chapitre~\ref{sec:PLNE_CECSP} et issues de la programmation linéaire
      mixte. Le modèle indexé par le temps pour le \CECSP~est comparé
      à ce même modèle auquel on a ajouté les inégalités déduites du
      raisonnement énergétique et présentées dans le
      paragraphe~\ref{sec:ER_TI}. Pour les modèles à événements, nous
      commençons par détailler l'algorithme utilisé pour séparer les
      inégalités de non-préemption (voir
      paragraphe~\ref{sec:nonPreem}). Les modèles sont ensuite comparés
      entre eux et l'ajout des inégalités et coupes décrites dans le
      paragraphe~\ref{sec:amelioration_OO} au modèle On/Off dans le cas du
      \CECSP~et du \RCPSP~est discuté. La dernière partie du paragraphe
      traitant de la programmation linéaire est consacrée à la comparaison
      des trois modèles.

      Le dernier paragraphe concerne les résultats des méthodes issues de
      la programmation par contraintes présentées dans le
      chapitre~\ref{sec:PPC_CECSP}. Dans un premier temps, nous
      décrivons le cadre dans lequel ces expérimentations ont été
      conduites: les algorithmes de propagation détaillés dans ce manuscrit
      sont inclus dans une méthode arborescente hybride. Une fois le cadre
      des expérimentations posé, nous comparons les trois méthodes permettant
      de calculer les intervalles d'intérêt du raisonnement
      énergétique. Enfin, la suite de cette partie détaille les résultats de
      cette méthode arborescente avec les différents algorithmes de
      propagation. 

    Dû à la difficulté du \CECSP, les expérimentations présentées
    dans cette partie ont majoritairement été conduites sur des
    instances dans lesquelles les activités ont des fonctions de
    rendement affines. Quelques résultats sur des instances de petite
    taille avec des fonctions de rendement concaves et affines par
    morceaux sont présentés à la fin de cette partie.}
    \vspace{0.5cm}
    \hrule
  \end{minipage}
\end{center}
\vspace*{\stretch{1}}

\chapter{Implémentations et Expérimentations}

Dans ce chapitre, consacré à la présentation des résultats
expérimentaux, nous allons, dans un premier temps, décrire les
jeux d'instances utilisés ainsi que certaines méthodes de
prétraitement qui sont appliquées à ces instances avant leur
résolution. 

Le paragraphe suivant sera quant à lui dédié aux résultats de la
programmation linéaire mixte et en nombres entiers. Nous présenterons
les performances des différents modèles présentés dans le
chapitre~\ref{sec:PLNE_CECSP} ainsi que l'impact des améliorations
proposées dans le paragraphe~\ref{sec:amelioration_modele} de ce même
chapitre. 

Le dernier paragraphe sera consacré à la présentation des performances
de la Programmation Par Contraintes où les performances des
algorithmes de filtrage présentés dans le chapitre~\ref{sec:PPC_CECSP}
seront comparées. 

Tous les tests ont été effectués avec un processeur Intel Core i7-4770
de 3.40GHz, 8 GB de RAM, tournant sous le système d’exploitation
Ubuntu 12.04 à 64-bits. Toutes les méthodes présentées ont été codées
en C++. Les programmes linéaires mixtes sont résolus avec le solveur
commercial ILOG-Cplex version 12.6. Le temps limite de résolution pour
chaque instance et chaque méthode a été fixé à 7200 secondes.

%TODO : fonction LPM et résultats time index RCPSP avc RE?
\section{Génération des instances et pré-traitement}

Ce paragraphe présente, dans un premier temps, la méthode utilisée
pour générer des ensembles d'instances de test pour le \CECSP. Nous
détaillons ensuite les jeux d'instances du \RCPSP sur lesquelles nous
avons évalué l'impact des améliorations proposées pour les modèles à
événements. Nous présenterons aussi une méthode de calcul des fenêtres
de temps pour le \RCPSP, appliqué sur les instances comme
pré-traitement à leur résolution.

\subsection{Instances du \CECSP}
\label{sec:instance_CECSP}
Les instances utilisées dans le cadre du \CECSP sont regroupées en
quatre familles selon les caractéristiques de leurs fonctions de
rendements et de l'énergie que les activités requièrent. Dans un
premier temps, les instances sont générées selon un modèle commun,
puis plusieurs types de modifications spécifiques sont appliquées sur
celles-ci afin de les séparées en différentes familles. 

Tout d'abord, cinq instances de $10$ et $60$ activités ainsi que dix
instances de respectivement $20$, $25$ et $30$ activités sont
générées. Pour toutes ces instances, la disponibilité de la ressource
est fixée à $R=10$ et les autres paramètres du problème sont générés
de façon aléatoire selon une loi uniforme et dans les intervalles
suivants:
\begin{itemize}
\item $W_i \in [1 , \frac{5}{4} * R]$,
\item $\bmin \in [0, \frac{1}{4} * W_i]$,
\item $\bmax \in [\bmin , 2 * \bmin {]}$,
\item $\ES \in [0,\frac{1}{2}*n]$ et
\item $\LE \in [ \EE, \EE + n ]$ avec $\EE = \ES +
\frac{W_i}{f_i(\bmax)}$.
\end{itemize}

Une première famille d'instance, appelée Famille 4, est construit en
utilisant comme fonction de rendement la fonction identité pour toutes
les activités, i.e. $\forall i \in \A,\ f_i(b_i(t))= b_i(t)$. Ensuite,
des fonctions de rendement affines sont générées aléatoirement pour
chaque activité. Pour cela, nous générons, pour chaque activité $i \in
\A$, les paramètres $a_i$ et $c_i$ de la fonctions $f_i$ suivant une
loi uniforme et telle que:
\begin{itemize}
\item $a_i \in [1,10]$ et
\item $c_i \in [1,10]$.
\end{itemize} La valeur de $W_i$ est ensuite modifiée à l'aide de ces
fonctions $f_i$. Les trois autres familles d'instances sont classées
suivant cette modification. Pour la première famille, appelée Famille
1, $W_i$ est généré aléatoirement, selon une loi uniforme, dans
l'intervalle $[0,f_i(W_i)]$ et pour la Famille 2 dans l'intervalle
$[f_i(W_i)/2, f_i(W_i)]$. Enfin, pour la Famille 3, la valeur
$f_i(W_i)$ est affectée $W_i$. 

Nous avons donc défini quatre familles d'instances pour le \CECSP. Ces
instances serviront à calculer les performances relatives des méthodes
présentées dans les chapitres précédents de ce manuscrit. 

Dans le prochain paragraphe, nous présentons les instances utilisées
pour évaluer les performances des améliorations des modèles du
\RCPSP~ainsi qu'une méthode de calcul des fenêtres de temps.
  
\subsection{Instances et pré-calcul des fenêtres de temps pour le
  \RCPSP}
\label{sec:instances_RCPSP}
\subsubsection{Génération des instances}
Les instances utilisées sont les instances définies par
Kone~\cite{theseOumar} pour évaluer la performances des modèles à
événements par rapport aux modèles indexés par le temps. Ces instances
sont générées à partir des instances à $30$ activités de la
PSPLIB~\cite{PSPLIB}. Ces instances, au nombre de $480$, sont des
instances à $4$ ressources renouvelables et avec des activités ayant
une durée aléatoire comprise entre $1$ et $10$ unités de temps. Ces
durée étant relativement courtes, les modèles indexés par le temps
sont particulièrement efficaces sur ces dernières (voir
paragraphe~\ref{sec:motiv_event_RCPSP}). L'auteur de~\cite{theseOumar}
a donc modifié ces instances afin d'allonger la durée des
activités. En effet, dans certaines applications pratiques (notamment
dans le domaine pharmaceutique ou de la pétrochimie), il arrive que
les activités aient des durées relativement longues.

Les instances de Kone sont crées selon le principe suivant: 
\begin{enumerate}
\item les $15$ premières activités non-fictives de l’instance sont
  sélectionnées (les autres activités ainsi que les contraintes de
  précédence qui leur sont adjacentes sont laissées de côté).
\item parmi les activités sélectionnées, celles sans prédécesseurs
  sont connectées à l’activité 0 et celles sans successeurs à l’activité
  $16$.
\item  $7$ activités parmi les $15$ sélectionnées sont ensuite
  choisies de façon aléatoire et leur durée est multipliée par un
  coefficient $25+b$, où $b$ est un nombre aléatoire généré entre $0$
  et  $1$.

  Les durées obtenues sont ensuite arrondies au nombre entier le plus
  proche. 
\end{enumerate}
$480$ instances de taille plus réduite sont obtenues ($n=15$), mais
avec des durées opératoires plus grandes, allant ainsi de $1$ à
approximativement $250$ unités de temps. 

\subsubsection{Calcul des fenêtres de temps}

Afin d'améliorer les performances des modèles pour le \RCPSP, nous
calculons par prétraitement, pour chaque activité, les fenêtres de
temps $[\ES,\LS{]}$ et $[\EE,\LE{]}$ dans lesquelles elle peut
respectivement commencer et finir. Pour calculer ces fenêtres de
temps, rappelons que, si chaque arc $(i,j)$ du graphe des précédences
$G$ est pondéré par $p_i$, la date de début au plus tôt de $i$, $\ES$
peut prendre la valeur du plus long chemin entre l'activité $0$ et
l'activité $i$ et la date de début au plus tard $\LS$ est égale à
$\ES[n+1]$, la date de début au plus tôt de l'activité terminale
$n+1$, moins la valeur du plus long chemin entre $i$ et $n+1$.

Notons que la date de fin au plus tard de l'activité fictive $n+1$
correspondant à une borne supérieure sur la date de fin du projet peut
être calculée au moyen d'une heuristique. L'heuristique utilisée pour
calculer cette valeur ici est la méthode d'ordonnancement parallèle
avec comme règle de priorité la plus petite date de fin au plus tard
des activités~\cite{heur_RCPSP}. 

En plus des calculs décrits ci-dessus, nous utilisons d'autres
techniques de déductions basées sur la propagation de contraintes. En
particulier, nous utilisons des techniques décrites dans la thèse de
Demassey~\cite{these_Sophie}. Parmi celles utilisées, nous trouvons
les techniques de sélection immédiate, Edge-Finding sur les cliques de
disjonction et de triplets symétriques. Ces techniques n'étant pas
détaillées ici, à l'exception du Edge-Finding au
paragraphe~\ref{sec:nrj_CUSP}, nous renvoyons le lecteur au chapitre 2
de~\cite{these_Sophie} pour une description de ces techniques.

\section{Performances de la Programmation Linéaire Mixte}
\label{sec:expe_PLNE}
Dans ce paragraphe, nous détaillons les résultats expérimentaux
obtenus par les modèles présentés dans le
chapitre~\ref{sec:PLNE_CECSP} et l'influence des techniques de
renforcement présentées dans ce même chapitre. Nous commençons, dans
un premier temps, par présenter les résultats obtenus pour le modèle
indexé par le temps puis nous détaillerons les résultats des modèles à événement.

\subsection{Modèles indexés par le temps}
Dans ce paragraphe, nous décrivons les performances du modèle indexé
par le temps pour le \CECSP. Les expérimentations ont été conduites
sur les quatre familles d'instances présentées dans le
paragraphe~\ref{sec:instance_CECSP} et avec une limite de temps de 100
secondes. Le modèle, avec pour objectif la minimisation de la
consommation totale de ressource, est dans un premier temps utilisé
pour résoudre les instances. Dans un second temps, nous ajoutons les
inégalités issues du raisonnement énergétiques décrites dans le
paragraphe~\ref{sec:ER_TI}. Ces inégalités sont seulement ajoutées au
n\oe ud racine de l'arbre de recherche. En effet, dû à leur grand
nombre ($5*|\I|*n$, avec $\I$ l'ensemble des intervalles d'intérêt du
raisonnement énergétique) , il est difficile, sans algorithme de
séparation, de les inclure dynamiquement pendant la recherche. Le
tableau~\ref{tab:TI_CECSP} présente ces résultats.

Dans ce tableau, nous avons comparé la qualité et le temps d'obtention
des premières solutions pour chaque famille d'instances et pour les
cas avec et sans coupes énergétiques. La qualité de la solution est
mesurée de la manière suivante: 
\[
  \text{gap}=100*\frac{|\text{Objectif final} - \text{Objectif } 1^{ère} \text{  Sol}|}{\text{Objectif final}}
\]
Nous avons aussi comparé le temps nécessaire à l'obtention de la
solution optimale (100 secondes si la solution trouvée n'est pas
optimale), la qualité de la solution, le nombre d'instances résolues
et le nombre d'entre elles résolues à l'optimum.

\begin{table}[!htb]
  \begin{center}\small
    \begin{tabularx}{\linewidth}{|Y|YY|YYYY|YY|YYYY|}
      \hline
      \multirow{2}{*}{\#act.} & \multicolumn{2}{c|}{$1^{ère}$ sol.}&
      \multicolumn{4}{c|}{sol. finale} & \multicolumn{2}{c|}{$1^{ère}$ 
        sol.}&
      \multicolumn{4}{c|}{sol. finale} \\ 
      \cline{2-13} 
                               & tps(s) & gap & tps & gap  &  \%opt.&\%solv.&
                                                                 tps
                                                                 & gap
                                                                 &
                                                                 tps
                                                                 & gap
                                                                 &
                                                                 \%opt. &\%solv.  \\ 
      \hline
      \multicolumn{7}{|l|}{Famille 1} & \multicolumn{6}{|l|}{Famille 2}\\
      \hline
      \multicolumn{7}{|r|}{sans coupes énergétiques} & \multicolumn{6}{|r|}{sans coupes énergétiques}\\
      \hline  
      $10 $&$ 0,058 $&$ 38 $&$ 45 $&$ 2,9$ &$ 60 $&$ 100$ &$ 0,075 $&$ 24 $&$ 100 $&$ 5,6$ &$ 0 $&$ 100$ \\ 
      $20 $&$ 0,25 $&$ 17 $&$ 100 $&$ 8,4$ &$ 0 $&$ 100$ & $ 0,37 $&$ 8,8 $&$ 100 $&$ 6,5$ &$ 0 $&$ 100$\\ 
      $25 $&$ 0,88 $&$ 21 $&$ 100 $&$ 9,4$ &$ 0 $&$ 100$ &$ 2 $&$ 12 $&$ 100 $&$ 5,8$ &$ 0 $&$ 100$ \\ 
      $30 $&$ 1,3 $&$ 35 $&$ 100 $&$ 10$ &$ 0 $&$ 100$&$ 3,1 $&$ 14 $&$ 100 $&$ 6,3$ &$ 0 $&$ 100$  \\ 
      \hline 
      \multicolumn{7}{|r|}{avec coupes énergétiques à la racine} &  \multicolumn{6}{|r|}{avec coupes énergétiques à la racine}\\
      \hline
      $10 $&$ 0,064 $&$ 3,4 $&$ 60 $&$ 2,6$ &$ 40 $&$ 100$ &$ 0,074 $&$ 9 $&$ 100 $&$ 4,1$ &$ 0 $&$ 100$\\ 
      $20 $&$ 0,31 $&$ 18 $&$ 100 $&$ 8,7$ &$ 0 $&$ 100$& $ 0,65 $&$ 9,4 $&$ 100 $&$ 6,2$ &$ 0 $&$ 100$  \\ 
      $25 $&$ 0,63 $&$ 17 $&$ 100 $&$ 9,6$ &$ 0 $&$ 100$ &$ 1,9 $&$ 12 $&$ 100 $&$ 6$ &$ 0 $&$ 100$\\ 
      $30 $&$ 1,6 $&$ 18 $&$ 100 $&$ 10$ &$ 0 $&$ 100$ &$ 5,8 $&$ 11 $&$ 100 $&$ 6,3$ &$ 0 $&$ 100$ \\ 
      \hline   
      \multicolumn{7}{|l|}{Famille 3} &  \multicolumn{6}{|l|}{Famille 4}\\
      \hline
      \multicolumn{7}{|r|}{sans coupes énergétiques}&  \multicolumn{6}{|r|}{sans coupes énergétiques}\\
      \hline  
    $10 $&$ 0,064 $&$ 4,1 $&$ 64 $&$ 1,8$ &$ 40 $&$ 100$ &$ 0,037 $&$ 0 $&$ 0,037 $&$ 0$ &$ 100 $&$ 100$\\ 
$20 $&$ 2,8 $&$ 6,8 $&$ 100 $&$ 3,9$ &$ 0 $&$ 100$ &$ 0,61 $&$ 0,72 $&$ 0,63 $&$ 0$ &$ 100 $&$ 100$\\ 
$25 $&$ 17 $&$ 4,5 $&$ 100 $&$ 4,5$ &$ 0 $&$ 100$ &$ 1,5 $&$ 0,31 $&$ 1,5 $&$ 0$ &$ 100 $&$ 100$\\ 
$30 $&$ 24 $&$ 6,4 $&$ 100 $&$ 4,1$ &$ 0 $&$ 80$ &$ 4,8 $&$ 0,07 $&$ 15 $&$ 0$ &$ 89 $&$ 89$ \\ 
      \hline 
      \multicolumn{7}{|r|}{avec coupes énergétiques à la racine} & \multicolumn{6}{|r|}{avec coupes énergétiques à la racine}\\
      \hline
$10 $&$ 0,063 $&$ 3,4 $&$ 80 $&$ 1,8$ &$ 20 $&$ 100$&$ 0,045 $&$ 1,1 $&$ 0,046 $&$ 0$ &$ 100 $&$ 100$  \\ 
$20 $&$ 2,5 $&$ 6,7 $&$ 100 $&$ 3,8$ &$ 0 $&$ 100$ &$ 0,56 $&$ 0,64 $&$ 0,71 $&$ 0$ &$ 100 $&$ 100$  \\ 
$25 $&$ 18 $&$ 5,8 $&$ 100 $&$ 4,4$ &$ 0 $&$ 100$ &$ 2,6 $&$ 0,072 $&$ 2,9 $&$ 0$ &$ 100 $&$ 100$ \\ 
$30 $&$ 24 $&$ 7,6 $&$ 100 $&$ 4,2$ &$ 0 $&$ 70$ &$ 6,4 $&$ 0 $&$ 6,4 $&$ 0$ &$ 100 $&$ 100$\\ 
      \hline   
    \end{tabularx}
  \end{center}
  \caption{Résultats du PLNE indexé par le temps du \CECSP~avec et
    sans coupes énergétiques.} 
  \label{tab:TI_CECSP}
\end{table}

Les résultats présentés dans le tableau~\ref{tab:TI_CECSP} montrent
que, dans la majorité des cas, le modèle avec les coupes énergétiques
obtient une première solution de meilleure qualité que celle obtenue
sans ces coupes. De plus, le temps de calcul de cette première
solution, bien que légèrement plus élevé dans le cas des coupés
énergétiques, est du même ordre de grandeur dans les deux cas.

Cependant, sauf dans le cas des instances à 30 activités de la Famille
4 où l'ajout des coupes permet un gain en termes de performances, les
coupes énergétiques ne produisent pas de gain significatif, ni en
termes de qualité de solution, ni en termes de temps de résolution. 

Malgré tout, les performances relatives des coupes énergétiques pour
le calcul des premières solutions pousse à poursuivre cette
investigation. La mise en place en place d'un algorithme de séparation
permettant de trouver les meilleures coupes à ajouter à chaque n\oe ud
de l'arbre de recherche devient donc une direction de recherche
encourageante dans un futur proche. 

\subsection{Modèles à événements}

Ce paragraphe, dédié à la présentation des résultats obtenus pour les
modèles à événements, commence par présenter l'algorithme de
séparation qui sera utilisé dans le modèle On/Off pour séparer les
inégalités de non-préemption. Cet algorithme a été conçu en
collaboration avec Tam{\'a}s Kis. Nous présenterons ensuite les
différents résultats obtenus pour les modèles Start/End et On/Off.

\subsubsection{Algorithme de séparation pour les inégalités de non
  préemption} 

L'idée principale de la procédure de séparation pour les inégalités de
non préemption (voir paragraphe~\ref{sec:nonPreem} est que, pour
chaque activité $i$, trouver la coupe à ajouter au modèle est
équivalent à trouver un plus long chemin dans un certain graphe. Ce
graphe orienté et acyclique est défini de la manière suivante:
\begin{itemize}
\item à chaque événement de l'ensemble $\E^*$=$\E \setminus\{2n\}$,
i.e. $\{1,\dots,2n-1\}$, on fait correspondre un sommet;
\item on ajoute au graphe un source et un puits, indexés
respectivement par $0$ et $2n$. Le graphe est donc composé de $2n+1$
sommets.
\item L'ensemble des arcs est divisé en trois catégories:
  \begin{itemize}
  \item[(i)] les {\it arcs de départ} relient le sommet source $0$ à
chaque sommet $u \in \{1,\dots,2n-3\}$;
  \item[(ii)] les {\it arcs intermédiaires} relient les sommets $u \in
\{1,\dots, 2n-3\}$ aux sommets $v \in \{u+2,\dots,2n-1\}$ et
  \item[(iii)] les {\it arcs terminaux} relient les sommets $u \in
\{3,\dots,2n-1\}$ au sommet puits $2n$;
  \end{itemize}
\end{itemize}

De plus, un coût $cost(u,v)$ est associé à chaque arc $(u,v)$
composant le graphe. Le coût d'un arc de départ est $0$, celui d'un
arc intermédiaire $(u,v)$ est $cost(u,v)= \overline{z}_{i,u} - \min\{
\overline{z}_{i,\ell}\ :\ \ell= u+1,\dots, v-1\}$ et celui d'un arc
terminal $(u,2n)$ est $\overline{z}_{i,u}$, avec $\overline{z}_{i,u}$
la valeur de la variable correspondante. La construction de ce graphe
est illustrée dans l'exemple~\ref{ex:algo_sep}.

\begin{ex}
\label{ex:algo_sep}
Considérons une instance à quatre activités. Le nombre d'événements
$|\E|$ est égal à $8$. Considérons, par exemple, l'activité $1$. Si
nous appliquons la transformation décrite ci-dessus, nous obtenons le
graphe à $9$ sommets et $30$ arcs décrits par la
figure~\ref{fig:algo_sep}. Sur ce graphe, seuls les coûts des arcs
terminaux et des arcs intermédiaires sortant du sommet $2$ sont
représentés. 
\begin{figure}[!htb]
  \centering
  \begin{tikzpicture}
    [  every node/.style={},%
    dot/.style={circle,fill=black,minimum size=4pt,inner sep=0pt,%
      outer sep=-1pt},
    cross/.style={path picture={ 
        \draw
        (path picture bounding box.south east) -- (path picture
        bounding box.north west) (path picture bounding box.south
        west) -- (path picture bounding box.north east); 
      }}]
    \tikzstyle{sommet}=[draw,circle,minimum width=0.5cm]
    \node[sommet] (O) at (0,0) {$0$};
    \node[sommet] (U) at (6,1) {$1$}; 
    \node[sommet] (D) at (-6,1) {$2$}; 
    \node[sommet] (T) at (2,1) {$3$}; 
    \node[sommet] (Q) at (2,7) {$4$}; 
    \node[sommet] (C) at (-6,7) {$5$}; 
    \node[sommet] (Si) at (6,7) {$6$}; 
    \node[sommet] (Se) at (-2,7) {$7$};
    \node[sommet] (H) at (0,9) {$8$};
    
    \draw[->,>=latex,blue] (T) --  node[pos= 0.86,sloped, above] {$
      \overline{z}_{1,3}$} (H) ;
    \draw[->,>=latex,blue] (Q) -- node[sloped,above] {$
      \overline{z}_{1,4}$} (H) ; 
    \draw[->,>=latex,blue] (C) --  node[sloped,above] {$
      \overline{z}_{1,5}$}(H) ; 
    \draw[->,>=latex,blue] (Si) -- node[sloped,above]
    {$ \overline{z}_{1,6}$} (H)  ;
    \draw[->,>=latex,blue] (Se)  -- node[sloped,above] {$
      \overline{z}_{1,7}$} (H) ;
    
    \draw[->,>=latex,red] (O) -- (U);
    \draw[->,>=latex,red] (O) -- (D);
    \draw[->,>=latex,red] (O) -- (T);
    \draw[->,>=latex,red] (O) -- (Q);
    \draw[->,>=latex,red] (O) -- (C);
    
    \draw[->,>=latex] (U) -- (T) ;
    \draw[->,>=latex] (U) -- (Q) ;
    \draw[->,>=latex] (U) -- (C) ;
    \draw[->,>=latex] (U) -- (Si) ;
    \draw[->,>=latex] (U) -- (Se) ;
    
    \draw[->,>=latex] (T) -- (C) ;
    \draw[->,>=latex] (T) -- (Si) ;
    \draw[->,>=latex] (T) -- (Se) ;
    
    \draw[->,>=latex] (Q) -- (Si) ;
    \draw[->,>=latex] (Q) -- (Se) ;
    
    \draw[->,>=latex] (C) -- (Se) ;
    
    \draw[->,>=latex,green] (D) --node[fill=white,sloped,above]
    {$ \overline{z}_{1,2} - \overline{z}_{1,3}$} (Q) ;
    \draw[->,>=latex,green] (D) -- node[fill=white,sloped,above]
    {$ \overline{z}_{1,2} -
      \min\{\overline{z}_{1,3},\overline{z}_{1,4}\}$}(C) ; 
    \draw[->,>=latex,green] (D) -- node[fill=white,sloped,above]
    {$ \overline{z}_{1,2}-
      \min\{\overline{z}_{1,3},\overline{z}_{1,4},\overline{z}_{1,5}\}$}(Si) 
    ; 
    \draw[->,>=latex,green] (D) -- node[fill=white,sloped,above]
    {$
      \overline{z}_{1,2}-\min\{\overline{z}_{1,3},\overline{z}_{1,4},\overline{z}_{1,5},\overline{z}_{1,6}\}$}(Se)
    ; 
    
  \end{tikzpicture}
  \caption{Création du graphe de l'algorithme de séparation pour les
      inégalités de non préemption}
    \label{fig:algo_sep}
  \end{figure}
\end{ex}

Pour séparer un vecteur $\overline{z}_i \in \mathbb{R}^{\E^*}$, nous
calculons un plus long chemin dans le graphe à l'aide de la
programmation dynamique. Pour cela, nous calculons, pour chaque $v \in
\{1,\dots,2n-1\}$, la valeur  du plus long chemin jusqu'à ce somment
$F(v)$ de la manière suivante:
\begin{equation}
  F(v) = \max\{ F(u) + cost (u,v) \ : \ u=1,\dots,v-2 \}
\end{equation}
avec $F(1)=F(2)=0$. 

Puis, on ajoute à ce plus long chemin, la valeur de l'arc servant à
rejoindre le puits. Cela revient à ajouter $\overline{z}_{i,v}$ à
chaque $F(v)$. Si cette valeur est supérieure à celle du
plus long chemin trouvé jusqu'à maintenant, nous remplaçons le plus
long chemin et sa valeur par le nouveau chemin trouvé.

\begin{ex}
Considérons le vecteur $\overline{z}_1$ suivant: $(0,2\ ;\ 0\ ;\
0,4\ ;\ 0,6\ ;\ 0,6\ ;\ 0\ ;\ 1\ ;\ 0,2)$.

Nous avons $F(1)=F(2)=0$. En effet, l'inégalité devant comporter au moins
trois éléments, $F(1)$ et $F(2)$ ne pourront conduire à une telle
inégalité. La longueur du plus long chemin est initialisée à $0$ et
correspond au chemin vide. De plus,nous avons: 
\begin{itemize}
\item $F(3) =  F(1) + cost (1,3) = F(1) + \overline{z}_{1,1} -
  \overline{z}_{1,2} = 0 + 0,2 - 0= 0,2 $. 
  
  Si nous calculons $F(3)+\overline{z}_{1,3}=0,2+0,4=0,6$. La
  longueur du plus long chemin est donc mise à jour et vaut
  $0,6$. Cette valeur correspond au chemin $\{1,2,3\}$.
\item $\begin{aligned}[t] 
    F(4) &=  \max \left\{
        \begin{array}{lcl}
          F(1) + cost (1,4) & = & F(1) + \overline{z}_{1,1}  -
                              \min\{\overline{z}_{1,2},\overline{z}_{1,3}\}  \\
          F(2) + cost (2,4) &= & F(2) + \overline{z}_{1,2} -
                              \overline{z}_{1,3}     
        \end{array} \right.\\
    &=  \max \left\{ 
        \begin{array}{lcl}
          0 + 0,2 - 0 &=& 0,2\\
          0 + 0 - 0,4 &= &-0,4 
        \end{array} \right.\\
    &=  0,2
  \end{aligned}$

  Et $F(4)+\overline{z}_{1,4}=0,2+0,6=0,8$. La
  longueur du plus long chemin est donc mise à jour et vaut
  $0,8$. Cette valeur correspond au chemin $\{1,2,4\}$.

\item $\begin{aligned}[t] 
    F(5) &=  \max \left\{
        \begin{array}{lcl}
          F(1) + cost (1,5) & = & F(1) + \overline{z}_{1,1}  -
                                  \min\{\overline{z}_{1,2},\overline{z}_{1,3},\overline{z}_{1,4}\} 
          \\ 
          F(2) + cost (2,5) &= & F(2) + \overline{z}_{1,2} -
                             \min\{\overline{z}_{1,3}, \overline{z}_{1,4}\}\\
          F(3) + cost (3,5) &= & F(3) + \overline{z}_{1,3} -
                             \overline{z}_{1,4} \\
        \end{array} \right.\\
      &=  \max \left\{ 
        \begin{array}{lcl}
          0 + 0,2 - 0 &=& 0,2 \\
          0 + 0 - 0,4 &= &-0,4 \\
          0,2 + 0.4 - 0,6 &= & 0 \\ 
        \end{array} \right.\\
    &=  0,2
  \end{aligned}$

  Si nous calculons $F(5)+\overline{z}_{1,5}=0,2+0,6=0,8$. Le plus
  long chemin n'est donc pas mis à jour. 

\item $F(6)= 0.2$ et $F(6)+\overline{z}_{1,6}=0,2+0=0,2$. Le plus long  
  chemin n'es pas mis à jour. 
\item $F(7) = F(4) +\overline{z}_{1,4} -
  \min\{\overline{z}_{1,5},\overline{z}_{1,6}\}= 0,2 + 0,6 - 0 =
  0,8$. De plus, $F(7)+\overline{z}_{1,7}=0,8+1=1,8$. La longueur du
  plus long est donc mise à jour, vaut $1,8$ et correspond au chemin $\{1,2,4,6,7\}$
\end{itemize}

{\'A} la fin de la procédure, le plus long chemin trouvé est donc
$\{1,2,4,6,7\}$ et l'inégalité correspondante, i.e. celle qui sera
ajoutée au modèle, est:
\[  z_{1,1} - z_{1,2} + z_{1,4} - z_{1,6} + z_{1,7} \le 1  \]
\end{ex}  

\subsubsection{Modèles Start/End}
\textcolor{red}{\LARGE pour ce paragraphe => que le texte}
Dans ce paragraphe, nous présentons les résultats obtenus pour le
modèle Start/End.  Les expérimentations ont été conduites
sur les instances de la Famille 1 présentée dans le
paragraphe~\ref{sec:instance_CECSP} et avec une limite de temps de
1000 secondes. Le modèle est utilisé avec l'objectif de minimisation de la
consommation totale de ressource. Le
tableau~\ref{tab:SE_CECSP} présente ces résultats.

Dans ce tableau, nous avons comparé la qualité et le temps d'obtention
des premières solutions pour chaque famille d'instances testées. Nous
avons aussi comparé le temps nécessaire à l'obtention de la 
solution optimale (7200 secondes si la solution trouvée n'est pas
optimale), la qualité de la solution, le nombre d'instances résolues
et le nombre d'entre elles résolues à l'optimum.

\begin{table}[!htb]
  \begin{center}\small
    \begin{tabularx}{\linewidth}{|Y|YY|YYYY|}
      \hline
      \multirow{2}{*}{\#act.} & \multicolumn{2}{c|}{$1^{ère}$ sol.}&
      \multicolumn{4}{c|}{sol. finale}\\
       	\cline{2-7} 
 & time(s) & gap & time & gap &\%solved &  \%opt \\ 
 \hline 
$10$ & $0,42$ & $95$ & $5200$ & $42$ & $100$ & $33 $\\ 
$20$ & $73$ & $80$ & $7200$ & $29$ & $100$ & $0 $\\ 
$25$ & $330$ & $94$ & $7200$ & $58$ & $100$ & $0 $\\ 
$30$ & $200$ & $82$ & $6500$ & $48$ & $100$ & $0 $\\ 
$60$ & $310$ & $140$ & $7200$ & $63$ & $100$ & $0 $\\ 
\hline 
    \end{tabularx}
  \end{center}
  \caption{Résultats du PLNE indexé par le temps du \CECSP~avec et
    sans coupes énergétiques.} 
  \label{tab:SE_CECSP}
\end{table}

Dans le tableau~\ref{tab:SE_CECSP}, nous pouvons voir que le modèle
Start/End permet de résoudre seulement une petite partie des instances
à l'optimum. Une solution est cependant trouvée pour toutes les
instances mais la qualité de ces dernières ne sont pas très bonne. De
meilleures solutions, obtenues avec le modèle On/Off, sont présentées
dans le paragraphe suivante. 

\subsubsection{Modèles On/Off}

\paragraph{Résultats du modèle On/Off pour le \CECSP}
Dans ce paragraphe, nous présentons les résultats obtenus pour le
modèle On/Off. Les expérimentations conduites pour ce modèle dans le
cadre du \CECSP~l'ont été
sur les instances de la Famille 4 (cf. tableau~\ref{tab:OO_f4}) et sur les
instances de la Famille 1 (cf. tableau~\ref{tab:OO_f1}) avec une
limite de temps de 1000 secondes.  

Pour la Famille 4, le tableau~\ref{tab:OO_f4} présente le pourcentage
d'instances résolues à l'optimum ainsi que le temps nécessaire à leur 
résolution pour différentes combinaisons de coupes ajoutées au
modèle. Dans le tableau, ces différentes inégalités sont représentées
de la manière suivantes: 
\begin{itemize}
\item {\it Sep.} ou {\it S.} représente les inégalités
 bornant supérieurement la distance entre deux événements;
\item {\it Date} ou {\it D.} les inégalités bornant supérieurement la
  date des événements;
\item {\it KP} les inégalités déduites du problème du sac-à-dos, et  
\item {\it $\overline{Preem.}$} ou {\it $\overline{P.}$} les
  inégalités de non-préemption.
\end{itemize}
Ces inégalités sont décrites dans le paragraphe~\ref{sec:maxDist}. 
 Les deux premiers ensembles d'inégalités sont ajoutés directement
 dans le modèle et sont aussi utilisées comme borne plus fine dans les
 contraintes du modèle (voir paragraphe~\ref{sec:maxDist}).


\begin{table}[!htb]
  \begin{center} 
    \begin{tabular}{|c|>{\centering\arraybackslash}p{1cm}>{\centering\arraybackslash}p{1cm}|>{\centering\arraybackslash}p{1cm}>{\centering\arraybackslash}p{1cm}|>{\centering\arraybackslash}p{1cm}>{\centering\arraybackslash}p{1cm}|>{\centering\arraybackslash}p{1cm}>{\centering\arraybackslash}p{1cm}|}
      \hline
      \multirow{2}{*}{\backslashbox{ineg.}{\#act.}}  &
                                                        \multicolumn{2}{c|}{10} & \multicolumn{2}{c|}{20} & \multicolumn{2}{c|}{25} & \multicolumn{2}{c|}{30}\\
      & tps(s) & \%opt & tps(s) & \%opt& tps(s) & \%opt& tps(s) & \%opt\\
      \hline
     $Aucune$ & $ 0,3 $&$ 100 $&$ 164,14 $&$ 90,9 $&$ 635,4 $&$ 55,6 $&$ 968 $&$ 10$\\
     $ Sep.$ &$ 0,6 $&$ 100 $&$ 182,2$&$ 90,9 $&$ 727,3 $&$ 55, 6 $&$ 851$&$ 20$ \\
     $ Date $&$ 0,5 $&$ 100 $&$ 167,3 $&$ 90,9 $&$ 629,5$&$ 88,9 $&$ 961,4$&$ 20 $\\
     $ KP$&$ 0,5 $&$ 100 $&$ 164,7 $&$ 90,9 $&$ 555,1$&$ 66,7 $&$ 845,3$&$ 20 $\\
     $ \overline{Preem.}$ &$ 0,3$&$ 100 $&$ 330,9 $&$ 72,7 $&$ 822,4 $&$ 44,4 $&$ 914,8$&$ 10$ \\
     %$ Sep\ \$&$\ Date $$&$ 100 $&$ $&$ 90,9 $&$ $&$ 66,7 $&$ $&$ 10 \\
     %$ Sep\ \$&$\ KP$$&$ 100 $&$ $&$ 81,8 $&$ $&$ 77,8 $&$ $&$ 20 \\
     %$ Sep\ \$&$\ \overline{Preem.}$ $&$ $&$ 100 $&$ $&$ 81,8 $&$ $&$ 33,3 $&$ $&$ 10 \\
     %$ Date\ \$&$\ KP$$&$ 100 $&$ $&$ 90,9$&$ 88,8 $&$ $&$ 10\\
     %$ Date\ \$&$\ \overline{Preem.}$ $&$100 $&$ $&$ 81,8 $&$ $&$ 66,7 $&$ $&$ 20 \\ 
     %$ KP\ \$&$\ \overline{Preem.}$ $&$ $&$ 100 $&$ $&$ 90,9 $&$ $&$ 33, 3 $&$ $&$ 10\\
     $ S.\ \&\ D.\ \&\ KP$&$ 0,6 $&$100 $&$ 154$&$ 90,9 $&$ 389,9 $&$ 77,8 $&$ 839,9$&$ 20$\\
     $ S.\ \&\ D.\ \&\ \overline{P.}$ &$ 0,6 $&$ 100 $&$ 179,8 $&$ 90,9 $&$
                                                                   454,4 $&$ 77,8 $&$ 813,8$&$ 60$ \\
     $ S.\ \&\ KP\ \&\ \overline{P.}$&$ 0,5 $&$100 $&$ 182,2$&$ 90,9 $&$759,6 $&$ 33,3 $&$ 924,9$&$ 20$\\
     $ D.\ \&\ KP\ \&\ \overline{P.} $&$ 0,5$&$ 100 $&$ 170$&$ 90,9 $&$ 705$&$ 66,7 $&$ 816$&$ 50$\\
     $ S.\ \&\ D.\ \&\ KP\ \&\ \overline{P.}$&$ 0,9 $&$ 100 $&$ 278,9 $&$ 81,8 $&$ 510$&$ 88,9 $&$ 802,8$&$ 50$\\
      \hline
    \end{tabular}
  \end{center}
  \caption{Résultats du modèle On/Off pour le \CECSP~avec différentes
    combinaisons de coupes (Famille 4).}
  \label{tab:OO_f4}
\end{table}

Dans le tableau~\ref{tab:OO_f4}, nous pouvons remarquer que le nombre
d'instances à $10$ et $20$ activités résolues est du même ordre de
grandeur pour toutes les combinaisons d'inégalités testées. Pour les
instances à $25$ activités, les résultats sont plus hétérogènes mais
les meilleurs résultats sont obtenus en combinant toutes les
inégalités. Enfin, pour les instances à $30$ activités, les meilleurs
résultats sont quant à eux obtenus en combinant seulement inégalités
de non-préemption et les deux ensembles d'inégalités portant sur les
dates des événements, i.e. {\it Sep} et {\it Date}. Cependant, les
résultats obtenus en combinant toutes les inégalités ne sont pas très
éloignés de ceux obtenant les meilleurs résultats.

Pour les instances de la Famille 1, seulement un petit nombre
d'instances sont résolues de manière optimale. C'est pourquoi nous
présentons seulement le pourcentage d'instances pour lesquelles une
solution a été trouvée et nous calculons la distance entre cette
solution et la meilleure borne inférieure trouvée à la fin du temps
imparti. Ces résultats sont présentés dans le
tableau~\ref{tab:OO_f1}. Les notations sont les mêmes que celles
utilisées dans le tableau~\ref{tab:OO_f4}.

\begin{table}
  \begin{center} 
    \begin{tabular}{|c|>{\centering\arraybackslash}p{1cm}>{\centering\arraybackslash}p{1.2cm}|>{\centering\arraybackslash}p{1cm}>{\centering\arraybackslash}p{1cm}|>{\centering\arraybackslash}p{1cm}>{\centering\arraybackslash}p{1cm}|>{\centering\arraybackslash}p{1cm}>{\centering\arraybackslash}p{1cm}|}
      \hline
      \multirow{2}{*}{\backslashbox{ineg.}{\#act.}}  &
                                                        \multicolumn{2}{c|}{10} & \multicolumn{2}{c|}{20} & \multicolumn{2}{c|}{25} & \multicolumn{2}{c|}{30}\\
      & \%feas. & gap &\%feas. & gap& \%feas. & gap& \%feas. & gap\\
      \hline
      $ Aucune$ &$ 100 $&$ 23,2 $&$ 81,8 $&$ 74,5 $&$ 22,2 $&$ 91,9 $&$ 0 $&$ 0$\\
      $ Sep$ &$ 100 $&$ <0,01 $&$ 100 $&$ 14,9 $&$ 44,4 $&$ 68,9 $&$ 0 $&$ 0$\\
      $ Date $ &$100$&$ 1,06 $&$ 100 $&$ 70,5 $&$ 33,3$&$ 88,1 $&$ 40$&$ 84,9 $\\
      $ KP$ &$ 100 $&$ 9,8 $&$ 100 $&$ 46,4 $&$ 77,78 $&$ 64,6 $&$ 25$&$ 70,7 $\\
      $ \overline{Preem.}$ &$ 100 $&$ 61,5 $&$ 81,8 $&$ 73,4 $&$ 0 $&$ 0 $&$ 0 $&$ 0$\\
      $ S.\ \&\ D.\ \&\ KP$&$ 100 $&$<0,01 $&$ 100$&$ 13,7 $&$ 100 $&$ 51,5 $&$ 50$&$ 70,6$\\
      $ S.\ \&\ D.\ \&\ \overline{P.}$ &$ 100 $&$ 4,77 $&$ 100 $&$ 45,5 $&$
                                                                   88,89 $&$ 75,1 $&$ 30$&$ 84,6 $\\
      $ S.\ \&\ KP\ \&\ \overline{P.}$ &$ 100 $&$ 8,67 $&$ 100 $&$ 68,3 $&$
                                                                   44,4
                $&$ 88,1 $&$ 10 $&$ 93$\\
     $ D.\ \&\ KP\ \&\ \overline{P.}$ &$ 100$&$ 23,8 $&$ 100$&$ 60,1 $&$ 83,3$&$ 79,8 $&$ 30$&$ 88,2$\\
     $ S.\ \&\ D.\ \&\ KP\ \&\ \overline{P.}$ &$ 100 $&$ 0,07 $&$ 100 $&$ 35,7 $&$ 100$&$ 66,9 $&$ 10$&$ 87,1$\\
      \hline
    \end{tabular}
  \end{center}
  \caption{Résultats du modèle On/Off pour le \CECSP~avec différentes
    combinaisons de coupes (Famille 1).}
  \label{tab:OO_f1}
\end{table}

Dans le tableau~\ref{tab:OO_f4}, nous pouvons voir que les meilleurs
résultats sont obtenus en combinant {\it Sep, Date} et {\it
KP}. Cependant, des résultats comparables sont obtenus en combinant
{\it Sep, Date} et {\it $\overline{Preem.}$} ou {\it Sep, Date, KP} et
{\it $\overline{Preem.}$}.


\paragraph{Résultats du modèle On/Off pour le \RCPSP}

Nous allons maintenant évaluer les performances relatives des
améliorations du modèle On/Off dans le cadre du \RCPSP. Pour ce faire,
les expérimentations ont été conduites sur les instances de Kone {\it
et al}~\cite{modele_RCPSP} et décrites dans le
paragraphe~\ref{sec:instances_RCPSP} avec un pré-calcul des fenêtres de
temps des activités aussi décrit dans le
paragraphe~\ref{sec:instances_RCPSP}. La limite de temps est fixée à
1000 secondes.

Pour ces instances, le tableau~\ref{tab:OO_PSP} présente le
pourcentage d'instances résolues à l'optimum ainsi que le temps
nécessaire à leur résolution pour différentes combinaisons de coupes
ajoutées au modèle. 

\begin{table}[!htb]
 \begin{center}
   \begin{tabular}{|c|cc|ccc|}
     \hline
       \multirow{2}{*}{\backslashbox{ineg.}{\#act.}} & \multicolumn{2}{c|}{$1^{ère}$ sol.}& \multicolumn{3}{c|}{Sol. finale}\\ 
	\cline{2-6}
     & time(s) & gap & time & gap &\%opt  \\ 
 \hline 
     $Aucune$ &$0,19$& $3,4 $&$ 34,8$ &$ 0 $& $100$\\
     $Sep.$ & $0,17 $& $3,2 $&$ 30,3$ &$ 0 $& $100$ \\ 
     $\overline{Preem.}$ & $0,17 $&$ 3,8$& $30,3$ & $0 $& $100 $ \\ 
     $Sep.\ \&\ \overline{Preem.}$ & $0,29$& $3,3$ & $ 33,1 $& $0$ & $100$ \\ 
\hline 
\end{tabular}
\end{center}
  \caption{Résultats du modèle On/Off pour le \RCPSP~avec différentes
    combinaisons de coupes.}
  \label{tab:OO_PSP}
\end{table}

Dans le tableau~\ref{tab:OO_PSP}, nous pouvons voir que les résultats
de l'ajout des coupes et inégalités a moins d'impact dans le cadre du
\RCPSP. Cependant, une amélioration des performances du modèle,
spécialement pour l'ajout des inégalités {\it Sep.} ou {\it
  $\overline{Preem.}$}, peut être notée. Cette amélioration est
moindre dans le cas où les deux ensembles d'inégalités sont ajoutés
simultanément. 

\subsection{Comparaison des différentes approches}

Dans ce paragraphe, nous allons effectuer des comparaisons entre
les différentes méthodes décrites ci-dessus. Dans un premier temps et
pour montrer que dans certains cas, le modèle indexé par le temps peut
produire des solutions sous-optimales, nous comparons les résultats de
ce dernier avec ceux du modèle On/Off (sans ajout de coupes
particulières) sur la Famille d'instances 1.   

Le tableau~\ref{CECSPMIPOBJ} présente ces résultats. Pour chacune des
formulations, la première colonne décrit le temps passé dans la
résolution du PLNE. La seconde colonne représente le pourcentage
d'instances résolues à l'optimum et la troisième colonne le nombre
d'instances pour lesquelles une solution réalisable a été
trouvé. Enfin, la dernière colonne du tableau montre la différence
moyenne entre la valeur de l'objectif retourné par les deux
modèles. Par exemple, la première valeur de cette colonne nous dit
que, pour les instances à $20$ activités, la valeur de l'objectif
retourné par le modèle indexé par le temps est, en moyenne, $10\%$
plus élevée que celle retournée par le modèle à événements. 

\begin{table}[ht] \centering
  \begin{tabular}{|c|ccc|ccc|c|}
    \hline
    & \multicolumn{3}{c|}{Modèle On/Off} &  \multicolumn{3}{c|}{Modèle
                                           indexé par le temps} &\\
    \hline
    \#act.&tps(s)&\%optimal&\%feas.&tps(s)&\%optimal
                              &\%feas.&\%dev. obj. \\
    \hline
    20 &7200 &0 &100 &7200 &0 &100 & 10,08\\
    25 &7200 &0 &44,44 &7200 &0 &66,67 & 10,11\\
    \hline
  \end{tabular}
  \caption{Comparaison du modèle On/Off et du modèle indexé par le
    temps pour le \CECSP.}
  \label{CECSPMIPOBJ}
\end{table}

Tout d'abord, nous pouvons remarquer que le modèle indexé par le temps
permet de trouver une solution réalisable pour plus d'instances que le
modèle On/Off. Cependant, aucun des deux modèles n'est capable de
résoudre les instances de manière optimale. Enfin, le modèle On/off
trouve clairement de meilleures solutions que le modèle indexé par le
temps ce qui tend à montrer que la considération du temps continu est
profitable en pratique pour économiser les ressources utilisées.

Les modèles présentés dans ce manuscrit ont de grandes difficultés à
trouver des solutions optimales. En particulier lorsque des fonctions
de rendement, même seulement affines, entrent en jeu. De plus, les
méthodes testées dans le paragraphe suivant, portant sur les techniques
issues de la programmation par contraintes, ne le sont que pour la
variante décisionnelle du \CECSP, i.e. sans objectif. De ce fait, nous
allons présenter les résultats des trois différents modèles dans le
cas où aucun objectif n'est présent. Le tableau~\ref{MIPresult}
présente ces résultats. 

Pour chacun des modèles, trois colonnes exposent les résultats. La
première correspond au temps nécessaire pour trouver une solution (si
une solution est trouvée). La seconde correspond au pourcentage
d'instance résolue et la dernière colonne correspond au nombre
d'instances prouvées réalisables. Cette dernière colonne est introduite
pour permettre de montrer que le modèle indexé par le temps peut
prouver l'infaisabilité d'une instance alors que les modèles a
événements trouvent une solution pour cette même instance prouvant
qu'elle est en fait réalisable. De ce fait, le modèle indexé par le
temps sur-contraint bien de manière significative le problème initial.

\begin{table}[ht] \centering
  \begin{tabular}{|c|ccc|ccc|ccc|}
\hline & \multicolumn{3}{c}{Indexé par le tps} & \multicolumn{3}{|c}{On/off} & \multicolumn{3}{|c|}{Start/End}\\ \hline
\#act. & tps (s) & \%solv. &
\%feas. & tps(s) & \%solv. &
\%feas. & tps(s) & \%solv. &
\%feas. \\ \hline 
    \multicolumn{10}{|c|}{Famille 1}\\
    \hline
   $ 10	$&$	0.03	$&$	100	$& $60$& $0,22$ & $100$ & $100$&$	0,71	$&$	100	$&$	100$	\\
   $ 20	$&$	0,08	$&$	100	$&$	54,5	$& $11,57$ & $100$ & $100$&$	355,4	$&$	100	$&$	100$	\\
    $25	$&$	0,22	$&$	100	$&$	66,7	$&$58,79$ & $100$ & $100	$&$	2226,7	$&$	77,8	$&$	77,8$	\\
    $30	$&$	0,25	$&$	100	$&$	60	$& $1582,9$ &
                                                                     $80$ & $80 $&$	6247,2	$&$	20	$&$	20$	\\
    $60	$&$	420	$&$	100	$&$	100	$&$4969,51$ & $80$ & $80	$&$	6219,8	$&$	20	$&$	20$	\\
    \hline 
    \multicolumn{10}{|c|}{Famille 3}\\
    \hline
   $ 10	$&$	0,04	$&$	100	$&$	100	$&$0,46$ & $100$ & $100$&$0,88	$&$	100	$&$	100$	\\
   $ 20	$&$	0,24	$&$	100	$&$	100	$&$658,362$ & $100$ & $100$&$	1430,6	$&$	90,9	$&$	90,9$	\\
   $ 25	$&$	0,45	$&$	100	$&$	88,9	$&$	1900,17$ & $77,78$ & $77,78	$&$	5816,1	$&$	33,3	$&$	33,3$	\\
   $ 30	$&$	1,38	$&$	100	$&$	100	$&$	2819,7$ & $80$ & $80$&$	6011,8	$&$	20	$&$	20$	\\
  $  60	$&$	328,7	$&$	100	$&$	80	$&$6358,15$ & $20$ & $20$&$	7200	$&$	0	$&$	0$	\\
    \hline 
    \multicolumn{10}{|c|}{Famille 4}\\
    \hline
   $ 10	$&$	0,01	$&$	100	$&$	100	$&$	0,23	$&$	100	$&$	100	$&$	0,7	$&$	100	$&$	100$	\\
  $  20	$&$	0,27	$&$	100	$&$	100	$&$	734,04	$&$	90,9	$&$	90,9	$&$	2995,9	$&$	63,6	$&$	63,6$	\\
   $ 25	$&$	0,7	$&$	100	$&$	100	$&$	2102,85	$&$	77,8	$&$	77,8	$&$	4833,6	$&$	44,4	$&$	44,4$	\\
   $ 30	$&$	1,57	$&$	100	$&$	100	$&$	4483,4	$&$	60	$&$	60	$&$	6485,9	$&$	20	$&$	20$	\\
   $ 60	$&$	224,72	$&$	100	$&$	60	$&$	7200	$&$	0	$&$	0	$&$	7200	$&$	0	$&$	0$	\\
    \hline
  \end{tabular}
  \caption{Comparaison des trois modèles de PLNE du \CECSP~sans
    fonction objectif.}
  \label{MIPresult}
\end{table} 

Comme nous pouvons le voir dans le tableau~\ref{MIPresult}, le
modèle indexé permet de trouver des solutions beaucoup plus rapidement
que les modèles à événements. Cependant, ce modèle ne peut être
employé comme une méthode de résolution exacte  puisque, pour
certaines instances, seules des solutions fractionnaires existent.
Ces résultats montrent ainsi encore que la résolution en temps continu
peut apporter des gains significatifs en pratique. 

Parmi les deux formulations basées sur les événements, la formulation
On/Off est celle qui fournit les meilleurs résultats en termes de
temps de calcul et de nombre d'instances résolues. 

\section{Performances de la Programmation Par Contraintes}
\label{sec:expe_PPC}
Dans ce paragraphe, nous présentons les résultats des expérimentations
portant sur les méthodes présentées dans le
chapitre~\ref{sec:PPC_CECSP}. Dans un premier temps, nous définissons
le cadre des expérimentations, notamment les algorithmes et
heuristiques de choix de variables utilisées. Puis, dans un second
temps, nous présenterons en détails les résultats numériques issus de
ces expérimentations. 

\subsection{Cadre des expérimentations}
\label{sec:hybridBB}
Pour mesurer les performances relatives des différents raisonnements
présentés dans le chapitre~\ref{sec:PPC_CECSP}, nous les intégrons
dans une procédure de branchement
hybride~\cite{Nattaf_Constraints,Nattaf_ORSpectrum}. Cette procédure
se divise en deux temps. Dans un premier temps, une méthode 
arborescente est utilisée afin de réduire la taille des domaines des
début et fin des activités jusqu'à ce que chaque domaine ait une
taille inférieure à un certain paramètre $\epsilon >0$. Une fois les
domaines réduits suffisamment, nous utilisons le modèle à événement
On/Off afin de fixer les dates de début et de fin et de calculer, pour
chaque activité $i$, sa fonction d'allocation de ressource
$b_i(t)$. Nous rappelons que, pour ce faire, seule la valeur de
$b_i(t)$ à chaque début et fin d'activité doit être calculée
(conséquence du théorème~\ref{theo_LPM_CECSP}).

La procédure de branchement est inspirée du travail de Carlier {\it et
  al.}~\cite{Carlier}. Au début de cette procédure, une activité peut
commencer (respectivement finir) à tout instant $t \in [\ES,\LS{]}$
(resp. $t \in [\EE,\LE{]}$). L'idée de l'algorithme est donc, à chaque
n\oe ud, de réduire la taille d'un de ces intervalles. Par exemple,
supposons que l'on ait choisi de réduire la taille du domaine de
$st_i$, alors deux nouveaux n\oe uds sont créés: le premier avec la
contrainte supplémentaire $st_i \in [\ES , (\ES+\LS) / 2]$ et le
second avec la contrainte $st_i \in ](\ES+\LS) / 2,\LS{]}$ (voir
figure~\ref{fig:branching}); et à chaque n\oe ud, un des raisonnements
présentés au chapitre~\ref{sec:PPC_CECSP} est appliqué.

\begin{figure}[!htb] 
  \centering
\begin{tikzpicture}
[yscale=0.6,xscale=0.8]
\tikzstyle{every node} = [align=center]
\node[ellipse,draw,text width=1.7cm] (R) at (0,0) {Problème $P$\\ $st_i \in [2,6]$};

\node[ellipse,draw,text width=1.7cm] (C1) at (-4,-4) {Problème $P1$\\ $st_i \in [2,4]$};
\node[ellipse,draw,text width=1.7cm] (C2) at (4,-4) {Problème $P2$\\ $st_i \in ]4,6]$};

\draw[->] (R.south west) -- (C1.north east);
\draw[->] (R.south east) -- (C2.north west);
\end{tikzpicture}

  \caption{Procédure de branchement du l'algorithme hybride du \CECSP.}
  \label{fig:branching}
\end{figure}

Cette procédure est répétée jusqu'à ce que tous les domaines de toutes
les variables soient plus petit qu'un certain paramètre $\epsilon
>0$. Quand ceci arrive, cela veut dire que nous sommes au niveau d'une
feuille de notre arbre de recherche et nous pouvons évaluer la
satisfiabilité du n\oe ud courant. Pour cela, nous utilisons le modèle
On/Off, le plus efficace dans le cadre d'une résolution exacte, pour
tester si une solution avec les contraintes supplémentaires définies
le long de la procédure de branchement existe.

Si une telle solution existe alors l'algorithme s'arrête et nous avons
trouvé une solution pour l'instance du \CECSP~testée. Sinon, nous
faisons marche arrière dans l'arbre de recherche afin d'évaluer
d'autres feuilles de l'arbre, i.e. d'autres solutions potentielles. 

Le parcours de l'arbre est fait en suivant un parcours en profondeur
et, à chaque n\oe ud, la variable dont on va réduire le domaine est
choisie selon l'heuristique choisissant la variable de plus petit
domaine.  De plus, la taille moyenne des domaines des variables étant
de $32$, les valeurs de $\epsilon$ testées sont $10$, $5$, $2.5$.

\subsection{Raisonnement énergétique}
\label{sec:expe_RE}

Dans un premier temps, nous allons commencer par comparer les
différentes méthodes de calcul des intervalles d'intérêt pour
l'algorithme de vérification de ce raisonnement, présentées dans le
paragraphe~\ref{sec:intervalle_CECSP}. 

\subsubsection{Comparaison des méthodes de calcul des intervalles
  d'intérêt pour l'algorithme de vérification du raisonnement
  énergétique}

Les résultats comparant les trois méthodes de calcul des intervalles
d'intérêt de l'algorithme de vérification du raisonnement énergétique
sont présentés dans le tableau~\ref{tab:intervalle_CECSP}. La première
colonne correspond à l'algorithme naïf de calcul d'intersection des
segments de définition des fonctions de consommation individuelle des
activités. La seconde colonne correspond quant à elle au calcul de ces
mêmes intersections à l'aide de l'algorithme de balayage de
Bentley-Ottmann. Enfin, la troisième colonne présente les résultats de
l'adaptation de la méthode de calcul de Derrien {\it et al.}~\cite{DP}
pour le problème cumulatif dans le cadre du \CECSP. Toutes ces
méthodes sont décrites dans le
paragraphe~\ref{sec:intervalle_CECSP}.

L'algorithme de balayage fait partie de la librairie C++  CGAL
\footnote{\textsc{Cgal}, {C}omputational {G}eometry {A}lgorithms
  {L}ibrary. http://www.cgal.org.}. Pour calculer ces performances,
nous appliquons l'algorithme de vérification du raisonnement
énergétique et les ajustements correspondant sur les intervalles de
l'algorithme de vérification seulement. Cet algorithme est appliqué
sur toutes les instances des Familles 1, 2, 3 et 4. Le temps est
représenté en millisecondes.

\begin{table}[ht] \centering
  \begin{tabular}{|>{\centering\arraybackslash}m{1.5cm}|>{\centering\arraybackslash}m{4cm}>{\centering\arraybackslash}m{4cm}>{\centering\arraybackslash}m{4cm}|}
    \hline \# tasks & méthode naïve & algorithme de balayage & adaptation
                                                               de
                                                               l'algorithme
                                                               de~\cite{DP}\\
    \hline 10 & 0,46 & 1,57 & 0,39 \\ 20 & 3,9 & 6,2 & 1,05 \\ 25 &
                                                                    7,18 & 7,50 & 1,73 \\ 30 & 11,54 & 11,78 & 3,06 \\ 60 & 45,97 &
                                                                                                                                    62,82 & 14,40 \\
    \hline
  \end{tabular}
  \caption{Comparaison des méthodes de calcul des intervalles
    d'intéret du raiosnnement énergétique.}
  \label{tab:intervalle_CECSP}
\end{table} 

La meilleure méthode permettant le calcul des intervalles d'intérêt de
l'algorithme de vérification du raisonnement énergétique est la
méthode adaptée de~\cite{DP}. En effet, un des avantages de cette
méthode était que le nombre d'intervalles à calculer était beaucoup
plus faible que dans les autres cas. Il paraît donc cohérent que cet
algorithme ait les meilleures performances. 

Cependant, il est moins naturel que l'algorithme par balayage ait de
moins bonnes performances que la méthode naïve. Ceci est dû au fait
que la complexité de l'algorithme de balayage dépend du nombre
d'intersections que l'algorithme doit calculer. Or, dans ce cas, ce
nombre est très grand, ce qui ralentit grandement le temps de calcul
de l'algorithme. 

Dans le reste des expérimentations conduites sur le raisonnement
énergétique, nous utiliserons donc la méthode de calcul adaptée de~\cite{DP}. 

\subsubsection{Intégration du raisonnement énergétique dans
  l'algorithme de branchement hybrides}

Dans ce paragraphe, nous présentons les résultats des expérimentations
faites pour le raisonnement énergétique. Ce raisonnement est intégré
dans l'algorithme de branchement hybride décrit dans le
paragraphe~\ref{sec:hybridBB}. Cet algorithme a été testé avec
l'heuristique de sélection de variable qui choisit la variable de plus
petit domaine.

Le tableau~\ref{tab:res_BB_ER} présente les résultats obtenus avec 
pour valeur de $\epsilon=2,5$ . Cette valeur est choisie car c'est celle
qui permettait d'obtenir les meilleurs résultats. Les autres valeurs
de paramètres testées étant: $10$ et $5$. Dans ce tableau, la première
colonne décrit le temps nécessaire pour résoudre les instances. Les
deuxième et troisième colonnes montrent le pourcentage de temps passé
dans la résolution du PLNE et dans l'arbre de recherche
respectivement. La quatrième colonne énonce le pourcentage d'instances
résolues. Les cinquième et sixièmes colonnes décrivent respectivement
 le nombre de n\oe ud contenus dans l'arbre de recherche et de programme
linéaire en nombres entiers résolus. Enfin, la dernière colonne
exhibe le nombre d'ajustements appliqués. 

\begin{table}[!htb]
  \begin{center}
    \begin{tabular}{|c|M{1.8cm}M{1.8cm}M{1.8cm}cccc|}
      \hline
      \#act. &  \multicolumn{7}{c|}{Méthode de branchement hybride}\\ 
             &  \multicolumn{7}{c|}{$\epsilon=2.5$} \\ 
      \hline 
      \multicolumn{8}{|l|}{Famille 1 }\\ 
      \hline 
             & Tps total(s) & Tps CPLEX(\%) & Tps arbre(\%) & \%solv.  & \#n\oe ud & \#PLNE  & \#adj. \\ 
      \hline 
$10$ & $0,09$ & $83,19$ & $16,81$ & $100$ & $25$ & $6$ & $6$ \\ 
$20$ & $660,21$ & $96,97$ & $3,03$ & $90$ & $3809$ & $20611$ & $2716$ \\ 
$25$ & $821,9$ & $98,3$ & $1,7$ & $88$ & $89$ & $10$ &  $95$ \\ 
$30$ & $112,58$ & $99,14$ & $0,86$ & $100$ & $102$ & $10$ &  $114$ \\ 
\hline 	
      \multicolumn{8}{|l|}{Famille 2 }\\ 
      \hline 
             & Tps total(s) & Tps CPLEX(\%) & Tps arbre(\%) & \%solv.  & \#n\oe ud & \#PLNE& \#adj. \\ 
      \hline 
$10$ & $1109,5$ & $81,2$ & $18,75$ & $100$ & $46783$ & $140276$ & $65446$ \\ 
$20$ & $728,04$ & $97,55$ & $2,45$ & $90$ & $1930$ & $9296$  & $3851$ \\ 
$25$ & $1825,15$ & $98,7$ & $1,3$ & $75$ & $495$ & $761$ &  $458$ \\ 
$30$ & $1971,81$ & $99,62$ & $0,38$ & $75$ & $150$ & $5$ &  $659$ \\ 
      \hline 	
      \multicolumn{8}{|l|}{Famille 4 }\\ 
      \hline 
             & Tps total(s) & Tps CPLEX(\%) & Tps arbre(\%) & \%solv.  & \#n\oe ud & \#PLNE  & \#adj. \\ 
      \hline 
$10$ & $0,19$ & $87,55$ & $12,45$ & $100$ & $25$ & $5$ &  $33$ \\ 
$20$ & $1617,07$ & $98,97$ & $1,03$ & $77$ & $68$ & $9$ & $246$ \\ 
$25$ & $104,9$ & $99,6$ & $0,4$ & $100$ & $66$ & $6$  & $252$ \\ 
$30$ & $1749,76$ & $99,72$ & $0,27$ & $77$ & $107$ & $9$ &  $424$ \\ 
      \hline 
    \end{tabular}
  \end{center}
  \caption{Résultats du raisonnement énergétique dans la méthode de
    branchement hybride pour le \CECSP.}
  \label{tab:res_BB_ER}
\end{table}

Tout d'abord, nous pouvons remarquer que l'algorithme de branchement
hybride permet de résoudre des instances jusqu'à $30$ activités en
moins de $7200$ secondes. De plus, nous pouvons voir qu'une bonne
partie du temps de résolution est passé dan s le programme linéaire
mixte. Enfin, le raisonnement énergétique permet de procéder à un
nombre important d'ajustements. 

\subsection{Raisonnements basés sur le Time-Table}
\label{sec:expe_TT}

Dans ce paragraphe, nous montrons l'intérêt d'ajouter le
raisonnement basé sur un problème de flot et sur le raisonnement
Time-Table. Pour ce faire, à chaque n\oe ud de l'arbre de recherche,
nous appliquons l'algorithme de Time-Table/Flot pour détecter une
incohérence. 

Pour ces expérimentations, nous avons aussi considéré l'heuristique
choisissant la variable de plus petit domaine. Le
tableau~\ref{tab:res_BB_ERFlot} présente les résultats obtenus avec
pour valeur de $\epsilon=2,5$. Les colonnes du tableau correspondent
à celle du tableau~\ref{tab:res_BB_ER}.

\begin{table}[!htb]
  \begin{center}
    \begin{tabular}{|c|M{1.8cm}M{1.8cm}M{1.8cm}ccc|}
      \hline
      \#act. &  \multicolumn{6}{c|}{Méthode de branchement hybride}\\ 
             &  \multicolumn{6}{c|}{$\epsilon=2.5$} \\ 
      \hline 
      \multicolumn{7}{|l|}{Famille 1 }\\ 
      \hline 
             & Tps total(s) & Tps CPLEX(\%) & Tps arbre(\%) & \%solv.  & \#n\oe ud & \#PLNE \\ 
      \hline 
$10$ & $0,12$ & $51,5$ & $48,5$ & $100$ & $25$ & $6$ \\ 
$20$ & $5,0$ & $88,7$ & $11,3$ & $100$ & $58$ & $12$ \\
$25$ & $26,6$ & $85,8$ & $14,2$ & $100$ & $79$ & $10$ \\
$30$ & $70,8$ & $96,69$ & $3,3$ & $100$ & $100$ & $11$ \\
\hline 	
      \multicolumn{7}{|l|}{Famille 2 }\\ 
      \hline 
             & Tps total(s) & Tps CPLEX(\%) & Tps arbre(\%) & \%solv.  & \#n\oe ud & \#PLNE\\
      \hline 
$10$ & $0,17$ & $63,0$ & $37,0$ & $100$ & $24$ & $6$ \\
$20$ & $9,03$ & $86,5$ & $13,5$ & $100$ & $59$ & $12$ \\
$25$ & $81,7$ & $94,1$ & $5,9$ & $100$ & $86$ & $10$ \\
$30$ & $118,9$ & $96,7$ & $3,3$ & $100$ & $106$ & $11$ \\
     \hline 	
\multicolumn{7}{|l|}{Famille 4 }\\ 
      \hline 
             & Tps total(s) & Tps CPLEX(\%) & Tps arbre(\%) & \%solv.  & \#n\oe ud & \#PLNE \\
      \hline 
$10$ & $0,29$ & $77,3$ & $22,7$ & $100$ & $27$ & $6$ \\
$20$ & $1476,1$ & $93,2$ & $6,8$ & $80$ & $98$ & $10$ \\
$25$ & $2568,5$ & $90,2$ & $9,8$ & $66$ & $18973$ & $8$ \\
$30$ & $3181,3$ & $98,3$ & $1,7$ & $60$ & $20908$ & $8$ \\
      \hline 
    \end{tabular}
  \end{center}
  \caption{Résultats du Time-Table basé sur les flots dans la méthode de
    branchement hybride pour le \CECSP.}
  \label{tab:res_BB_ERFlot}
\end{table}

L'algorithme de branchement, combiné au raisonnement basé sur les
flots, permet de résoudre des instances du \CECSP~avec $30$ activités en
moins de $7200$ secondes. De plus, nous pouvons voir que, comme dans
le cas du raisonnement énergétique, une bonne
partie du temps de résolution est aussi passé dans le programme linéaire
mixte. Le raisonnement basé sur les flots permet aussi d'obtenir de
meilleurs résultats que le raisonnement énergétique pour les Familles 2
et 1 et de moins bons résultats pour la Famille 4. La recherche de
propriétés inhérentes au problème permettant de savoir quel 
raisonnement appliqué est une direction de recherche intéressante. 

\subsection{Comparaison des différentes approches}

Dans ce paragraphe, nous allons comparer les résultats obtenus avec le
modèle On/Off avec ceux obtenus par la méthode de branchement
hybride. Ces résultats sont décrits dans le
tableau~\ref{tab:comp_OOBB}.

Dans ce tableau, les deux premières colonnes présentent les résultats
obtenus par le modèle On/Off et les deux suivantes ceux obtenus par
la méthode de branchement hybride.   
\begin{table}[!htb]
  \begin{center}
    \begin{tabular}{|c|cc|cc|}
      \hline
      \#tasks & \multicolumn{2}{c|}{Modèle On/Off}&
                                                    \multicolumn{2}{c|}{Algo. hybride}\\ 
      \hline 
              & tps(s) &\%solv. & tps & \%solv.\\ 
      \hline
      \multicolumn{5}{|c|}{Famille 1}\\
      \hline 
      $10 $& $0,22$ & $100$ & $0,12$ &  $100$ \\ 
      $20 $& $11,56$ & $100$ & $5,0$ & $100$ \\ 
      $25 $& $58,79$ & $100$ & $26,6$ &$100 $ \\ 
      $30 $& $1582,9$ & $80$ & $70,8$ & $100 $ \\  
      \hline 
      \multicolumn{5}{|c|}{Famille 2}\\
      \hline 
      $10 $& $0,29$ & $100$ & $0,17$ &$100 $ \\ 
      $20 $& $27,2$ & $100$ & $9,03$&$100 $ \\ 
      $25 $& $380,20$ & $100$ & $81,7$ &$100 $ \\ 
      $30 $& $2634,3$ & $70$ & $118,9$ &$100 $ \\  
      \hline 
    \end{tabular}
  \end{center}
  \caption{Comparaison des résultats du modèle On/Off et de ceux de la
    méthode de branchement hybride pour le \CECSP.}
  \label{tab:comp_OOBB}
\end{table}

Dans le tableau~\ref{tab:comp_OOBB}, nous pouvons voir que, pour les
Familles 1 et 2, l'algorithme de branchement hybride permet de
résoudre plus d'instances que le modèle On/Off en un temps moins
important. Dans certains cas, la résolution devient $20$ fois plus
rapide grâce à l'utilisation de cette méthode. 

Nous allons maintenant présenter les quelques résultats obtenus pour
les instances de la famille $LP$ (avec des fonctions de rendement
concaves et affines par morceaux). Cependant, comme seules les
instances à $10$ ont pu être résolues. Ce sont donc seulement ces
résultats que nous présentons. Le tableau~\ref{tab:BB_LPM} présente
ces résultats avec une limite de temps fixée à $3600$ secondes et
$\epsilon=5$. La première ligne du tableau décrit les résultats de la
méthode arborescente hybride avec le raisonnement basé sur les flots,
la seconde avec le raisonnement énergétique et la dernière avec les
deux raisonnements combinés. Pour chaque ligne, la première colonne
correspond au temps nécessaire à la résolution des instances. La
seconde exhibe le temps passé dans l'arbre de branchement. Enfin, la
troisième et quatrième colonne présentent respectivement le nombre de
PLNE résolus et le nombre de n\oe uds dans l'arbre.


\begin{table}[!htb]
\centering 
\begin{tabular}{|c|cccc|}
\hline
 & Tps Total (s)  & Tps arbre (s) & \#PLNE & \#n\oe ud\\
\hline
TTFlot &3100,8 &0,01&	1	& 8\\
RE&2315,9&0,07&	14,8	&40\\
RE + TTFlot &1709,9&	0,14&	9,25	&37,75\\
\hline
\end{tabular}
\caption{Résultats de la méthode de branchement arborescente sur les
  instances avec fonctions de rendement concaves et affines par
  morceaux. }
\label{tab:BB_LPM}
\end{table}

Dans le tableau~\ref{tab:BB_LPM}, nous pouvons voir que le
raisonnement basé sur l'algorithme de flot est moins efficace
lorsqu'il est utilisé sans le raisonnement énergétique mais lorsque
les deux raisonnements sont combinés ils deviennent plus efficace que
l'un ou l'autre utilisé seul. 

Nous pouvons également remarquer est que, lors de la
résolution, la plupart du temps est passé dans la recherche d'une
solution au PLNE. De ce fait, l'amélioration des performances de cette
méthode devra obligatoirement passer par une amélioration des
performances du PLNE. 

Pour les instances de la Famille $L$, une solution a été trouvée pour
seulement $75\%$ des instances. Donc, quand on compare ces résultats
avec ceux obtenus pour la Famille $LP$, on peut conclure que dans un
cas sur quatre, approximer la fonction de rendement par une fonction
affine rend l'instance infaisable. Ceci permet, malgré la difficulté
de résolution de ces instances, de justifier la considération de
fonctions de rendement concaves et affines par morceaux.





\chapter*{Conclusion\markboth{CONCLUSION}{}}

Dans cette partie, nous avons présenté les résultats obtenus par les
expérimentations que nous avons conduites. Ces expérimentations ont
permis de valider et de comparer les méthodes présentées dans ce
manuscrit. 

Dans un premier temps, nous avons présenté la procédure utilisée pour
générer des instances hétérogènes du \CECSP~et les caractéristiques
des instances utilisées dans les expérimentations sur le \RCPSP. Pour
ce dernier, nous expliquons aussi comment sont précalculées les
fenêtres de temps dans lesquelles les activités doivent s'exécuter. 

Nous présentons ensuite les résultats obtenus par le modèle indexé par
le temps du \CECSP. Une comparaison des résultats avec et sans coupes
énergétiques est réalisée montrant l'intérêt de ces coupes. En effet,
même si l'ajout de ces coupes augmente le temps de résolution du
modèle, cela permet de trouver une première solution de meilleure
qualité que pour le modèle sans ces coupes. En effet, le nombre
important d'inégalités ajoutées au modèle rend la résolution plus
délicate mais ces dernières rendent le modèle plus fort. La mise en
place d'un algorithme de séparation permettant d'ajouter seulement une
partie de ces inégalités à chaque n\oe ud de l'arbre de branchement est
donc une piste de recherche intéressante. 

Le paragraphe suivant détaille les résultats obtenus par les modèles à
événement. Nous avons présenté, dans un premier temps, un algorithme
polynomial utilisé pour séparer les inégalités de non-préemption définies dans le
paragraphe~\ref{sec:nonPreem}. Ensuite, les résultats obtenus par le
modèle Start/End pour résoudre le \CECSP~sont détaillés. Ce dernier,
malgré le fait qu'il ait de meilleure relaxation que le modèle On/Off,
ne permet pas de résoudre autant d'instances car il comprend deux fois
plus de variables binaires.

Les expérimentations conduites sur le modèle On/Off pour résoudre le
\CECSP~ont permis de valider les ensembles d'inégalités et de coupes
définies dans le paragraphe~\ref{sec:amelioration_modele}. En effet,
plusieurs sous-ensembles de ces inégalités ont successivement été
ajoutées au modèle et une comparaison des résultats a été effectuée,
montrant que les meilleures performances étaient obtenus quand ces
inégalités étaient ajoutées au modèle. De plus, l'intérêt de l'ajout
d'une partie de ces inégalités dans le cadre de la résolution du
\RCPSP~a aussi été démontré.

Enfin, une comparaison des résultats obtenus pour les trois modèles
est effectuée dans le paragraphe suivant permettant de démontrer que
le modèle indexé par le temps ne peut être utilisé comme une méthode
de résolution exacte pour le \CECSP. De plus, cette comparaison a
de nouveau permis de montrer que les résultats obtenus avec le modèle
On/Off était meilleurs que ceux obtenus par le modèle Start/End. 

Le paragraphe suivant est consacré aux résultats obtenus par les
algorithmes de filtrage pour le \CECSP. Ces derniers sont inclus dans
une méthode de branchement hybride, couplant règle de branchement et
modèle On/Off. Les expérimentations conduite sont
permis de montrer que cet algorithme obtenait de meilleurs résultats
que le modèle On/Off seul. De plus, l'intérêt de l'algorithme du
raisonnement énergétique et de l'algorithme de vérification basé sur
les flots présentés dans le chapitre~\ref{sec:PLNE_CECSP}. En effet,
pour certaines familles d'instances l'algorithme de flot obtient de
meilleurs résultats et, pour d'autres familles, c'est le raisonnement
énergétique qui les obtient. 

Enfin, la fin de cette partie compare les deux algorithmes de
filtrage sur des instances comportant des fonctions de rendement
concaves et affines par morceaux. Pour ces instances, difficiles à
résoudre, nous appliquons l'algorithme de branchement hybride avec un
des deux raisonnements puis avec les deux. Ceci nous permet de
démontrer que, dans ce cas, l'algorithme de vérification basé sur les
flots obtient de meilleures performances lorsqu'il est couplé avec le
raisonnement énergétique. La dernière chose que ces expérimentations
nous permettent de conclure est que l'approximation des fonctions de
rendement par des fonctions affines peut conduire à l'infaisabilité de
l'instance alors que l'approximation par des fonctions concaves et
affines par morceaux permet de trouver une solution. Ce dernier point
permet, malgré la difficulté de résolution de ces instances, de
justifier la considération de fonctions de rendement concaves et
affines par morceaux. 







%
\clearemptydoublepage%
\chapter*{Conclusions et Perspectives\markboth{CONCLUSIONS ET PERSPECTIVES}{}}

Beaucoup de problèmes d'ordonnancement cumulatifs définis dans la
littérature souffrent d'une limitation imposant que chaque activité
aille une durée et une consommation de ressource fixe durant son 
exécution. Il arrive cependant, dans de nombreux cas pratique, que
cette limitation empêche la modélisation correcte du problème. De ce
fait, de nouveaux problèmes permettant de modéliser la malléabilité
des activités, i.e. ayant une durée et une consommation de ressource
variable, ont été introduits~\cite{DDH,NK,FT,Kis,BLN}. Cette thèse
introduit un nouveau problème appartenant à cette classe, le \CECSP. 
Une comparaison des différents problèmes existant permettant de
modéliser des activités malléables a été réalisée et a permis de
montrer que le \CECSP~est très différents de ces derniers. 

La principale difficulté du \CECSP~repose sur la combinaison entre
ressource continue et malléabilité des activités. En effet, ces deux
caractéristiques impliquent que les activités peuvent prendre des
formes quasi quelconques. De ce fait, un des premiers travail effectué
dans cette thèse a été une étude détaillée du problème. Cette étude a
permis de décrire des cas particuliers du \CECSP~pouvant être résolus
en temps polynomial mais aussi a permis, dans certains cas, la mise en
place d'une propriété permettant de décomplexifier le problème en
caractérisant les différentes formes que peut prendre une activité. 
Cette simplification du problème a ensuite été utilisée pour mettre en
place des méthodes de résolution pour le \CECSP, souvent adaptées de
méthodes existantes dans le cadre des problèmes d'ordonnancement
cumulatifs. 

Les méthodes de résolution dédiées au \CECSP~et décrites dans ce
manuscrit sont regroupées en deux catégories: les méthodes issues de
la programmation par contraintes et adaptées des méthodes définies
pour la contrainte cumulative et les méthodes issues de la
programmation linéaire mixte et en nombres entiers adaptées des
méthodes définies pour le \RCPSP. 

Plusieurs des techniques de programmation par contraintes utilisées dans le
cadre de la résolution de la contrainte cumulative ont décrites et, en
particulier, les principaux algorithmes de filtrage qui lui sont
dédiés. Une partie de ces algorithmes a été adaptés au cas du
\CECSP. C'est le cas, par exemple, du raisonnement énergétique qui
prend une part importante de ce manuscrit. Ce raisonnement comptant
parmi les plus forts dans le cas de la contrainte cumulative a été le
premier à avoir été adapté. De plus, les récents travaux de Derrien
{\it et al.} permettant l'accélération de ce raisonnement ont aussi pu
être adaptés. D'autres raisonnements comme le Time-Table, le
raisonnement disjonctif et le raisonnement Time-Table disjonctif ont
aussi pu être transformés afin de s'appliquer dans le cas du
\CECSP. Enfin, un nouvel algorithme de détection d'incohérence
utilisant un programme linéaire basé sur un problème de flot couplé
avec le raisonnement Time-Table a été présenté. Les expérimentations
conduites sur ces raisonnements -- inclus à l'intérieur d'une méthode
de branchement hybride -- ont permis de montrer que ce nouvel
algorithme de vérification permet la détection d'un plus grand nombre
d'incohérences que le raisonnement énergétique à lui seul.

Différents modèles de programmation linéaire en nombres entiers pour le
\RCPSP~ont été décrits dans le chapitre~\ref{sec:PLNE_RCPSP}. Parmi
ces modèles, le premier repose sur une formulation indexée par le temps
et les deux autres sur des formulations basées sur les événements. Ces
dernières ont montré leur efficacité dans le cas où l'horizon de temps
des modèles devient très grand. De plus, contrairement aux modèles
indexés par le temps, ils permettent de modéliser des dates de début
et de fin d'activités continue. De ce fait, nous avons adapté ces
modèles au cas du \CECSP. Un modèle indexé par le temps est aussi
détaillé. En effet, ces modèles ont prouvé leur efficacité dans le
cadre du \RCPSP.

Pour chacun des modèles présentés, des inégalités valides et/ou des
techniques de coupes ont été présenté afin de renforcer ces
derniers. Des inégalités directement déduites du raisonnement
énergétique sont introduites pour les modèles indexés par le temps du
\CECSP~et du \RCPSP. Pour les modèles à événements, une comparaison
des relaxations linéaires des deux modèles présentés est
effectuée. Des inégalités permettant de renforcer le modèle ayant les
moins bonnes relaxations, le modèle On/Off, sont présentées. Ces
inégalités, appelées inégalités de non-préemption, ont été montré
comme appartenant à l'enveloppe convexe du polyèdre formé par
l’ensemble de toutes les affectations possibles pour les variables
binaires correspondant à une seule activité. Enfin, plusieurs autres
ensembles d'inégalités pour les modèles à événements ont été présentés
et les performances relatives à l'ajout de différentes combinaisons de
ces inégalités au modèle ont été détaillées. 

Malgré tout, le \CECSP~reste un problème difficile, en particulier
dans sa forme générale. En effet, es résultats présentés dans cette
thèse nous ont permis de résoudre des instances allant jusqu'à $60$
activités pour la version décisionnelle de ce problème et seulement
jusqu'à $30$ pour la version ayant pour objectif la minimisation de la
consommation de la ressource. 

Le \CECSP~est un nouveau problème pour lequel de nombreux travaux
restent à faire. Parmi les perspectives directes des résultats
présentés dans ce manuscrit, on trouve:
\paragraph{La considération de fonction de rendement plus générale.} Même si les fonctions concaves permettent une plus grande
  liberté d'expression du problème, elles restent insuffisante pour
  modéliser certains problèmes réels. Dans un premier temps, la
  considération de fonctions de rendement convexes est une
  continuation naturelle de ce travail. Cependant, ce n'est toujours
  pas suffisant dans certains cas. En effet, beaucoup de fonctions 
  de rendement ne sont ni totalement concave, ni totalement convexe
  mais sont convexes sur certains intervalles et concaves sur les
  autres intervalles. 

\paragraph{La mise en place d'algorithme de filtrage plus
  performants.} Qu'il s'agisse d'algorithme plus performant en termes
de temps de calcul ou plus performant en termes de filtrage, cette
direction de recherche est une perspective importante. L'adaptation
d'autres algorithmes de filtrage mis en place pour la contrainte
cumulative semble une piste intéressante mais l'accélération du temps
de calcul de ces derniers est fait au prix d'un filtrage de moins
bonne qualité. Ceci peut s'avérer critique dans le cas du \CECSP~où
les algorithmes de filtrage pour la contrainte cumulative sont
beaucoup plus faibles. A l'inverse, la mise en place d'algorithmes
dédiés au problème ayant un pouvoir de filtrage plus grand sera
probablement effectuée au prix d'une augmentation importante du temps 
de calcul. 

\paragraph{Amélioration des modèles de programmation linéaire mixte.}
Dans un premier temps, les modèles décrits dans ce manuscrit
pourraient encore être renforcés. En effet, les inégalités
énergétiques décrites pour le modèle indexé par le temps sont un moyen
intéressant de coupler des techniques issues de la programmation par
contraintes à la programmation linéaire mixte. D'autres inégalités de
ce type pourraient être définies ou de meilleures techniques
d'intégration de ces inégalités dans le processus de résolution
pourraient être mises en place. 

L'amélioration des modèles à événement est aussi une piste de
recherche importante puisque, contrairement aux modèles indexés par le
temps, ces derniers permettent d'obtenir des solutions exactes pour le
\CECSP~mais au prix d'un temps de calcul beaucoup plus important. Pour
renforcer ces modèles, d'autres jeux d'inégalités, plus fortes,
peuvent être mises en place. Par exemple, des inégalités décrivant des
facettes du polyèdre formé par l'ensemble des solutions de chaque
modèle peuvent être exhibées.

\paragraph{L'utilisation d'autres méthodes de résolution.}
Les techniques décrites dans ce manuscrit utilisent principalement des
techniques de programmation linéaire mixte et de programmation par
contraintes. Cependant, d'autres paradigmes existent et des techniques
issues de ces derniers pourraient être considérées. De même, des
techniques appliquées sur des problèmes connexes pourraient aussi être
adaptées, quitte à discrétiser le problème. Parmi ces techniques, on
pourra trouver les techniques de génération de colonnes, de
branch-and-cut, la programmation dynamique ou des techniques utilisant
des règles de priorités, etc. 









%
\clearemptydoublepage%
\bibliographystyle{alpha}
\bibliography{main_file,JFPC,IESM}

\begin{bibunit}[alpha]
\nocite{main_file}
\nocite*
\putbib[main_file]
\end{bibunit}
%
\clearemptydoublepage%
\appendix
\part*{Annexes}

\chapter{Annexe1 - Article JFPC}  

Nous considérerons un problème d'ordonnancement cumulatif dans
lequel les tâches ont une durée et un profil de consommation de
ressource variable. Ce profil, qui peut varier en fonction du temps, est
une variable de décision du problème dont dépend la durée de la tâche
associée. 
Pour ce problème NP-difficle, nous présentons un modèle de programmation
par contraintes et un modèle de programmation linéaire en nombres
entiers (PLNE). De plus, des inégalités valides déduites de la
programmation par contraintes  viennent renforcer le PLNE. 
Ces modèles sont ensuite comparés par le biais d'expérimentations.


\section{Introduction}

Nous étudions un problème d’ordonnancement avec ressource continue et
contraintes énergétiques, le Continuous Energy-Constrained Scheduling
Problem (CECSP). Dans ce problème, un ensemble de tâches ${\cal
A}=\{1,\dots,n\}$ utilisant une ressource continue et cumulative de
capacité limitée $B$ doit être ordonnancé.  La quantité de ressource
nécessaire à l'exécution d'une tâche n'est pas fixée mais - le profil
de consommation de cette dernière est une fonction $b_i(t)$ définie
pour tout  $t \in \mathbb{R}$\footnote{Le domaine de définition de la
fonction peut être réduit mais, pour faciliter les notations, nous
supposons qu'elle est définie pour tout $t \in \mathbb{R}$.} - doit
être déterminé. Une fois la tâche commencée et jusqu'à sa date de fin,
la fonction $b_i(t)$ doit être comprise entre une valeur maximale,
$\bmax$, et minimale, $\bmin$.   

De plus, la consommation, à un instant $t$, d'une partie de
la ressource permet la production d'une certaine quantité d'énergie
et, une tâche finit lorqu'elle a reçu une énergie $W_i$. Cette
énergie est calculée par le biais d'une fonction de rendement $f_i$, 
propre à chaque tâche. Dans cet article, ces fonctions sont supposées
continues, croissantes, affines et peuvent être exprimées de la
manière suivante: 
 
\noindent
{\scriptsize
  $f_i(b)=\left\{
    \begin{array}{ll} 0 & \quad \text{si }b=0\\ a_i*b+c_i &\quad
                                                            \text{si }\bmin=0\text{ et }b \in ]\bmin,\bmax] \\ a_i*b+c_i &\quad
                                                                                                                           \text{si }\bmin\neq 0 \text{ et }b \in [\bmin,\bmax]
    \end{array} \right.$ }

\noindent
avec $a_i>0$ et $c_i \geq -a_i*\bmin $ pour
s'assurer que $f_i(b) \geq 0,\ \forall b \in \inter[\bmin][\bmax]$.

Dans la suite, nous dénotons par $\underline{s_i}$ et $\overline{s_i}$
la date de début au plus tôt et au plus tard de $i$ et par
$\underline{e_i}$ et $\overline{e_i}$ la date de fin au plus tôt et au
plus tard de $i$.

Pour trouver une solution pour le CECSP, nous devons déterminer, pour
chaque tâche $i \in {\cal A}$, sa date de début $s_i$, sa date de fin
$e_i$ et sa fonction d'allocation de ressource $b_i(t)$, $\forall t
\in {\cal T}=[\min_{i\in{\cal A}} \underline{s}_i, \max_{i\in {\cal
    A}} \overline{e}_i]$. De plus, ces variables doivent satisfaire
les contraintes suivantes:
{\scriptsize
\begin{eqnarray} 
  \underline{s}_i\le s_i < e_i \le \overline{e}_i & & \forall i \in
{\cal A} \label {tw_CECSP}\\
  b_i^{min} \le b_i(t) \le b_i^{max} & & \forall i \in {\cal A},\
\forall t \in [s_i,e_i] \label {bminmax_CECSP}\\
  b_i(t)=0 & & \forall i \in {\cal A},\ \forall t \not\in
[s_i,e_i] \label {b0_CECSP}\\
  \int_{s_i}^{e_i} f_i(b_i(t))dt =W_i & & \forall i \in {\cal
A} \label{nrj_CECSP}\\
  \sum_{i \in {\cal A}} b_i(t) \le B & & \forall t \in {\cal
T} \label{res_CECSP}
\end{eqnarray}}
L'objectif auquel nous nous sommes intéressés est la minimisation de
la consommation totale de la ressource. Dans~\cite{Nattaf2015}, les
auteurs montrent que trouver une solution admissible pour le CECSP est
déjà un problème NP-complet. 

De plus, une instance ayant des données seulement entières peut
n'avoir que des solutions à valeurs dans
$\mathbb{R}$~\cite{Nattaf2015}. Cependant, une dilatation de
l'instance, i.e. multiplier les données par un certain coefficient
$\alpha$, permet de palier à ce problème. 
De ce fait et dans le but de résoudre des instances entières, nous
nous sommes intéressés, dans un premier temps, à la version discrète
du CECSP, le DECSP (Discrete Energy Constrained Scheduling
Problem). Dans ce problème, toutes les données sont supposées entières
et les domaines de chaque variable ne contiennent
que des valeurs entières, i.e. $s_i,\ e_i,\ b_i(t) \in \mathbb{N}$ et
$b_i(t)$ est défini $\forall t \in {\cal T}_{\cal D}=\{\min_{i\in{\cal
A}} \underline{s}_i,\dots, \max_{i\in {\cal A}} \overline{e}_i\}$. 

Pour ce problème, nous présentons un modèle de
programmation par contraintes (PPC) permettant l'utilisation des
algorithmes de propagation mis en place pour la contrainte
cumulative, notamment~\cite{Gay2015}.  Un modèle de
programmation linéaire en nombres entiers (PLNE) est aussi
présenté. Ce modèle est ensuite renforcé à l'aide d'inégalités
valides déduites du raisonnement énergétique~\cite{Lopez1990}. 
Ces deux modèles sont ensuite testés sur des instances à données
entières, avec et sans dilatation.

\section{Modèle de programmation par contrainte}

Pour modéliser le DECSP à l'aide de la PPC, nous divisons chaque tâche
$i$ en deux sous-tâches $i_{min}$ et $i_{preem}$. La première,
$i_{min}$, est une tâche ayant une consommation de ressource fixe,
égale à $b_i^{min}$, et une durée variable $p_i$. Cette tâche
représente la quantité de ressource obligatoirement consommée par une
activité durant son exécution, i.e. $\bmin$.  La seconde, $i_{preem}$
est une tâche préemptive optionnelle, consommant une quantité variable
de ressource comprise entre $0$ et $\bmax-\bmin$ et devant s'exécuter
en même temps que $i_{min}$. Cette tâche est elle-même divisée en
sous-tâches $i_{preem}^\ell,\ \ell \in \{1,\dots, e_i-s_i\}={\cal L}_i$. Notons
que $|{\cal L}_i|=e_i - s_i \le \lceil \frac{W_i}{f_i(\bmin)}\rceil$. 
\begin{ex}
Considérons la tâche possédant les attributs suivant:  $\underline{s_i}= 0$,
$\overline{e_i}=6$, $W_i=28$, $b^{min}=1$,
$b^{max}=5$ et $f_i(b)=2b+1$. La Figure~\ref{fig:ex:PPC} présente un
ordonnancement de cette tâche (à droite) et l'ordonnancement
correspondant donné par le modèle (à gauche) avec $b_{i^2_{preem}}=0$. 
\begin{figure}[!htb]
  \centering
  \begin{tikzpicture}
[xscale=0.5,yscale=0.46]
\node (O) at (0,0) {};
\draw (0,0) rectangle (4,1) node[midway] {$i_{min}$};
\draw (0,1) rectangle (1,5) node[midway,rotate=-90] {$i_{preem}^1$};
\draw (2,1) rectangle (3,3) node[midway,rotate=-90] {$i_{preem}^3$};
\draw (3,1) rectangle (4,3) node[midway,rotate=-90] {$i_{preem}^4$};

\draw[->,>=latex] (0,0) -- (4.5,0);

\foreach \i in {0,...,4}
{
  \draw (\i,-0.3) -- (\i,0);
  \draw (-0.3,\i) -- (0,\i)node[left=0.2cm] {\footnotesize \i};
}

\draw (7,5) -- (7,1) -- (8,1) -- (8,3)  -- node[below=0.5cm,midway] {\Large $i$}(10,3)  --  (10,1)
-- (10,0)--  (6,0) -- (6,5) -- cycle;
\draw[->,>=latex] (6,0) -- (10.5,0);

\foreach \i in {0,...,4}
{
  \draw (\i+6,-0.3) -- (\i+6,0);
  \draw (5.7,\i) -- (6,\i) node[left=0.2cm] {\footnotesize \i}; 
}    
  \end{tikzpicture}
  \caption{Exemple de solution du modèle PPC.}
  \label{fig:ex:PPC}
\end{figure}
\end{ex}
Le problème du DECSP peut alors être formulé à l'aide des variables:
{\scriptsize
\begin{itemize}
\item $i_{min}=\{s_{i_{min}}, e_{i_{min}}, b_{i_{min}},p_{i_{min}}\},\ 
  \forall i \in {\cal A}$ 
\item $i_{preem}^\ell=\{s_{i_{preem}^\ell},e_{i_{preem}^\ell},
  b_{i_{preem}^\ell},p_{i_{preem}^\ell}\}$, $\forall i \in {\cal A},\
  \ell \in {\cal L}_i$.
\end{itemize}
et des contraintes:
\begin{enumerate}  
\item $\forall (i,l) \in {\cal A} \times {\cal L}_i:\
  e_{i_{preem}^\ell} = s_{i_{preem}^{\ell+1}}$ 
\vspace{0.2cm}
\item $\forall i \in {\cal A}:\  s_{i_{min}} = s_{i_{preem}^1}$ et $ e_{i_{preem}^{|{\cal L}_i|}} = e_{i_{min}}$   
\vspace{0.2cm}
\item $\forall t \in  {\cal T}:\\ \sum\limits_{\substack{i \in {\cal A} \\ t \in [s_{i_{min}},e_{i_{min}}[}} b_{i_{min}}
  + \sum\limits_{\substack{(i,l) \in {\cal A}\times {\cal L}_i\\ t \in
      [s_{i_{preem}^\ell},e_{i_{preem}^\ell}[}}
  b_{i_{preem}^\ell} \le B$ 
\vspace{0.2cm}
\item $\forall i \in {\cal A}:\ \sum_{l \in {\cal L}_i} \left(f_i(b_{i_{preem}^\ell})
    (e_{i_{preem}^\ell}-s_{i_{preem}^\ell})\right) +
  f_i(b_{i_{min}}) (e_{i_{min}}-s_{i_{min}}) \ge W_i $
\end{enumerate}}
La première contrainte permet d'ordonner les sous-tâches de
$i_{preem}$. Ceci dans le but de faciliter la modélisation des autres
contraintes. La seconde contrainte modélise le fait que $i_{preem}$
commence et finit en même temps que $i_{min}$. La troisième contrainte
assure que la capacité de la ressource n'est pas excédée en sommant, à
un instant $t$, les consommations minimales des tâches en cours ainsi
que les consommations des sous-tâches préemptives en cours. Enfin, la
quatrième contrainte permet de s'assurer que chaque tâche reçoit au
moins l'énergie requise $W_i$. 

Un des avantages de cette formulation est qu'elle permet l'utilisation
des algorithmes de propagation mis en places pour la contrainte
cumulative tels que le time-table classique~\cite{Baptiste2001},
disjonctif~\cite{Gay2015} ou associé au edge-finding~\cite{Vilim2011},
le raisonnement disjonctif~\cite{Baptiste2001}, ou encore le
raisonnement énergétique~\cite{Lopez1990}. Cependant, certains de ces
raisonnements peuvent être adaptés pour 
prendre en compte l'ensemble du problème. C'est le cas, par exemple,
du raisonnement énergétique détaillé ci-dessous.

\subsection{Raisonnement énergétique}


Ce paragraphe présente un algorithme de propagation pour le
DECSP basé sur le raisonnement énergétique défini pour le
CECSP~\cite{Nattaf2015}.  L'adaptation de ce raisonnement au cas
discret est quasi-directe. Cependant, nous rappelons les bases de
celui-ci car nous l'utiliserons dans la suite pour déduire des
inégalités valides pour le PLNE.

Le principe du raisonnement énergétique est de comparer la quantité de
ressource disponible dans un intervalle avec la quantité minimale de
ressource consommée par toutes les tâches dans cet intervalle.

Les configurations pour lesquelles la quantité de ressource requise
par une tâche $i$ dans l'intervalle $[t_1,t_2[$ est minimale
correspondent toujours à une configuration où la tâche reçoit le
maximum d'énergie possible, i.e. est ordonnancée à $\bmax$, en dehors
de $[t_1,t_2[$, tout en respectant les contraintes
\eqref{tw_CECSP}--\eqref{res_CECSP}. Ceci correspond donc à une des
configurations suivantes:
\begin{itemize}
\item la tâche est calée à gauche: ordonnancée à $\bmax$ durant
$[\underline{s_i},t_1[$;
\item la tâche est calée à droite: ordonnancée à $\bmax$ durant
$[t_2,\overline{e_i}[$;
\item la tâche est centrée: ordonnancée à $\bmax$ durant
$[\underline{s_i},t_1[ \cup [t_2,\overline{e_i}[$ ou ordonnancée à
$\bmin$ durant $[t_1,t_2[$.
\end{itemize} En effet, dans le dernier cas, il peut arriver
qu'ordonnancer la tâche à $\bmax$ dans $[\underline{s_i},t_1[ \cup
[t_2,\overline{e_i}[$ implique que la quantité d'énergie restant à
apporter à la tâche dans $[t_1,t_2[$ ne soit pas suffisante pour
ordonnancer la tâche à $\bmin$ durant $[t_1,t_2[$. Or, ceci
impliquerait une violation de la
contrainte~\eqref{bminmax_CECSP}. Dans ce cas, la tâche est donc
ordonnancée à $\bmin$ durant l'intervalle $[t_1,t_2[$. Alors la
quantité de ressource requise par la tâche $i$ dans $[t_1,t_2[$ est la
quantité minimale requise par ces configurations.  

Les intervalles $[t_1,t_2[$ sur lesquels appliquer ce test pour le
CECSP sont décrits dans~\cite{Nattaf2015}. Pour le DECSP, nous devons
considérer les projections de ces intervalles sur les entiers,
i.e. $[a,b[ \rightarrow [\lfloor a \rfloor, \lceil b \rceil[$.  Les
ajustements pour le CECSP s'adaptent aussi naturellement au DECSP à
l'aide de cette même projection.

\section{Modèle de programmation linéaire en nombres entiers}
\label{MIP}
\subsection{Modèle}
La formulation proposée dans cet article est une formulation indexée par
le temps. Elle est adaptée de la formulation décrite
dans~\cite{Nattaf2015}. 
Dans ces formulations, l'horizon de temps est divisé en
intervalles de taille 1 et est défini par: ${\cal T}_{\cal D}$. 
Pour chaque activité $i \in {\cal A}$ et pour chaque instant $t \in
{\cal T}_{\cal D}$, nous définissons deux variables binaires $x_{it}$
et $y_{it}$ pour modéliser le début et la fin des activités. La
variable $x_{it}$ (resp. $y_{it}$) prendra la valeur $1$ si et
seulement si l'activité $i$ commence (finit) à l'instant $t$.
Pour modéliser la consommation de ressource et l'apport en énergie,
nous introduisons deux variables, $b_{it}$ et $w_{it}$ qui
représentent respectivement la quantité de
ressource consommée par l'activité $i$ dans la période de temps
$t$ et l'énergie reçue par cette même activité durant cette période. 

Par manque de place, ce modèle n'est pas entièrement décrit ici mais
nous décrivons les contraintes permettant de lier les variables
$b_{it}$ et $w_{it}$, i.e. permettant de calculer l'énergie apportée à
$i$ dans la période $t$, $w_{it}$ , en fonction de la consommation de
ressource $b_{it}$. Nous donnons aussi le nombre de variables et de
contraintes du modèle.

Les contraintes liant $b_{it}$ et $w_{it}$, $\quad \forall t\in {\cal T}_{\cal D},\
\forall i \in {\cal A}$ sont les suivantes:
\begin{equation} \small w_{it}=a_ib_{it}+c_i\left(\sum_{\tau=\underline{s_i}}^t
x_{i\tau}-\sum_{\tau=\underline{s_i}+1}^t y_{i\tau}\right)  \label{conv_CECSP_TI}
\end{equation} Cette contrainte nous permet de modéliser la fonction
de rendement $f_i,\ \forall i \in {\cal A}$. En effet,
$\left(\sum_{\tau=\underline{s_i}}^t x_{i\tau}-\sum_{\tau=\underline{s_i}+1}^t
y_{i\tau}\right) $ est égale à $1$ si et seulement si l'activité $i$
est en cours à l'instant $t$.  Dans ce cas là, la valeur de l'énergie
apportée à $i$ est bien $w_{it}=a_ib_{it}+c_i$. Le second cas se
produit quand l'activité $i$ n'est pas en cours à $t$. Dans ce cas,
$b_{it}=0$ implique $w_{it}=0$.

Le modèle possède donc $2n|{\cal T}_{\cal D}|$ variables binaires,
$2n|{\cal T}_{\cal D}|$ variables continues et au plus $3n+|{\cal
T}_{\cal D}|*(6n+1)$ contraintes.

\section{Inégalités valides basées sur le raisonnement énergétique}
Ce paragraphe décrit des inégalités valides déduites du raisonnement
énergétique pour le PLNE. Soit ${\cal R}$ l'ensemble des intervalles
d'intérêt pour le raisonnement énergétique.
{\scriptsize
\begin{align}
&(x_{i\underline{s}_i} + y_{i\overline{e}_i} -1 ) \, \bb + \sum_{j\neq i}
  \bb[j] \le \nonumber\\ 
&  B(t_2-t_1) \quad \forall i \in {\cal A},\ \forall
  [t_1,t_2] \in {\cal R}
\label{both}\\[2mm] 
&(x_{i\underline{s}_i} + \sum_{t=t_1}^{t_2}y_{it} -1) \, \bb + \sum_{j\neq i}
  \bb[j] \le \nonumber \\
&  B(t_2-t_1) \quad \forall i \in {\cal A},\ \forall
  [t_1,t_2] \in {\cal R}
\label{left}\\[2mm] 
&(\sum_{t=t_1}^{t_2}x_{it} + y_{i\overline{e}_i}-1) \, \bb + \sum_{j\neq i}
  \bb[j] \le \nonumber\\
&  B(t_2-t_1) \quad \forall i \in {\cal A},\ \forall
  [t_1,t_2] \in {\cal R}
\label{right}\\[2mm] 
&(1-\sum_{t<t_1}x_{it} - \sum_{t>t_2}y_{it}) \, \bb + \sum_{j\neq i}
  \bb[j] \le \nonumber \\
&  B(t_2-t_1) \quad \forall i \in {\cal A},\ \forall
  [t_1,t_2] \in {\cal R}
\label{total}\\[2mm] 
&(\sum_{t\le t_1}x_{it} + \sum_{t\ge t_2}y_{it} -1 ) \, \bb  \le
B(t_2-t_1) \nonumber\\
&\forall i \in {\cal A},\ \forall
[t_1,t_2] \in {\cal R}
\label{min}
\end{align}
}
L'inégalité~\eqref{both} correspond au cas où la tâche est centrée et est
ordonnancée à $\bmax$ durant $[\underline{s_i},t_1] \cup
[t_2,\overline{e_i}]$. En effet, cette inégalité n'est active que dans
le cas où $(x_{i\underline{s}_i} + y_{i\overline{e}_i} -1 )= 1 \Rightarrow
\left[x_{i\underline{s}_i}=1\land y_{i\overline{e}_i}=1\right]$. Or,
ceci implique que la tâche commence à $\underline{s_i}$ et finit
$\overline{e_i}$. Donc, la ressource disponible dans $[t_1,t_2[$ doit
être suffisante pour donner la quantité de ressource minimale requise
par $i$ dans $[t_1,t_2[$ dans cette configuration.  Dans tous les
autres cas, l'inégalité devient $\sum_{j\neq i} \bb[j] \le B(t_2-t_1)$
ou $\sum_{j\neq i} \bb[j] - \bb \le B(t_2-t_1)$.

Les inégalités \eqref{left}, \eqref{right}, \eqref{total}, \eqref{min}
correspondent respectivement au cas où $i$ est calée à gauche, $i$ est
calée à droite, $i$ est complètement incluse dans $[t_1,t_2]$, $i$ est
exécutée à $\bmin$ durant l'intervalle $[t_1,t_2]$ et sont déduites de
la même façon que~\eqref{both}.  Ces inégalités seront ajoutées au
modèle indexé par le temps décrit à la section~\ref{MIP} pour
renforcer ce dernier. 
 
\section{Résultats Expérimentaux}

Nous avons testé les différentes méthodes de résolution proposées dans
cet article sur les instances de~\cite{Nattaf2015}. Les
expérimentations ont été conduites sous le système d'exploitation
Ubuntu 64-bit 12.04 et les résultats sont calculés au moyen d'un
processeur 4-core, 8 thread Core (TM) i7-4770 CPU et de 8GB de mémoire
RAM. 

Le modèle de PLNE est résolu à l'aide de IBM Cplex 12.6 avec 2 threads
et une limite de temps de 100 secondes. Les inégalités déduites du
raisonnement énergétique sont calculées avant la résolution du PLNE et
ajoutées statiquement au modèle. Ceci augmente la taille du modèle de
$5|{\cal R}|n$ contraintes (avec $|{\cal R}| \in O(n^2)$). 

Le tableau~\ref{table1} décrit les résultats du PLNE. 

L'ajout des inégalités du raisonnement énergétique permet de résoudre
les instances à $25$ tâches de manières plus efficace. Cependant, elles
ralentissent le modèle pour les instances à $20$ ou $30$ tâches mais
la perte de rapidité dans ce cas là est beaucoup moins élevée que le
gain fait sur les instances à $25$. Une poursuite de recherche
intéressante serait d'essayer d'ajouter ces contraintes pendant la
résolution du PLNE en tant que coupes. 

Le modèle de PPC est résolu avec IBM CP Optimizer 12.6. Le
tableau~\ref{table2} décrit les résultats du modèle de PPC.  Le modèle 
de PPC est testé sans ajout du raisonnement énergétique présentés dans
cet article mais le modèle utilise les propagateurs du solveur. Des
résultats expérimentaux plus détaillées seront proposés lors de la
conférence.
\begin{table}
\centering
    \footnotesize
    \begin{tabular}{|c|c|cc|cc|}
      \hline
       & & \multicolumn{2}{c|}{$1^{ère}$ sol.} & \multicolumn{2}{c|}{fin algo.}\\
      \hline
    &  \#tâches & temps(s) &  écart & temps  & \%opt. \\
      \hline
      DEF	&	20	&	5.37	&	7.85	&	75.4	&	0.25\\
      ER	&	20	&	8.4	&	10.6	&	78.9	&	0.22\\
      \hline
      DEF	&	25	&	4.6	&	4.4	&	83.8	&	0.17\\
      ER	&	25	&	0.06	&	3.86	&	60.1	&	0.4\\
      \hline
      DEF	&	30	&	0.99	&	7.18	&	75.19	&	0.25\\
      ER	&	30	&	5.66	&	7.53	&	75.8	&	0.25\\
      \hline
    \end{tabular}
  \caption{Résultats du PLNE avec et sans inégalités valides: ER et
    DEF resp. (TL 1000s)}
  \label{table1}
\end{table}


\begin{table}
    \footnotesize
\centering
    \begin{tabular}{|c|cc|cc|}
      \hline
      & \multicolumn{2}{c|}{$1^{ère}$ sol.} & \multicolumn{2}{c|}{fin algo.}\\
      \hline
      \#tasks & time & deviation & time lim. &  \%solved \\
      \hline
20	&0.19&	34.1&	100&	95\\
25	&0.3	&47.9&	100	&91\\
30	&0.42	&43.1	&100	&95\\
      \hline
    \end{tabular}
  \caption{Résultats du modèle PPC}
\label{table2}
\end{table}

Les résultats montrent l'intérêt des inégalités valides ajoutées au
PLNE. Le modèle de PPC ne permet pas de prouver l'optimalité des solutions
trouvées mais a des résultats similaires au PLNE. En effet, dans
presque tous les cas, le modèle de PPC trouve une solution aussi bonne
que le PLNE, sans toutefois prouver son optimalité.

Les méthodes présentées ont aussi été testées sur des instances
dilatées dans le but de garantir l'existence de solutions entières.  
La dilatation est effectuée de la manière suivante. Soit $\alpha$ le
plus petit commun 
multiple à tous les $\bmin$ et $\bmax$. Alors, la dilatation
consiste à multiplier $\overline{e}_i,\ \underline{e}_i,\
\overline{s}_i,\ \underline{s}_i$ et $W_i$ par $\alpha$. Ces
expérimentations n'ont pas donné de résultats dû à la grande taille
de ces modèles. Les modèles continus pourraient donc rester  la
seule alternative pour obtenir des solutions optimales dans le cas où
la solution est réelle.

Parmi les poursuites de recherche possibles, on trouve l'amélioration
des modèles avec la réduction du nombre de variable et/ou de contraintes et la
mise en place d'algorithmes de propagation dédiés.

\begin{thebibliography}{99}  
\bibitem{Baptiste2001} P. Baptiste, C. Le Pape and W. Nuijten (2001).
\emph{Constraint-based scheduling}, Kluwer Academic Publishers,
Boston/Dordrecht/London.
  
\bibitem{Lopez1990} J. Erschler and P. Lopez (1990).  Energy-based
approach for task scheduling under time and resources constraints.
\emph{2nd International Workshop on Project Management and
Scheduling}, pp. 115-121, Compi{\`e}gne, France.
  
\bibitem{Gay2015}
S. Gay, R. Hartert and P.Schaus (2015).
Time-Table Disjunctive Reasoning for the Cumulative Constraint.
\emph{CPAIOR 2015, Proceedings}, pp 157--172.

\bibitem{Nattaf2015} M. Nattaf, C. Artigues, P. Lopez and D. Rivreau
(2015).  Energetic reasoning and mixed-integer linear programming for
scheduling with a continuous resource and linear efficiency functions.
\emph{OR Spectrum}, pp 1--34.

\bibitem{Vilim2011}
P. Vil{\'i}m (2011).
Timetable Edge Finding Filtering Algorithm for Discrete Cumulative Resources.
\emph{CPAIOR 2011, Proceedings}, pp 230--245.


\end{thebibliography}

% Vous pouvez aussi générer vos références en utilisant BibTeX
% Pour cela, remplacez l'ensemble de l'environnement ``thebibliographie''
% par la commande ``\bibliography{biblio}''
% où biblio.bib est le nom de votre fichier de références bibtex.

\end{document}











\chapter{A batch sizing and scheduling problem on parallel machines with
different speeds, maintenance operations, setup times and energy
costs}
\chaptermark{Papier IESM}
\label{ann:IESM}

This paper considers a production scheduling problem in a Chilean
company from the metalworking industry. This company produces steel
balls of different diameters on parallel production lines. There are
different types of production lines and each production line may have a
different speed for producing each diameter. Furthermore a setup time
occurs when changing the diameter produced on each machine. Besides
these production and setup operations, maintenance operations have to
be scheduled. These electrical machines yield high energy
demands. It is therefore crucial to minimize total energy consumption,
which depends on batch/machine assignment, and maximum demand on peak
hours. We consider the batch sizing and scheduling problem involving
electricity costs in a non-uniform parallel machine context. Given a
demand for each family of steel balls, the problem consists in
splitting the demand in sublots (batches) that have to be assigned and
scheduled on the parallel machines together with the required
maintenance operations. The goal is to complete the schedule before a
common deadline while minimizing electricity costs. We propose to
tackle this problem through mixed integer linear programming. We
propose a global formulation and a two-phase
matheuristic. Computational results on realistic instances are
provided.

\section{Introduction}

This paper considers a production scheduling problem in a Chilean
company from the metalworking industry. The problem and the industrial
context was described in \cite{Urrutia:Thesis:2014}. This company
produces steel balls on parallel production lines. The steel balls are
obtained by roll forming or forging from a raw material consisting of
metal bars. The steel balls are mainly used in the copper and gold
mining industry for mineral grinding, i.e. for reducing the size of
the mineral particles to a maximal granularity that permits to remove
the major part of impurities from the mineral. There are several types
of steel balls to produce, each corresponding to a different ball
diameter. There are two types of production lines : roll formers for
small diameters and forges for larger diameters (although medium
diameter balls can be produced by both production line types). Each
production line may have a different speed for producing each
diameter. Furthermore a setup time occurs when changing the diameter
produced on each machine. Besides these production and setup
operations, maintenance operations have to be scheduled. These
machines are electrical and the production process results in high
energy demands. As mentioned in \cite{Urrutia:Thesis:2014}, up to 50\%
of the production cost in such a manufacturing process can be due to
electricity consumption. In Chile, as in many other places, different
electricity rates are applied for peak hours and off-peak hours. So it
can be crucial for metalworking companies to control the electricity
demand during peak hours. Planning the maintenance and setup
operations during the peak hours can be intuitively a policy that
favors electricity cost decrease.

In this paper we consider a batch sizing and scheduling problem
involving electricity costs in a non-uniform parallel machine
context. Given a demand for each family of steel balls, the problem
consists in splitting the demand in sublots (batches) that have to be
assigned and scheduled on the parallel machines together with the
required maintenance operations. The goal is to complete the schedule
before a common deadline while minimizing electricity costs. We
propose to tackle this problem through a mixed integer linear
programming approach.

Section \ref{sec:prob} presents the batch sizing and scheduling
problem. Section \ref{sec:review} is devoted to a brief presentation
of the related work. Section \ref{sec:milp} gives the proposed
mixed-integer linear programming (MILP) formulation. Section
\ref{sec:simple} gives a simplified MILP formulation that ignores the
peak costs and that serves as a basis for a matheuristic. Section
\ref{sec:exp} presents the considered realistic problem instances and
the results obtained by our approach. Section \ref{sec:concl} draws
concluding remarks and directions for future work.

\section{The industrial scheduling problem}
\label{sec:prob}

The problem involves a set ${\cal J}$ of $n$ lots (jobs) to be
scheduled on a set ${\cal M}$ of $m$ machines. There is total a
production demand $D_j$ for each job $j\in{\cal J}$ to be fulfilled
during the scheduling horizon given by a time interval $[0,T]$. Each
job $j\in{\cal J}$ can be split into a maximum number of $\lfloor
D_j/\epsilon \rfloor$ sublots (batches), where $\epsilon$ is the
minimum batch size. The machines able to produce a job $j$ are
gathered in set ${\cal M}_j$. For a given job $j\in{\cal J}$, a
machine $k\in {\cal M}_j$ has a production speed $v_{j}^{k}$. Whenever
a batch of a job $j\in{\cal J}$ is scheduled immediately after a batch
of job $i\in{\cal J}$ on a machine $k\in{\cal M}$, a setup time
$s_{ij}^k\geq 0$ is necessary on the machine. Each machine is assumed
to have a constant electrical power demand $w_k$. There are $H$ peak
periods within the horizon, each being defined by an interval
$[a_h,f_h)$ with $a_h<f_{h}$ for $h=1,\ldots,H$ and $f_h<a_{h+1}$ for 
  $h=1,\ldots,H-1$. Concerning cost minimization, we considered a
  weighted sum problem where $\alpha$ denotes the weight of the total
  energy consumption while $\beta$ denotes the weight of the maximum
  consumption during peak hours.

A solution consists in determining:
\begin{itemize}
\item A set of batches ${\cal B}_j$ with $|{\cal B}_j|=B_j\geq 1$, for
  each job $j\in{\cal J}$,
\item A production amount $q_{j,b}\geq \epsilon$, for each batch
  $b\in{\cal B}_j$,
\item The assignment $a_{j,b}$ of each batch $b \in {\cal B}_j$, i.e. 
  the line on which batch $b \in {\cal B}_j$ is produced,
\item A start time $S_{j,b}$, for each job $j\in{\cal J}$ and for each
  batch $b\in{\cal B}_j$.
\end{itemize}
Let ${\cal B}=\cup_{j\in{\cal J}}\{(j,b)|b\in{\cal B}_j\}$ denote the
set of pairs (job, batch index). Let ${\cal A}_k=\{(j,b)|a_{j,b}=k\}$ 
the set of batches assigned to machine $k\in{\cal M}$.

The problem constraints can be stated as follows:

%\noindent One and only one machine has to be assigned to each batch.
%\begin{eqnarray}{C}
%\left(\sum_{k\in{\cal M}}A_k=|{\cal B}|\right)\wedge
%\left(\mathop{\cup}_{k\in{\cal M}}{\cal A}_k={\cal
%  B}\right)\label{ct:1}
%\end{eqnarray}
A batch can only be assigned on an authorized machine.
\begin{eqnarray}
a_{j,b} \in {\cal M}_j \quad \forall j\in {\cal J},\forall b\in{\cal B}_j
\label{ct:2}
\end{eqnarray}

\noindent The demand must be satisfied for each job.
\begin{eqnarray}
\sum_{b\in{\cal B}_j} q_{j,b} = D_j\quad\forall j\in{\cal
  J}\label{ct:4}
\end{eqnarray}

\noindent The duration $p_{j,b}$ of a batch $b\in {\cal B}_j$ is equal to 
its production divided by the speed of its assigned machine.
\begin{eqnarray}
p_{j,b}=q_{j,b}/v_{j}^{a_{j,b}}\quad\forall
j\in{\cal J},\forall b\in{\cal B}_j\label{ct:5}
\end{eqnarray}


\noindent No batches assigned to the same machine may overlap and the
setup time between two consecutive batches on the same machine must be
respected.
\begin{eqnarray}{Cr}
S_{j,b}\geq S_{i,b'} + p_{i,b'}
+ s_{i,j}^{a_{j,b}} & \nonumber
\\ \forall (i,j) \in{\cal
  J}^2,\forall (b,b')\in{\cal B}_i\times{\cal B}_j \mbox{ such that } 
a_{i,b'}=a_{j,b}\label{ct:6}
\end{eqnarray}

\noindent All production must fit in the time horizon.
\begin{eqnarray}
S_{j,b}+p_{j,b}\leq T \quad \forall (j,b)\in{\cal B}\label{ct:7}
\end{eqnarray}

We need not introduce maintenance operations in the model. We consider
that there is a subset ${\cal J}^M$ of the jobs ${\cal J}$ that
correspond to maintenance operations. They concern a single machine
($|{\cal M}_j|=1$). The maintenance duration is given by $D_j$ and we
have $v^k_j=1$ for $k\in{\cal M}_j$. We denote by ${\cal B}^M$ the set
of batches corresponding to maintenance operations. We introduce last
the constraints that each maintenance job cannot be splitted:
\begin{eqnarray}
|{\cal B}_j|=1\quad\forall j\in{\cal J}^M\label{ct:8}
\end{eqnarray}
We assume that there is no required setup time between a maintenance
job and a production job. However me make the realistic assumption
that the maintenance duration is such that $D_j\geq s_{ii'}^k$, $\forall
j\in{\cal J}^M$, $\forall i,i'\in {\cal J}\setminus{\cal J}^M, i\neq
i', \forall k \in {\cal M}$, i.e. is larger than any other setup time. 
Furthermore we assume that all other setup times satisfy the triangle inequality
\begin{eqnarray}
s_{ij}^k\geq s_{il}^k+s_{lj}^k\quad\forall k\in{\cal M},\forall
i,j,l\in{\cal J}\setminus{\cal J}^M,i\neq j\neq l\nonumber
\end{eqnarray}

A solution is feasible if and only if it satisfies constraints
\ref{ct:2}--\ref{ct:8}.


There are two components linked to energy consumption. As the machines
have different speeds and different power consumptions, the first part
of the objective is to minimize the total electricity consumption $C$.
\begin{eqnarray}
C=\sum_{k\in{\cal M}} w_k\sum_{(j,b)\in{\cal A}_k\setminus{\cal
    B}^M}p_{j,b}\label{eq:obj2}
\end{eqnarray}
We are also interested in the maximum demand over the peak
periods. Let us define for convenience the set of batches in process
at a given time point $t$
\begin{eqnarray}
{\cal B}(t)=\{(j,b)\in{\cal B}|S_j,b\leq t <
S_{j,b}+p_{j,b}\}\label{eq:obj4}
\end{eqnarray}
Then the maximum demand over peak hours is equal to
\begin{eqnarray}
P_{\max}=\max_{h\in{\cal H}}\max_{t\in[a_h,b_h)}\sum_{k\in{\cal
      M},\exists(j,b)\in{\cal B}(t)\cap{\cal A}_k\setminus{\cal
      B}^M}w_k\label{eq:obj5}
\end{eqnarray}
Remark that maintenance operations are excluded in the computation of
these costs.  Then the problem objective can be written
\begin{eqnarray}
\min \alpha C + \beta P_{\max}\label{eq:obj}
\end{eqnarray}


Note that the problem is NP-hard as the problem of finding a solution
has the decision variant of the parallel machine problem as special
case.  The above-defined constraints can be used to check the
feasibility and cost of a solution in polynomial time but they cannot
be implemented as is in a mathematical programming solver.

We present a small example and its optimal solution. Consider $n=3$
jobs on $m=2$ lines with demands $D_1=10$, $D_2=12$, $D_3=6$. There is
one peak period $h=1$ with $a_h=2$ and $b_h=4$ and a common deadline
of $T=6$ for all jobs and maintenance operations. There are two
maintenance operations ${\cal J}^M=\{4,5\}$ of duration $D_4=2$ on
machine $1$ and $D_5=1$ on machine $2$ to be performed also before the
deadline. Machine power demands are $w_1=w_2=10$.  For each job,
authorized machines are ${\cal M}_1=\{1,2\}$, ${\cal M}_2=\{1\}$,
${\cal M}_3=\{2\}$, ${\cal M}_3=\{2\}$, ${\cal M}_4=\{1\}$, ${\cal
  M}_5=\{2\}$.  Setup and speed matrices are respectively equal to
$$(s_{i,j}^1)_{i,j\in{\cal J}}=(s_{i,j}^2)_{i,j\in{\cal
    J}}=\left[\begin{array}{ccc}0&2&1\\1 & 0 &
    1\\1&1&0\end{array}\right]$$
$$(v_j^k)_{j\in{\cal J},k\in{\cal
    M}}=\left[\begin{array}{cc}2&3\\6&0\\0&2\end{array}\right]$$
Figure \ref{fig_smallex} presents an optimal solution for this example
where ${\cal B}_1=\{1,2\}$ (job $1$ is split into two batches), ${\cal
  B}_2={\cal B}_3=1$ (jobs $2$ and $3$ are scheduled in one batch).
Batch machine assignments are given by $a_{1,1}=M_1,\ a_{1,2}=M_2,\
a_{2,1}=M_1$ and $a_{3,1}=M_2$. Productions
inside each batch are given by $q_{1,1}=4$, $q_{1,2}=6$, $q_{2,1}=12$,
$q_{3,1}=6$. According to the assigned machine speeds, the batch
duration are given by $p_{1,1}=q_{1,1}/v_1^1=2$,
$p_{1,2}=q_{1,2}/v_1^2=2$, $p_{2,1}=q_{2,1}/v_2^1=2$,
$p_{3,1}=q_{3,1}/v_3^2=3$. The start times of the different batches
and of the maintenance operations (in gray) are displayed in the Gantt
chart. According to the machine power demand, total consumption of
machine 1 is $4\times w_1=40$ and total consumption of machine 2 is
$5\times 10=50$. Hence we obtain a total energy consumption of 90 and
a maximal demand on peak hours of 10. We observe that batch splitting
has been necessary to meet the deadline and minimize energy
consumption, while maintenance has been used to avoid large demands
during peak hours.
\begin{figure}[htbp]
\centering 
\begin{tikzpicture}[scale=0.50]
    \begin{scope}[thick,font=\scriptsize]
\draw[->] (0,0) -- coordinate (x axis mid) (9,0);
    	\foreach \x in {0,...,8}
     		\draw (\x,1pt) -- (\x,-3pt)
			node[anchor=north] {\x};

\node at (-1,0.5) {$M_2$};
\draw (0,0) rectangle (2,0.8) node [pos=0.5] {(1,2)};
\draw[fill=gray] (2,0) rectangle (3,0.8);
\draw (3,0) rectangle (6,0.8) node [pos=0.5] {(3,1)};

\node at (-1,2.5) {$M_1$};
\draw (0,2) rectangle (2,2.8) node [pos=0.5] {(1,1)};
\draw[fill=gray] (2,2) rectangle (4,2.8);
\draw (4,2) rectangle (6,2.8) node [pos=0.5] {(2,1)};

%\draw (2,0) rectangle (3,0.8) node [pos=0.5] {2};
%\draw[fill=gray] (3,0) rectangle (5,0.8);
%\draw (5,0) rectangle (6,0.8) node [pos=0.5] {3} ;
%\draw[fill=gray] (6,0) rectangle (7,0.8);
%\draw (7,0) rectangle (8,0.8) node [pos=0.5] {1};

\draw (2,0) -- (2,3);
\draw (4,0) -- (4,3);
%\node at (1.5,0.4) {$J_1$};
%\node at (4,0.4) {$J_2$};
\end{scope}
 \end{tikzpicture}  
\caption{Optimal solution of a simple instance}
\label{fig_smallex}
\end{figure}


Hence, after a brief presentation of the related work, we will
introduce a mixed-linear integer programming formulation.


\section{Related work and modeling issues}
\label{sec:review}

A vast literature considers upper level lot-sizing and scheduling
problems in parallel machine environment with setup times (such as in
\cite{james2011single,xiao2013mip}) involving, among others, inventory
costs and also using discrete time mixed-integer linear programming
formulations. First, the discrete time models do not fit our problem
as the duration of the sublot is a function of the sublot size (a
continuous variable) and the machine speed for the family of balls
corresponding to the sublot. Furthermore we do not have holding costs
in our model. More generally, our problem is in fact related to the
(operational) production scheduling level where it is assumed that
lots of products aiming at satisfying the demand in a given period
have already been constituted at the (tactical) planning level.

Lot streaming (also called lot splitting) is a known technique to
improve the performance of production scheduling environment at the
operational level \cite{dauzere1997lot,chang2005comprehensive} by
allowing to split the jobs across the different production stages or
lines.  In our case, the lot streaming process in simplified by the
fact that we have a single stage. This is also the reason why we
rather use the ``batch sizing'' term rather than lot streaming or
splitting (we borrowed this term from \cite{Hazir2014}).

The possibility of splitting the jobs (lots/batches) adds another
dimension to the decisions besides assignment of the jobs to the
machine and sequencing/scheduling of the job operations. Many lot
streaming/batch sizing and scheduling problems have been considered in
the literature for different scheduling context, including parallel
machine scheduling for which efficient algorithms are available
\cite{serafini1996scheduling,xing2000parallel}. However, these first
studies did not consider the presence of setup times that can be
needed on a machine between two sublots. This feature, that occurs in
our case when the ball diameter is changed, puts obviously a limit to
the interest of systematic job splitting.  Several approaches can be
found for parallel machine lot streaming/batch sizing problems with
setup times
\cite{Yalaoui2003,tahar2006linear,beraldi2008rolling}. Compared to
these previous works, our study is concerned with electricity cost
minimization objectives. In manufacturing, energy constraints and
costs are becoming crucial. Consequently, energy-related objectives
have been recently considered both in lot-sizing problems
\cite{absi2013lot} and in production scheduling problems
\cite{mouzon2007operational,artigues2013energy,german2015}. However to
our knowledge, reducing the energy consumption cost by an integrated
management of lot streaming/batch sizing, setup and maintenance
scheduling in a parallel machine scheduling environment is a novel
approach.

In this paper, one of the objective is to come up with a mixed-integer
linear programming formulation (MILP) of the problem. In scheduling
problems, there are standardly three categories of MILP formulations
\cite{queyranne1994}: the continuous time formulations with sequencing
variables, the continuous time formulations with positional (or
event-based) variables and the discrete time formulations. In the
considered problem, the continuous lot splitting possibility and the
variable machine speed combined with large time horizons would render
a discrete time formulation (such as the ones generally used in the
lot-sizing literature) impractical. Therefore we need a continuous
time formulation. Now looking at the conceptual continuous time
formulation (\ref{ct:2}--\ref{eq:obj}), most non-linear constraints
could be linearized in a standard way yielding a continuous time (also
called disjunctive) formulation with sequencing variables
$x_{j,b,i,b'}\in\{0,1\}$ where $x_{j,b,i,b'}=1$ if and only if batch
$(i,b')$ is scheduled after batch $(j,b)$ on the same machine. In
\cite{neron2001}, the authors consider such a formulation for a simple
identical parallel machine scheduling problem with job-splitting and
sequence-dependent setup times with the makespan criterion. In this
simple case splitting a job is only useful if all sublots are
scheduled on different machines, hence the authors consider binary
sequencing variables of the type $x_{i,j,k}$ where $x_{i,j,k}=1$ is
the sublot of job $i$ in machine $k$ is sequenced before the sublot of
job $j$ in machine $k$.  However in our case, it can be necessary to
have two batches of the same job on the same machine, especially due
to the interest of avoiding production during the peak hours. Among
many other configurations Fig.~\ref{fig_contrex-1batch} shows an
optimal solution for a $1$ machine and $3$ jobs example ($m=1$ and
$n=3$) with setup times $s_{12}=s_{21}=s_{13}=s_{31}=1$,
$s_{23}=s_{32}=2$, unit machine speed, unit machine power, demands
$D_1=2$, $D_2=D_3=1$, deadline $T=8$ and weights $\alpha=0$,
$\beta=1$. Setup times are shown in gray.  The optimal solution (of
cost $0$) needs to split job 1 in two batches to have no production
within the peak hour (interval marked with a P).


\begin{figure}[htbp]
\centering \begin{tikzpicture}[scale=0.50]
    \begin{scope}[thick,font=\scriptsize]
\draw[->] (0,0) -- coordinate (x axis mid) (9,0);
    	\foreach \x in {0,...,8}
     		\draw (\x,1pt) -- (\x,-3pt)
			node[anchor=north] {\x};
\draw (0,0) rectangle (1,0.8) node [pos=0.5] {1};
\draw[fill=gray] (1,0) rectangle (2,0.8);
\draw (2,0) rectangle (3,0.8) node [pos=0.5] {2};
\draw[fill=gray] (3,0) rectangle (5,0.8);
\draw (5,0) rectangle (6,0.8) node [pos=0.5] {3} ;
\draw[fill=gray] (6,0) rectangle (7,0.8);
\draw (7,0) rectangle (8,0.8) node [pos=0.5] {1};
\node at (1.5,1.5) {O};
\node at (4,1.5) {P};
\node at (6.5,1.5) {O};

\draw (3,0) -- (3,2);
\draw (5,0) -- (5,2);
%\node at (1.5,0.4) {$J_1$};
%\node at (4,0.4) {$J_2$};
\end{scope}
 \end{tikzpicture}  
\caption{Two batches of a job on the same machine}
\label{fig_contrex-1batch}
\end{figure}


Another issue for modeling the problem in continuous time is the
linearization of the expressions (\ref{eq:obj4}, \ref{eq:obj5})
defining the maximum demand over the continuous horizon. An important
remark is that the electricity consumption only changes at a batch
start or end time.  This clearly brings us to consider an event-based
model. These models consider a finite number of events and use an
event-indexed continuous variable to associate a start time to each
event and a binary variable to assign a batch start or end
time. Consequently, we can also associate a continuous variable
representing the total electricity consumption at a given event, that
shall not change until the next event.  Note that event-based models
were previously used for lot streaming and scheduling in hybrid
flow-shops \cite{defersha2012mathematical} and are also widely used in
batch scheduling in the process industry \cite{floudas2005mixed}. They
were also more recently introduced in resource-constrained project
scheduling problems \cite{KALM10d} and in scheduling problems under
energy constraints \cite{nataff2015}.

\section{A continuous time event-based mixed-integer linear programming formulation}
\label{sec:milp}

The event-based model we propose for the problem, is based on the fact
that energy consumption can only change at the beginning or at the end
of each batch. Hence we fix for each job its maximum number of batches
$B_j$ and we consider a set ${\cal E}$ of $2\sum_{j=1}^nB_j$
events. Now the start-end event-based formulation of the problem
involves the following decision variables (variables with domain
$[0,1]$ are continuous variables that will be set binary by the
constraints involving other binary variables).

\begin{itemize}
\item $t_e\geq 0$ is the time of event $e\in{\cal E}$.
\item $x_{j,b}^e\in\{0,1\}$ is equal to $1$ iff batch $(j,b)\in{\cal
  B}$ starts at event $e\in{\cal E}$.
\item $y_{j,b}^e\in\{0,1\}$ is equal to $1$ iff batch $(j,b)\in{\cal
  B}$ ends at event $e\in{\cal E}$.
\item $a_{j,b}^k\in\{0,1\}$ is equal to $1$ iff batch $(j,b)\in{\cal
  B}$ is assigned to machine $k\in{\cal M}$.
\item $x_{j,b}^{e,k}\in[0,1]$ auxiliary variable equal to product
  $x_{j,b}^ea_{j,b}^k$.
\item $y_{j,b}^{e,k}\in[0,1]$ auxiliary variable equal to product
  $y_{j,b}^ea_{j,b}^k$.
\item $o_{e,k}\in[0,1]$ is equal to $1$ iff machine $k\in{\cal M}$ is
  on at event $e\in{\cal E}$.
\item $p_{j,b}\geq 0$ is the production of batch $(j,b)\in{\cal B}$.
\item $P_{e,h}\in\{0,1\}$ models the relative positioning of
  intervals $[t_e,t_{e+1})$ and peak hour interval $[a_h,f_h)$ in the
      sense that $P_{e,h}=0\Rightarrow t_e\geq f_h$.
\item $P_{h,e}\in\{0,1\}$ models the relative positioning of
  intervals $[t_e,t_{e+1})$ and peak hour interval $[a_h,f_h)$ in the
      sense that $P_{h,e}=0\Rightarrow t_{e+1}\leq a_h$.
\item $P_e^k\in[0,1]$ models the fact that event $e$ is active on a peak hour on machine
  $k$ (i.e. $[t_e,t_{e+1})$ has a non-empty intersection with a peak
  hour interval and machine $k$ is on at event $e$).
\item $W_{j,b}\geq 0$ total electrical consumption of batch $(j,b)$.
\end{itemize}

The constraints can be expressed as follows. $M$ denotes a
sufficiently large integer.
\begin{eqnarray}
t_e\leq t_{e+1}\quad\forall e\in{\cal E}\setminus\{|{\cal
  E}|\}\label{ct:see:1}
\end{eqnarray}
Each batch $(j,b)$ has to be scheduled at most once.
\begin{eqnarray}
\sum_{e\in{\cal E}}x_{j,b}^e\leq1 \quad \forall (j,b)\in{\cal B}
\end{eqnarray}
\begin{eqnarray}
\sum_{e\in{\cal E}}y_{j,b}^e\leq 1 \quad \forall (j,b)\in{\cal B}
\end{eqnarray}
A started batch must be ended, and reciprocally.
\begin{eqnarray}
\sum_{e\in{\cal E}}y_{j,b}^e=\sum_{e\in{\cal E}}x_{j,b}^e \quad
\forall (j,b)\in{\cal B}
\end{eqnarray}
The end event of a batch cannot be lower than its start event.
\begin{eqnarray}
\sum_{f\in{\cal E},f\geq e}x^f_{j,b}\geq \sum_{f\in{\cal E},f\geq
  e}y^f_{j,b}\quad \forall (j,b)\in{\cal B}, \forall e\in{\cal E}
\end{eqnarray}
A batch has to be assigned to a machine to have a non-zero production
(as $B_j$ is the maximum number of batch for job $j$, there can be
some unassigned batches).
\begin{eqnarray}
p_{j,b}\leq\sum_{k\in{\cal M}_j}a_{j,b}^kD_j\quad \forall
(j,b)\in{\cal B}
\end{eqnarray}
A batch assigned to a machine must have a minimum size.
\begin{eqnarray}
p_{j,b}\geq\sum_{k\in{\cal M}_j}a_{j,b}^k\epsilon\quad \forall
(j,b)\in{\cal B}
\end{eqnarray}

A batch has to be assigned to at most one machine among the candidate
machines.
\begin{eqnarray}
\sum_{k\in{\cal M}_j}a_{j,b}^k\leq 1\quad \forall (j,b)\in{\cal B}
\end{eqnarray}
The following constraints are set on auxiliary variables
$x_{j,b}^{e,k}$ and $y_{j,b}^{e,k}$ to linearize products
$x_{i,b}^ea_{j,b}^k$ and $y_{i,b}^ea_{j,b}^k$.
\begin{eqnarray}
x_{j,b}^{e,k}\geq x_{j,b}^e+a_{j,b}^k-1\quad\forall (j,b)\in{\cal
  B},\forall e\in{\cal E},\forall k\in{\cal M}
\end{eqnarray}
\begin{eqnarray}
x_{j,b}^{e,k}\leq x_{j,b}^e\quad\forall (j,b)\in{\cal B},\forall
e\in{\cal E},\forall k\in{\cal M}
\end{eqnarray}
\begin{eqnarray}
x_{j,b}^{e,k}\leq a_{j,b}^k\quad\forall (j,b)\in{\cal B},\forall
e\in{\cal E},\forall k\in{\cal M}
\end{eqnarray}
\begin{eqnarray}
y_{j,b}^{e,k}\geq y_{j,b}^e+a_{j,b}^k-1\quad\forall (j,b)\in{\cal
  B},\forall e\in{\cal E},\forall k\in{\cal M}
\end{eqnarray}
\begin{eqnarray}
y_{j,b}^{e,k}\leq x_{j,b}^e\quad\forall (j,b)\in{\cal B},\forall
e\in{\cal E},\forall k\in{\cal M}
\end{eqnarray}
\begin{eqnarray}
y_{j,b}^{e,k}\leq a_{j,b}^k\quad\forall (j,b)\in{\cal B},\forall
e\in{\cal E},\forall k\in{\cal M}
\end{eqnarray}
A machine is on at an event $e$ depending on its status at the
preceding event and/or start or completion of a task on this machine
(we define also $o_{0,k}=\sum_{(j,b) \in {\cal B}}x_{j,b}^{0,k}$,
$\forall k\in{\cal M}$).
\begin{eqnarray}{Cr}
o_{e,k}=o_{e-1,k}-\sum_{(j,b)\in{\cal B}} y_{j,b}^{e,k} \nonumber\\ +
\sum_{(j,b)\in{\cal B}} x_{j,b}^{e,k}&\forall e\in{\cal E},\forall
k\in{\cal M}
\end{eqnarray}
%A machine can only run one batch at a time.
%\begin{eqnarray}
%o_{e,k}\leq 1\quad \forall e\in{\cal E},\forall k\in{\cal M}
%\end{eqnarray}
The common deadline must be satisfied.
\begin{eqnarray}
t_{|{\cal E}|}\leq T\label{eq25}
\end{eqnarray}
Setup times must be satisfied between two consecutive batches.
\begin{eqnarray}
t_f\geq t_e + s_{ij}^k ( x_{j,b}^{f,k}+y_{i,b'}^{e,k}-1 ) \quad
\forall e,f\in{\cal E}, f\geq e,\nonumber\\ \quad \quad \quad
\quad\forall (j,b),(i,b')\in{\cal B},(j,b)\neq(i;b'), \forall
k\in{\cal M}
\end{eqnarray}
Batches can be assigned only on authorized machines.
\begin{eqnarray}
a_{j,b}^k=0\quad\forall(j,b)\in{\cal B},\forall k\in{\cal
  M}\setminus{\cal M}_j
\end{eqnarray}
End and start time events of a batch have to be spaced according to
the batch duration on the assigned machine.
\begin{eqnarray}
t_f\geq t_e+p_{j,b}/v_{j,k}-M
(2-x_{j,b}^{e,k}-y_{j,b}^{f,k})\nonumber\\ \quad \quad \quad
\quad\forall e,f\in{\cal E}, f\geq e,\forall (j,b)\in{\cal B},\forall
k\in{\cal M}
\end{eqnarray}
\begin{eqnarray}
t_f\leq t_e+p_{j,b}/v_{j,k}+M
(2-x_{j,b}^{e,k}-y_{j,b}^{f,k})\nonumber\\ \quad \quad \quad
\quad\forall e,f\in{\cal E}, f\geq e,\forall (j,b)\in{\cal B},\forall
k\in{\cal M}
\end{eqnarray}
The demand of each job must be satisfied.
\begin{eqnarray}
\sum_{b\in{\cal B}_j} p_{j,b}=D_j\quad\forall j\in{\cal J}
\end{eqnarray}
The following constraints set the relative positioning of event $e$
and peak hour period $h$, in the sense that if $P_{e,h}=0$,
intersection of $[t_e,t_e+1)$ and peak hour interval $[a_h,f_h)$ is
    empty because $t_e\geq f_h$
\begin{eqnarray}
t_e\geq f_h(1-P_{e,h})\quad\forall e\in{\cal E},\forall h\in{\cal H}
\end{eqnarray}
The following constraints set the relative positioning of event $e$
and peak hour period $h$, in the sense that if $P_{h,e}=0$,
intersection of $[t_e,t_e+1)$ and peak hour interval $[a_h,f_h)$ is
    empty because $t_{e+1}\leq a_h$
\begin{eqnarray}
t_{e+1}-a_h \leq T P_{h,e}\quad\forall e\in{\cal E},\forall h\in{\cal
  H}
\end{eqnarray}



Machine $k$ is marked as active during peak hour at event $e$ whenever
machine $k$ is on at event $e$ and interval $[t_e,t_{e+1})$ intersects
  peak hour interval $[a_h,f_h)$.
\begin{eqnarray}
P_{e}^k\geq P_{e,h}+P_{h,e}+o_{e,k}-2\nonumber\\ \quad \quad \quad
\quad\quad \quad \quad \quad \forall e\in{\cal E},\forall h\in{\cal
  H},\forall k\in{\cal M}
\end{eqnarray}
The maximum demand objective is defined by the following constraints.
\begin{eqnarray}
P_{\max}\geq \sum_{k\in{\cal M}} w_kP_{e}^k\quad\forall e\in{\cal E}
\end{eqnarray}
The consumption of a batch depends on its duration and on the power of
its assigned machine.
\begin{eqnarray}
W_{j,b}\geq w_k p_{j,b} / v_{j,k} - M (1-a_{j,b}^k)\nonumber\\ \quad
\quad \quad \quad\quad \quad \quad \quad\forall (j,b)\in{\cal
  B}\setminus{\cal B}^M,\forall k\in{\cal M}
\end{eqnarray}

The objective function aims at minimizing the total consumption and
the maximum demand on peak hours.
\begin{eqnarray}
\min \alpha\sum_{(j,b)\in{\cal B}\setminus{\cal B}^M}W_{j,b} + \beta
P_{\max}\label{ct:objsee1}
\end{eqnarray}

The MILP model made of constraints (\ref{ct:see:1}-\ref{ct:objsee1})
is denoted as (MILP1). We now present an alternative model that
contains more binary variables but less constraints. Product variables
$x_{j,b}^{e,k}$ and $y_{j,b}^{e,k}$ can be considered as binary and
assignment variables $a^k_{j,b}$, as well as start and end variables
$x_{j,b}^e$ and $y_{j,b}^e$ can be eliminated by setting:
$$ a^k_{j,b}=\sum_{e\in{\cal E}}x_{j,b}^{e,k}\quad\forall(j,b)\in{\cal
  B},\forall k\in{\cal M}$$
$$ x_{j,b}^e=\sum_{k\in{\cal M}}x_{j,b}^{e,k}\quad\forall(j,b)\in{\cal
  B},\forall e\in{\cal E}$$
$$ y_{j,b}^e=\sum_{k\in{\cal M}}y_{j,b}^{e,k}\quad\forall(j,b)\in{\cal
  B},\forall e\in{\cal E}$$

The obtained model is denoted by (MILP2).

If we come back to Figure \ref{fig_smallex} example. We can use 5
global events with $t_1=0$, $t_2=2$, $t_3=3$, $t_4=4$ and $t_5=6$. For
MILP2, we have $x_{1,1}^{1,1}=y_{1,1}^{2,1}=1$,
$x_{1,2}^{1,2}=y_{1,2}^{2,2}=1$, $x_{2,1}^{4,1}=y_{2,1}^{5,1}=1$,
$x_{3,1}^{3,2}=y_{3,1}^{5,2}=1$ and for maintenance operations
$x_{4,1}^{2,1}=y_{4,1}^{4,1}=1$ and
$x_{5,1}^{3,2}=y_{5,1}^{5,2}=1$. For peak computation the only
concerned event is $e=3$ and we have $P_3^2=1$ because
$P_{3,1}+P_{1,3}+o_{3,2}=1$ (the event is inside the peak and the
machine is on).

\section{A simplified model and a matheuristic}
\label{sec:simple}

The models (MILP1) and (MILP2) proposed in Section \ref{sec:milp} have
several drawbacks. First they have a large number of binary variables
and constraints, especially constraints involving big-$M$ coefficient
that significantly weaken the LP relaxation. Second, finding a
solution that ends before the common deadline can be difficult and
even impossible in practice.

If we ignore the maximum consumption during peak hours, we can
simplify the model, transforming the event-based model into a
positional model.  We consider a set ${\cal E}_k$ of maximum
$\sum_{j=1}^nB_j$ positions on each machine $k$ and the following
reduced set of variables, including a tardiness variable for each job
aiming at relaxing the hard common deadline and including it into the
objective function.
\begin{itemize}
\item $t_{e,k}\geq 0$ is the start time of batch scheduled at position
  $e\in{\cal E}_k$ on machine $k\in{\cal M}$.
\item $x_{j,b}^{e,k}\in\{0,1\}$ is equal to $1$ if batch
  $(j,b)\in{\cal B}$ is assigned to position $e$ on machine $k$
\item $p_{j,b}^{e,k}\geq 0$ is the production of batch $(j,b)\in{\cal
  B}$ if it is scheduled at position $e$ on machine $k$.
\item $T_k\geq 0$ is the tardiness of machine $k\in{\cal M}$ in the
  case it finishes its production after the common deadline.
\end{itemize}
The problem constraints can then be expressed as follows.  The start
times of tasks at each position are ordered.
\begin{eqnarray}
t_{e,k}\leq t_{e+1,k}\quad\forall k\in{\cal M},\forall e\in{\cal
  E}_k\setminus\{|{\cal E}_k|\}\label{ct:simpl:1}
\end{eqnarray}
Each batch $(j,b)$ has to be scheduled at most once.
\begin{eqnarray}
\sum_{ k\in{\cal M}_j}\sum_{e\in{\cal E}_k}x_{j,b}^{e,k}\leq 1 \quad
\forall (j,b)\in{\cal B}
\end{eqnarray}
Each event on each machine $(e,k)$ corresponds to at most one batch.
\begin{eqnarray}
\sum_{(j,b)\in{\cal B}} x_{j,b}^{e,k}\leq 1 \quad \forall k\in{\cal
  M}_j,\ \forall e\in{\cal E}_k
\end{eqnarray}
The production of a batch $(j,b)$ at a position of a machine is 0 if
the batch is not assigned at this position on this machine.
\begin{eqnarray}
p_{j,b}^{e,k}\leq D_jx_{j,b}^{e,k}\quad \forall (j,b)\in{\cal
  B},\forall k\in{\cal M}_j, \forall e\in{\cal E}_k
\end{eqnarray}
A batch assigned to a position on a machine must have a minimum size
of $\epsilon$.
\begin{eqnarray}
p_{j,b}^{e,k}\geq \epsilon x_{j,b}^{e,k}\quad \forall (j,b)\in{\cal
  B},\forall k\in{\cal M}_j, \forall e\in{\cal E}_k
\end{eqnarray}
The demand of each job must be satisfied.
\begin{eqnarray}
\sum_{b\in{\cal B}_j} \sum_{ k\in{\cal M}_j}\sum_{e\in{\cal E}_k}
p^{e,k}_{j,b}=D_j\quad\forall j\in{\cal J}
\end{eqnarray}
Setup times must be satisfied between two consecutive batches.
\begin{eqnarray}
t_{e+1,k}\geq t_{e,k} + \sum_{(j,b)\in{\cal B}} p_{i,b}^{e,k} /
v_{i,k} + s_{ij}^k (
\sum_{b=1}^{B_i}x_{i,b}^{e,k}+\sum_{b'=1}^{B_j}x_{j,b'}^{e+1,k}-1 )
\nonumber\\\quad \forall e\in{\cal E}_k\setminus|{\cal E}_k|, \forall
i,j\in {\cal J}, i\neq j, \forall k\in{\cal M}
\end{eqnarray}
The following constraint computes the violation of the deadline
(tardiness) on machine k (in case a job is assigned to the last event,
its duration must be added).
\begin{eqnarray}
T_k\geq t_{|{\cal E}_k|} +\sum_{(j,b)\in{\cal B}}p_{j,b}^{e,k}- T
\quad \forall k\in{\cal M}
\end{eqnarray}

The objective function aims at minimizing a weighted sum of the total
tardiness and the total consumption.
\begin{eqnarray}
\min \gamma \sum_{k\in{\cal M}}T_k+\sum_{(j,b)\in{\cal B}} \sum_{
  k\in{\cal M}_j}\sum_{e\in{\cal E}_k} w_k p_{j,b}^{e,k} / v_{j,k}
\end{eqnarray}
where $\gamma$ is a coefficient large enough to ensure that the total
tardiness will be minimized in priority.

This model, denoted as (MILP3) is greatly simplified, however the
solution ignores peak hours.  Consequently in a second phase, The
solution of (MILP3) can serve as a basis of a matheuristic for further
improvement of the computed peak cost. To that purpose we compute a
global event set ${\cal E}'=\cup_{k\in{\cal M}}{\cal E}_k$ and we sort
the events according to their time $t_e$.  Then we can accelerate the
solution time of (MILP1) or (MILP2) by preassigning the start and end
times of activities to the event set according to the solution found
by (MILP3).

\section{Computational experiments}
\label{sec:exp}

The experiments are conducted on an Intel Core i7-4770 processor with
4 cores and 8 gigabytes of RAM under the 64-bit Ubuntu 12.04 operating
system. We use CPLEX 12.6 with 1 thread for solving the models. The
total time limit of 3050 seconds for solving both (MILP2) and (MILP3)
models: 3000 seconds for (MILP3) and 50 seconds for (MILP2).

The instances are extracted from the real instances of
\cite{Urrutia:Thesis:2014} and are as follows. For these instances,
the number of machines is equal to $3$, the time horizon is equal to
the number of hours in each month (e.g. in January, this number is
$31\times 24= 744$) and the machine consumption is constant and equal
to $10 MW$ for each machine.

Each instance has $10$ different lots and the setup time and the
speed of each machine are described by Table~\ref{table1} and
Table~\ref{table2} respectively. In Table~\ref{table2}, whenever the
speed of the machine is equal to zero then the lots can not be
scheduled on this machine.

\begin{table}[!htb]
  \begin{center}
    \normalsize
    \begin{tabular}{|c|cccccccccc|}
      \hline 
      \backslashbox{i}{j} & 1 & 2 & 3 & 4 & 5 & 6 & 7 & 8 & 9 &
      10\\ 
      \hline 
      1&0& 4& 5& 6& 7& 8& 1& 1& 0& 1\\
      2&4& 0 &4& 5& 6& 7& 1& 1& 0& 1\\ 
      3&5 &4 &0 &4 &5 &6 &1 &1 &1 &1 \\
      4&6 &5 &4 &0 &4 &5 &0 &2 &2 &2 \\
      5&7 &6 &5 &4 &0 &4 &1 &3 &3 &3 \\
      6&8 &7 &6 &5 &4 &0 &1 &4 &3 &4 \\
      7&4 &4 &5 &4 &4 &5 &0 &5 &2 &5 \\
      8&4 &4 &5 &6 &7 &8 &5 &0 &2 &5 \\
      9&4 &4 &5 &6 &7 &8 &5 &3 &0 &3 \\
      10&4 &4 &5& 6& 7& 8 &5 &5 &3 &0\\
      \hline
    \end{tabular}
    \vspace{0.1cm}
    \caption{setup times between batches}
    \label{table1}
  \end{center}
\end{table}

\begin{table}[!htb]
  \normalsize
  \begin{center}
    \begin{tabular}{|c|ccc|}
      \hline \backslashbox{lot}{line} & 1 & 2 & 3\\ \hline 1 & 0 & 4 &
      0\\ 2 & 0 & 5.1 & 0\\ 3 & 0 & 6 & 0\\ 4 & 8.1 & 8.9 & 0\\ 5 &
      9.5 & 9 & 0\\ 6 & 11 & 8.8 & 0\\ 7 & 10.6 & 0 & 4\\ 8 & 0 & 0 &
      5.4\\ 9 & 0 & 0 & 8.5\\ 10 & 0 & 0 & 8.1\\ \hline
    \end{tabular}
    \vspace{0.1cm}
    \caption{machine speeds for each lot}
    \label{table2}
  \end{center}
\end{table}

For each instance, the maintenance duration is equal to $120$ for each
machine. As in Chile, peak hours occurs from 6pm to 11pm, we set the
peak hours to these values in our instances.
\begin{table*}[t]
  \small
  \centering
  \begin{tabular}{|c|cccc|cccc|}
    \hline 
    & \multicolumn{4}{|c|}{MILP3}&
                                   \multicolumn{4}{|c|}{MILP2}\\ 
    \hline 
    \#batch & consump. & tard. & gap & time(s) &
                                                 peak & tard. & consump. & gap\\ 
    \hline
    \rule[-5pt]{0pt}{10pt}
    2	&	16711,54	&	19,39	&	49,56	&	25,2	&	30	&	18,41	&	16692,6	&	59,88	\\
    \rule[-5pt]{0pt}{10pt}
    3	&	16703,83	&	21,43	&	49,71	&	25,49	&	30	&	22,44	&	16699,8	&	71,21	\\
    \rule[-5pt]{0pt}{10pt}
    4	&	16743,19	&	27,41	&	56,8	&	29,22	&	30	&	26,85	&	16750,71	&	97,18	\\
    \hline
  \end{tabular}
  \caption{Results of experiments for (MILP2+MILP3)}
  \label{table4}
\end{table*}

Unfortunately, both model (MILP1) and (MILP2) can not be used alone to
find an optimal solution. Indeed, the great number of constraints and
variables prevent the solver from finding a solution, even a feasible
one. Therefore, in order to solve the problem, we use first 
(MILP3) and then, we use the solution to help (MILP2) to find a
feasible solution.

At the beginning of the procedure, i.e. before using (MILP3), we
use a simple heuristic to help the model finding a solution. This
heuristic considers only one batch for each different lot and schedule
this batch on the least loaded machine. This heuristic is described by
algorithm~\ref{algo1}. Maintenance operations are randomly set in such
a way that, on each machine, the end of the operation occurs before
the deadline, i.e. the end of the planning horizon $T$.

\begin{algorithm}[english]
  \caption{Simple heuristic for finding a simple solution}
\label{algo1}
  $machLoad_k=0,\ \forall k \in {\cal M}$ 
  
  $aff_j=-1,\ \forall j \in {\cal J} $
  \For {every lot $j$}{
    
    $minLoad \gets \infty $ 
    
    $LeastLoadedMach \gets -1$ 
    \For {all machine $k$} {
      \If {$k \in {\cal M}_j$ and $machLoad_k<minLoad$} {
        
        $minLoad \gets machLoad_k$
        
        $LeastLoadedMach \gets k$ 
      }
    }
    
    $aff_j=LeastLoadedMach$ 
    
    $machLoaded_{aff_j}=machLoaded_{aff_j}+D_j/v_j^k$
  }
\end{algorithm}


After solving (MILP3), we use the solution found as a
basis solution of (MILP2). To make sure this solution is feasible,
we use a relaxation of (MILP2). Indeed, the solution of (MILP3) may
violate the deadline (constraints~(\ref{eq25})). Therefore, we remove
this constraint from the model and add the tardiness in the
objective. Then, the objective function of (MILP2) becomes:
\begin{eqnarray}
\min \alpha\sum_{(j,b)\in{\cal B}\setminus{\cal B}^M}W_{j,b} + \beta
P_{\max} + \gamma T_{ard}\label{ct:objsee1000}
\end{eqnarray}

where $T_{ard} \ge T-t_{|E|}$ and $T_{ard} \ge 0$. Coefficients
$\alpha,\ \beta$ and $\gamma$ are set in such a way that priority is
given to the minimization of the tardiness and then to the
minimization of the maximum peak hour consumption and of the total
consumption equivalently. Note that this relaxation can be seen 
as a bi-objective optimization problem.


Experiments have been done on $10$ instances and for different numbers
of batches, i.e. the number of sublots for each lot: $2$, $3$ and
$4$. 

Results are displayed in Table~\ref{table4}. The first column 
displays the number of batches. The second and seventh column show the
total consumption found by (MILP2) and (MILP3) respectively. The third
and eighth column present the tardiness of the solution returned by
(MILP2) and (MILP3) respectively. The fourth and ninth column display
the gap of the solution found by (MILP2) and (MILP3)
respectively. The fifth column presents the time needed to solve
(MILP2). For (MILP3), since none of the instances are solved
optimally, this time is equal to 3000s and is not displayed in the
table. For each number of batches, the first model solved 50\% of the
instances optimally. Finally, the maximum peak of consumption at the end of the
algorithm (after (MILP3)) is shown in column six.


As we can see, the model solves all the instances but not
optimally. (MILP2) solves half of the instances to optimal and (MILP3)
reduces either the total consumption or the tardiness for $2$ and $3$
batches. Another remark can be made about the results of the
experiments. Indeed, the peak of consumption is very high. This
is mostly due to the large size of the second model.

Table~\ref{table3}\footnote{in Appendix.} presents the detailed
results of experiments on eight instances. The column are as in
Table~\ref{table4}.
 
\section{Conclusion}
\label{sec:concl}

In this paper, we consider a real industrial problem. For this
problem, we present three different models. Both of them solve exactly
the problem (MILP1 and MILP2) whereas the last one only minimizes the
total consumption (MILP3). To solve the problem, we use first the
(MILP3) and we use the solution to help the (MILP2) to improve the
maximum consumption during peak hours.

The third model is quite efficient as it solved most of the instances very quickly but the second model does not improve the maximum consumption within
the time limit allowed. Therefore, an important amount of work is left
to be done. There are multiple research direction to pursue. Indeed,
an efficient solution method minimizing the maximum consumption during
peak hours needs to be developed. For this purpose, several heuristics
can be tested in order to improve the quality of the solution in the
beginning of the resolution or between the two MILP models.

Another development can be made by improving the modeling of the
problem with a Mixed Integer Linear Program in order to decrease the
number of constraints and/or variables to simplify the solution
procedure. Finally, we aim at designing extended formulations and to
develop column generation procedures.

\section*{Acknowledgment}

This work was funded by ECOS--CONICYT French-Chile cooperation program
C13E04.

\section*{Bibliography}
\sectionmark{Bibliography}
\begin{bibunit}[alpha]
\nocite{IESM}
\nocite*
\putbib[IESM]
\end{bibunit}

\newpage
\section*{Appendix}
\sectionmark{Appendix}
~


\begin{table}[!ht]
\small
    \hspace{2cm}
    \begin{tabular}{|c|c|cccc|cccc|}
      \hline 
      & & \multicolumn{4}{|c|}{MILP3}&
                                                        \multicolumn{4}{|c|}{MILP2}\\ 
      \hline 
      instance & \#batch & consump. & tard. & gap & time(s) &
      peak & tard.  & consump. & gap\\ 
      \hline
\rule[-7pt]{0pt}{15pt}  
      1	&	2	&	15409	&	0	&	0
                         &	0,69	&	30	&	0
                                                  &	15341,2	&
                                                                  49,44
      \\
\rule[-7pt]{0pt}{15pt} 
1	&	3	&	15388,6	&	0	&	0
                         &	0,86	&	30	&	0
                                                  &	15388,6	&
                                                                  49,6
      \\
\rule[-7pt]{0pt}{15pt} 
1	&	4	&	15341,8	&	0	&	0
                         &	1,63	&	30	&	0
                                                  &	15341,8	&
                                                                  100
      \\
      \hline
\rule[-7pt]{0pt}{15pt} 
2	&	2	&	17198	&	69,95	&	99,81
                         &	50	&	30	&	66,18
                                                  &	17207,4	&
                                                                  82,69
      \\
\rule[-7pt]{0pt}{15pt} 
2	&	3	&	17190	&	78,15	&	99,82	&	50	&	30	&	77,08	&	17192,7	&	91,38	\\
\rule[-7pt]{0pt}{15pt} 
2	&	4	&	17150,3	&	100,18	&	99,85	&	50	&	30	&	100,09	&	17150,6	&	96,26	\\
      \hline
\rule[-7pt]{0pt}{15pt} 
3	&	2	&	16798,1	&	28,11	&	99,41	&	50	&	30	&	28,11	&	16798,1	&	71,95	\\
\rule[-7pt]{0pt}{15pt} 
3	&	3	&	16798,1	&	28,11	&	99,41	&	50	&	30	&	28,11	&	16798,1	&	82,73	\\
\rule[-7pt]{0pt}{15pt} 
3	&	4	&	16798,1	&	28,11	&	99,41	&	50	&	30	&	28,11	&	16798,1	&	99,35	\\
      \hline
\rule[-7pt]{0pt}{15pt} 
4	&	2	&	17577,2	&	49,56	&	99,65	&	50	&	30	&	49,56	&	17465,9	&	78,76	\\
\rule[-7pt]{0pt}{15pt} 
4	&	3	&	17577,2	&	49,56	&	99,65	&	50	&	30	&	50,56	&	17522,2	&	88,13	\\
\rule[-7pt]{0pt}{15pt} 
4	&	4	&	17579,7	&	49,56	&	99,65	&	50	&	30	&	49,56	&	17557,2	&	100	\\
      \hline
\rule[-7pt]{0pt}{15pt} 
5	&	2	&	16253,9	&	0	&	0	&	0,26	&	30	&	0	&	16253,9	&	48,33	\\
\rule[-7pt]{0pt}{15pt} 
5	&	3	&	16253,9	&	0	&	0	&	0,58	&	30	&	14,31	&	16254,2	&	64,97	\\
\rule[-7pt]{0pt}{15pt} 
5	&	4	&	16253,9	&	0	&	0	&	1,57	&	X	&	X	&	X	&	X	\\
      \hline
\rule[-7pt]{0pt}{15pt} 
6	&	2	&	16624,3	&	0	&	0	&	0,28	&	30	&	0	&	16614,3	&	47,8	\\
\rule[-7pt]{0pt}{15pt} 
6	&	3	&	16624,3	&	0	&	0	&	1,37	&	30	&	0	&	16614,3	&	71,23	\\
\rule[-7pt]{0pt}{15pt} 
6	&	4	&	16624,3	&	0	&	0	&	1,35	&	30	&	0	&	16629,3	&	100	\\
      \hline
\rule[-7pt]{0pt}{15pt} 
7	&	2	&	18604,5	&	7,45	&	97,58	&	50	&	30	&	3,42	&	18632,7	&	50,02	\\
\rule[-7pt]{0pt}{15pt} 
7	&	3	&	18576,2	&	15,62	&	98,83	&	50	&	30	&	9,44	&	18593,1	&	64,27	\\
\rule[-7pt]{0pt}{15pt} 
7	&	4	&	18576,2	&	13,99	&	98,69	&	50	&	30	&	10,16	&	18646,1	&	84,66	\\
      \hline
\rule[-7pt]{0pt}{15pt} 
8	&	2	&	15227,3	&	0	&	0	&	0,36	&	30	&	0	&	15227,3	&	50,01	\\
\rule[-7pt]{0pt}{15pt} 
8	&	3	&	15222,3	&	0	&	0	&	1,04	&	30	&	0	&	15235,2	&	57,38	\\
\rule[-7pt]{0pt}{15pt} 
8	&	4	&	15131,9	&	0	&	0
                         &	1,57	&	30	&	0
                                                  &	15131,9	&
                                                                  100
      \\
\hline
    \end{tabular}
\centering
~

  \caption{Detailed results of experiments for (MILP2+MILP3)}
  \label{table3}
\end{table}



\newpage%
\thispagestyle{empty}

\vspace{2cm}
\noindent\hrulefill

{\bf Auteur: }Margaux NATTAF

{\bf Titre:} Ordonnancement sous contraintes d'énergie

{\bf Directeur de thèse: }Christian ARTIGUES et Pierre LOPEZ

Soutenue le 18/10/2016 au LAAS-CNRS (Toulouse) 

\noindent\hrulefill

{\bf Résumé: }Les problèmes d’ordonnancement à contraintes de
ressource ont été largement étudiés dans la littérature. Cependant,
dans la plupart des cas, il est supposé que les activités ont une
durée fixe et nécessitent une quantité constante de la ressource
durant toute leur exécution.  Dans cette thèse, nous nous proposons de
traiter un problème d'ordonnancement dans lequel les tâches ont une
durée et un profil de consommation de ressource variables. Ce profil,
qui peut varier en fonction du temps, est une variable de décision du
problème dont dépend la durée de la tâche associée. Par ailleurs, la
considération de fonctions de rendement linéaires et non linéaires
pour la représentation de l’utilisation des ressources complexifie le
problème et permet de modéliser de manière réaliste les transferts de
ressources énergétiques. Pour ce problème NP-complet, nous présentons
plusieurs propriétés permettant de dériver des modèles et méthodes de
résolution. Ces méthodes de résolution sont divisées en deux
parties. La première partie visualise ce problème du point de vue de
la Programmation Par Contraintes et plusieurs méthodes dérivées de ce
paradigme sont détaillées dont le développement du raisonnement
énergétique sur le problème étudié. La seconde partie de la thèse est
dédiée à des approches de Programmation Linéaire Mixte et plusieurs
modèles, notamment un modèle à temps continu basé sur les événements,
ainsi que des analyses théoriques et des techniques d’amélioration de
ces modèles sont présentés. Enfin, des expérimentations viennent
appuyer les résultats présentés dans ce manuscrit.

\noindent\hrulefill

{\bf Mots-clés: } ordonnancemment, énergie, recherche arborescente et
locale, programmation mathématique, propagation de
contraintes, complexité



\noindent\hrulefill

{\bf Discipline administrative: }Informatique

\noindent\hrulefill

{\bf Adresse du laboratoire :} 

Laboratoire d'Analyse et d'Architecture des Systèmes

7, avenue du Colonel Roche 

BP 54200 

31031 Toulouse cedex 4, France

\end{document}