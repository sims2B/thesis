\chapter*{Conclusion\markboth{CONCLUSION}{}}

Dans cette partie, nous avons présenté les résultats obtenus par les
expérimentations que nous avons conduites. Ces expérimentations ont
permis de valider et de comparer les méthodes présentées dans ce
manuscrit. 

Dans un premier temps, nous avons présenté la procédure utilisée pour
générer des instances hétérogènes du \CECSP~et les caractéristiques
des instances utilisées dans les expérimentations sur le \RCPSP. Pour
ce dernier, nous expliquons aussi comment sont précalculées les
fenêtres de temps dans lesquelles les activités doivent s'exécuter. 

Nous présentons ensuite les résultats obtenus par le modèle indexé par
le temps du \CECSP. Une comparaison des résultats avec et sans coupes
énergétiques est réalisée montrant l'intérêt de ces coupes. En effet,
même si l'ajout de ces coupes augmente le temps de résolution du
modèle, cela permet de trouver une première solution de meilleure
qualité que pour le modèle sans ces coupes. En effet, le nombre
important d'inégalités ajoutées au modèle rend la résolution plus
délicate mais ces dernières rendent le modèle plus fort. La mise en
place d'un algorithme de séparation permettant d'ajouter seulement une
partie de ces inégalités à chaque n\oe ud de l'arbre de branchement est
donc une piste de recherche intéressante. 

La sous-section suivante détaille les résultats obtenus par les modèles à
événements. Nous avons présenté, dans un premier temps, un algorithme
polynomial utilisé pour séparer les inégalités de non-préemption
définies dans la sous-section~\ref{sec:nonPreem}. Ensuite, les
résultats obtenus par le modèle Start/End pour résoudre le \CECSP~sont
détaillés. Ce dernier, bien qu'il ait de meilleures relaxations que le
modèle On/Off, ne permet pas de résoudre autant d'instances car il
comprend deux fois plus de variables binaires.

Les expérimentations conduites sur le modèle On/Off pour résoudre le
\CECSP~ont permis de valider les ensembles d'inégalités et de coupes
définies dans la section~\ref{sec:amelioration_modele}. En effet,
plusieurs sous-ensembles de ces inégalités ont successivement été
ajoutées au modèle et une comparaison des résultats a été effectuée,
montrant que les meilleures performances étaient obtenues quand ces
inégalités étaient ajoutées au modèle. De plus, l'intérêt de l'ajout
d'une partie de ces inégalités dans le cadre de la résolution du
\RCPSP~a aussi été démontré.

Enfin, une comparaison des résultats obtenus pour les trois modèles
a été effectuée dans le paragraphe suivant permettant de vérifier
expérimentalement que
le modèle indexé par le temps ne peut pas être utilisé comme une méthode
de résolution exacte pour le \CECSP. De plus, cette comparaison a
de nouveau permis de montrer que les résultats obtenus avec le modèle
On/Off était meilleurs que ceux obtenus par le modèle Start/End. 

La section suivante a été consacré aux résultats obtenus par les
algorithmes de filtrage pour le \CECSP. Ces derniers sont inclus dans
une méthode de branchement hybride, couplant règle de branchement et
modèle On/Off. Les expérimentations conduites ont permis de montrer
que cet algorithme obtenait de meilleurs résultats que le modèle
On/Off seul. De plus, l'intérêt et la complémentarité de l'algorithme
du raisonnement énergétique et celui de l'algorithme de vérification
basé sur les flots présentés dans le chapitre~\ref{sec:PLNE_CECSP} ont
été montrés. En effet, pour certaines familles d'instances
l'algorithme de flot obtient de meilleurs résultats et, pour d'autres
familles, c'est le raisonnement énergétique qui les obtient.

Enfin, cette partie a comparé les deux algorithmes de
filtrage sur des instances comportant des fonctions de rendement
concaves et affines par morceaux. Pour ces instances, difficiles à
résoudre, nous appliquons l'algorithme de branchement hybride avec un
des deux raisonnements puis avec les deux. Ceci nous permet de
démontrer que, dans ce cas, l'algorithme de vérification basé sur les
flots obtient de meilleures performances lorsqu'il est couplé avec le
raisonnement énergétique. La dernière information que ces expérimentations
nous permettent d'obtenir est que l'approximation des fonctions de
rendement par des fonctions affines peut conduire à l'infaisabilité de
l'instance alors que l'approximation par des fonctions concaves et
affines par morceaux permet de trouver une solution. Ce dernier point
permet, malgré la difficulté de résolution de ces instances, de
justifier la considération de fonctions de rendement concaves et
affines par morceaux. 