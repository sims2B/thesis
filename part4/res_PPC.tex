\section{Performances de la Programmation Par Contraintes}

Dans ce paragraphe, nous présentons les résultats des experimentations
portant sur les méthodes présentées dans le
chapitre~\ref{sec:PPC_CECSP}. Dans un premier temps, nous définissons
le cadre des expérimentations, notamment les algorithmes et
heuristiques de choix de variables utilisées. Puis, dans un second
temps, nous présenterons en détails les résultats numériques issus de
ces expérimentations. 

\subsection{Cadre des expérimentations}

Pour mesurer les performances relatives des différents raisonnements
présentés dans le chapitre~\ref{sec:PPC_CECSP}, nous les intégrons
dans une procédure de branchement
hybride~\cite{Nattaf_Constraints,Nattaf_ORSpectrum}. Cette procédure
se divise en deux temps. Dans un premier temps, une méthode 
arborescente est utilisée afin de réduire la taille des domaines des
début et fin des activités jusqu'à ce que chaque domaine ait une
taille inférieure à un certain paramètre $\epsilon >0$. Une fois les
domaines réduits suffisamment, nous utilisons le modèle à événement
On/Off afin de fixer les dates de début et de fin et de calculer, pour
chaque activité $i$, sa fonction d'allocation de ressource
$b_i(t)$. Nous rappelons que, pour ce faire, seule la valeur de
$b_i(t)$ à chaque début et fin d'activité doit être calculée
(conséquence du théorème~\ref{theo_LPM_CECSP}).

La procédure de branchement est inspirée du travail de Carlier{\it et
  al.}~\cite{Carlier}. Au début de cette procédure, une activité peut
commencer (respectivement finir) à tout instant $t \in [\ES,\LS{]}$
(resp. $t \in [\EE,\LE{]}$). L'idée de l'algorithme est donc, à chaque
noeud, de réduire la taille d'un de ces intervalles. Par exemple,
supposons que l'on ait choisi de réduire la taille du domaine de
$st_i$, alors deux nouveaux noeuds sont créés: le premier avec la
contrainte supplémentaire $st_i \in [\ES , (\ES+\LS) / 2]$ et le
second avec la contrainte $st_i \in ](\ES+\LS) / 2,\LS{]}$ (voir
figure~\ref{fig:branching}); et à chaque noeud, un des raisonnements
présentés au chapitre~\ref{PPC_CECSP} est appliqué.

\begin{figure}[!htb] 
  \centering
\begin{tikzpicture}
[yscale=0.6,xscale=0.8]
\tikzstyle{every node} = [align=center]
\node[ellipse,draw,text width=1.7cm] (R) at (0,0) {Problème $P$\\ $st_i \in [2,6]$};

\node[ellipse,draw,text width=1.7cm] (C1) at (-4,-4) {Problème $P1$\\ $st_i \in [2,4]$};
\node[ellipse,draw,text width=1.7cm] (C2) at (4,-4) {Problème $P2$\\ $st_i \in ]4,6]$};

\draw[->] (R.south west) -- (C1.north east);
\draw[->] (R.south east) -- (C2.north west);
\end{tikzpicture}

  \caption{Procédure de branchement du l'algorithme hybride du \CECSP.}
  \label{fib:branching}
\end{figure}

Cette procédure est répétée jusqu'à ce que tous les domaines de toutes
les variables soient plus petit qu'un certain paramètre $\epsilon
>0$. Quand ceci arrive, cela veut dire que nous sommes au niveau d'une
feuille de notre arbre de recherche et nous pouvons évaluer la
satisfiabilité du noeud courant. Pour cela, nous utilisons le modèle
On/Off, le plus efficace dans le cadre d'une résolution exacte, pour
tester si une solution avec les contraintes supplémentaires définies
le long de la procédure de branchement existe.

Si une telle solution existe alors l'algorithme s'arrête et nous avons
trouvé une solution pour l'instance de \CECSP~testée. Sinon, nous
faisons marche arrière dans l'arbre de recherche afin d'évaluer
d'autres feuilles de l'arbre, i.e. d'autres solutions potentielles. 

Le parcours de l'arbre est fait en suivant un parcours en profondeur
et, à chaque noeud,  la variable dont on va réduire le domaine est
choisie selon l'une des deux heuristiques suivantes: 
\begin{itemize}
\item nous choisissons la variable de plus petit domaine ou 
\item la variable est choisie de manière aléatoire.
\end{itemize}
De plus, la taille moyenne des domaines des variables étant de $32$,
les valeurs de $\epsilon$ testées sont $10$, $5$, $2.5$, $1$.

\subsection{Raisonnement énergétique}
\label{sec:expe_RE}

\subsection{Raisonnements basés sur le Time-Table}
\label{sec:expe_TT}

