\section{Performances de la Programmation Par Contraintes}
\label{sec:expe_PPC}
Dans ce paragraphe, nous présentons les résultats des experimentations
portant sur les méthodes présentées dans le
chapitre~\ref{sec:PPC_CECSP}. Dans un premier temps, nous définissons
le cadre des expérimentations, notamment les algorithmes et
heuristiques de choix de variables utilisées. Puis, dans un second
temps, nous présenterons en détails les résultats numériques issus de
ces expérimentations. 

\subsection{Cadre des expérimentations}
\label{sec:hybridBB}
Pour mesurer les performances relatives des différents raisonnements
présentés dans le chapitre~\ref{sec:PPC_CECSP}, nous les intégrons
dans une procédure de branchement
hybride~\cite{Nattaf_Constraints,Nattaf_ORSpectrum}. Cette procédure
se divise en deux temps. Dans un premier temps, une méthode 
arborescente est utilisée afin de réduire la taille des domaines des
début et fin des activités jusqu'à ce que chaque domaine ait une
taille inférieure à un certain paramètre $\epsilon >0$. Une fois les
domaines réduits suffisamment, nous utilisons le modèle à événement
On/Off afin de fixer les dates de début et de fin et de calculer, pour
chaque activité $i$, sa fonction d'allocation de ressource
$b_i(t)$. Nous rappelons que, pour ce faire, seule la valeur de
$b_i(t)$ à chaque début et fin d'activité doit être calculée
(conséquence du théorème~\ref{theo_LPM_CECSP}).

La procédure de branchement est inspirée du travail de Carlier{\it et
  al.}~\cite{Carlier}. Au début de cette procédure, une activité peut
commencer (respectivement finir) à tout instant $t \in [\ES,\LS{]}$
(resp. $t \in [\EE,\LE{]}$). L'idée de l'algorithme est donc, à chaque
noeud, de réduire la taille d'un de ces intervalles. Par exemple,
supposons que l'on ait choisi de réduire la taille du domaine de
$st_i$, alors deux nouveaux noeuds sont créés: le premier avec la
contrainte supplémentaire $st_i \in [\ES , (\ES+\LS) / 2]$ et le
second avec la contrainte $st_i \in ](\ES+\LS) / 2,\LS{]}$ (voir
figure~\ref{fig:branching}); et à chaque noeud, un des raisonnements
présentés au chapitre~\ref{sec:PPC_CECSP} est appliqué.

\begin{figure}[!htb] 
  \centering
\begin{tikzpicture}
[yscale=0.6,xscale=0.8]
\tikzstyle{every node} = [align=center]
\node[ellipse,draw,text width=1.7cm] (R) at (0,0) {Problème $P$\\ $st_i \in [2,6]$};

\node[ellipse,draw,text width=1.7cm] (C1) at (-4,-4) {Problème $P1$\\ $st_i \in [2,4]$};
\node[ellipse,draw,text width=1.7cm] (C2) at (4,-4) {Problème $P2$\\ $st_i \in ]4,6]$};

\draw[->] (R.south west) -- (C1.north east);
\draw[->] (R.south east) -- (C2.north west);
\end{tikzpicture}

  \caption{Procédure de branchement du l'algorithme hybride du \CECSP.}
  \label{fig:branching}
\end{figure}

Cette procédure est répétée jusqu'à ce que tous les domaines de toutes
les variables soient plus petit qu'un certain paramètre $\epsilon
>0$. Quand ceci arrive, cela veut dire que nous sommes au niveau d'une
feuille de notre arbre de recherche et nous pouvons évaluer la
satisfiabilité du noeud courant. Pour cela, nous utilisons le modèle
On/Off, le plus efficace dans le cadre d'une résolution exacte, pour
tester si une solution avec les contraintes supplémentaires définies
le long de la procédure de branchement existe.

Si une telle solution existe alors l'algorithme s'arrête et nous avons
trouvé une solution pour l'instance du \CECSP~testée. Sinon, nous
faisons marche arrière dans l'arbre de recherche afin d'évaluer
d'autres feuilles de l'arbre, i.e. d'autres solutions potentielles. 

Le parcours de l'arbre est fait en suivant un parcours en profondeur
et, à chaque noeud, la variable dont on va réduire le domaine est
choisie selon l'heuristique choisissant la variable de plus petit
domaine.  De plus, la taille moyenne des domaines des variables étant
de $32$, les valeurs de $\epsilon$ testées sont $10$, $5$, $2.5$.

\subsection{Raisonnement énergétique}
\label{sec:expe_RE}

Dans un premier temps, nous allons commencer par comparer les
différentes méthodes de calcul des intervalles d'ntérêt pour
l'algorithme de vérification de ce raisonnement présentées dans le
paragraphe~\ref{sec:intervalle_CECSP}. 

\subsubsection{Comparaison des méthodes de calcul des intervalles
  d'intérêt pour l'algorithme de vérification du raisonnement
  énergétique}

Les résultats comparant les trois méthodes de calcul des intervalles
d'intérêt de l'algorithme de vérification du raisonnement énergétique
sont présentés dans le tableau~\ref{tab:intervalle_CECSP}. La première
colonne correspond à l'algorithme naïf de calcul d'intersection des
segments de définition des fonctions de consommation individuelle des
activités. La seconde colonne correspond quant à elle au calcul de ces
mêmes intersections à l'aide de l'algorithme de balayage de
Bentley-Ottmann. Enfin, la troisième colonne présente les résultats de
l'adaptation de la méthode de calcul de Derrien {\it et al.}~\cite{DP}
pour le problème cumulatif dans le cadre du \CECSP. Toutes ces
méthodes sont décrites dans le
paragraphe~\ref{sec:intervalle_CECSP}.

L'algorithme de balayage fait partie de la librairie C++  CGAL
\footnote{\textsc{Cgal}, {C}omputational {G}eometry {A}lgorithms
  {L}ibrary. http://www.cgal.org.}. Pour calculer ces performances,
nous appliquons l'algorithme de vérification du raisonnement
énergétique et les ajustements correspondant sur les intervalles de
l'algorithme de vérifiaction seulement. Cet algorithme est appliqué
sur toutes les instances des Familles 1, 2, 3 et 4. Le temps est
représenté en millisecondes.

\begin{table}[ht] \centering
  \begin{tabular}{|>{\centering\arraybackslash}m{1.5cm}|>{\centering\arraybackslash}m{4cm}>{\centering\arraybackslash}m{4cm}>{\centering\arraybackslash}m{4cm}|}
    \hline \# tasks & méthode naïve & algorithme de balayage & adaptation
                                                               de
                                                               l'algorithme
                                                               de~\cite{DP}\\
    \hline 10 & 0,46 & 1,57 & 0,39 \\ 20 & 3,9 & 6,2 & 1,05 \\ 25 &
                                                                    7,18 & 7,50 & 1,73 \\ 30 & 11,54 & 11,78 & 3,06 \\ 60 & 45,97 &
                                                                                                                                    62,82 & 14,40 \\
    \hline
  \end{tabular}
  \caption{Comparaison des méthodes de calcul des intervalles
    d'intéret du raiosnnement énergétique.}
  \label{tab:intervalle_CECSP}
\end{table} 

La meilleure méthode permettant le calcul des intervalles d'intérêt de
l'algorithme de vérification du raisonnement énergétique est la
méthode adaptée de~\cite{DP}. En effet, un des avantages de cette
méthode était que le nombre d'intervalles à calculer était beaucoup
plus faible que dans les autres cas. Il paraît donc cohérent que cet
algorithme ait les mailleures performances. 

Cependant, il est moins naturel que l'algorithme par balayage ait de
moins bonnes performances que la méthode naïve. Ceci est du au fait
que la complexité de l'algorithme de balayage dépend du nombre
d'intersection que l'algorithme doit calculer. Or, dans ce cas, ce
nombre est très grand, ce qui ralentit grandement le temps de calucl
de l'algorithme. 

Dans le reste des expérimentations conduites sur le raisonnement
énergétique, nous utiliserons donc la méthode de calcul de~\cite{DP}. 

\subsubsection{Intégration du raisonnement énergétique dans
  l'algorithme de branchement hybrides}

Dans ce paragraphe, nous présentons les résultats des expérimentations
faites pour le raisonnement énergétique. Ce raisonnement est intégré
dans l'algorithme de branchement hybride décrit dans le
paragraphe~\ref{sec:hybridBB}. Cet algorithme a été testé avec
l'heuristique de sélection de variable qui choisit la variable de plus
petit domaine.

Le tableau~\ref{tab:res_BB_ER} présente les résultats obtenus avec 
pour valeur de $\epsilon$ $??$ . Cette valeur est choisie car c'est celle
qui permettait d'obtenir les meilleurs résultats. Les autres valeurs
de paramètres testées étant: $??$. Dans ce tableau, la première
colonne décrit le temps nécessaire pour résoudre les instances. Les
deuxièmes et troisième colonnes montrent le pourcentage de temps passé
dans la résolution du PLNE et dans l'arbre de recherche
respectivement. La quatrième colonne énonce le pourcentage d'instances
résolues. Les cinquième et sixièmes colonnes décrivent respectivement
le nombre de noeud contenus dans l'arbre de recherche et de programme
linéaire en nombre entiers résolus. Enfin, les deux dernières colonnes
exhibent le nombre de fois où l'algorithme de vérification a détecté
une incohérence et le nombre d'ajustements appliqués. 

\begin{table}[!htb]
  \begin{center}
    \begin{tabular}{|c|M{1.8cm}M{1.8cm}M{1.8cm}ccccc|}
      \hline
      \#act. &  \multicolumn{8}{c|}{Méthode de branchement hybride}\\ 
             &  \multicolumn{8}{c|}{$\epsilon=5$} \\ 
      \hline 
      \multicolumn{9}{|l|}{Famille 1}\\ 
      \hline 
             & Tps total(s) & Tps CPLEX(s) & Tps arbre(s) & \%solv.  & \#n\oe ud & \#PLNE &\#fail. & \#adj. \\ 
      \hline 
      $10$ & $0.28$ & $4.77$ & $0.22$ & $100$ & $0$ & $5$ & $0$ & $0$ \\ 
      $20$ & $25.9$ & $9.92$ & $0.069$ & $100$ & $1$ & $10$ & $0$ & $1$ \\ 
      $25$ & $143.7$ & $6.98$ & $0.01$ & $100$ & $1$ & $7$ & $0$ & $1$ \\ 
      $30$ & $1643.9$ & $5.74$ & $0.00$ & $87$ & $2$ & $6$ & $0$ & $4$ \\ 
      $0$ & $0$ & $0$ & $0$ & $0$ & $0$ & $0$ & $0$ & $0$ \\ 

      \hline 	
      \multicolumn{9}{|l|}{Famille 2}\\ 
      \hline 
             & Tps total(s) & Tps CPLEX(s) & Tps arbre(s) & \%solv.  & \#n\oe ud & \#PLNE& \#fail. & \#adj. \\ 
      \hline 
      $10$ & $0.28$ & $4.7$ & $0.2$ & $100$ & $0$ & $5$ & $0$ & $0$ \\ 
      $20$ & $25.2$ & $9.92$ & $0.06$ & $100$ & $1$ & $10$ & $0$ & $1$ \\ 
      $25$ & $143.7$ & $6.98$ & $0.01$ & $100$ & $1$ & $7$ & $0$ & $1$ \\ 
      $30$ & $1643.9$ & $5.74$ & $0.00$ & $87$ & $2$ & $6$ & $0$ & $4$ \\ 
      $0$ & $0$ & $0$ & $0$ & $0$ & $0$ & $0$ & $0$ & $0$ \\ 
      
      \hline 	\multicolumn{9}{|l|}{Famille 4}\\ 
      \hline 
             & Tps total(s) & Tps CPLEX(s) & Tps arbre(s) & \%solv.  & \#n\oe ud & \#PLNE &\#fail. & \#adj. \\ 
      \hline 
      $10$ & $0.28$ & $4.77$ & $0.22$ & $100$ & $0$ & $5$ & $0$ & $0$ \\ 
      $20$ & $25.91$ & $9.92$ & $0.06$ & $100$ & $1$ & $10$ & $0$ & $1$ \\ 
      $25$ & $143.76$ & $6.98$ & $0.01$ & $100$ & $1$ & $7$ & $0$ & $1$ \\ 
      $30$ & $1643.9$ & $5.74$ & $0.0$ & $87$ & $2$ & $6$ & $0$ & $4$ \\ 
      $0$ & $0$ & $0$ & $0$ & $0$ & $0$ & $0$ & $0$ & $0$ \\ 
      \hline 
    \end{tabular}
  \end{center}
  \caption{Résultats du raisonnement énergétique dans la méthode de
    branchement hybride pour le \CECSP.}
  \label{tab:res_BB_ER}
\end{table}

\subsection{Raisonnements basés sur le Time-Table}
\label{sec:expe_TT}

Dans ce paragraphe, nous montrons l'intérêt d'ajouter le
raisonnement basé sur un problème de flot et sur le raisonnement
Time-Table. Pour ce faire, nous le couplons avec un autre raisonnement
présenté dans ce manuscrit, le raisonnement énergétique. C'est-à-dire
qu'à chaque noeud de l'arbre de recherche, au lieu d'appliquer
seulement le raisonnement énergétique comme dans le paragraphe
précédent, nous appliquons l'algorithme de Time-Table/Flot et, dans
le cas où le test ne prouve pas l'infaisabilité du sous-problème, on
applique alors le raisonnement énergétique (le checker et les
ajustements). 

Pour ces expérimentations, nous avons aussi considéré l'heuristique
choisissant la variable de plus petit domaine.   Le
tableau~\ref{tab:res_BB_ERFlot} présente les résultats obtenus avec  
pour valeur de $\epsilon$ $??$. Les colonnes du tableau correspondent
à celle du tableau~\ref{tab:res_BB_ER}. 

\begin{table}[!htb]
  \begin{center}
    \begin{tabular}{|c|M{1.8cm}M{1.8cm}M{1.8cm}ccccc|}
      \hline
      \#act. &  \multicolumn{8}{c|}{Méthode de branchement hybride}\\ 
             &  \multicolumn{8}{c|}{$\epsilon=5$} \\ 
      \hline 
      \multicolumn{9}{|l|}{Famille 1}\\ 
      \hline 
             & Tps total(s) & Tps CPLEX(s) & Tps arbre(s) & \%solv.  & \#n\oe ud & \#PLNE &\#fail. & \#adj. \\ 
      \hline 
      $10$ & $0.28$ & $4.77$ & $0.22$ & $100$ & $0$ & $5$ & $0$ & $0$ \\ 
      $20$ & $25.9$ & $9.92$ & $0.069$ & $100$ & $1$ & $10$ & $0$ & $1$ \\ 
      $25$ & $143.7$ & $6.98$ & $0.01$ & $100$ & $1$ & $7$ & $0$ & $1$ \\ 
      $30$ & $1643.9$ & $5.74$ & $0.00$ & $87$ & $2$ & $6$ & $0$ & $4$ \\ 
      $0$ & $0$ & $0$ & $0$ & $0$ & $0$ & $0$ & $0$ & $0$ \\ 

      \hline 	
      \multicolumn{9}{|l|}{Famille 2}\\ 
      \hline 
             & Tps total(s) & Tps CPLEX(s) & Tps arbre(s) & \%solv.  & \#n\oe ud & \#PLNE& \#fail. & \#adj. \\ 
      \hline 
      $10$ & $0.28$ & $4.7$ & $0.2$ & $100$ & $0$ & $5$ & $0$ & $0$ \\ 
      $20$ & $25.2$ & $9.92$ & $0.06$ & $100$ & $1$ & $10$ & $0$ & $1$ \\ 
      $25$ & $143.7$ & $6.98$ & $0.01$ & $100$ & $1$ & $7$ & $0$ & $1$ \\ 
      $30$ & $1643.9$ & $5.74$ & $0.00$ & $87$ & $2$ & $6$ & $0$ & $4$ \\ 
      $0$ & $0$ & $0$ & $0$ & $0$ & $0$ & $0$ & $0$ & $0$ \\ 
      
      \hline 	\multicolumn{9}{|l|}{Famille 4}\\ 
      \hline 
             & Tps total(s) & Tps CPLEX(s) & Tps arbre(s) & \%solv.  & \#n\oe ud & \#PLNE &\#fail. & \#adj. \\ 
      \hline 
      $10$ & $0.28$ & $4.77$ & $0.22$ & $100$ & $0$ & $5$ & $0$ & $0$ \\ 
      $20$ & $25.91$ & $9.92$ & $0.06$ & $100$ & $1$ & $10$ & $0$ & $1$ \\ 
      $25$ & $143.76$ & $6.98$ & $0.01$ & $100$ & $1$ & $7$ & $0$ & $1$ \\ 
      $30$ & $1643.9$ & $5.74$ & $0.0$ & $87$ & $2$ & $6$ & $0$ & $4$ \\ 
      $0$ & $0$ & $0$ & $0$ & $0$ & $0$ & $0$ & $0$ & $0$ \\ 
      \hline 
    \end{tabular}
  \end{center}
  \caption{Résultats du Time-Table basé sur les flots dans la méthode de
    branchement hybride pour le \CECSP.}
  \label{tab:res_BB_ERFlot}
\end{table}

\subsection{Comparaison des différentes approches}

Dans ce paragraphe, nous allons comparer les résultats obtenus avec le
modèle On/Off avec ceux obtenus par la méthode de branchement
hybride. Ces résultats sont décrits dans le
tableau~\ref{tab:comp_OOBB}.

Dans ce tableau, les trois premières colonnes présentent les résultats
obtenus par le modèle On/Off et les trois suivantes ceux obtenus par
la méthode de branchement hybride.   
\begin{table}[!htb]
  \begin{center}
    \begin{tabular}{|c|ccc|}
      \hline
      \#tasks & \multicolumn{3}{c|}{Modèle On/Off}\\ 
      \hline 
              & time(s) &\%solv. &\%feas. \\ 
      \hline
      \multicolumn{4}{|c|}{Famille 1}\\
      \hline 
      $10 $& $0,22$ & $100$ & $100 $ \\ 
      $20 $& $11,56$ & $100$ & $100 $ \\ 
      $25 $& $58,79$ & $100$ & $100 $ \\ 
      $30 $& $1582,9$ & $80$ & $80 $ \\ 
      $60 $& $4969,51$ & $80$ & $80 $ \\ 
      \hline 
      \multicolumn{4}{|c|}{Famille 2}\\
      \hline 
      $10 $& $0,29$ & $100$ & $100 $ \\ 
      $20 $& $27,2$ & $100$ & $100 $ \\ 
      $25 $& $380,20$ & $100$ & $100 $ \\ 
      $30 $& $2634,3$ & $70$ & $70 $ \\ 
      $60 $& $6104,6$ & $50$ & $50 $ \\ 
      \hline 
      \multicolumn{4}{|c|}{Famille 3}\\
      \hline 
      $10 $& $0,46$ & $100$ & $100 $ \\ 
      $20 $& $658,4$ & $100$ & $100 $ \\ 
      $25 $& $1900,2$ & $77,78$ & $77,78 $ \\ 
      $30 $& $2819,7$ & $80$ & $80 $ \\ 
      $60 $& $6358,1$ & $20$ & $20 $ \\ 
      \hline 
    \end{tabular}
  \end{center}
  \caption{Comparaison des résultats du modèle On/Off et de ceux de la
    méthode de branchement hybride pour le \CECSP.}
  \label{tab:comp_OOBB}
\end{table}