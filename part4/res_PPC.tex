\section{Performances de la Programmation Par Contraintes}
\label{sec:expe_PPC}
Dans cette section, nous présentons les résultats des expérimentations
portant sur les méthodes présentées dans le
chapitre~\ref{sec:PPC_CECSP}. Dans un premier temps, nous définissons
le cadre des expérimentations, notamment les algorithmes et
heuristiques de choix de variables utilisées. Puis, dans un second
temps, nous présenterons en détail les résultats numériques issus de
ces expérimentations. 

\subsection{Cadre des expérimentations}
\label{sec:hybridBB}
Pour mesurer les performances relatives des différents raisonnements
présentés dans le chapitre~\ref{sec:PPC_CECSP}, nous les intégrons
dans une procédure de branchement
hybride~\cite{Nattaf_Constraints,Nattaf_ORSpectrum}. Cette procédure
se divise en deux temps. Dans un premier temps, une méthode 
arborescente est utilisée afin de réduire la taille des domaines des
début et fin des activités jusqu'à ce que chaque domaine ait une
taille inférieure à un certain paramètre $\epsilon >0$. Une fois les
domaines réduits suffisamment, nous utilisons le modèle à événements 
On/Off afin de fixer les dates de début et de fin et de calculer, pour
chaque activité $i$, sa fonction d'allocation de ressource
$b_i(t)$. Nous rappelons que, pour ce faire, seule la valeur de
$b_i(t)$ à chaque début et fin d'activité doit être calculée
(conséquence du théorème~\ref{theo_LPM_CECSP}).

La procédure de branchement est inspirée du travail de Carlier {\it et
  al.}~\cite{Carlier}. Au début de cette procédure, une activité peut
commencer (respectivement finir) à tout instant $t \in [\ES,\LS{]}$
(resp. $t \in [\EE,\LE{]}$). L'idée de l'algorithme est donc, à chaque
n\oe ud, de réduire la taille d'un de ces intervalles. Par exemple,
supposons que l'on ait choisi de réduire la taille du domaine de
$st_i$, alors deux nouveaux n\oe uds sont créés: le premier avec la
contrainte supplémentaire $st_i \in [\ES , (\ES+\LS) / 2]$ et le
second avec la contrainte $st_i \in ](\ES+\LS) / 2,\LS{]}$ (voir
figure~\ref{fig:branching}); et à chaque n\oe ud, un des raisonnements
présentés au chapitre~\ref{sec:PPC_CECSP} est appliqué.

\begin{figure}[!htb] 
  \input{part4/branching.tex}
  \caption{Procédure de branchement du l'algorithme hybride du \CECSP.}
  \label{fig:branching}
\end{figure}

Cette procédure est répétée jusqu'à ce que tous les domaines de toutes
les variables soient plus petits qu'un certain paramètre $\epsilon
>0$. Quand ceci arrive, cela veut dire que nous sommes au niveau d'une
feuille de notre arbre de recherche et nous pouvons évaluer la
satisfiabilité du n\oe ud courant. Pour cela, nous utilisons le modèle
On/Off, le plus efficace dans le cadre d'une résolution exacte, pour
tester si une solution avec les contraintes supplémentaires définies
le long de la procédure de branchement existe.

Si une telle solution existe alors l'algorithme s'arrête et nous avons
trouvé une solution pour l'instance du \CECSP~testée. Sinon, nous
faisons marche arrière dans l'arbre de recherche afin d'évaluer
d'autres feuilles de l'arbre, i.e. d'autres solutions potentielles. 

Le parcours de l'arbre est fait en suivant un parcours en profondeur
et, à chaque n\oe ud, la variable dont on va réduire le domaine est
choisie selon l'heuristique choisissant la variable de plus petit
domaine.  De plus, la taille moyenne des domaines des variables étant
de $32$, les valeurs de $\epsilon$ testées sont $10$, $5$, $2.5$.

\subsection{Raisonnement énergétique}
\label{sec:expe_RE}

Dans un premier temps, nous allons commencer par comparer les
différentes méthodes de calcul des intervalles d'intérêt pour
l'algorithme de vérification de ce raisonnement, présentées dans la
sous-section~\ref{sec:intervalle_CECSP}. 

\subsubsection{Comparaison des méthodes de calcul des intervalles
  d'intérêt pour l'algorithme de vérification du raisonnement
  énergétique}

Les résultats comparant les trois méthodes de calcul des intervalles
d'intérêt de l'algorithme de vérification du raisonnement énergétique
sont présentés dans le tableau~\ref{tab:intervalle_CECSP}. La première
colonne correspond à l'algorithme naïf de calcul d'intersection des
segments de définition des fonctions de consommation individuelle des
activités. La seconde colonne correspond quant à elle au calcul de ces
mêmes intersections à l'aide de l'algorithme de balayage de
Bentley-Ottmann. Enfin, la troisième colonne présente les résultats de
l'adaptation de la méthode de calcul de Derrien {\it et al.}~\cite{DP}
pour le problème cumulatif dans le cadre du \CECSP. Toutes ces
méthodes sont décrites dans la sous-section~\ref{sec:intervalle_CECSP}.

L'algorithme de balayage fait partie de la librairie C++  CGAL
\footnote{\textsc{Cgal}, {C}omputational {G}eometry {A}lgorithms
  {L}ibrary. http://www.cgal.org.}. Pour calculer ces performances,
nous appliquons l'algorithme de vérification du raisonnement
énergétique et les ajustements correspondant sur les intervalles de
l'algorithme de vérification seulement. Cet algorithme est appliqué
sur toutes les instances des Familles 1, 2, 3 et 4. Le temps est
représenté en millisecondes.

\begin{table}[ht] \centering
  \begin{tabular}{|>{\centering\arraybackslash}m{1.5cm}|>{\centering\arraybackslash}m{4cm}>{\centering\arraybackslash}m{4cm}>{\centering\arraybackslash}m{4cm}|}
    \hline \# act. & méthode naïve & algorithme de balayage & adaptation
                                                               de
                                                               l'algorithme
                                                               de~\cite{DP}\\
    \hline 10 & 0,46 & 1,57 & 0,39 \\ 20 & 3,9 & 6,2 & 1,05 \\ 25 &
                                                                    7,18 & 7,50 & 1,73 \\ 30 & 11,54 & 11,78 & 3,06 \\ 60 & 45,97 &
                                                                                                                                    62,82 & 14,40 \\
    \hline
  \end{tabular}
  \caption{Comparaison des méthodes de calcul des intervalles
    d'intéret du raisonnement énergétique.}
  \label{tab:intervalle_CECSP}
\end{table} 

La meilleure méthode permettant le calcul des intervalles d'intérêt de
l'algorithme de vérification du raisonnement énergétique est la
méthode adaptée de~\cite{DP}. En effet, un des avantages de cette
méthode était que le nombre d'intervalles à calculer était beaucoup
plus faible que dans les autres cas. Il paraît donc cohérent que cet
algorithme ait les meilleures performances. 

Cependant, il est moins naturel que l'algorithme par balayage ait de
moins bonnes performances que la méthode naïve. Ceci est dû au fait
que la complexité de l'algorithme de balayage dépend du nombre
d'intersections que l'algorithme doit calculer. Or, dans ce cas, ce
nombre est très grand, ce qui ralentit grandement le temps de calcul
de l'algorithme. 

Dans le reste des expérimentations conduites sur le raisonnement
énergétique, nous utiliserons donc la méthode de calcul adaptée de~\cite{DP}. 

\subsubsection{Intégration du raisonnement énergétique dans
  l'algorithme de branchement hybride}

Dans ce paragraphe, nous présentons les résultats des expérimentations
faites pour le raisonnement énergétique. Ce raisonnement est intégré
dans l'algorithme de branchement hybride décrit dans la
sous-section~\ref{sec:hybridBB}. Cet algorithme a été testé avec 
l'heuristique de sélection de variable qui choisit la variable de plus
petit domaine.

Le tableau~\ref{tab:res_BB_ER} présente les résultats obtenus avec 
pour valeur de $\epsilon=2,5$ . Cette valeur est choisie car c'est celle
qui permettait d'obtenir les meilleurs résultats. Les autres valeurs
de paramètres testées étant: $10$ et $5$. Dans ce tableau, la première
colonne décrit le temps nécessaire pour résoudre les instances. Les
deuxième et troisième colonnes montrent le pourcentage de temps passé
dans la résolution du PLNE et dans l'arbre de recherche
respectivement. La quatrième colonne énonce le pourcentage d'instances
résolues. Les cinquième et sixième colonnes décrivent respectivement
le nombre de noeuds contenus dans l'arbre de recherche et de
programmes linéaires en nombres entiers résolus. Enfin, la dernière
colonne exhibe le nombre d'ajustements appliqués.

\begin{table}[!htb]
  \begin{center}
    \begin{tabular}{|c|M{1.8cm}M{1.8cm}M{1.8cm}cccc|}
      \hline
      \#act. &  \multicolumn{7}{c|}{Méthode de branchement hybride}\\ 
             &  \multicolumn{7}{c|}{$\epsilon=2.5$} \\ 
      \hline 
      \multicolumn{8}{|l|}{Famille 1 }\\ 
      \hline 
             & Tps total(s) & Tps CPLEX(\%) & Tps arbre(\%) & \%solv.  & \#noeuds & \#PLNE  & \#adj. \\ 
      \hline 
$10$ & $0,09$ & $83,19$ & $16,81$ & $100$ & $25$ & $6$ & $6$ \\ 
$20$ & $660,21$ & $96,97$ & $3,03$ & $90$ & $3809$ & $20611$ & $2716$ \\ 
$25$ & $821,9$ & $98,3$ & $1,7$ & $88$ & $89$ & $10$ &  $95$ \\ 
$30$ & $112,58$ & $99,14$ & $0,86$ & $100$ & $102$ & $10$ &  $114$ \\ 
\hline 	
      \multicolumn{8}{|l|}{Famille 2 }\\ 
      \hline 
             & Tps total(s) & Tps CPLEX(\%) & Tps arbre(\%) & \%solv.
                                            & \#noeuds & \#PLNE& \#adj. \\ 
      \hline 
$10$ & $1109,5$ & $81,2$ & $18,75$ & $100$ & $46783$ & $140276$ & $65446$ \\ 
$20$ & $728,04$ & $97,55$ & $2,45$ & $90$ & $1930$ & $9296$  & $3851$ \\ 
$25$ & $1825,15$ & $98,7$ & $1,3$ & $75$ & $495$ & $761$ &  $458$ \\ 
$30$ & $1971,81$ & $99,62$ & $0,38$ & $75$ & $150$ & $5$ &  $659$ \\ 
      \hline 	
      \multicolumn{8}{|l|}{Famille 4 }\\ 
      \hline 
             & Tps total(s) & Tps CPLEX(\%) & Tps arbre(\%) & \%solv.  & \#noeuds & \#PLNE  & \#adj. \\ 
      \hline 
$10$ & $0,19$ & $87,55$ & $12,45$ & $100$ & $25$ & $5$ &  $33$ \\ 
$20$ & $1617,07$ & $98,97$ & $1,03$ & $77$ & $68$ & $9$ & $246$ \\ 
$25$ & $104,9$ & $99,6$ & $0,4$ & $100$ & $66$ & $6$  & $252$ \\ 
$30$ & $1749,76$ & $99,72$ & $0,27$ & $77$ & $107$ & $9$ &  $424$ \\ 
      \hline 
    \end{tabular}
  \end{center}
  \caption{Résultats du raisonnement énergétique dans la méthode de
    branchement hybride pour le \CECSP.}
  \label{tab:res_BB_ER}
\end{table}

Tout d'abord, nous pouvons remarquer que l'algorithme de branchement
hybride permet de résoudre des instances jusqu'à $30$ activités en
moins de $2000$ secondes. De plus, nous pouvons voir qu'une bonne
partie du temps de résolution est passé dans le programme linéaire
mixte. Enfin, le raisonnement énergétique permet de procéder à un
nombre important d'ajustements. 

\subsection{Raisonnements basés sur le Time-Table}
\label{sec:expe_TT}

Dans cette sous-section, nous montrons l'intérêt d'ajouter le
raisonnement basé sur un problème de flot et sur le raisonnement
Time-Table. Pour ce faire, à chaque noeuds de l'arbre de recherche,
nous appliquons l'algorithme de Time-Table/Flot pour détecter une
incohérence. 

Pour ces expérimentations, nous avons aussi considéré l'heuristique
choisissant la variable de plus petit domaine. Le
tableau~\ref{tab:res_BB_ERFlot} présente les résultats obtenus avec
pour valeur de $\epsilon=2,5$. Les colonnes du tableau correspondent
à celle du tableau~\ref{tab:res_BB_ER}.

\begin{table}[!htb]
  \begin{center}
    \begin{tabular}{|c|M{1.8cm}M{1.8cm}M{1.8cm}ccc|}
      \hline
      \#act. &  \multicolumn{6}{c|}{Méthode de branchement hybride}\\ 
             &  \multicolumn{6}{c|}{$\epsilon=2.5$} \\ 
      \hline 
      \multicolumn{7}{|l|}{Famille 1 }\\ 
      \hline 
             & Tps total(s) & Tps CPLEX(\%) & Tps arbre(\%) & \%solv.  & \#noeuds & \#PLNE \\ 
      \hline 
$10$ & $0,12$ & $51,5$ & $48,5$ & $100$ & $25$ & $6$ \\ 
$20$ & $5,0$ & $88,7$ & $11,3$ & $100$ & $58$ & $12$ \\
$25$ & $26,6$ & $85,8$ & $14,2$ & $100$ & $79$ & $10$ \\
$30$ & $70,8$ & $96,69$ & $3,3$ & $100$ & $100$ & $11$ \\
\hline 	
      \multicolumn{7}{|l|}{Famille 2 }\\ 
      \hline 
             & Tps total(s) & Tps CPLEX(\%) & Tps arbre(\%) & \%solv.  & \#noeuds & \#PLNE\\
      \hline 
$10$ & $0,17$ & $63,0$ & $37,0$ & $100$ & $24$ & $6$ \\
$20$ & $9,03$ & $86,5$ & $13,5$ & $100$ & $59$ & $12$ \\
$25$ & $81,7$ & $94,1$ & $5,9$ & $100$ & $86$ & $10$ \\
$30$ & $118,9$ & $96,7$ & $3,3$ & $100$ & $106$ & $11$ \\
     \hline 	
\multicolumn{7}{|l|}{Famille 4 }\\ 
      \hline 
             & Tps total(s) & Tps CPLEX(\%) & Tps arbre(\%) & \%solv.  & \#noeuds & \#PLNE \\
      \hline 
$10$ & $0,29$ & $77,3$ & $22,7$ & $100$ & $27$ & $6$ \\
$20$ & $1476,1$ & $93,2$ & $6,8$ & $80$ & $98$ & $10$ \\
$25$ & $2568,5$ & $90,2$ & $9,8$ & $66$ & $18973$ & $8$ \\
$30$ & $3181,3$ & $98,3$ & $1,7$ & $60$ & $20908$ & $8$ \\
      \hline 
    \end{tabular}
  \end{center}
  \caption{Résultats du Time-Table basé sur les flots dans la méthode de
    branchement hybride pour le \CECSP.}
  \label{tab:res_BB_ERFlot}
\end{table}

L'algorithme de branchement, combiné au raisonnement basé sur les
flots, permet de résoudre des instances du \CECSP~avec $30$ activités en
moins de $7200$ secondes. De plus, nous pouvons voir que, comme dans
le cas du raisonnement énergétique, une bonne
partie du temps de résolution est aussi passé dans le programme linéaire
mixte. Le raisonnement basé sur les flots permet aussi d'obtenir de
meilleurs résultats que le raisonnement énergétique pour les Familles 2
et 1 et de moins bons résultats pour la Famille 4. La recherche de
propriétés inhérentes au problème permettant de savoir quel 
raisonnement appliquer est une direction de recherche intéressante. 

\subsection{Comparaison des différentes approches}

Dans cette sous-section, nous allons comparer les résultats obtenus avec le
modèle On/Off avec ceux obtenus par la méthode de branchement
hybride. Ces résultats sont décrits dans le
tableau~\ref{tab:comp_OOBB}.

Dans ce tableau, les deux premières colonnes présentent les résultats
obtenus par le modèle On/Off et les deux suivantes ceux obtenus par
la méthode de branchement hybride.   
\begin{table}[!htb]
  \begin{center}
    \begin{tabular}{|c|cc|cc|}
      \hline
      \#act. & \multicolumn{2}{c|}{Modèle On/Off}&
                                                    \multicolumn{2}{c|}{Algo. hybride}\\ 
      \hline 
              & tps(s) &\%solv. & tps & \%solv.\\ 
      \hline
      \multicolumn{5}{|c|}{Famille 1}\\
      \hline 
      $10 $& $0,22$ & $100$ & $0,12$ &  $100$ \\ 
      $20 $& $11,56$ & $100$ & $5,0$ & $100$ \\ 
      $25 $& $58,79$ & $100$ & $26,6$ &$100 $ \\ 
      $30 $& $1582,9$ & $80$ & $70,8$ & $100 $ \\  
      \hline 
      \multicolumn{5}{|c|}{Famille 2}\\
      \hline 
      $10 $& $0,29$ & $100$ & $0,17$ &$100 $ \\ 
      $20 $& $27,2$ & $100$ & $9,03$&$100 $ \\ 
      $25 $& $380,20$ & $100$ & $81,7$ &$100 $ \\ 
      $30 $& $2634,3$ & $70$ & $118,9$ &$100 $ \\  
      \hline 
    \end{tabular}
  \end{center}
  \caption{Comparaison des résultats du modèle On/Off et de ceux de la
    méthode de branchement hybride pour le \CECSP.}
  \label{tab:comp_OOBB}
\end{table}

Dans le tableau~\ref{tab:comp_OOBB}, nous pouvons voir que, pour les
Familles 1 et 2, l'algorithme de branchement hybride permet de
résoudre plus d'instances que le modèle On/Off en un temps moins
important. Dans certains cas, la résolution devient $20$ fois plus
rapide grâce à l'utilisation de cette méthode. 

Nous allons maintenant présenter les quelques résultats obtenus pour
les instances de la famille $LP$ (avec des fonctions de rendement
concaves et affines par morceaux). Cependant, comme seules les
instances à $10$ activités ont pu être résolues, nous présentons
uniquement ces résultats. Le tableau~\ref{tab:BB_LPM} présente
ces résultats avec une limite de temps fixée à $3600$ secondes et
$\epsilon=5$. La première ligne du tableau décrit les résultats de la
méthode arborescente hybride avec le raisonnement basé sur les flots,
la seconde avec le raisonnement énergétique et la dernière avec les
deux raisonnements combinés. Pour chaque ligne, la première colonne
correspond au temps nécessaire à la résolution des instances. La
seconde exhibe le temps passé dans l'arbre de branchement. Enfin, la
troisième et quatrième colonne présentent respectivement le nombre de
PLNE résolus et le nombre de noeuds dans l'arbre.


\begin{table}[!htb]
\centering 
\begin{tabular}{|c|cccc|}
\hline
 & Tps Total (s)  & Tps arbre (s) & \#PLNE & \#noeuds\\
\hline
TTFlot &3100,8 &0,01&	1	& 8\\
RE&2315,9&0,07&	14,8	&40\\
RE + TTFlot &1709,9&	0,14&	9,25	&37,75\\
\hline
\end{tabular}
\caption{Résultats de la méthode de branchement arborescente sur les
  instances avec fonctions de rendement concaves et affines par
  morceaux. }
\label{tab:BB_LPM}
\end{table}

Dans le tableau~\ref{tab:BB_LPM}, nous pouvons voir que le
raisonnement basé sur l'algorithme de flot est moins efficace
lorsqu'il est utilisé sans le raisonnement énergétique mais lorsque
les deux raisonnements sont combinés ils deviennent plus efficace que
l'un ou l'autre utilisé seul. 

Nous pouvons également remarquer est que, lors de la
résolution, la plupart du temps est passé dans la recherche d'une
solution au PLNE. De ce fait, l'amélioration des performances de cette
méthode devra obligatoirement passer par une amélioration des
performances du PLNE. 

Pour les instances de la Famille $L$, une solution a été trouvée pour
seulement $75\%$ des instances. Donc, quand on compare ces résultats
avec ceux obtenus pour la Famille $LP$, on peut conclure que, dans un
cas sur quatre, approximer la fonction de rendement par une fonction
affine rend l'instance infaisable. Ceci permet, malgré la difficulté
de résolution de ces instances, de justifier la considération de
fonctions de rendement concaves et affines par morceaux.




