%TODO : fonction LPM et résultats time index RCPSP avc RE?
\section{Génération des instances et pré-traitement}

Ce paragraphe présente, dans un premier temps, la méthode utilisée
pour générer des ensembles d'instances de test pour le \CECSP. Nous
détaillons ensuite les jeux d'instances du \RCPSP sur lesquelles nous
avons évalué l'impact des améliorations proposées pour les modèles à
événements. Nous présenterons aussi une méthode de calcul des fenêtres
de temps pour le \RCPSP, appliqué sur les instances comme
pré-traitement à leur résolution.

\subsection{Instances du \CECSP}
\label{sec:instance_CECSP}
Les instances utilisées dans le cadre du \CECSP sont regroupées en
quatre familles selon les caractéristiques de leurs fonctions de
rendements et de l'énergie que les activités requièrent. Dans un
premier temps, les instances sont générées selon un modèle commun,
puis plusieurs types de modifications spécifiques sont appliquées sur
celles-ci afin de les séparées en différentes familles. 

Tout d'abord, cinq instances de $10$ et $60$ activités ainsi que dix
instances de respectivement $20$, $25$ et $30$ activités sont
générées. Pour toutes ces instances, la disponibilité de la ressource
est fixée à $R=10$ et les autres paramètres du problème sont générés
de façon aléatoire selon une loi uniforme et dans les intervalles
suivants:
\begin{itemize}
\item $W_i \in [1 , \frac{5}{4} * R]$,
\item $\bmin \in [0, \frac{1}{4} * W_i]$,
\item $\bmax \in [\bmin , 2 * \bmin {]}$,
\item $\ES \in [0,\frac{1}{2}*n]$ et
\item $\LE \in [ \EE, \EE + n ]$ avec $\EE = \ES +
\frac{W_i}{f_i(\bmax)}$.
\end{itemize}

Une première famille d'instance, appelée Famille 4, est construit en
utilisant comme fonction de rendement la fonction identité pour toutes
les activités, i.e. $\forall i \in \A,\ f_i(b_i(t))= b_i(t)$. Ensuite,
des fonctions de rendement affines sont générées aléatoirement pour
chaque activité. Pour cela, nous générons, pour chaque activité $i \in
\A$, les paramètres $a_i$ et $c_i$ de la fonctions $f_i$ suivant une
loi uniforme et telle que:
\begin{itemize}
\item $a_i \in [1,10]$ et
\item $c_i \in [1,10]$.
\end{itemize} La valeur de $W_i$ est ensuite modifiée à l'aide de ces
fonctions $f_i$. Les trois autres familles d'instances sont classées
suivant cette modification. Pour la première famille, appelée Famille
1, $W_i$ est généré aléatoirement, selon une loi uniforme, dans
l'intervalle $[0,f_i(W_i)]$ et pour la Famille 2 dans l'intervalle
$[f_i(W_i)/2, f_i(W_i)]$. Enfin, pour la Famille 3, la valeur
$f_i(W_i)$ est affectée $W_i$. 

Nous avons donc défini quatre familles d'instances pour le \CECSP. Ces
instances serviront à calculer les performances relatives des méthodes
présentées dans les chapitres précédents de ce manuscrit. 

Dans le prochain paragraphe, nous présentons les instances utilisées
pour évaluer les performances des améliorations des modèles du
\RCPSP~ainsi qu'une méthode de calcul des fenêtres de temps.
  
\subsection{Instances et pré-calcul des fenêtres de temps pour le
  \RCPSP}
\label{sec:instances_RCPSP}
\subsubsection{Génération des instances}
Les instances utilisées sont les instances définies par
Kone~\cite{theseOumar} pour évaluer la performances des modèles à
événements par rapport aux modèles indexés par le temps. Ces instances
sont générées à partir des instances à $30$ activités de la
PSPLIB~\cite{PSPLIB}. Ces instances, au nombre de $480$, sont des
instances à $4$ ressources renouvelables et avec des activités ayant
une durée aléatoire comprise entre $1$ et $10$ unités de temps. Ces
durée étant relativement courtes, les modèles indexés par le temps
sont particulièrement efficaces sur ces dernières (voir
paragraphe~\ref{sec:motiv_event_RCPSP}). L'auteur de~\cite{theseOumar}
a donc modifié ces instances afin d'allonger la durée des
activités. En effet, dans certaines applications pratiques (notamment
dans le domaine pharmaceutique ou de la pétrochimie), il arrive que
les activités aient des durées relativement longues.

Les instances de Kone sont crées selon le principe suivant: 
\begin{enumerate}
\item les $15$ premières activités non-fictives de l’instance sont
  sélectionnées (les autres activités ainsi que les contraintes de
  précédence qui leur sont adjacentes sont laissées de côté).
\item parmi les activités sélectionnées, celles sans prédécesseurs
  sont connectées à l’activité 0 et celles sans successeurs à l’activité
  $16$.
\item  $7$ activités parmi les $15$ sélectionnées sont ensuite
  choisies de façon aléatoire et leur durée est multipliée par un
  coefficient $25+b$, où $b$ est un nombre aléatoire généré entre $0$
  et  $1$.

  Les durées obtenues sont ensuite arrondies au nombre entier le plus
  proche. 
\end{enumerate}
$480$ instances de taille plus réduite sont obtenues ($n=15$), mais
avec des durées opératoires plus grandes, allant ainsi de $1$ à
approximativement $250$ unités de temps. 

\subsubsection{Calcul des fenêtres de temps}

Afin d'améliorer les performances des modèles pour le \RCPSP, nous
calculons par prétraitement, pour chaque activité, les fenêtres de
temps $[\ES,\LS{]}$ et $[\EE,\LE{]}$ dans lesquelles elle peut
respectivement commencer et finir. Pour calculer ces fenêtres de
temps, rappelons que, si chaque arc $(i,j)$ du graphe des précédences
$G$ est pondéré par $p_i$, la date de début au plus tôt de $i$, $\ES$
peut prendre la valeur du plus long chemin entre l'activité $0$ et
l'activité $i$ et la date de début au plus tard $\LS$ est égale à
$\ES[n+1]$, la date de début au plus tôt de l'activité terminale
$n+1$, moins la valeur du plus long chemin entre $i$ et $n+1$.

Notons que la date de fin au plus tard de l'activité fictive $n+1$
correspondant à une borne supérieure sur la date de fin du projet peut
être calculée au moyen d'une heuristique. L'heuristique utilisée pour
calculer cette valeur ici est la méthode d'ordonnancement parallèle
avec comme règle de priorité la plus petite date de fin au plus tard
des activités~\cite{heur_RCPSP}. 

En plus des calculs décrits ci-dessus, nous utilisons d'autres
techniques de déductions basées sur la propagation de contraintes. En
particulier, nous utilisons des techniques décrites dans la thèse de
Demassey~\cite{these_Sophie}. Parmi celles utilisées, nous trouvons
les techniques de sélection immédiate, Edge-Finding sur les cliques de
disjonction et de triplets symétriques. Ces techniques n'étant pas
détaillées ici, à l'exception du Edge-Finding au
paragraphe~\ref{sec:nrj_CUSP}, nous renvoyons le lecteur au chapitre 2
de~\cite{these_Sophie} pour une description de ces techniques.
